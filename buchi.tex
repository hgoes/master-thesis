\section{Büchi-Automaten}
Büchi-Automaten stellen eine Erweiterung von endlichen Automaten auf unendliche Eingaben dar\cite{buchibasics}.
Anders als endliche Automaten, die eine Eingabe akzeptieren, wenn die Ausführung in einem finalen Zustand endet, akzeptiert ein Büchi-Automat eine Eingabe genau dann, wenn die Ausführung unendlich oft einen finalen Zustand erreicht.

Formal ist ein Büchi-Automat ein Tupel
\[ A = (Q,\Sigma,\delta,q_0,F) \]
wobei die Symbole folgende Bedeutung haben:
\begin{itemize}
  \item $Q$ ist die Menge der Zustände des Automaten.
  \item Das Alphabet $\Sigma$, über dem der Automat definiert ist.
  \item $\delta : Q\times\Sigma\rightarrow Q$ ist die Übergangsfunktion des Automaten.
  \item $q_0\subseteq Q$ ist die Startzustandsmenge.
  \item $F\subseteq Q$ ist die Finalzustandsmenge.
\end{itemize}
Ein Wort $a_0a_1a_2\dots$ wird nun also genau dann akzeptiert, wenn es eine Folge von Zuständen $q_0q_1q_2\dots$ gibt, wobei stets gilt $\delta(q_n,a_n)=q_{n+1}$ und in der mindestens ein Zustand aus $F$ unendlich oft vorkommt.