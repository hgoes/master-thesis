\section{Büchi-Automaten}
Büchi-Automaten stellen eine Erweiterung von endlichen Automaten auf unendliche Eingaben dar~\cite{buchibasics}.
Sie wurden erstmals 1966 von dem Mathematiker Julius Richard Büchi eingeführt~\cite{buchi}.
Anders als endliche Automaten, die eine Eingabe akzeptieren, wenn die Ausführung in einem finalen Zustand endet, akzeptiert ein Büchi-Automat eine Eingabe genau dann, wenn die Ausführung unendlich oft einen finalen Zustand erreicht.

Formal ist ein Büchi-Automat ein Tupel
\[ A = (Q,\Sigma,\delta,\mu,q_0,F) \]
wobei die Symbole folgende Bedeutung haben:
\begin{itemize}
  \item $Q$ ist die Menge der Zustände des Automaten.
  \item Die Menge von atomaren Aussagen $\Sigma$, die gültig oder ungültig sein können.
  \item $\delta\subseteq Q\times Q$ ist die Übergangsrelation des Automaten.
  \item $\mu : Q\rightarrow2^\Sigma$ gibt an, welche Aussagen in einem Zustand gelten müssen.
  \item $q_0\subseteq Q$ ist die Startzustandsmenge.
  \item $F\subseteq Q$ ist die Finalzustandsmenge.
\end{itemize}
Eine Folge von Aussagen $a_0a_1a_2\dots$ wird nun also genau dann akzeptiert, wenn es eine Folge von Zuständen $q_0q_1q_2\dots$ gibt, wobei stets gilt $q_n\delta q_{n+1}$ und in der mindestens ein Zustand aus $F$ unendlich oft vorkommt.
Außerdem muss jede Aussage $a_n$ mit der Menge von Aussagen $\mu(q_n)$ kompatibel sein, das heißt $a_n\in \mu(a_n)$.

\subsection{Verallgemeinerter Büchi-Automat}
\label{sec:gba}
Für das Übersetzungsverfahren in Abschnitt \ref{sec:ltl-translation} sind Büchi-Automaten wegen ihrer Beschränkung auf eine Finalzustandsmenge ungeeignet.
Algorithmen wie \cite{Gerth95simpleon-the-fly} verwenden daher eine abgewandelte Form von Büchi-Automaten.

Ein Verallgemeinerter Büchi-Automat\footnote{Im englischen als "`generalized buchi automaton"' bezeichnet und als \emph{GBA} abgekürzt.} unterscheidet sich von einem normalen Bü\-chi-Au\-to\-ma\-ten durch die Definition des Akzeptanzverhaltens.
Während ein Büchi-Automat akzeptiert, wenn unendlich oft ein Finalzustand betreten wird, ist $F$ hier eine Menge von Finalzustandsmengen ($F\subseteq\mathcal{P}(Q)$).
Der Automat akzeptiert nun, wenn aus jeder Finalzustandsmenge mindestens ein Zustand unendlich oft betreten wird.

Ein verallgemeinerter Büchi-Automat lässt sich in einen normalen Büchi-Automaten übersetzen, indem man für jede Finalzustandsmenge eine "`Ebene"' einführt.
Jede Ebene enthält die gleichen Zustände und Transitionen wie der ursprüngliche Automat, nur dass beim Verlassen von den Finalzuständen der aktuellen Finalzustandsmenge die nächste Ebene betreten wird.
Dadurch wird erreicht, dass ein Zyklus nur dann zustande kommt, wenn ein Zustand aus allen Finalzustandsmengen betreten wird.

Für einen verallgemeinerten Büchi-Automaten $(Q,\Sigma,\delta,\mu,q_0,F)$ konstruiert man also einen Büchi-Automaten $(Q',\Sigma,\delta',\mu',q_0',F')$.
Ohne Beschränkung der Allgemeinheit kann $F=\{f_0,f_1,\dots,f_{n-1}\}$ angenommen werden.
Die neue Zustandsmenge enthält nun eine Kopie der ursprünglichen Zustände für jede Finalzustandsmenge.
\[ Q' = \{ (q,f)\ |\ q\in Q, f\in F \} \]
Die Übergangsfunktion $\delta'$ kopiert zunächst alle ursprünglichen Übergänge und fügt dann Übergänge zwischen den Ebenen hinzu, wenn der Ursprungszustand in der aktuellen Finalzustandsmenge liegt.
\[ \delta' = \{ ((q_1,f_i),(q_2,f_j))\ |\ (q_1,q_2)\in\delta, (q_1\in f_i\land j=(i+1)\bmod n)\lor (q_1\not\in f_i\land i=j)\} \]
Die atomaren Aussagen, die in den Zuständen gelten müssen werden einfach übernommen.
\[ \mu'((q,f)) = \mu(q) \]
Die Finalzustandsmenge besteht aus den Zuständen, die in der Finalzustandsmenge der aktuellen Ebene liegen.
\[  F' = \{ (q,f)\ |\ f\in F, q\in f \} \]