

Die Implementierung besteht zum einen aus der eigentlichen Anwengung -- \emph{gtl} -- und zum anderen aus verschiedenen Bibliotheken, die zusätzlich entwickelt werden.
Diese sind:
\begin{itemize}
\item \emph{language-promela} -- Stellt Datenstrukturen für den Promela-Syntax bereit, formatiert Promela-Quelltext für die Ausgabe und parst Promela-Code.
\item \emph{language-scade} -- Ein Parser und Code-Generator für den SCADE-Syntax.
\item \emph{bdd} -- Eine Bibliothek, die binäre Entscheidungsdiagramme ("`binary decision diagrams"' -- BDD) implementiert.
\end{itemize}

\begin{figure}[h]
  \centering
  \begin{tikzpicture}
  \node[tape,draw,tape bend top=none] (gtl) at (0,0) {GTL};
  \node[rectangle,draw] (parser) at (0,-1) {Parser};
  \node[rounded rectangle,draw] (gtl ast) at (0,-2) {GTL AST};
  \node[rectangle,draw] (type checker) at (0,-3) {Type checker};
  \node[tape,draw,tape bend top=none] (scade) at (4,-3) {SCADE};
  \node[rounded rectangle,draw] (gtl spec) at (0,-4) {GTL Spec};
  \node[tape,draw,tape bend top=none] (promela1) at (-3,-5) {Promela};
  \node[cylinder,draw,shape border rotate=90,aspect=0.25] (cudd) at (-5,-6) {CUDD};
  \node[cloud,draw,cloud ignores aspect,inner sep=0em] (dyn bdd) at (-3,-8) {\begin{tabular}{c}Dynamic\\BDD\\Verifier\end{tabular}};
  \node[tape,draw,tape bend top=none] (promela2) at (0,-5) {Promela};
  \node[rectangle,draw] (kcg) at (4,-4) {KCG};
  \node[tape,draw,tape bend top=none] (c) at (4,-5) {C};
  \node[cloud,draw,cloud ignores aspect,inner sep=0em] (native) at (0,-8) {\begin{tabular}{c}Native\\Verifier\end{tabular}};
  \node[tape,draw,tape bend top=none] (scade2) at (2,-5) {\begin{tabular}{c}SCADE\\Testnode\end{tabular}};
  \node[cloud,draw,cloud ignores aspect,inner sep=0em] (scade verifier) at (6,-7) {\begin{tabular}{c}SCADE\\Verifier\end{tabular}};
  \draw[->] (gtl) -- (parser);
  \draw[->] (parser) -- (gtl ast);
  \draw[->] (gtl ast) -- (type checker);
  \draw[->] (scade) -- (type checker);
  \draw[->] (type checker) -- (gtl spec);
  \draw[->] (gtl spec) -| (promela1);
  \draw[->] (gtl spec) -- (promela2);
  \draw[->] (gtl spec) -| (scade2);
  \draw[->] (cudd) -| (dyn bdd);
  \draw[->] (promela1) -- (dyn bdd);
  \draw[->] (scade) -- (kcg);
  \draw[->] (kcg) -- (c);
  \draw[->] (promela2) -- (native);
  \draw[->] (c) |- (native);
  \draw[->] (scade) -| (scade verifier);
  \draw[->] (scade2) |- (scade verifier);
\end{tikzpicture}

  \caption{GTL Implementierung}
  \label{fig:gtl_implementation}
\end{figure}

Abbildung \ref{fig:gtl_implementation} zeigt den Datenfluss der \emph{gtl}-Anwendung.
Zunächst wird mithilfe des Parsers eine textuelle GTL-Repräsentation in einen abstrakten Syntax-Baum\footnote{englisch: abstract syntax tree, AST} transformiert.
Der Parser wird im Abschnitt \ref{module:Language.GTL.Parser} beschrieben, der Syntax-Baum in \ref{module:Language.GTL.Parser.Syntax}.
Daraufhin wird der Syntax-Baum an die Typüberprüfung weiter gereicht.
Diese extrahiert die Typinformationen aus den verwendeten synchronen Komponenten und überprüft, ob alle Kontrakte und Verifikationsformeln wohl-getypt sind (Siehe Abschnitt \ref{sec:sos}).
Für das SCADE-Backend müssen also die Dateien mit den Beschreibungen der SCADE-Komponenten geparst werden, die Modelle in dem entstandenen Syntax-Baum gefunden werden und die SCADE-Typen in GTL-Typen umgewandelt werden.
Daraufhin wird der Syntax-Baum in eine Instanz des \emph{GTLSpec}-Datentyps umgewandelt (Beschrieben in Abschnitt \ref{module:Language.GTL.Model}).

Ab hier entscheidet sich nun, welche Transformation vom Benutzer gewählt wurde.
Für die SCADE-Verifikation der Komponenten wird für jede Komponente ein SCADE-Observer erzeugt, der dann zusammen mit dem Quelltext der Modelle mit dem SCADE Design-Verifier geprüft wird (Beschrieben in Abschnitt \ref{module:Language.GTL.Backend.Scade}).

Für die C-Übersetzung wird der SCADE Code-Generator KCG aufgerufen, der wie in Abschnitt \ref{sec:c_integration} beschrieben C-Code für alle Komponenten liefert.
Es wird dann Promela-Code generiert, der die einzelnen C-Code-Modelle vereint.
Die Implementierung dieses Verfahrens wird in Abschnitt \ref{module:Language.GTL.PromelaCIntegration} genauer beschrieben.

Für die Übersetzung der Kontrakte mithilfe von binären Entscheidungsdiagrammen, wie in Abschnitt \ref{sec:bdd} beschrieben, wird Promela-Code generiert und dann gegen die CUDD-Bibliothek gelinkt.
Genauere Details des Verfahrens sind in Abschnitt \ref{module:Language.GTL.PromelaDynamicBDD} angegeben.