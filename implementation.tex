Die Implementierung besteht zum einen aus der eigentlichen Anwengung -- \emph{gtl} -- und zum anderen aus verschiedenen Bibliotheken, die zusätzlich entwickelt werden.
Diese sind:
\begin{itemize}
\item \emph{language-promela} -- Stellt Datenstrukturen für den Promela-Syntax bereit, formatiert Promela-Quelltext für die Ausgabe und parst Promela-Code.
\item \emph{language-scade} -- Ein Parser und Code-Generator für den SCADE-Syntax.
\item \emph{bdd} -- Eine Bibliothek, die binäre Entscheidungsdiagramme ("`binary decision diagrams"' -- BDD) implementiert.
\end{itemize}

\begin{figure}[h]
  \centering
  \begin{tikzpicture}
  \node[tape,draw,fill=d1!40,tape bend top=none] (gtl) at (0,0) {GTL};
  \node[rectangle,draw,fill=d2!40] (parser) at (0,-1) {Parser};
  \node[rounded rectangle,draw,fill=d3!40] (gtl ast) at (0,-2) {GTL AST};
  \node[rectangle,draw,fill=d2!40] (type checker) at (0,-3) {Type checker};
  \node[tape,draw,fill=d1!40,tape bend top=none] (scade) at (4,-3) {SCADE};
  \node[rounded rectangle,draw,fill=d3!40] (gtl spec) at (0,-4) {GTL Spec};
  \node[tape,draw,fill=d1!40,tape bend top=none] (promela1) at (-3,-5) {Promela};
  \node[cylinder,draw,fill=d5!40,shape border rotate=90,aspect=0.25] (cudd) at (-5,-6) {CUDD};
  \node[cloud,draw,fill=d4!40,cloud ignores aspect,inner sep=0em] (dyn bdd) at (-3,-8) {\begin{tabular}{c}Dynamic\\BDD\\Verifier\end{tabular}};
  \node[tape,draw,fill=d1!40,tape bend top=none] (promela2) at (0,-5) {Promela};
  \node[rectangle,draw,fill=d2!40] (kcg) at (4,-4) {KCG};
  \node[tape,draw,fill=d1!40,tape bend top=none] (c) at (4,-5) {C};
  \node[cloud,draw,fill=d4!40,cloud ignores aspect,inner sep=0em] (native) at (0,-8) {\begin{tabular}{c}Native\\Verifier\end{tabular}};
  \node[tape,draw,fill=d1!40,tape bend top=none] (scade2) at (2,-5) {\begin{tabular}{c}SCADE\\Testnode\end{tabular}};
  \node[cloud,draw,fill=d4!40,cloud ignores aspect,inner sep=0em] (scade verifier) at (6,-7) {\begin{tabular}{c}SCADE\\Design\\Verifier\end{tabular}};
  \draw[->] (gtl) -- (parser);
  \draw[->] (parser) -- (gtl ast);
  \draw[->] (gtl ast) -- (type checker);
  \draw[->] (scade) -- (type checker);
  \draw[->] (type checker) -- (gtl spec);
  \draw[->] (gtl spec) -| (promela1);
  \draw[->] (gtl spec) -- (promela2);
  \draw[->] (gtl spec) -| (scade2);
  \draw[->] (cudd) -| (dyn bdd);
  \draw[->] (promela1) -- (dyn bdd);
  \draw[->] (scade) -- (kcg);
  \draw[->] (kcg) -- (c);
  \draw[->] (promela2) -- (native);
  \draw[->] (c) |- (native);
  \draw[->] (scade) -| (scade verifier);
  \draw[->] (scade2) |- (scade verifier);
\end{tikzpicture}

  \caption{GTL Implementierung}
  \label{fig:gtl_implementation}
\end{figure}

Abbildung \ref{fig:gtl_implementation} zeigt den Datenfluss der \emph{gtl}-Anwendung.
Zunächst wird mithilfe des Parsers eine textuelle GTL-Repräsentation in einen abstrakten Syntax-Baum\footnote{englisch: abstract syntax tree, AST} transformiert.
Der Parser wird im Abschnitt \ref{module:Language.GTL.Parser}, Modul "`\verb|Language.GTL.Parser|"' beschrieben, der Syntax-Baum in \ref{module:Language.GTL.Parser.Syntax}, Modul "`\verb|Language.Parser.Syntax|"'.
Daraufhin wird der Syntax-Baum an die Typüberprüfung weiter gereicht.
Diese extrahiert die Typinformationen aus den verwendeten synchronen Komponenten und überprüft, ob alle Kontrakte und Verifikationsformeln wohl-getypt sind (Siehe Abschnitt \ref{sec:sos}).
Für das SCADE-Backend müssen also die Dateien mit den Beschreibungen der SCADE-Komponenten geparst werden, die Modelle in dem entstandenen Syntax-Baum gefunden werden und die SCADE-Typen in GTL-Typen umgewandelt werden.
Daraufhin wird der Syntax-Baum in eine Instanz des \emph{GTLSpec}-Datentyps umgewandelt (Beschrieben in Abschnitt \ref{module:Language.GTL.Model} und implementiert durch das Modul "`\verb|Language.GTL.Model|"').

Ab hier entscheidet sich nun, welche Transformation vom Benutzer gewählt wurde.
Für die SCADE-Verifikation der Komponenten wird für jede Komponente ein SCADE-Observer erzeugt, der dann zusammen mit dem Quelltext der Modelle mit dem SCADE Design-Verifier geprüft wird (Beschrieben in Abschnitt \ref{module:Language.GTL.Backend.Scade}, Modul "`\verb|Language.GTL.Backend.Scade|"').
Die Verwendung von anderen synchronen Formalismen wird durch das Interface in Modul "`\verb|Language.GTL.Backend|"' in Abschnitt \ref{module:Language.GTL.Backend} ermöglicht.

Für die C-Übersetzung wird der SCADE Code-Generator KCG aufgerufen, der wie in Abschnitt \ref{sec:c_integration} beschrieben C-Code für alle Komponenten liefert.
Es wird dann Promela-Code generiert, der die einzelnen C-Code-Modelle vereint.
Die Implementierung dieses Verfahrens wird in Abschnitt \ref{module:Language.GTL.PromelaCIntegration}, Modul "`\verb|Language.GTL.PromelaCIntegration|"' genauer beschrieben.

Für die Übersetzung der Kontrakte mithilfe von binären Entscheidungsdiagrammen, wie in Abschnitt \ref{sec:bdd} beschrieben, wird Promela-Code generiert und dann gegen die CUDD-Bibliothek gelinkt.
Das Verfahren wird im Modul "`\verb|Language.GTL.PromelaDynamicBDD|"' in Abschnitt \ref{module:Language.GTL.PromelaDynamicBDD} genauer beschrieben.
\section{Backend}
\begin{figure}
  \centering
  \includegraphics[scale=0.3]{impl-backend}
  \caption{UML Diagramm der Backend-Implementierung}
  \label{fig:backend-uml}
\end{figure}
Die Backend-Implementierung stellt eine Möglichkeit zur Verfügung, synchrone Verifikationsformalismen in das GTL-Tool zu integrieren.
Dazu steht die Klasse "`\emph{GTLBackend}"' (Siehe Abschnitt \ref{module:Language.GTL.Backend}) bereit.
Abbildung \ref{fig:backend-uml} zeigt das UML-Diagramm des Paketes\footnote{Da UML nicht für die Modellierung funktionaler Sprachen wie Haskell ausgelegt ist, sind einige Syntax-Elemente sehr liberal verwendet. Die Abbildung ist nicht als formal korrektes UML-Diagramm gedacht sondern als Orientierungshilfe.}.
Instanzen der Klasse müssen eine Funktion \emph{backendName} bereit stellen, die den Namen des Backends zurück gibt (Diese Information wird benutzt um das richtige Backend für ein gegebenes Modell zu finden).
Wird ein Modell gefunden, dass dieses Backend verwendet, so wird die Initialisierungs-Funktion \emph{initBackend} mit den übergebenen Modellparametern aufgerufen.
Diese muss eine Instanz des assoziierten Datentyps "`\emph{GTLBackendModel}"' zurück liefern.

Mit dem initialisierten Backend-Modell können nun verschiedene Aktionen ausgeführt werden:
\begin{description}
\item[Typüberprüfung] Die Funktion \emph{typeCheckInterface} überprüft und vervollständigt eine gegebene Typbindung.
  Hierfür müssen die Typen der Variablen aus der Backend-Abhängigen Beschreibungssprache extrahiert werden und mit den bereits im Modell spezifizierten Variablen verglichen werden.
  Existiert hierbei kein Konflikt, so werden die Typbindungen vereinigt und zurück gegeben.
\item[C-Übersetzung] Mithilfe der Funktion \emph{cInterface} wird ein Wert vom Typ "`\emph{CInterface}"' zurück gegeben, mit dem es möglich sein muss, das entsprechende Modell nach C zu übersetzen.
  Es wird hierbei davon ausgegangen, dass die generierte C-Struktur einer Zustandsmaschine ähnelt.
  Der Datentype "`\emph{CInterface}"' stellt hierbei folgende Informationen zur Verfügung:
  \begin{itemize}
  \item Die einzubindenden Header-Dateien (\emph{cIFaceIncludes}).
  \item Die Typen der benötigten Zustandsvariablen (\emph{cIFaceStateType}).
  \item Die Typen der Eingabe-Variablen (\emph{cIFaceInputType}).
  \item Eine Funktion um einen Aufruf zu generieren, der die Zustandsmaschine initialisiert (\emph{cIFaceStateInit}).
    Die Funktion erhält die Namen der Zustandsvariablen als Parameter.
  \item Eine weitere Funktion um einen Aufruf zu generieren, der einen Schritt in der Zustandsmaschine durchführt (\emph{cIFaceIterate}).
    Als Parameter werden die Namen von Zustands- und Eingabevariablen übergeben.
  \item Funktionen um einzelne Variablen aus den Zustands- bzw. Eingabe-Variablen zu extrahieren (\emph{cIFaceGetOutputVar} bzw. \emph{cIFaceGetInputVar}).
  \item Eine Funktion um GTL-Typen nach C zu übersetzen (\emph{cIFaceTranslateType}).
  \end{itemize}
\item[Lokale Verifikation] Um die Korrektheit der Kontrakte für eine Komponente sicherzustellen kann die Funktion \emph{backendVerify} verwendet werden.
  Gegeben eine LTL-Formel muss diese in der Lage sein zu entscheiden, ob das Modell den Kontrakt einhält oder nicht.
  Dazu wird ein Wert vom Typ "`\emph{Maybe Bool}"' zurück gegeben.
  Ist der Wert "`\emph{Nothing}"', so ist die Verifikation unentscheidbar, ansonsten gibt der \emph{Bool}-Wert an, ob die Verifikation erfolgreich war.
\end{description}
\section{Modell}
\begin{figure}
  \centering
  \includegraphics[scale=0.3]{impl-model}
  \caption{UML-Diagramm der Modell-Implementierung}
  \label{fig:model-impl}
\end{figure}
Im Modul "`\emph{Language.GTL.Model}"' (Siehe auch Abschnitt \ref{module:Language.GTL.Model}) wird das Datenmodell des GTL-Tools implementiert.
Abbildung \ref{fig:model-impl} zeigt die verwendeten Datenstrukturen.

Die Klasse "`\emph{GTLSpec}"' stellt eine vollständig geparste und auf Typfehler überprüfte GTL-Datei dar.
Das Attribut \emph{gtlSpecModels} enthält eine Abbildung von Modellnamen auf Werte vom Typ "`\emph{GTLModel}"' und repräsentiert alle in der Spezifikation angegebenen Modelle.
Die Verifikationsformel ist im Attribut \emph{gtlSpecVerify} angegeben.
Die Formel ist über ein Paar von Strings spezifiziert, da alle Variablennamen qualifiziert sein müssen.
Über das Attribut \emph{gtlSpecConnections} werden die Variablenverbindungen spezifiziert und als Viertupel repräsentiert, wobei die Komponenten des Tupels folgende Bedeutung besitzen:
\begin{enumerate}
\item Name des Modells der Ausgabe-Variable.
\item Name der Ausgabe-Variable.
\item Name des Modells der Eingabe-Variable.
\item Name der Eingabe-Variable.
\end{enumerate}

Jedes Modell enthält dabei die folgenden Attribute:
\begin{itemize}
\item Der Kontrakt des Modells ist in dem Attribut \emph{gtlModelContract} abgelegt.
\item Das Backend des Modells wird in dem Attribut \emph{gtlModelBackend} gespeichert.
  Der Typ des Backends wird dabei aus dem in der Modell-Datei spezifizierten String hergeleitet.
\item Ein- und Ausgabe-Variablen mit ihren entsprechenden Typen sind in den Attributen \emph{gtlModelInput} sowie \emph{gtlModelOutput} hinterlegt.
\item Initialisierungswerte für Variablen sind in \emph{gtlModelDefaults} zu finden.
\end{itemize}
\section{Ausdrücke}
\begin{figure}
  \centering
  \includegraphics[scale=.3]{impl-expression}
  \caption{UML-Diagramm für Ausdrücke}
  \label{fig:uml-expr}
\end{figure}
Ausdrücke werden in der GTL-Sprache für Kontrakte und Verifikationsziele verwendet.
Ein Ausdruck wird durch eine Instanz des Datentyps "`\emph{Expr}"' repräsentiert, der über den Namenstypen der Variablen $v$ und den Typen des Ausdrucks $a$ parametrisiert ist.
Abbildung \ref{fig:uml-expr} zeigt eine Übersicht über die involvierten Datentypen.
Ein Ausdruck kann dabei eine der folgenden Formen annehmen:
\begin{itemize}
\item Es kann sich um eine einfache Variable handeln (\emph{ExprVar}).
  Diese wird mit ihrem Namen initialisiert und kann einen beliebigen Typen haben.
\item Konstanten können mit dem Konstruktor \emph{ExprConst} erstellt werden.
  Die resultierenden Ausdrücke haben dann denselben Typen wie die Konstante selbst.
\item Arithmetik-Operationen über zwei Int-Ausdrücken werden durch den \emph{ExprBinInt}-Konstruktor ermöglicht.
\item Boolesche binäre Ausdrücke werden durch den \emph{ExprBinBool}-Konstruktor repräsentiert.
\item Zwei numerische Ausdrücke lassen sich mit dem Konstruktor \emph{ExprRel} in eine Relation setzten.
  Der resultierende Type ist dann \emph{Bool}.
\item Die Aussage "`Variable $v$ ist (nicht) Element der Menge $x$"' lässt sich mit dem Konstruktor \emph{ExprElem} realisieren.
\item Negation wird mithilfe des Konstruktors \emph{ExprNot} bereit gestellt.
\item Die temporalen Operatoren "`always"' und "`next"' stehen durch die Konstruktoren \emph{ExprAlways} und \emph{ExprNext} bereit.
\end{itemize}
