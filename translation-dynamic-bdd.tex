\section{Abstraktion durch dynamische BDD}
Um die Probleme der statischen BDD-Übersetzung zu lösen, bietet es sich an, die Berechnung der BDDs erst während der Verifikation durchzuführen.
Der Vorteil ist, dass man nun auch mehrere Variablen miteinander in Beziehung setzten kann und auch Modelle verifizieren kann, die Zyklen enthalten.
Es kann allerdings passieren, dass bei der Verifikation mehrfach die gleiche BDD Operation ausgeführt wird; dies lässt sich aber durch den Einsatz eines Operationscaches vermeiden.

Für die Implementierung der dynamischen BDDs wurde die C-Bibliothek \emph{CUDD}
\footnote{Die Abkürzung steht für {\bf C}olorado {\bf U}niversity {\bf D}ecision {\bf D}iagram Package.
  Ein Benutzerhandbuch, sowie die Quellen sind online erhältlich\cite{cudd}.
  Die Bibliothek ist unter einer BSD-artigen Lizenz veröffentlicht.
}
gewählt, da diese im Gegensatz zu anderen Bibliotheken Zugriff auf ihre internen Schnittstellen bietet und somit sehr gut zu erweitern ist.