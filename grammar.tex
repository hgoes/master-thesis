Eine globale Deklaration ist entweder eine Modell-Deklaration, eine Verbindungsdeklaration oder eine Formel, die das Gesamtsystem erfüllen muss.
\begin{grammar}
  <declaration> ::= <model\_decl>
  \alt <connect\_decl>
  \alt <verify\_decl>
\end{grammar}
Eine Modell Deklaration besteht aus dem Schlüsselwort \emph{model}, gefolgt von dem Namen des synchronen Formalismus, in dem das Modell definiert ist (beispielsweise \emph{scade}).
Danach folgt der Name der Komponente und eine Liste von Argumenten, die spezifisch für den synchronen Formalismus angeben, wie das Modell zu laden ist.
Als letztes kommt der Rumpf der Modelldeklaration, der die Kontrakte spezifiziert, die das Modell erfüllt.
\begin{grammar}
  <model\_decl> ::= `model' `[' <id> `]' <id> `(' (<argument> (`,' <argument>)*)? `)' <model\_contract>
\end{grammar}
Ein Kontrakt ist eine Liste von LTL-Formeln, die die Komponente erfüllt oder Initialisierungswerten für die Variablen der Komponente.
Erfüllt das Modell keine Formel, so kann die Modell-Deklaration auch mit einem Semikolon beendet werden.
\begin{grammar}
  <model\_contract> ::= `{' (<contract\_body> `;')* `}'
  \alt `;'

  <contract\_body> ::= <formula>
  \alt `init' <id> <int>
\end{grammar}
Eine Verbindungsdeklaration besteht aus dem Schlüsselwort \emph{connect} und der Angabe einer Variable der Quellkomponente und der Zielkomponente.
Die Variable wird dabei durch den Namen der Komponente, einen Punkt und den Namen der Variable spezifiziert.
\begin{grammar}
  <connect\_decl> ::= `connect' <id> `.' <id> <id> `.' <id> `;'
\end{grammar}
Formeln, die über das Gesamtsystem verifiziert werden sollen, lassen sich in \emph{verify}-Blöcken spezifizieren.
Diese enthalten eine Liste von Formeln.
\begin{grammar}
  <verify\_decl> ::= `verify' `{' (<formula> `;')* `}'
\end{grammar}
Eine Formel kann verschiedene Formen annehmen:
\begin{itemize}
\item Als atomare Aussage besteht sie aus einer Variable.
\item Als Relation setzt sie zwei Ausdrücke in eine Beziehung ("`kleiner"', "`größer"', "`gleich"' etc.).
\item Um anzugeben, dass eine Variable einen Wert aus einer Menge annehmen kann, lässt sich das \emph{in}-Schlüsselwort verwenden.
\item Formeln können mit den logischen Operatoren \emph{and}, \emph{or}, \emph{implies} sowie \emph{not} wieder zu neuen Formeln zusammen gesetzt werden.
\item Mithilfe der temporallogischen Operatoren \emph{always}, \emph{next} und \emph{finally} können neue Formeln aus anderen gebildet werden.
\item Das Konstrukt \emph{exists} erlaubt die Bindung einer Variable an den aktuellen Wert einer anderen und somit die Referenzierung von früheren Werten.
\end{itemize}
\begin{grammar}
  <formula> ::= <var>
  \alt <expr> `<' <expr>
  \alt <expr> `>' <expr>
  \alt <expr> `<=' <expr>
  \alt <expr> `>=' <expr>
  \alt <expr> `=' <expr>
  \alt <var> `in' `{' (<lit> (`,' <lit>)*)? `}'
  \alt `not' <formula>
  \alt <formula> `and' <formula>
  \alt <formula> `or' <formula>
  \alt <formula> `implies' <formula>
  \alt `always' <formula>
  \alt `next' <formula>
  \alt `finally' <int> <formula>
  \alt `exists' <id> `=' <lit> `:' <formula>
  \alt `(' <formula> `)'
  
  <lit> ::= <int>
  \alt <var>
\end{grammar}
Ein Ausdruck ist entweder eine numerische Konstante, eine Variable oder eine arithmetische Operation (Addition, Subtraktion, Multiplikation oder Division werden unterstützt) von zwei Ausdrücken.
\begin{grammar}
  <expr> ::= <lit>
  \alt <expr> `+' <expr>
  \alt <expr> `-' <expr>
  \alt <expr> `*' <expr>
  \alt <expr> `/' <expr>
  \alt `(' <expr> `)'
\end{grammar}
Eine Variable kann unqualifiziert sein und nur ihren Namen angeben, oder zusätzlich noch den Namen der Komponente aus der sie stammt angeben und damit als qualifiziert gelten.
\begin{grammar}  
  <var> ::= <id>
  \alt <id> `.' <id>
  
  <id> ::= (`a'-`z' `A'-`Z' `0'-`9')+
  
  <int> ::= (`0'-`9')+
\end{grammar}
