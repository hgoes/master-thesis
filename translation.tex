Dieser Abschnitt beschäftigt sich mit der Übersetzung der GTL in verschiedene Zielformalismen.
Im Rahmen dieser Arbeit wurden drei verschiedene Übersetzungsmethoden entwickelt und zwei Optimierungsstrategien implementiert:
\begin{itemize}
\item Die erste Übersetzungsmethode verwendet die von SCADE generierten C-Modelle um ein Gesamtsystem in Promela zusammen zu setzen.
\item Die Kontrakte der Komponenten können mithilfe des SCADE-Design-Verifiers überprüft werden.
\item Das Kontraktsystem kann nach Promela übersetzt werden.
  \begin{itemize}
  \item Statische BDD können verwendet werden, um den Zustandsraum bei der Verifikation zu verkleinern.
  \item Dynamische BDD dienen dem selben Zweck, haben aber weniger Einschränkungen.
  \end{itemize}
\end{itemize}
Zunächst wird eine allgemeine Konstruktion angegeben, mit der ein GALS-System nach Promela übersetzt werden kann.
Diese wird dann verwendet, um die anderen Übersetzungsmethoden zu erklären.
Dazu verwendet die allgemeine Konstruktion drei Übersetzungsfunktionen $\llbracket\rrbracket_C$, $\llbracket\rrbracket_A$ und $\llbracket\rrbracket_D$, die von den konkreten Übersetzungen bereit gestellt werden müssen.
Da die SCADE-Übersetzung keinen Promela-Code generiert, verwendet sie als einzige Übersetzung auch nicht die allgemeine Übersetzungsmethode.