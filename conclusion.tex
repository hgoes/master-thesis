Die vorliegende Arbeit zeigt den aktuellen Entwicklungsstand der GTL-Spezifikationssprache (April 2011).
Im Rahmen des \emph{VerSyKo}-Projekts wird jedoch weiter an Verifikationsalgorithmen für verteilte GALS-Architekturen geforscht.
Die Ergebnisse dieser Forschung werden sehr wahrscheinlich großen Einfluss auf die weitere Entwicklung der GTL-Sprache haben.
Unter anderem werden folgende Gebiete untersucht:
\begin{itemize}
\item Inkorporation von zeit-basierter Modelierung und Verifikation.
  Hierbei soll die Verifikation von Realzeit-Eigenschaften ermöglicht werden.
\item Vervollständigung des Datenmodells.
  Momentan unterstützt die GTL Sprache nur einen Bruchteil der Datentypen der zugrundeliegenden Formalismen.
  Eine Erweiterung der Datentypen ermöglicht die Verifikation von realistischeren Modellen.
\item Verwendung von anderen synchronen Formalismen.
  Der Verifikationsalgorithmus des GTL-Programms erlaubt die gleichzeitige Verwendung von verschiedenen synchronen Formalismen zur Spezifikation von Komponenten.
  Die Erweiterung um Formalismen außer SCADE ist ein weiteres Ziel der Entwicklung.
\end{itemize}
