\section{LTL Übersetzung}
\label{sec:ltl-translation}
Um LTL Formeln einfacher übersetzen zu können und zu kanonisieren, werden diese in Büchi-Automaten übersetzt.
Diese Übersetzung benutzt den in \cite{Gerth95simpleon-the-fly} beschriebenen Algorithmus.

Da der Übersetzungsalgorithmus keine \emph{always}-Konstrukte zulässt, müssen diese zunächst mit der folgenden Identität transformiert werden:
\[ \textbf{always}\ \varphi = \lnot\top U \lnot\varphi \]
Außerdem müssen für den Algorithmus alle Negationen so weit wie möglich nach innen geschoben werden, bis sie nur noch vor atomaren Aussagen stehen.
Um die Größe der Formeln nicht unnötig zu erhöhen wird dual zum \emph{until}-Operator $U$ der Operator $V$ eingeführt, der über die folgende Identität definiert ist:
\[ \varphi V\psi = \lnot (\lnot\varphi U\lnot\psi) \]

Für die Konstruktion des Büchi-Automaten wird ein Graph aufgebaut, der Schrittweise erweitert wird, bis der Büchi-Automat vollständig ist.
Die Knoten des Graphen benötigen die folgenden Felder:
\begin{description}
\item[Name] Ein eindeutiger Bezeichner für den Knoten.
  Es wird vorausgesetzt, dass es eine Funktion \proc{NewName} existiert, die bei jedem Aufruf einen neuen, eindeutigen Namen zurück gibt.
\item[Incoming] Gibt die Knoten an, die eine Kante in diesen Knoten besitzen.
  Das Symbol \const{init} wird verwendet, um anzuzeigen, dass der Knoten initial ist.
\item[New] Eine Liste von Formeln, die noch nicht bearbeitet wurde.
\item[Old] Die Liste der Formeln, die bereits abgearbeitet wurden.
\item[Next] Formeln, die in allen Nachfolgeknoten gelten müssen.
\end{description}
Der Anfangsknoten hat einen beliebigen Namen, das Symbol \const{Init} in der \emph{Incoming}-Menge und die gesamte zu übersetzende Formel als \emph{New}-Feld.
Die restlichen Felder sind leer.
Der zentrale Bestandteil des Algorithmus ist die \emph{expand}-Funktion.
Diese nimmt einen Knoten und die Menge aller bisher generierten Knoten und erstellt durch rekursive Aufrufe ihrer selbst die resultierende Knotenmenge.

\begin{codebox}
\Procname{$\proc{Expand}(\id{Node},\id{NodeSet})$}
\li \If $\attrib{Node}{New}\isequal\emptyset$ \Then
\li \If $\exists N\in NodeSet: \attrib{N}{Old}\isequal\attrib{Node}{Old} \mbox{ and } \attrib{N}{Next}\isequal\attrib{Node}{Next}$ \Then
\li $\attrib{N}{Incoming}\gets \attrib{N}{Incoming}\cup\attrib{Node}{Incoming}$
\li \Return \id{NodeSet}
\li \Else \Return $\proc{Expand}([\id{Name}\gets \proc{NewName}(),\id{Incoming}\gets \{\attrib{Node}{Name}\},$
\Startalign{\Return $\proc{Expand}([$}
\> $\id{New}\gets\attrib{Node}{Next}, \id{Old}\gets\emptyset, \id{Next}\gets\emptyset],$\\
\> $\{\id{Node}\}\cup \id{NodeSet})$
\Stopalign
\li \Else
\li $\eta\gets \attrib{Node}{New}[0]$
\li $\attrib{Node}{New}\gets\attrib{Node}{New}\setminus\{\eta\}$
\li \If $\eta\isequal P\mbox{ or }\eta\isequal\lnot P\mbox{ or }\eta\isequal\top\mbox{ or }\eta\isequal\bot$ \Then
\li \If $\eta\isequal\bot\mbox{ or }\lnot\eta\in\attrib{Node}{Old}$
\li \Then \Return \id{NodeSet}
\li \Else
\li $\attrib{Node}{Old}\gets \attrib{Node}{Old}\cup\{\eta\}$
\li \Return $\proc{Expand}(\id{Node},\id{NodeSet})$
\End
\li \ElseIf $\eta\isequal\varphi U\psi\mbox{ or }\eta\isequal\varphi V\psi\mbox{ or }\eta\isequal\varphi\lor\psi$ \Then
\li $\id{Node1}\gets [\id{Name}\gets\proc{NewName}(),\id{Incoming}\gets\attrib{Node}{Incoming},$
\Startalign{$\id{Node1}\gets [$}
\> $\id{New}\gets\attrib{Node}{New}\cup(\{\proc{New1}(\eta)\}\setminus\attrib{Node}{Old}),$\\
\> $\id{Old}\gets\attrib{Node}{Old}\cup\{\eta\},\id{Next}\gets\attrib{Node}{Next}\cup\{\proc{Next1}(\eta)\}]$
\Stopalign
\li $\id{Node2}\gets [\id{Name}\gets\proc{NewName}(),\id{Incoming}\gets\attrib{Node}{Incoming},$
\Startalign{$\id{Node2}\gets [$}
\> $\id{New}\gets\attrib{Node}{New}\cup(\{\proc{New2}(\eta)\}\setminus\attrib{Node}{Old}),$\\
\> $\id{Old}\gets\attrib{Node}{Old}\cup\{\eta\},\id{Next}\gets\attrib{Node}{Next} ]$
\Stopalign
\li \Return $\proc{Expand}(\id{Node2},\proc{Expand}(\id{Node1},\id{NodeSet}))$

\li \ElseIf $\eta\isequal\varphi\land\psi$\Then
\li \Return $\proc{Expand}([\id{Name}\gets\attrib{Node}{Name},\id{Incoming}\gets\attrib{Node}{Incoming},$
\Startalign{\Return $\proc{Expand}([$}
\> $\id{New}\gets\attrib{Node}{New}\cup(\{\varphi,\psi\}\setminus\attrib{Node}{Old}),$\\
\> $\id{Old}\gets\attrib{Node}{Old}\cup\{\eta\},\id{Next}\gets\attrib{Node}{Next}],\id{NodeSet})$
\Stopalign

\li \ElseIf $\eta\isequal \textbf{next }\varphi$ \Then
\li \Return $\proc{Expand}([\id{Name}\gets\attrib{Node}{Name},\id{Incoming}\gets\attrib{Node}{Incoming},$
\Startalign{\Return $\proc{Expand}([$}
\> $\id{New}\gets\attrib{Node}{New},\id{Old}\gets\attrib{Node}{Old}\cup\{\eta\},$\\
\> $\id{Next}\gets\attrib{Node}{Next}\cup\{\varphi\}],\id{NodeSet})$
\Stopalign
\End
\End
\End
\end{codebox}
Die Hilfsfunktionen \proc{New1}, \proc{New2} und \proc{Next1} sind hierbei über Tabelle \ref{tab:helper_funcs} definiert.

\begin{table}[h]
  \centering
  \begin{tabular}{|l|l|l|l|}
    \hline
    $\eta$ & \proc{New1} & \proc{Next1} & \proc{New2}\\
    \hline
    \hline
    $\varphi U\psi$ & $\{\varphi\}$ & $\{\varphi U\psi\}$ & $\{\psi\}$\\
    \hline
    $\varphi V\psi$ & $\{\psi\}$ & $\{\varphi V\psi\}$ & $\{\varphi,\psi\}$\\
    \hline
    $\varphi\lor\psi$ & $\{\varphi\}$ & $\emptyset$ & $\{\psi\}$\\
    \hline
  \end{tabular}
  \caption{Hilfsfunktionen für den LTL Übersetzungsalgorithmus}
  \label{tab:helper_funcs}
\end{table}

Um eine gegebene Formel $\varphi$ zu übersetzen, ruft man die Funktion also wie folgt auf:
\begin{align*}
  \proc{Expand}([&\id{Name}\gets\proc{NewName}(),\\
                 &\id{Incoming}\gets\const{Init},\\
                 &\id{New}\gets\{\varphi\},\\
                 &\id{Old}\gets\emptyset,\\
                 &\id{Next}\gets\emptyset],\emptyset) = ns
\end{align*}
Per Konstruktion ist am Ende des Durchlaufs das \id{New}-Feld aller Knoten leer.
Die atomaren Aussagen in den \id{Old}-Feldern der Knoten stellen die Aussagen dar, die in dem entsprechenden Zustand des Büchi-Automaten wahr sein müssen.
Um den Graphen zu konstruieren, müssen noch die die Informationen der \id{Incoming}-Felder verwendet werden, um die Folgezustände von jedem Zustand zu berechnen.
Die \id{Next}-Felder enthalten keine relevanten Informationen für den Büchi-Automaten.


Es ergibt sich also ein verallgemeinerter Büchi-Automat $(Q,\Sigma,\delta,\mu,q_0,F)$ mit
\begin{align*}
  Q = {}& \{ \attrib{Node}{Name}\ |\ \id{Node}\in ns \}\\
  \delta = {} & \{ (\attrib{Node1}{Name},\attrib{Node2}{Name})\ |\\
          & \id{Node1}\in ns,\id{Node2}\in ns, \attrib{Node1}{Name}\in \attrib{Node2}{Incoming}\}\\
  \mu(n) = {} & \{ f\ |\ \id{Node}\in ns, \attrib{Node}{Name} = n, f\in\attrib{Node}{Old}, \id{atom}(f) \}\\
  q_0 = {} & \{ \attrib{Node}{Name}\ |\ \id{Node}\in ns, \const{Init}\in\attrib{Node}{Incoming} \}\\
  F = {} & \{ \{ \attrib{Node}{Name}\ |\ \id{Node}\in ns, \mu U\psi \not\in \attrib{Node}{Old}\lor \psi\in\attrib{Node}{Old} \}\ |\\
     & \mu U\psi\in\id{subformula}(\varphi) \}
\end{align*}
Dieser kann nun mithilfe der in Abschnitt \ref{sec:gba} definierten Methode in einen normalen Büchi-Automaten umgewandelt werden.

\subsection{Übersetzung des Existenzquantors}
Mithilfe des Existenzquantors ist es möglich, Werte zu einem bestimmten Zeitpunkt zu binden und zu einem späteren Zeitpunkt wieder zu verwenden
\footnote{Eine sehr viel generellere Definition des Existenzquantors findet sich in \cite{exists_quantor}.
  Es wird jedoch kein allgemeiner Übersetzungsalgorithmus angegeben, weshalb in dieser Arbeit eine sehr eingeschränkte Version verwendet wird.
}.
Beispielsweise gibt die Formel
\[ \textbf{exists}\ u = \mathit{Engine}.\mathit{state} : \textbf{next}\ u = \mathit{Engine}.\mathit{state} \]
an, dass die Variable "`$\mathit{Engine}.\mathit{state}$"' im nächsten Zustand den gleichen Wert wie im aktuellen besitzen soll.

Um dieses Konstrukt zu übersetzen kann man Variablen einführen, die die Geschichte von Variablen speichern.
Muss also beispielsweise der Wert einer Variable überprüft werden, der drei Schritte in der Vergangenheit liegt, so müssen drei neue Variablen in das Modell eingefügt werden, die die letzten drei Werte der Variablen speichern.

Die Anzahl der Schritte, die in die Vergangenheit geschaut werden muss ist allerdings nicht immer endlich.
Betrachten wir die leicht abgewandelte Formel
\[ \textbf{exists}\ u = \mathit{Engine}.\mathit{state} : \textbf{always}\ u = \mathit{Engine}.\mathit{state} \]
die aussagt, dass der Wert einer Variable für immer konstant bleiben soll, so ist leicht einzusehen, dass man für die Verifikation unendlich viele Geschichtsvariablen einführen müsste.
Verwendet man also diese Übersetzung, so muss man die Verwendung des Existenzquantors einschränken:
Die durch den Quantor gebundene Variable darf nicht innerhalb eines \emph{always}-Operators vorkommen.

Der Algorithmus zur Übersetzung ist in Abbildung \ref{fig:translate_history} angegeben.
\begin{figure}[h]
\begin{codebox}
\Procname{$\proc{TranslateHistory}(\id{expr},\id{vars})$}
\li \If $\id{expr}\isequal (\textbf{exists}\ u=x : f)$ \Then
\li $\attrib{vars[u]}{var}\gets x$
\li $\attrib{vars[u]}{level}\gets 0$
\li \Return $\proc{TranslateHistory}(f,\id{vars})$
\li \ElseIf $\id{expr}\isequal v$\>\>\>\>\>\>\Comment$v$ ist Variable
\Then
\li \If $\id{vars}[v]\isequal\const{Nil}$\Then
\li \Return $v$
\li \Else \Return $\id{vars}[v]$
\End
\li \ElseIf $\id{expr}\isequal (\textbf{next}\ f)$\Then
\li $\forall v: \attrib{vars[v]}{level}\gets \attrib{vars[v]}{level}+1$
\li \Return $(\textbf{next}\ \proc{TranslateHistory}(\id{f},\id{vars}))$
\li \ElseIf $\id{expr}\isequal (\textbf{always} f)$\Then
\li \Return $(\textbf{always}\ \proc{TranslateHistory}(\id{f},\emptyset))$
\li \ElseIf $\id{expr}\isequal (\id{lhs}\ \id{op}\ \id{rhs})$\>\>\>\>\>\>\Comment $\id{op}$ ist $\land$, $\lor$, $<$, $>$ oder eine andere Relation
\Then
\li \Return ($\proc{TranslateHistory}(\id{lhs},\id{vars})\ \id{op}\ \proc{TranslateHistory}(\id{rhs},\id{vars})$)
\End
\end{codebox}
\caption{Übersetzung von Geschichtsvariablen}
\label{fig:translate_history}
\end{figure}