\section{SPIN}
\label{sec:spin}
SPIN ist ein Model-Checker für asynchrone Software-Modelle, die in der Sprache Promela ({\bf Pro}cess {\bf Me}ta {\bf La}nguage) definiert sind~\cite{spinbook}.
Das Werkzeug verwendet "`Explicit-State Model Checking"'-Techniken, berechnet also jeden möglichen Zustand des Systems und überprüft, ob die zu verifizierenden Eigenschaften erfüllt sind.
SPIN unterstützt eine Reihe von Optimierungstechniken, darunter
\begin{itemize}
\item "`Partial order reduction"' um den Zustandsraum von Modellen zu verkleinern~\cite{partial_order_reduction}.
\item Zustandskompressionstechniken, die den Speicherbedarf von Verifikationen senken können~\cite{spin_state_compression}.
\item Nutzung von Multi-Prozessor-Systemen zur Geschwindigkeitsverbesserung~\cite{spin_multi_core}.
\end{itemize}
Es werden sowohl die Verifikation von \emph{Safety}-Eigenschaften ("`Es passiert nie etwas schlimmes"') als auch von \emph{Liveness}-Eigenschaften ("`Es passiert immer mal wieder etwas gutes"') unterstützt.
Die Verifikation von Modellen erfolgt nicht direkt durch SPIN selbst, sondern es wird Code für einen domänenspezifischen Modell-Checker generiert.