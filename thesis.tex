\documentclass[masterarbeit]{thesis}

\usepackage[bookmarks=true,colorlinks=false]{hyperref} % PDF-Inhaltsverzeichnis und Links

\usepackage[utf8]{inputenc}
\usepackage[ngerman]{babel} % Deutsche Silbentrennung

\usepackage[T1]{fontenc}
\usepackage[lf]{venturis} % Eine schöne Schriftart

\usepackage{amsmath}
\usepackage{graphicx} % For includegraphics
\usepackage[nounderscore]{syntax} % Syntax im BNF-Stil (Niemals das "nounderscore entfernen, das macht alles kaputt!)
\usepackage{amssymb} % Für \mathbb
\usepackage[numbers]{natbib} % Literaturverzeichnis
\usepackage{tikz} % Für Grafiken und Diagramme
\usetikzlibrary{automata,positioning,arrows,fit,shapes,decorations.pathmorphing}
\usepackage{listings} % Für Quellcode-Listings (Promela,GTL etc.)
\usepackage{stmaryrd} % Für \llbracket,\rrbracket
\usepackage{semantic} % Strukturelle Operationelle Semantik
\usepackage[all]{xy} % XY-Matrix Umgebung für Diagramme
\usepackage{clrscode3e} % Für Pseudo-Code im Cormen-Stil
\usepackage{amsthm} % Für Notation, Bemerkung etc.

\usepackage{setspace}
\usepackage{relsize}

\usepackage{generated/haddock}

%\definecolorset{RGB}{}{}{d1,11,64,12;d2,39,140,41;d3,77,57,153;d4,89,49,16;d5,217,124,49}
%\definecolorset{RGB}{}{}{d1,255,192,146;d2,204,169,96;d3,153,112,26;d4,210,255,247;d5,96,204,155}
\definecolorset{RGB}{}{}{d1,204,102,34;d2,46,46,51;d3,147,153,138;d4,229,201,148;d5,255,244,188}

\bibliographystyle{alphadin}

\title{Verifikation von GALS Systemen}
\author{Henning Günther}
\keywords{GALS, verification, SPIN, modeling}

\lstdefinelanguage{gtl}{
  morekeywords={model,and,or,next,not,connect,verify,always,contract,init,input,output,state,automaton,instance,transition},
  sensitive=true,
  morestring=[b]",
  otherkeywords={.,=>}
}

\newcommand{\bdd}[1]{\vcenter{\hbox{\includegraphics[scale=.4]{#1}}}}

\definecolor{tubsrot}{cmyk}{.1,1,.8,0}

\newtheorem{notation}{Notation}
\newtheorem{definition}[notation]{Definition}
\newtheorem{beispiel}[notation]{Beispiel}

\newcommand{\docUni}{Technische Universität Braunschweig}
\newcommand{\docInstitut}{Institut für Theoretische Informatik}
\newcommand{\docType}{Master-Arbeit}
\newcommand{\docPruefer}{Prof. Dr. Jiří Adámek}
\newcommand{\docBetreuer}{Dr. Stefan Milius}
\newcommand{\docId}{2860185}

\makeatletter
\hypersetup{%
  pdftitle=\@title,
  pdfsubject=\docType,
  pdfauthor=\@author,
  pdfkeywords=\@keywords
}
\makeatother

\begin{document}
\pagenumbering{roman}
\maketitle

\pagestyle{headings}
\makeatletter
\erklaerung{\@author}
\makeatother

\begin{abstract}
  \thispagestyle{headings}
  \pdfbookmark[0]{Zusammenfassung}{abstract}
  Diese Arbeit stellt einen Formalismus zur Spezifikation von GALS Systemen vor, implementiert einen Algorithmus zur Verifikation und zeigt Möglichkeiten für Optimierungen auf.

  \\[5pt]
  {\bf \keywordsname :} \makeatletter\@keywords\makeatother
\end{abstract}
\cleardoublepage
\pdfbookmark[0]{\contentsname}{content}
\tableofcontents
\listoffigures
\listoftables
\cleardoublepage
\pagenumbering{arabic}
\chapter{Einleitung}
Die Verifikation von GALS-Systemen steht vor zwei Problemen:
\begin{itemize}
\item Zum einen fehlen Werkzeuge, die sowohl synchrone wie auch asynchrone Systeme unterstützen.
  Formalismen, die für synchrone Systeme entworfen sind, haben in der Regel keine Möglichkeit, Asynchronität oder Nichtdeterminismus zu modellieren.
  Asynchrone Formalismen bieten zwar meist Unterstützung für synchrone Systeme, allerdings in den meisten Fällen nicht sehr elegant oder performant.
\item Zum anderen können GALS-Systeme beträchtliche Größen annehmen und damit ohne die manuelle Einführung von Vereinfachungen und Abstraktionen nur noch schwer in absehbarer Zeit zu verifizieren sein.
  Das manuelle Einführen von Abstraktionen ist aber sehr schwierig und kann bei ungenauen Abstraktionen zu schwer auffindbaren Fehlern in der Verifikation führen.
  Automatisch zu überprüfen, ob eine gegebene Abstraktion korrekt ist, ist in keinem in dieser Arbeit untersuchtem Formalismus vorgesehen.
\end{itemize}
Diese Arbeit versucht beide Probleme durch die Einführung eines neuen Formalismus zu lösen, der die Vorteile von synchronen und asynchronen Formalismen vereint und zusätzlich die Möglichkeit bietet, sichere Abstraktionen durch Kontrakte einzuführen.

\section{Andere Arbeiten}
Die Idee, GALS-Systeme durch eine Kombination von synchronen und asynchronen Sprachen zu verifizieren, wird in verschiedenen Arbeiten behandelt:
\begin{enumerate}
\item Damien Thivolle und Hubert Garavel erklären die grundsätzliche Herangehensweise und zeigen anhand der Verifikation eines Kommunikationsprotokolls die Vorteile dieses Ansatzes~\cite{gals_sam}.
Es wird erklärt, wie sich synchrone Komponenten als Funktionen ansehen lassen und wie sie sich in ein asynchrones Verifikationstool einbinden lassen.
Im Gegensatz zu dieser Arbeit wird aber kein neuer Formalismus für die Spezifikation von GALS-Systemen eingeführt.
Auch Techniken zur Optimierung wie sie in dieser Arbeit behandelt werden fehlen.
\item Ein Artikel von Doucet et al. beschreibt die Übersetzung des synchronen Formalismus SIGNAL nach Promela~\cite{gals_signal}.
  Die Verifikation wird dann vollständig von SPIN übernommen und nicht wie in dieser Arbeit aufgeteilt in einen synchronen und asynchronen Teil.
\item "`Multiclock Esterel"' erweitert die Beschreibungssprache von SCADE um die Möglichkeit, mehr als einen Takt für verschiedene Komponenten anzugeben~\cite{multiclock_esterel}.
  Der Formalismus bleibt allerdings vollständig deterministisch und jede Komponente lässt sich in eine äquivalente SCADE Komponente übersetzen.
\item Der Formalismus "`Communicating Reactive State Machines"' wird in einer Arbeit verwendet, um GALS Systeme zu modelieren und zu verifizieren~\cite{gals_crsm}.
\item Das "`IF Toolset"' führt einen Formalismus zur Spezifikation von asynchronen Systemen ein, der wie der hier vorgestellte Formalismus auf der Komposition von Komponenten durch Verbindungen basiert~\cite{if_toolset}.
  Der Schwerpunkt der Arbeit liegt in der Bereitstellung eines Formalismus, der mächtig genug ist, um die vielen verschiedenen Design-Formalismen zu vereinen.
\end{enumerate}

\chapter{GALS-Architekturen}
\label{sec:gals}
Ein GALS System -- GALS steht für "`Globally asynchronous, locally synchronous"' -- besteht aus mehreren synchronen Komponenten, die asynchron miteinander kommunizieren.
Die synchronen Komponenten lassen sich dabei als Mealy-Automaten auffassen\footnote{Mächtigere Formalismen wie beispielsweise Turing-Maschinen sind ungeeignet, da sie einen unendlichen Zustandsraum aufweisen können und daher mit aktuellen Model-Checkern nicht überprüft werden können.}.
In einem Mealy-Automaten hängt die Ausgabe sowohl von der Eingabe als auch dem aktuellen Zustand ab~\cite{Mealy}.
Mealy-Automaten sind so allgemein formuliert, dass sich jedes in einem synchronen Formalismus formulierte Modell in einen bisimularen Mealy-Automaten transformieren lässt.
Ein Beispiel für einen Mealy-Automaten ist in Abbildung \ref{fig:mealy} dargestellt.
Transitionen sind hier in der Form $i/o$ angegeben, wobei $i$ die Eingabe und $o$ die Ausgabe bezeichnen.
\begin{figure}[h]
  \centering
  \begin{tikzpicture}
  [initial text=,
  shorten >=1pt,
  >=stealth',
  every state/.style={draw=d1,very thick,fill=d1!40}
  ]
  \node[state,initial] (q0) at (0,2) {$q_0$};
  \node[state] (q1) at (2,2) {$q_1$};
  \node[state] (q2) at (1,0) {$q_2$};
  
  \path[->] (q0) edge[bend left] node[above] {$a/b$} (q1)
                 edge node[left] {$b/a$} (q2)
            (q1) edge node[below] {$a/c$} (q0)
                 edge[bend left] node[right] {$b/a$} (q2)
            (q2) edge node[left] {$a/b$} (q1)
                 edge[loop below] node {$b/c$} (q2);
\end{tikzpicture}

  \caption{Mealy-Automat}
  \label{fig:mealy}
\end{figure}

Ein Schritt im Automaten bedeutet, dass das aktuelle Eingabe-Symbol gelesen wird und die Transition gewählt wird, die vom aktuellen Zustand ausgeht und mit dem aktuellen Eingabe-Symbol übereinstimmt.
Das Ziel der Transition ist der neue aktuelle Zustand des Automaten und das Ausgabe-Symbol der Transition wird zur Ausgabe hinzugefügt.
Der Beispiel-Automat in Abbildung \ref{fig:mealy} hat also beispielsweise die folgende Ablauf-Sequenz:

\[ \xymatrix {
     q_0\ar[r]^a &  q_1 \ar[r]^a & q_0 \ar[r]^a & q_1 \ar[r]^b & q_2\ar[r] & \dots
   } \]
Der Automat transformiert damit die Eingabe "`$aaab\dots$"' in die Ausgabe "`$bcba\dots$"'.
Der Beispiel-Automat verwendet als Ein- und Ausgabe-Symbole Zeichen und stellt damit einen Aus- und Eingabekanal zur Verfügung.
Benutzt man stattdessen $n$-Tupel als Eingabe und $m$-Tupel als Ausgaben, erhält man einen Automaten, der $n$ Eingabe- und $m$ Ausgabe-Kanäle bietet.
In jedem Schritt wird dann aus jedem Eingabekanal ein Zeichen gelesen und in jeden Ausgabekanal eins geschrieben.

Für die vollständige Beschreibung eines Mealy-Automaten mit Kanälen ist die Repräsentation durch Tupel ausreichend, für eine übersichtliche Notation und Darstellung jedoch nicht geeignet, da die Anzahl der aufzuschreibenden Transitionen exponentiell mit der Anzahl der Eingabe-Kanäle wächst\footnote{Hat man $n$ Eingabe-Kanäle, die jeweils $m$ verschiedene Zeichen enthalten können, so benötigt jeder Zustand $m^n$ ausgehende Transitionen, da der Automat für jede Eingabe eine gültige Transition aufweisen sollte.}.
Daher ist es sinnvoll, für die Spezifikation von Mealy-Automaten die Tupel-Komponenten mit Variablen zu benennen und Bedingungen (engl. \emph{conditions}) über diese Variablen an die Transitionen zu schreiben.
Eine Bedingung ist hierbei eine boolesche Formel, die nur die Variablen der Tupelkomponenten enthält.
Ist die Formel für eine Belegung der Variablen wahr, so gilt die Transition auch für das entsprechende Tupel.

Benennt man beispielsweise die Komponenten der Tupel $\mathbb{N}\times\mathbb{B}\times\mathbb{B}$ mit $\alpha$, $\beta$ und $\gamma$, so beschreibt die Bedingung $\alpha=3\land\gamma=0$ die Tupel $(3,0,0)$ und $(3,1,0)$.
Eine Transition $\alpha=3\land\gamma=0/(1,2)$ beschreibt also eigentlich die Transitionen $(3,0,0)/(1,2)$ und $(3,1,0)/(1,2)$.
Diese Art der Notation ist hilfreich, wenn für eine Transition die Werte von bestimmten Kanälen keine Rolle spielen.
Abbildung \ref{fig:mealy2} zeigt einen Beispiel-Automaten, der Tupel der Form $\mathbb{B}\times\mathbb{N}$ als Eingaben verwendet.

\begin{figure}[h]
  \centering
  \begin{tikzpicture}
  [initial text=,
  shorten >=1pt,
  >=stealth',
  every state/.style={draw=blue!50,very thick,fill=blue!20}
  ]
  \node[state,initial] (q0) at (0,0) {$q_0$};
  \node[state] (q1) at (3,0) {$q_1$};
  
  \path[->] (q0) edge[bend left] node[above] {$\mathit{on}\land x<5/1$} (q1)
                 edge[loop above] node {$\lnot\mathit{on}\lor x\geq 5/3$} (q0)
            (q1) edge[bend left] node[below] {$x\geq 5/0$} (q0)
                 edge[loop right] node {$x<5/2$} (q1);
\end{tikzpicture}

  \caption{Mealy-Automat mit Bedingungen}
  \label{fig:mealy2}
\end{figure}

Um ein GALS-System zu erstellen, kann man nun die Ausgabekanäle von Automaten mit den Eingabekanälen von anderen Automaten verbinden und so ein Netzwerk aus Mealy-Automaten erhalten.
Ist eine Ausgabe eines Automaten $A$ mit einer Eingabe eines Automaten $B$ verbunden, so verwendet $B$ als Eingabe in jedem Schritt die letzte von $A$ produzierte Ausgabe.

Abbildung \ref{fig:mealy3} zeigt ein solches System.
Das Gesamtsystem hat sämtliche unverbundenen Aus- und Eingaben als Aus- und Eingaben.
Ein Schritt des Gesamtsystems wird nun durch einen Schritt eines beliebigen Mealy-Automaten realisiert.
Durch diese Eigenschaft wird das GALS-System nicht-deterministisch und asynchron.
Ein Problem ergibt sich nun, wenn die Quelle einer Verbindung noch keinen Schritt durchgeführt hat: in diesem Fall gibt es keinen letzten Wert für die Verbindung und die Eingabe des Ziel-Automaten ist undefiniert.
Dieser Fall kann durch die Definition von Default-Werten oder durch das Verbot, Schritte mit Automaten durchzuführen, deren Eingaben (teilweise) undefiniert sind gelöst werden.
In dieser Arbeit wird das Problem durch die Einführung von impliziten oder expliziten Default-Werten gelöst.
\begin{figure}[h]
  \centering
  \begin{tikzpicture}
  [initial text=,
  shorten >=1pt,
  >=stealth',
  every state/.style={draw=blue!50,very thick,fill=blue!20}
  ]
  %\node[draw,red!50,fill=red!20,very thick,minimum height=1cm,minimum width=4cm] (c1) at (0.25,0.5) {};
  
  %\draw[color=red!50,fill=red!20,very thick] (-1,1) -- (-1,-1) -- (3,-1) -- (3,1) -- (-1,1);
  \path[draw,red!50,fill=red!20,very thick] (-1,1) rectangle (2.6,-1);
  \path[draw,red!50,fill=red!20,very thick] (2.6,-1) rectangle (3.3,1);
  \node[state,initial] (q0) at (0,0) {$q_0$};
  \node[state] (q1) at (2,0) {$q_1$};
  \path[->] (q0) edge[bend left] node[above] {$/0$} (q1)
            (q1) edge node[below] {$/1$} (q0);
  \node at (2.3,0.7) {$P_1$};
  \node (tout) at (2.95,0) {out};
  \begin{scope}[shift={(5.5,-0.15)}]
    \path[draw,red!50,fill=red!20,very thick] (-1.2,-1.8) rectangle (4,0.8);
    \path[draw,red!50,fill=red!20,very thick] (-1.2,-1.8) rectangle (-1.8,-0.5);
    \path[draw,red!50,fill=red!20,very thick] (-1.2,-0.5) rectangle (-1.8,0.8);
    \path[draw,red!50,fill=red!20,very thick] (4,-1.8) rectangle (4.7,0.8);
    \node[state,initial] (p0) at (0,0) {$q_0$};
    \node[state] (p1) at (3,0) {$q_1$};
  
    \path[->] (p0) edge node[above] {$\mathit{on}\land x<5/1$} (p1)
                   edge[loop below] node {$\lnot\mathit{on}\lor x\geq 5/3$} (p0)
              (p1) edge[bend left] node[below] {$x\geq 5/0$} (p0)
                   edge[loop below] node {$x<5/2$} (p1);
    \node at (-1.5,-1.15) {x};
    \node (ton) at (-1.5,0.15) {on};
    \node at (3.7,0.5) {$P_2$};
    \node at (4.35,-0.5) {res};
  \end{scope}
  \path[->,thick] (tout) edge (ton);
\end{tikzpicture}

  \caption{Aus Mealy-Automaten zusammen gesetztes GALS-System}
  \label{fig:mealy3}
\end{figure}

Das in Abbildung \ref{fig:mealy3} gezeigte System hat also nur die Eingabe $x$ und die Ausgabe $\mathit{res}$.
Der Gesamtzustand des Systems ist nun durch den Zustand der Verbindungen und den aktuellen Zustand der Prozesse definiert.
Der Startzustand des abgebildeten Systems kann also beispielsweise durch $(p_0,q_0,\bot)$ dargestellt werden, wobei $\bot$ angibt, dass die Verbindung zwischen $\mathit{out}$ und $\mathit{on}$ einen undefinierten Wert besitzt.
Eine Transition wird dargestellt durch ein Tupel, dass die Ein- und Ausgaben des Systems sowie den Mealy-Automaten, der den aktuellen Schritt durchführt, enthält.
Die Transition $(x=4,P_1)/3$ gibt also beispielsweise an, dass die Eingabe $x$ den Wert $4$, die Ausgabe $\mathit{res}$ den Wert $3$ besitzt und der Automat $P_1$ einen Schritt durchführt hat.
Das resultierende Transitionssystem für das GALS-System aus Abbildung \ref{fig:mealy3} ist beispielsweise in Abbildung \ref{fig:gals_trans} dargestellt.

\begin{figure}[h]
  \centering
  \begin{tikzpicture}
  [initial text=,
  shorten >=1pt,
  >=stealth',
  every state/.style={draw=d1,very thick,fill=d1!40}
  ]
  \node[state,initial] (q00u) at (0,0) {$p_0,q_0,\bot$};
  \node[state] (q100) at (3,0) {$p_1,q_0,0$};
  \node[state] (q001) at (3,-3) {$p_0,q_0,1$};
  \node[state] (q011) at (6,-3) {$p_0,q_1,1$};
  \node[state] (q110) at (6,0) {$p_1,q_1,0$};

  \path[->] (q00u) edge node[above] {$P_1/\bot$} (q100)
            (q100) edge[loop above] node {$P_2/3$} (q100)
                   edge node[right] {$P_1/\bot$} (q001)
            (q001) edge[bend left] node[left] {$P_1/\bot$} (q100)
                   edge[bend left] node[above] {$x<5,P_2/1$} (q011)
            (q011) edge[loop right] node {$x<5,P_2/2$} (q011)
                   edge[bend left] node[below] {$x\geq 5,P_2/0$} (q001)
                   edge node[left] {$P_1/\bot$} (q110)
            (q110) edge[loop above] node {$x<5,P_2/2$} (q110)
                   edge[bend right] node[above] {$x\geq 5,P_2/0$} (q100)
                   edge[bend left] node[right] {$P_1/\bot$} (q011);
\end{tikzpicture}

  \caption{Transitionssystem des GALS Systems aus \ref{fig:mealy3}}
  \label{fig:gals_trans}
\end{figure}

Möchte man verhindern, dass die Ausgaben des Gesamtsystems häufig einen undefinierten Wert ($\bot$) annehmen, so kann man beispielsweise definieren, dass eine undefinierte Ausgabe bedeutet, dass der letzte Wert der Ausgabe erhalten bleibt.
Um das zu erreichen nimmt man die Ausgaben des Systems mit in den Zustand des Gesamtsystems.
Dies vergrößert natürlich den Zustandsraum des resultierenden Systems, aber reduziert undefinierte Ausgaben des Gesamtsystems.

\section{Formale Definition}
\begin{notation}
  Die Projektion, die für eine gegebene Indexmenge $I$ die entsprechenden Elemente aus einem Tupel $X$ auswählt, ohne die Reihenfolge zu verändern wird mit
  \[ X |_I \]
  abgekürzt.
  Es gilt also beispielsweise:
  \[ (a,b,c,d) |_{\{0,2\}} = (a,c) \]
\end{notation}
\label{sec:gals_formal_definition}
Eine synchrone Komponente mit $n$ Eingängen und $m$ Ausgängen lässt sich als ein modifizierter Mealy-Automat $\mathcal{A} = (Q,\Sigma,\Omega,\delta,q_0)$ darstellen:
\begin{itemize}
  \item $Q$ ist eine (endliche) Zustandsmenge.
  \item $\Sigma = \Sigma_0\times\dots\times\Sigma_n$ ist die Menge der Eingabesymbole.
    Da der Automat $n$ Eingänge besitzt, ist die Eingabe ein Tupel aus $n$ Symbolen.
  \item $\Omega = \Omega_0\times\dots\times\Omega_m$ ist die Menge der Ausgabesymbole.
  \item $\delta : Q\times\Sigma\rightarrow Q\times\Omega$ ist die Übergangsfunktion, die einen Zustand und eine Eingabe auf einen neuen Zustand und eine Ausgabe abbildet.
    Da es sich um eine Funktion und keine Relation handelt, ist der Automat deterministisch.
  \item $q_0\in Q$ ist der Startzustand des Automaten.
\end{itemize}

\begin{definition}
  Ein GALS-System ist ein Tripel $\mathcal{G}=(A,p,C)$ mit $A$ als einer Menge von Automatennamen, $p$ als eine Funktion, die Automatennamen einen konkreten Mealy-Automaten zuordnet und $C\subseteq (A\times\mathbb{N})\times(A\times\mathbb{N})$ das die Verbindungen zwischen Ein- und Ausgaben der Automaten definiert.
\end{definition}
Ein Tupel $(a,n,b,m)\in C$ bezeichnet also eine Verbindung der $n$-ten Ausgabe des Mealy-Automaten $p(a)$ mit der $m$-ten Eingabe des Mealy-Automaten $p(b)$.

Um die Referenzierung von Automatenkomponenten zu ermöglichen, benutzen wir folgende Konvention:
\[ p(a) = (Q^a,\Sigma^a,\Omega^a,\delta^a,q_0^a) \]
Eine Aus- oder Eingabe ist spezifiziert durch den Automaten und den Index im Ein- oder Ausgabetupel.
Für eine Verbindung $((a,i),(b,j))\in C$ muss immer gelten, dass die Ausgabesymbole den Eingabesymbolen entsprechen, also $\Omega_i^a = \Sigma_j^b$ gilt.

\begin{notation}
  Es kann ohne Beschränkung der Allgemeinheit davon ausgegangen werden, dass eine Ordnung auf den Automatennamen existiert, die bei der Erstellung von Eingabe- und Ausgabetupeln eingehalten wird.
  Dann wird die Projektion, die aus einem solchen Tupel $X$ alle Elemente selektiert, die mit dem Automatennamen $a$ indiziert sind, mit
  \[ X|^a \]
  notiert.
  Es gilt dann also beispielsweise:
  \begin{align*}
    s &\in \Sigma^a_0\times\Sigma^a_2\times\Sigma^b_1\times\Sigma^b_2\\
    s &= (a,b,c,d)\\
    s|^b &= (c,d)
  \end{align*}
\end{notation}

Die Ein- und Ausgaben des GALS Systems ergeben sich aus den Automaten des Systems sowie den Verbindungen.
Ist eine Eingabe eines Automaten mit keiner Ausgabe verbunden, so ist sie automatisch eine Eingabe des Gesamtsystems.
%Die Indexmenge der Eingaben lässt sich also schreiben als
%\[ I_a = \{ n\ |\ n\in\mathbb{N}, \lnot\exists ((X,i),(Y,j))\in C: Y=a\land j=n \} \]
Die Eingaben $I(\mathcal{G})$ des Gesamtsystems sind also
\[ I(\mathcal{G}) = \prod \{ \Sigma^a_n\ |\ a\in A, n\in \mathbb{N}, \lnot\exists ((X,i),(Y,j))\in C: Y=a\land j=n \} \]
Für die Ausgaben $O(\mathcal{G})$ gilt entsprechend
\[ O(\mathcal{G}) = \prod \{ \Omega^a_n\ |\ a\in A, n\in \mathbb{N}, \lnot\exists ((X,i),(Y,j))\in C: X=a\land i=n \} \]
Der Zustandsraum $S(\mathcal{G})$ des Systems ergibt sich dann aus den Zuständen der einzelnen Automaten ($S_Q(\mathcal{G})$) sowie dem aktuellen Inhalt der Verbindungen ($S_C(\mathcal{G})$):
\begin{align*}
  S_Q(\mathcal{G}) &= \prod_{a\in A} Q^a\\
  S_C(\mathcal{G}) &= \prod_{((a,i),\_)\in C} \Sigma^a_i\\
  S(\mathcal{G}) &= S_Q(\mathcal{G})\times S_C(\mathcal{G})
\end{align*}
Da die Eingaben für einen konkreten Automaten sich nun entweder im Zustandsraum $S(\mathcal{G})$ oder in den globalen Eingaben $I(\mathcal{G})$ befinden können, definiert man eine Familie von Hilfsfunktionen
\begin{align*}
  \pi_I^a &\in S(\mathcal{G})\times I(\mathcal{G})\rightarrow Q^a\times\Sigma^a\\
  \pi_I^a((q,s),i) &= ((q|^a,s|^a),i|^a)
\end{align*}
für $a\in\mathcal{A}$, die die benötigten Parameter für den gegebenen Automaten $a$ extrahieren.
\begin{notation}
  Um alle zum Automaten $a$ gehörigen Komponenten eines Tupels $X$ unter berücksichtigung der Reihenfolge durch alle Komponenten aus dem Tupel $Y$ zu ersetzen, wird die folgende Notation verwendet:
  \[ X[a\mapsto Y] \]
  Es gilt:
  \[ \left.X[a\mapsto Y]\right|_a = Y \]
  und
  \[ \forall b\in A,b\neq a: \left.X[a\mapsto Y]\right|_b = X|_b \]
\end{notation}
Analog definiert man eine weitere Familie von Funktionen
\begin{align*}
  \pi_O^a &\in S(\mathcal{G})\times Q^a\times\Omega^a\rightarrow S(\mathcal{G})\times O(\mathcal{G})\\
  \pi_O^a ((q,s),q',o) &= ((q[a\mapsto q'],s[a\mapsto o]),(\bot,\dots,\bot)[a\mapsto o])
\end{align*}
die Ausgaben und Zustand eines Automaten zurück in den Zustandsvektor des Gesamtsystems sowie die Gesamtausgaben schreibt.
%Dabei gilt
%\[ s'|_{\{n\}}^b = \left\{\begin{array}{lr}
%    o|_{\{m\}} & \exists ((a,m),(b,n))\in C\\[10pt]
%    s|_{\{n\}}^b & \textrm{sonst}
%  \end{array}\right. \]
Um die Notation zu erleichtern, wird zusätzlich noch eine Zustandsübergangsfunktion $\lambda$ definiert, die den Zustandsübergang des GALS Systems bei Ausführung eines Automatenschritts angibt.
\begin{align*}
  \lambda &\in S(\mathcal{G})\times I(\mathcal{G})\times A\rightarrow S(\mathcal{G})\times O(\mathcal{G})\\
  (q,i,a) &\mapsto \pi_O^a(q,\delta^a(\pi_I^a(q,i)))
\end{align*}
Der Initialzustand $\alpha(\mathcal{G})\in\mathcal{S}(\mathcal{G})$ ergibt sich als Kombination aller Initialzustände der Komponenten und den undefinierten Verbindungen
\[ \alpha(\mathcal{G}) = (\langle q_0^a\rangle_{a\in A},(\bot,\dots,\bot)) \]

Das System in Abbildung \ref{fig:mealy3} wird formal beispielsweise so dargestellt:
\[ \mathcal{G} = (\{a_1,a_2\},p,\{ (a_1,0),(a_2,0) \}) \]
Wobei die Funktion $p$ definiert ist als:
\begin{eqnarray*}
  p : a_1&\mapsto& (\{p_0,p_1\},\emptyset,\mathbb{B},\delta^1,p_0)\\
      a_2&\mapsto& (\{q_0,q_1\},\mathbb{B}\times\mathbb{N},\mathbb{N},\delta^2,q_0)
\end{eqnarray*}
Wobei $\delta^1$ und $\delta^2$ wie folgt definiert sind:
\begin{eqnarray*}
  \delta^1 : p_0 &\mapsto& (p_1,0)\\
             p_1 &\mapsto& (p_0,1)\\
  \delta^2 : (q_0,\lnot\mathit{on}\lor x\geq 5) &\mapsto& (q_0,3)\\
             (q_0,\mathit{on}\land x<5) &\mapsto& (q_1,1)\\
             (q_1,x<5) &\mapsto& (q_1,2)\\
             (q_1,x\geq 5) &\mapsto& (q_0,0)
\end{eqnarray*}
Daraus ergeben sich die abgeleitetenden Eingaben, Ausgaben und Zustandsmengen zu:
\begin{eqnarray*}
  I(\mathcal{G}) &=& \mathbb{N}\\
  O(\mathcal{G}) &=& \mathbb{N}\\
  S(\mathcal{G}) &=& \{p_0,p_1\}\times\{q_0,q_1\}\times\mathbb{B}
\end{eqnarray*}

\section{Semantik}
\label{sec:semantic}
Man kann nun das maximale Transitionssystem des GALS-Systems wie folgt über die Transitionsrelation $T$ definieren:
\[ \xymatrix { s \ar[r]^{i/o}_>T & s' } \Leftrightarrow \exists a\in A: \lambda(s,i,a) = (s',o) \]
Je nach dem, was man für Systeme betrachtet ist dieses Transitionssystem aber nicht unbedingt realistisch:
Es schließt beispielsweise nicht aus, dass immer nur Schritte von einer Komponente durchgeführt werden, während alle anderen Komponenten nie einen Schritt machen (sie "`verhungern"' sozusagen).
Viele Formeln, die bei einer realistischen Ausführung des Systems erfüllt sind können so Fehler produzieren und eine formale Verifikation unmöglich machen.
Daher werden in diesem Abschnitt verschiedene Ausführungsarten mit ihren Vor- und Nachteilen vorgestellt.
Dazu wird zunächst informell die Ausführungsart beschrieben und danach formal die Herleitung des entsprechenden Transitionssystems erklärt.

\subsection{Synchrone Ausführung}
Bei der synchronen Ausführung führen alle Systeme gleichzeitig ihren Berechnungsschritt aus.
Da eine echte Gleichzeitigkeit aber von den in dieser Arbeit verwendeten Verifikationsformalismen nicht unterstützt wird, muss sie dadurch angenähert werden, dass die Komponenten zwar nacheinander ihre Schritte ausführen, aber dies  immer in der gleichen Reihenfolge tun.

\begin{figure}[h]
  \centering
  \begin{tikzpicture}
    \node[fill=d1,draw=black,minimum height=1cm,minimum width=0.5cm,label=below:$P_1$] at (0,0) {};
    \node[fill=d2,draw=black,minimum height=1cm,minimum width=0.5cm,label=below:$P_2$] at (0.5,0) {};
    \node[fill=d3,draw=black,minimum height=1cm,minimum width=0.5cm,label=below:$P_3$] at (1,0) {};

    \node[fill=d1,draw=black,minimum height=1cm,minimum width=0.5cm,label=below:$P_1$] at (1.5,0) {};
    \node[fill=d2,draw=black,minimum height=1cm,minimum width=0.5cm,label=below:$P_2$] at (2,0) {};
    \node[fill=d3,draw=black,minimum height=1cm,minimum width=0.5cm,label=below:$P_3$] at (2.5,0) {};

    \node at (3.25,0) {\dots};
    
  \end{tikzpicture}
  \caption{Synchrone Ausführung}
  \label{fig:synchronized_execution}
\end{figure}

In Abbildung \ref{fig:synchronized_execution} wird eine mögliche Ausführung der drei Prozesse $P_1$, $P_2$ und $P_3$ gezeigt.
Andere Ausführungsmöglichkeiten unterscheiden sich nur durch die Reihenfolge, in der die Prozesse ihren Berechnungsschritt ausführen.
Für ein System mit $n$ Prozessen gibt es also $n!$ Möglichkeiten der Ausführung.

Vorteile dieser Architektur sind eine extrem einfache Implementierung, sowie wenige Ausführungsreihenfolgen, die bei der Verifikation in Betracht gezogen werden müssen.
Das zu verifizierende Zustandsmodell des Systems hat also sehr viel weniger Zustände als die der anderen Architekturen.
Der Nachteil ist jedoch, dass es für viele Szenarien sehr unrealistisch ist, perfekte Synchronität zu fordern.
In Kommunikationsnetzen bedeuten schon minimale Verzögerungen bei der Zustellung von Nachrichten, dass Prozesse nicht mehr echt synchron laufen, selbst wenn ihre Uhren genau gleich gehen.

Formal lässt sich das Transitionssystem für die synchrone Ausführung herleiten, indem man den Automaten, der als nächstes ausgeführt werden soll in den Zustand übernimmt.
Sei ohne Beschränkung der Allgemeinheit $A=\{P_0,P_1,\dots,P_{n-1}\}$.
%Außerdem muss man eine Funktion $f : A\rightarrow A$ angeben, die die Ausführungsreihenfolge festlegt.
Der Zustandsvektor des Gesamtsystems ist dann $(a,s)$, wobei $a\in A$ ein Automatenname und $s\in S(\mathcal{G})$ der Zustand des Systems ist.
Die Zustandsübergangsrelation ergibt sich dann wie folgt:
\[ \xymatrix { (P_j,s) \ar[r]^{l_i/l_o} & (P_k,s') } \Leftrightarrow k=(j+1)\bmod n\land \lambda(s,l_i,a) = (s',l_o) \]

\subsection{Vollständig asynchrones System}
%In der vollständig asynchronenen Semantik können Komponenten zu jedem Zeitpunkt, unabhängig von dem Ausführungsstand der anderen, einen Schritt ausführen.
%Das bedeutet also, dass auch Extremfälle wie der, bei dem nur eine Komponente die gesamte Zeit Schritte ausführt, berücksichtigt werden.
%Diese Semantik deckt zwar jede Ausführungsreihenfolge der Komponenten ab, ist aber nicht unbedingt realistisch.
Im Gegensatz zur synchronen Architektur steht die vollständig asynchrone: 
Hier kann zu jedem Zeitpunkt jeder Prozess unabhängig vom Fortschritt der anderen einen Berechnungsschritt ausführen.
Eine mögliche asynchrone Ausführung von drei Prozessen ist in Abbildung \ref{fig:asynchronous_execution} gezeigt.
\begin{figure}[h]
  \centering
  \begin{tikzpicture}
    \node[fill=d2,draw=black,minimum height=1cm,minimum width=0.5cm,label=below:$P_2$] at (0,0) {};
    \node[fill=d2,draw=black,minimum height=1cm,minimum width=0.5cm,label=below:$P_2$] at (0.5,0) {};
    \node[fill=d2,draw=black,minimum height=1cm,minimum width=0.5cm,label=below:$P_2$] at (1,0) {};
    \node[fill=d3,draw=black,minimum height=1cm,minimum width=0.5cm,label=below:$P_3$] at (1.5,0) {};
    \node[fill=d3,draw=black,minimum height=1cm,minimum width=0.5cm,label=below:$P_3$] at (2,0) {};
    \node[fill=d3,draw=black,minimum height=1cm,minimum width=0.5cm,label=below:$P_3$] at (2.5,0) {};
    \node[fill=d1,draw=black,minimum height=1cm,minimum width=0.5cm,label=below:$P_1$] at (3,0) {};
    \node[fill=d3,draw=black,minimum height=1cm,minimum width=0.5cm,label=below:$P_3$] at (3.5,0) {};
    \node at (4.25,0) {\dots};
  \end{tikzpicture}
  \caption{Asynchrone Ausführung}
  \label{fig:asynchronous_execution}
\end{figure}

Eine asynchrone Architektur löst das Problem der synchronen Architektur, indem sie sämtliche theoretisch mögliche Verschachtelungen der Ausführungen der Prozesse bei der Verifikation berücksichtigt.
Dies führt aber zu zwei neuen Problemen:
Zum einen nimmt die Zustandsgröße des Systems eventuell gewaltig zu; $n$ Prozesse haben nach $m$ Ausführungsschritten bereits $n^m$ mögliche Ausführungen.
Zwar führen meist viele unterschiedliche Verschachtelungen zu den selben Zuständen, in diesem Fall kann die Technik der "`partial order reduction"'~\cite{partial_order_reduction} verwendet werden, aber im schlimmsten Fall führt eine vollständig asynchrone Architektur zu einer gewaltigen Zustandsexplosion.
Das zweite Problem ist, dass diese Architektur auch extrem unrealistische Ausführungen in Erwägung zieht:
Ein Prozess kann zum Beispiel immer rechnen, während ein anderer nie zum Zuge kommt.
In der Verifikation können so Fehlerzustände erkannt werden, die in der Realität nie vorkommen.

\begin{figure}[h]
  \centering
  \includegraphics[scale=.5]{async}
  \caption{Mögliche Ausführungspfade eines asynchronen Systems}
  \label{fig:asynchronous_paths}
\end{figure}
\subsection{Asynchrone Ausführung mit Fairness}
Um das Problem zu umgehen, dass einzelne Prozesse "`verhungern"', also nie einen Rechenschritt ausführen dürfen, kann man so genannte \emph{Fairness}-Kriterien definieren:
Diese besagen, dass nur Ausführungen für die Verifikation in Betracht gezogen werden, in denen jeder Prozess immer mal wieder an die Reihe kommt.
Viele Verifikationsformalismen unterstützen die Modellierung von Fairness durch die Definition von Zuständen, die immer mal wieder erreicht werden müssen, damit die Ausführung in Betracht gezogen wird.

Da Fairness-Eigenschaften aber etwas umständlich zu formulieren sind und für den Rest der Arbeit keine Bedeutung haben, wird das resultierende Transitionssystem hier nicht explizit angegeben.
Mehr zum Thema Fairness kann in \cite{model_checking_fairness} nachgelesen werden.
\subsection{Asynchrone Ausführung mit Schranken}
%Um die Probleme der vollständig asynchronen Ausführung zu umgehen kann man die Asynchronität soweit begrenzen, dass die Anzahl der ausgeführten Schritte nie um mehr als einen festen Wert divergiert.
Obwohl das Hinzufügen von Fairness-Eigenschaften die Fälle entfernt, in denen ein Prozess niemals zur Ausführung kommt, so werden immer noch extrem unrealistische Szenarien betrachtet:
In der Realität wird es beispielsweise nie vorkommen, dass ein Prozess nur ein mal einen Berechnungsschritt durchführt, während ein anderer im gleichen Zeitraum 1000 ausführt.
Wesentlich realistischere Ausführungen erreicht man, wenn einzelnen Prozessen nur für einen gewissen Zeitraum erlaubt, mehr oder weniger Schritte als die anderen auszuführen.
Hierzu zählt man die Berechnungsschritte, die jeder Prozess bereits ausgeführt hat und überprüft, dass zu jedem Zeitpunkt der Verifikation die Bedingung "`Keine zwei Prozesse liegen um mehr als $x$ Berechnungsschritte von einander entfernt"' erfüllt ist.

Genau wie Fairness-Eigenschaften ist diese Eigenschaft jedoch auch formal unhandlich zu definieren, nicht relevant für den Rest der Arbeit und daher weg gelassen.
\section{Kontrakte}
\label{sec:contracts}
Eine Komponente in einem GALS-System besitzt durch ihre Spezifikation als Mealy-Automat eine Menge von Verhaltensweisen.
Jede Verhaltensweise ist eine Kombination aus Eingaben und Ausgaben.
Betrachtet man den Raum aller Verhaltensweisen, also aller Kombinationen von Eingaben und Ausgaben, so nimmt jede Komponente einen Teilraum ein.

Formuliert man nun mithilfe von LTL-Formeln Bedingungen an das Gesamtsystem aus Komponenten, so spezifiziert man für jede Komponente einen neuen Raum, nämlich den des \emph{erlaubten} Verhaltens.
Erfüllt das Gesamtsystem die LTL-Formeln, so ist das Verhalten jeder Komponente ein Teilraum des erlaubten Verhaltens.

Ein Problem in der formalen Verifikation ist, dass der Automat, der eine Komponente repräsentiert, sehr komplex seien kann.
Die Verifikation benötigt dann sehr viel Speicher, um jeden Zustand der Kompontente zu erfassen.
Es ist aber häufig möglich, einen ähnlichen Automaten zu finden, der wesentlich kleiner ist und trotzdem jedes Verhalten zeigt, dass die ursprüngliche Komponente besaß.
Dieser Automat kann auch mehr Verhalten besitzen, vorausgesetzt, dieses Verhalten liegt immer noch innerhalb des erlaubten Verhaltens.
Ein solcher Automat heißt \emph{Kontrakt} und lässt sich beispielsweise als LTL-Formel darstellen.
Der erläuterte Zusammenhang zwischen \emph{Verhalten}, \emph{erlaubtem Verhalten} und \emph{Kontrakten} ist in Abbildung \ref{fig:contracts} illustriert.
\begin{figure}[h]
  \centering
  \begin{tikzpicture}
    \fill[fill=d1!40,draw=d1,thick] decorate[decoration={snake}] { (-0.4,-0.6) circle (2.4) };
    \fill[fill=d2!40,draw=d2,thick] decorate[decoration={random steps,segment length=2mm,amplitude=2mm}] {(0,0) circle (1)};
    \draw[dash pattern=on 2pt off 2pt,rotate=45] (0,0) ellipse (1.5 and 1.8);
    \node at (0,0) {Verhalten};
    \node at (-1,-2.2) {\begin{tabular}{c}Erlaubtes\\Verhalten\end{tabular}};
    \node at (-0.2,1.4) {Kontrakt};
  \end{tikzpicture}
  \caption{Verhalten und Kontrakte}
  \label{fig:contracts}
\end{figure}

Ist ein vereinfachender Kontrakt-Automat gefunden, so muss die formale Verifikation zunächst beweisen, dass der Kontrakt von der ursprünglichen Komponente eingehalten wird.
Dies kann beispielsweise festgestellt werden, indem der Kontrakt in den Formalismus der Komponente übersetzt wird und dort verifiziert wird.
Ist gesichert, dass alle Kontrakte von den Komponenten erfüllt werden, so werden die Kontrakte verwendet, um das Gesamtsystem zu verifizieren.
Sind die Kontrakte allerdings zu locker formuliert, spezifizieren also mehr Verhalten als die zu verifizierende Formel erlaubt, so werden bei der Verifikation Fehler gefunden, die bei einer normalen Verifikation ohne Kontrakte nicht auftreten würden.
Eine Lösung für dieses Problem wird in Abschnitt \ref{sec:error-refinement} vorgestellt.

Die Formulierung von Kontrakten stellt einen Balance-Akt dar:
Formuliert man die Kontrakte zu scharf, so hat der resultierende Kontrakt-Automat ähnlich viele Zustände wie die Komponente und es gibt keinen Gewinn durch die Verwendung von Kontrakten.
Wird der Kontrakt jedoch zu lose formuliert so bekommt der Kontrakt-Automat viele Transitionen und zeigt Verhalten, dass die Verifikation der Systemeigenschaft unmöglich machen.

Um gute Kontrakte zu formulieren, muss der Anwender zwischen relevanten und irrelevanten Verhaltensweisen unterscheiden.
Ist beispielsweise für die zu verifizierende Eigenschaft unerheblich, welchen konkreten Wert eine Variable aufweist, sondern nur wichtig, dass der Wert eine bestimmte Eigenschaft erfüllt, so lässt sich häufig ein Kontrakt finden, der den Wert der Variable nicht-deterministisch auf einen Wert setzt.
\chapter{Grundlagen}
\label{sec:basics}
In diesem Kapitel werden für diese Arbeit relevante Konzepte kurz zusammengefasst und erklärt.
Abschnitte über Konzepte, die dem Leser bereits bekannt sind können also ohne Probleme übersprungen werden.
\section{SPIN}
SPIN ist ein Model-Checker für asynchrone Software-Modelle, die in der Sprache Promela ({\bf Pro}cess {\bf Me}ta {\bf La}nguage) definiert sind~\cite{spinbook}.
Das Werkzeug verwendet "`Explicit-State Model Checking"'-Techniken, berechnet also jeden möglichen Zustand des Systems und überprüft, ob die zu verifizierenden Eigenschaften erfüllt sind.
SPIN unterstützt eine Reihe von Optimierungstechniken, darunter
\begin{itemize}
\item "`Partial order reduction"' um den Zustandsraum von Modellen zu verkleinern~\cite{partial_order_reduction}.
\item Zustandskompressionstechniken, die den Speicherbedarf von Verifikationen senken können~\cite{spin_state_compression}.
\item Nutzung von Multi-Prozessor-Systemen zur Geschwindigkeitsverbesserung~\cite{spin_multi_core}.
\end{itemize}
Es werden sowohl die Verifikation von \emph{Safety}-Eigenschaften ("`Es passiert nie etwas schlimmes"') als auch von \emph{Liveness}-Eigenschaften ("`Es passiert immer mal wieder etwas gutes"') unterstützt.
Die Verifikation von Modellen erfolgt nicht direkt durch SPIN selbst, sondern es wird Code für einen domänenspezifischen Modell-Checker generiert.
\section{SCADE}
\label{sec:scade}
SCADE ist ein Formalismus und Werkzeug zur Erstellung sicherheitskritischer Softwaresysteme\footnote{SCADE wird von der Firma \emph{Esterel Technologies} vertrieben, zu finden unter \url{http://esterel-technologies.com}}.
Zentraler Bestandteil ist die auf der synchronen Sprache Lustre~\cite{lustre} aufbauende Beschreibungssprache.
Die Sprache ist Datenfluss-orientiert, es werden also Gleichungen für die Ein- und Ausgabevariablen formuliert.
Es stehen aber auch andere Mechanismen, wie zum Beispiel \emph{Safe State-Machines}~\cite{ssm} zur Verfügung.

Außerdem beinhaltet die SCADE Suite noch einen Model-Checker, den \emph{Design Verifier}, mit dessen Hilfe sich Eigenschaften von Modellen nachprüfen lassen\footnote{Entwickelt von der Firma \emph{Prover}, zu finden unter \url{http://prover.com}}.
Der Design Verifier hat die Einschränkung, dass sich keine Liveness-Eigenschaften mit ihm nachweisen lassen.
Aussagen wie "`Die Ausgangsvariable $x$ wird irgendwann einmal wahr"' lassen sich also nicht beweisen.
Stattdessen werden solche Aussagen in SCADE beispielsweise verschärft zu "`Die Ausgangsvariable $x$ wird innerhalb von $t$ Ausführungsschritten einmal wahr"'.
\section{LTL -- Linear temporal logic}
\label{sec:ltl}
LTL-Formeln können benutzt werden, um das zeitabhängige Verhalten von Systemen zu beschreiben~\cite{principles_of_model_checking}.
Der LTL-Formalismus wurde von Amir Pnueli~\cite{ltlbasics} erfunden.
Die LTL-Logik stellt eine Erweiterung der Aussagenlogik dar, so dass nicht nur Aussagen über den jetzigen Zustand getroffen werden können, sondern auch über noch folgende Zustände.
Die Aussagenlogik wird um folgende Operatoren erweitert:
\begin{itemize}
\item Der \emph{next}-Operator (Auch mit $\bigcirc$ bezeichnet) sagt aus, dass eine Formel im nächstens Zustand gilt.
  Die Formel $\bigcirc\varphi$ spezifiziert also Pfade der Form
  \[ \xymatrix @R=0em {
      \bullet \ar[r] & \bullet \ar[r] & \bullet \ar@{-->}[r] &\\
      & \varphi & &
  }
    \]
\item Mit dem \emph{always}-Operator ($\square$) versehene Formeln gelten sowohl im aktuellen wie auch in allen folgenden Zuständen.
  Die Formel $\square\varphi$ erlaubt also alle Pfade der Form
  \[ \xymatrix @R=0em {
      \bullet \ar[r] & \bullet \ar[r] & \bullet \ar@{-->}[r] &\\
      \varphi & \varphi & \varphi &
  }
    \]
\item Der \emph{finally}-Operator ($\diamond$) gibt an, dass eine Formel irgendwann in der Zukunft einmal gelten wird.
  Die Formel $\diamond\varphi$ spezifiziert zum Beispiel Pfade der Form
  \[ \xymatrix @R=0em {
    \bullet \ar[r] & \bullet \ar@{-->}[r] & \bullet \ar[r] & \bullet \ar@{-->}[r] & \\
    & & \varphi & &
  } \]
\item Formeln die gelten sollen, bis eine bestimmte Bedingung erfüllt ist, lassen sich mit dem \emph{until}-Operator ($U$) angeben.
  Wird beispielsweise gefordert, dass die Formel $\varphi$ gilt, bis $\psi$ gilt, so lässt sich dies schreiben als $\varphi U\psi$.
  Ein Beispielpfad für diese Formel ist
  \[ \xymatrix @R=0em {
    \bullet \ar[r] & \bullet \ar@{-->}[r] & \bullet \ar[r] & \bullet \ar@{-->}[r] & \\
    \varphi & \varphi & \varphi & \psi &
  } \]
\end{itemize}
Um die vollständige Mächtigkeit von LTL zu erreichen reicht es allerdings auch schon, nur die Operatoren $\bigcirc$ und $U$ zu haben, denn der \emph{finally}-Operator lässt sich ausdrücken als
\[ \diamond\varphi = \top U \varphi \]
und der \emph{always}-Operator als
\[ \square\varphi = \lnot (\top U \lnot\varphi) \]
Alle anderen Operatoren sind also zwar nützlich, aber nicht benötigt.
\section{Büchi-Automaten}
Büchi-Automaten stellen eine Erweiterung von endlichen Automaten auf unendliche Eingaben dar\cite{buchibasics}.
Anders als endliche Automaten, die eine Eingabe akzeptieren, wenn die Ausführung in einem finalen Zustand endet, akzeptiert ein Büchi-Automat eine Eingabe genau dann, wenn die Ausführung unendlich oft einen finalen Zustand erreicht.

Formal ist ein Büchi-Automat ein Tupel
\[ A = (Q,\Sigma,\delta,\mu,q_0,F) \]
wobei die Symbole folgende Bedeutung haben:
\begin{itemize}
  \item $Q$ ist die Menge der Zustände des Automaten.
  \item Die Menge von atomaren Aussagen $\Sigma$, die gültig oder ungültig sein können.
  \item $\delta\subseteq Q\times Q$ ist die Übergangsrelation des Automaten.
  \item $\mu : Q\rightarrow\Sigma$ gibt an, welche Aussagen in einem Zustand gelten müssen.
  \item $q_0\subseteq Q$ ist die Startzustandsmenge.
  \item $F\subseteq Q$ ist die Finalzustandsmenge.
\end{itemize}
Eine Folge von Aussagen $a_0a_1a_2\dots$ wird nun also genau dann akzeptiert, wenn es eine Folge von Zuständen $q_0q_1q_2\dots$ gibt, wobei stets gilt $q_n\delta q_{n+1}$ und in der mindestens ein Zustand aus $F$ unendlich oft vorkommt.
Außerdem muss jede Aussage $a_n$ mit der Menge von Aussagen $\mu(q_n)$ kompatibel sein.

\subsection{Verallgemeinerter Büchi-Automat}
\label{sec:gba}
Ein Verallgemeinerter Büchi-Automat\footnote{Im englischen als "`generalized buchi automaton"' bezeichnet und als \emph{GBA} abgekürzt.} unterscheidet sich von einem normalen Büchi-Automaten durch die Definition des Akzeptanzverhaltens.
Während ein Büchi-Automat akzeptiert, wenn unendlich oft ein Finalzustand betreten wird, ist $F$ hier eine Menge von Finalzustandsmengen ($F\subseteq\mathcal{P}(Q)$).
Der Automat akzeptiert nun, wenn aus jeder Finalzustandsmenge mindestens ein Zustand unendlich oft betreten wird.

Ein verallgemeinerter Büchi-Automat lässt sich in einen normalen Büchi-Automaten übersetzen, indem man für jede Finalzustandsmenge eine "`Ebene"' einführt.
Jede Ebene enthält die gleichen Zustände und Transitionen wie der ursprüngliche Automat, nur dass beim Verlassen von den Finalzuständen der aktuellen Finalzustandsmenge die nächste Ebene betreten wird.
Dadurch wird erreicht, dass ein Zyklus nur dann zustande kommt, wenn ein Zustand aus allen Finalzustandsmengen betreten wird.

Für einen verallgemeinerten Büchi-Automaten $(Q,\Sigma,\delta,\mu,q_0,F)$ konstruiert man also einen Büchi-Automaten $(Q',\Sigma,\delta',\mu',q_0',F')$.
Dazu benötigt man zunächst eine Hilfsfunktion $i : F\rightarrow F$, die zyklisch durch alle Finalzustandsmengen geht.
\begin{align*}
  Q' &= \{ (q,f)\ |\ q\in Q, f\in F \}\\
  \delta' &= \{ ((q_1,f_1),(q_2,f_2))\ |\ (q_1,q_2)\in\delta, (q_1\in f_1\land i(f_1)=f_2)\lor f_1=f_2\}\\
  \mu'((q,f)) &= \mu(q)\\
  F' &= \{ (q,f)\ |\ f\in F, q\in f \}
\end{align*}

\section{LTL Übersetzung}
Um LTL Formeln einfacher übersetzen zu können und zu kanonisieren, werden diese in Büchi-Automaten übersetzt.
Diese Übersetzung benutzt den in \cite{Gerth95simpleon-the-fly} beschriebenen Algorithmus.

Da der Übersetzungsalgorithmus keine \emph{always}-Konstrukte zulässt, müssen diese zunächst mit der folgenden Identität transformiert werden:
\[ \textbf{always}\ \varphi = \lnot\top U \lnot\varphi \]
Außerdem müssen für den Algorithmus alle Negationen so weit wie möglich nach innen geschoben werden, bis sie nur noch vor atomaren Aussagen stehen.
Um die Größe der Formeln nicht unnötig zu erhöhen wird dual zum \emph{until}-Operator $U$ der Operator $V$ eingeführt, der über die folgende Identität definiert ist:
\[ \varphi V\psi = \lnot (\lnot\varphi U\lnot\psi) \]

Für die Konstruktion des Büchi-Automaten wird ein Graph aufgebaut, der Schrittweise erweitert wird, bis der Büchi-Automat vollständig ist.
Die Knoten des Graphen benötigen die folgenden Felder:
\begin{description}
\item[Name] Ein eindeutiger Bezeichner für den Knoten.
  Es wird vorausgesetzt, dass es eine Funktion \emph{new\_name} existiert, die bei jedem Aufruf einen neuen, eindeutigen Namen zurück gibt.
\item[Incoming] Gibt die Knoten an, die eine Kante in diesen Knoten besitzen.
  Das Symbol \emph{init} wird verwendet, um anzuzeigen, dass der Knoten initial ist.
\item[New] Eine Liste von Formeln, die noch nicht bearbeitet wurde.
\item[Old] Die Liste der Formeln, die bereits abgearbeitet wurden.
\item[Next] Formeln, die in allen Nachfolgeknoten gelten müssen.
\end{description}
Der Anfangsknoten hat einen beliebigen Namen, das Symbol \emph{init} in der \emph{Incoming}-Menge und die gesamte zu übersetzende Formel als \emph{New}-Feld.
Die restlichen Felder sind leer.
Der zentrale Bestandteil des Algorithmus ist die \emph{expand}-Funktion.
Diese nimmt einen Knoten und die Menge aller bisher generierten Knoten und erstellt durch rekursive Aufrufe ihrer selbst die resultierende Knotenmenge.

\begin{codebox}
\Procname{$\proc{Expand}(Node,NodeSet)$}
\li \If $\attrib{Node}{New}\isequal\emptyset$ \Then
\li \If $\exists N\in NodeSet: \attrib{N}{Old}\isequal\attrib{Node}{Old} \mbox{ and } \attrib{N}{Next}\isequal\attrib{Node}{Next}$ \Then
\li $\attrib{N}{Incoming}\gets \attrib{N}{Incoming}\cup\attrib{Node}{Incoming}$
\li \Return $NodeSet$
\li \Else \Return $\proc{Expand}([Name\gets \proc{NewName}(),Incoming\gets \{\attrib{Node}{Name}\},$
\Startalign{\Return $\proc{Expand}([$}
\> $New\gets\attrib{Node}{Next}, Old\gets\emptyset, Next\gets\emptyset],$\\
\> $\{Node\}\cup NodeSet)$
\Stopalign
\li \Else
\li $\eta\gets \attrib{Node}{New}[0]$
\li $\attrib{Node}{New}\gets\attrib{Node}{New}\setminus\{\eta\}$
\li \If $\eta\isequal P\mbox{ or }\eta\isequal\lnot P\mbox{ or }\eta\isequal\top\mbox{ or }\eta\isequal\bot$ \Then
\li \If $\eta\isequal\bot\mbox{ or }\lnot\eta\in\attrib{Node}{Old}$
\li \Then \Return $NodeSet$
\li \Else
\li $\attrib{Node}{Old}\gets \attrib{Node}{Old}\cup\{\eta\}$
\li \Return $\proc{Expand}(Node,NodeSet)$
\End
\li \ElseIf $\eta\isequal\varphi U\psi$ \Then
\li $Node1\gets [Name\gets\proc{NewName}(),Incoming\gets\attrib{Node}{Incoming},$
\Startalign{$Node1\gets [$}
\> $New\gets\attrib{Node}{New}\cup(\{\proc{New1}(\eta)\}\setminus\attrib{Node}{Old}),$\\
\> $Old\gets\attrib{Node}{Old}\cup\{\eta\},Next\gets\attrib{Node}{Next}\cup\{\proc{Next1}(\eta)\}]$
\Stopalign
\li $Node2\gets [Name\gets\proc{NewName}(),Incoming\gets\attrib{Node}{Incoming},$
\Startalign{$Node2\gets [$}
\> $New\gets\attrib{Node}{New}\cup(\{\proc{New2}(\eta)\}\setminus\attrib{Node}{Old}),$\\
\> $Old\gets\attrib{Node}{Old}\cup\{\eta\},Next\gets\attrib{Node}{Next} ]$
\Stopalign
\li \Return $\proc{Expand}(Node2,\proc{Expand}(Node1,NodeSet))$

\li \ElseIf $\eta\isequal\varphi\land\psi$\Then
\li \Return $\proc{Expand}([Name\gets\attrib{Node}{Name},Incoming\gets\attrib{Node}{Incoming},$
\Startalign{\Return $\proc{Expand}([$}
\> $New\gets\attrib{Node}{New}\cup(\{\varphi,\psi\}\setminus\attrib{Node}{Old}),$\\
\> $Old\gets\attrib{Node}{Old}\cup\{\eta\},Next\gets\attrib{Node}{Next}],NodeSet)$
\Stopalign

\li \ElseIf $\eta\isequal X\varphi$ \Then
\li \Return $\proc{Expand}([Name\gets\attrib{Node}{Name},Incoming\gets\attrib{Node}{Incoming},$
\Startalign{\Return $\proc{Expand}([$}
\> $New\gets\attrib{Node}{New},Old\gets\attrib{Node}{Old}\cup\{\eta\},$\\
\> $Next\gets\attrib{Node}{Next}\cup\{\varphi\}],NodeSet)$
\Stopalign
\End
\End
\End
\end{codebox}
Die Hilfsfunktionen \proc{New1}, \proc{New2} und \proc{Next1} sind hierbei über Tabelle \ref{tab:helper_funcs} definiert.

\begin{table}[h]
  \centering
  \begin{tabular}{|l|l|l|l|}
    \hline
    $\eta$ & \proc{New1} & \proc{Next1} & \proc{New2}\\
    \hline
    \hline
    $\varphi U\psi$ & $\{\varphi\}$ & $\{\varphi U\psi\}$ & $\{\psi\}$\\
    \hline
    $\varphi V\psi$ & $\{\psi\}$ & $\{\varphi V\psi\}$ & $\{\varphi,\psi\}$\\
    \hline
    $\varphi\lor\psi$ & $\{\varphi\}$ & $\emptyset$ & $\{\psi\}$\\
    \hline
  \end{tabular}
  \caption{Hilfsfunktionen}
  \label{tab:helper_funcs}
\end{table}

\section{Binäre Entscheidungsdiagramme}
Versucht man, in einem Programm Funktionen genau wie Daten zu behandeln, so stößt man auf verschiedene Probleme:
\begin{itemize}
\item Wird die Funktion durch ihren Code repräsentiert, so ist die Darstellung nicht nur abhängig von der Wahl der Programmiersprache, sondern auch uneindeutig, da es in einer Turing-vollständigen Sprache unendlich viele Quelltexte gibt, die eine gegebene Funktion kodieren.
\item Verwendet man zur Repräsentation die Wertetabelle der Funktion, so ist zwar eine eindeutige Kodierung sichergestellt, allerdings kann schon die Kodierung einer Funktion mit sehr kleinem Wertebereich enorm viel Speicher veranschlagen (Die Kodierung einer Funktion von 32-bit Integer nach Bool würde zum Beispiel schon 0.5 GB Daten benötigen).
\end{itemize}
Um diese Probleme zu lösen, kann man binäre Entscheidungsdiagramme (BDD)\footnote{In der englisch-sprachigen Literatur als "`binary decision diagrams"' bezeichnet und mit BDD abgekürzt} verwenden.
Diese lassen sich verwenden, um Funktionen der Form
\[ f : \mathbb{B}^n\rightarrow\mathbb{B} \]
eindeutig zu kodieren (Wobei $n$ beliebig, aber endlich ist).

Ein binäres Entscheidungsdiagramm ist ein gerichteter, azyklischer Graph des folgenden Aufbaus:
\begin{itemize}
\item Den einfachsten Fall stellen die Diagramme dar, die nur aus den Symbolen $\top$ oder $\bot$ bestehen (Abbildung \ref{fig:easy_bdd}).
  \begin{figure}[!h]
    \centering
    \begin{tabular}{cc}
      \includegraphics[scale=.5]{top} & \includegraphics[scale=.5]{bot}
    \end{tabular}
    \caption{Die einfachsten zwei Entscheidungsdiagramme}
    \label{fig:easy_bdd}
  \end{figure}
  Diese repräsentieren die Funktion, die für alle Eingaben wahr ist ($\top$) und die Funktion, die für alle Eingaben unwahr ist ($\bot$).
\item Möchte man die Funktion, die sich, falls die Variable $\alpha$ wahr ist, wie die Funktion $f_1$ und ansonsten wie $f_2$ verhält, kodieren, so erstellt man einen neuen Knoten mit der Bezeichnung $\alpha$ und verbindet ihn mit einer durchzogenen Linie mit dem Diagramm für $f_1$ und mit einer gestrichelten Linie mit dem Diagramm von $f_2$ (Abbildung \ref{fig:con_bdd}).
  \begin{figure}[!h]
    \centering
    \includegraphics[scale=.5]{con}
    \caption{Zusammengesetztes Entscheidungsdiagramm}
    \label{fig:con_bdd}
  \end{figure}
\end{itemize}
Mit diesen einfachen Konstruktionsregeln lassen sich nun beliebige Funktionen konstruieren.
Beispielsweise kann die Funktion $(\alpha\land\beta)\lor \gamma$ wie in Abbildung \ref{fig:example1_bdd} kodiert werden.
\begin{figure}[h]
  \centering
  \includegraphics[scale=.5]{example1}
  \caption{Entscheidungsdiagramm der Funktion $(\alpha\land\beta)\lor \gamma$}
  \label{fig:example1_bdd}
\end{figure}
Diese Art der Kodierung hat nun aber das folgende Problem: Sie ist nicht eindeutig.
Beispielsweise stellt das Diagramm in Abbildung \ref{fig:example2_bdd} die selbe Funktion dar.
\begin{figure}[h]
  \centering
  \includegraphics[scale=.5]{example2}
  \caption{Äquivalentes Entscheidungsdiagramm der Funktion $(\alpha\land\beta)\lor \gamma$}
  \label{fig:example2_bdd}
\end{figure}
Das liegt daran, dass das Diagramm in Abbildung \ref{fig:example1_bdd} viele redundante Knoten enthält:
Beide Äste des linken $\gamma$-Knoten führen zum gleichen Knoten, dass heißt an dieser Stelle spielt die Belegung der Variablen keine Rolle.
Die anderen beiden $\gamma$-Knoten sind äquivalent, da ihre Kinderknoten gleich sind.
Entfernt man alle diese Redundanzen, so erhält man das \emph{reduzierte} Entscheidungsdiagramm in Abbildung \ref{fig:example3_bdd}.
\begin{figure}[h]
  \centering
  \includegraphics[scale=.5]{example3}
  \caption{Reduziertes Entscheidungsdiagramm der Funktion $(\alpha\land\beta)\lor \gamma$}
  \label{fig:example3_bdd}
\end{figure}

Die Bedingung, dass das Diagramm keine redundanten Knoten aufweisen darf, reicht allerdings noch nicht aus, um eine Eindeutigkeit zu erzwingen, wie das Diagramm in Abbildung \ref{fig:example4_bdd} zeigt.
\begin{figure}[h]
  \centering
  \includegraphics[scale=.5]{example4}
  \caption{Äquivalentes reduziertes Entscheidungsdiagramm der Funktion $(\alpha\land\beta)\lor \gamma$}
  \label{fig:example4_bdd}
\end{figure}

Um dieses Problem zu lösen, kann man fordern, dass die Diagramme zusätzlich auch \emph{geordnet} sind, dass heißt es gibt eine totale Ordnung auf den Variablen und Variablen höherer Ordnung haben Verbindungen zu Variablen niedriger Ordnung, aber nicht umgekehrt.
Das Diagramm in Abbildung \ref{fig:example3_bdd} hat also die Ordnung $\alpha > \beta > \gamma$, während in Abbildung \ref{fig:example4_bdd} die Ordnung $\gamma > \alpha > \beta$ eingehalten wird.

Die so eingeführten \emph{geordneten}, \emph{reduzierten} binären Entscheidungsdiagramme haben damit eine Reihe von Vorteilen gegenüber anderen Funktionskodierungen:
\begin{itemize}
\item Sie sind eindeutig, jede Funktion hat genau ein Entscheidungsdiagramm.
  Außerdem sind sie schon eindeutig über ihren Anfangsknoten definiert, so dass ein Test auf Gleichheit in konstanter Zeit möglich ist (Zum Vergleich: Bei der Kodierung als Wertetabelle benötigt man $2^n$ Vergleiche wobei $n$ die Anzahl der Variablen ist und bei der Kodierung als Quelltext ist ein Test auf Äquivalenz in vielen Fällen prinzipiell unmöglich\footnote{Gezeigt durch die Unentscheidbarkeit des Halteproblems\cite{halteproblem}}).
\item Eine Auswertung der Funktion ist effizient möglich:
  An jedem Knoten wird entschieden, ob die entsprechende Variable wahr oder falsch ist und der entsprechende Ast verfolgt.
  Endet man bei dem Knoten $\top$, so ist das Ergebnis der Funktion wahr, endet man bei $\bot$, so ist es falsch.
\item Für viele Funktionen ist das entsprechende Entscheidungsdiagramm sehr klein.
  Allerdings lässt sich zeigen, dass es Funktionen gibt, für die keine effiziente Repräsentation als Entscheidungsdiagramm existiert, wie zum Beispiel die Multiplikationsfunktion\cite{Bryant98onthe}.
\item Speichert man mehrere Entscheidungsdiagramme, so kann man den Speicherbedarf enorm verringern, indem man gleiche Knoten zwischen Diagrammen teilt.
\end{itemize}

In dem Rest dieser Arbeit sind mit dem Begriff "`Entscheidungsdiagramm"' immer geordnete, reduzierte und geteilte Entscheidungsdiagramme gemeint.
Mehr zu Entscheidungsdiagrammen im allgemeinen lässt sich in \cite{knuth2011computer} finden.
\section{Datentypen als Entscheidungsdiagramme}
Da Enscheidungsdiagramme binäre Funktionen auf boolesche Werte kodieren, lassen sie sich auch verwenden, um Mengen von endlichen Datentypen zu kodieren.
Das ist möglich, weil sich jeder endliche Datentyp binär kodieren lässt (Notfalls durch Durchnummerierung aller möglichen Werte).
Die entsprechende Funktion gibt also \emph{wahr} zurück, wenn sich das Element, dass der binären Kodierung entspricht sich in der Menge befindet.

Ist also ein endlicher Datentyp $T$ zusammen mit einer Kodierungsfunktion $c : T\rightarrow \{0,1\}^n$ gegeben, so lässt sich ein BDD für eine beliebige Menge $Q\subseteq T$ konstruieren, indem man für jedes Element $q\in Q$ der Menge das BDD konstruiert, dass die Funktion repräsentiert, die nur bei dem Wert $c(q)$ wahr ist.
Die so generierten BDD werden dann per Disjunktion zu einem BDD zusammengefügt.
Diese Konstruktion ist aber extrem ineffizient, da sehr viele BDDs erstellt und sofort wieder verworfen werden.
Effizienter ist es, das BDD mit einem \emph{divide-and-conquer}-Algorithmus zu erstellen:
Dieser erhält eine Menge der noch zu kodierenden Werte sowie die Bitposition, an der gerade kodiert wird (beginnend bei Bit null).
Der Algorithmus teilt nun die Menge in zwei Mengen:
Die eine enthält die Werte deren Kodierung an der aktuellen Bitposition eine null aufweisen, die andere Menge enthält die restlichen Werte.
Auf diesen Mengen wird der Algorithmus nun rekursiv aufgerufen.
Die vollständige Implementierung ist in Abbildung \ref{alg:create_bdd} zu sehen.

\begin{figure}[h]
\centering
\begin{codebox}
\Procname{$\proc{CreateBDD}(Q,\id{Pos})$}
\li \If $Q\isequal\emptyset$\Then
\li \Return $\proc{Leaf}(0)$
\End
\li \If $\id{Pos}\isequal n$\Then
\li \Return $\proc{Leaf}(1)$
\End
\li $Q_l\gets \{ v\ |\ v\in Q, c(v)[\id{Pos}]\isequal 0 \}$
\li $Q_r\gets \{ v\ |\ v\in Q, c(v)[\id{Pos}]\isequal 1 \}$
\li \Return $\proc{Node}(\id{Pos},\proc{CreateBDD}(Q_l,\id{Pos}+1),\proc{CreateBDD}(Q_r,\id{Pos}+1))$
\end{codebox}
\caption{Eine Funktion, um endliche Mengen in BDDs umzuwandeln}
\label{alg:create_bdd}
\end{figure}

Um ein BDD nun wieder in eine Menge zu verwandeln, kann man das Diagramm rekursiv durchlaufen.
Die entsprechende Funktion erhält das aktuell betrachtete Teildiagramm sowie die Bitposition und den binären Wert der bis zu dieser Position kodiert wurde.
Enspricht das Diagramm dem Nullblatt, so bedeutet dies, dass vom Teildiagramm keine Werte kodiert werden, es wird also die leere Menge zurück gegeben.
Ist die Bitposition am Ende des kodierbaren Bereichs angelangt, so wird der aktuelle Wert von dem Teildiagramm kodiert und damit als einzelnes Element dekodiert und zurück gegeben.
Entspricht der aktuelle Hauptknoten des Diagramms nicht der aktuellen Bitposition, so ist der Wert an der aktuellen Bitposition egal, das Ergebnis ist also die Vereinigung der Aufrufe mit Wert null und eins an der aktuellen Bitposition.
Ansonsten wird der linke Ast des Knotens mit dem Wert eins belegt, der rechte mit null und die Ergebnisse der rekursiven Aufrufe vereinigt.
Der vollständige Algorithmus ist in Abbildung \ref{alg:decode_bdd} angegeben.

\begin{figure}[h]
\begin{codebox}
\Procname{$\proc{DecodeBDD}(\id{Tree},\id{Pos},\id{Value})$}
\li \If $\id{Tree}\isequal\proc{Leaf}(0)$\Then
\li \Return $\emptyset$
\li \ElseIf $\id{Pos}\isequal n$\Then
\li \Return $\{c^{-1}(\id{Value})\}$
\li \ElseIf $\id{Tree}\isequal\id{Leaf}(1)\mbox{ or }\attrib{Tree}{id}<\id{Pos}$\Then
\li \Return $\proc{DecodeBDD}(\id{Tree},\id{Pos}+1,\id{Value}\|(1<<\id{Pos}))\ \cup$
\Startalign{\Return}
\> $\proc{DecodeBDD}(\id{Tree},\id{Pos}+1,\id{Value})$
\Stopalign
\li \Else
\li \Return $\proc{DecodeBDD}(\attrib{Tree}{left},\id{Pos}+1,\id{Value}\|(1<<\id{Pos}))\ \cup$
\Startalign{\Return}
\> $\proc{DecodeBDD}(\attrib{Tree}{right},\id{Pos}+1,\id{Value})$
\Stopalign
\End
\end{codebox}
\caption{Algorithmus um BDDs in Mengen umzuwandeln}
\label{alg:decode_bdd}
\end{figure}

Im folgenden sollen sowohl Datentypen wie auch Algorithmen auf ihnen in BDD-Form realisiert werden.
\subsection{Integer}
Ganze Zahlen mit oder ohne Vorzeichen lassen sich sehr leicht binär kodieren.
Kodiert man natürliche Zahlen beispielsweise mit den drei Bits $b_0$, $b_1$ und $b_2$ (Wobei $b_0$ das niederwertigste Bit kodiert), so lässt sich die Menge $\{2,3,5\}$ auffassen als das Diagramm in Abbildung \ref{fig:example_set1_bdd}.
\begin{figure}[h]
  \centering
  \includegraphics[scale=.5]{example_set1}
  \caption{Entscheidungsdiagramm der Menge $\{2,3,5\}$}
  \label{fig:example_set1_bdd}
\end{figure}

Verwendet man Mengen, um verschiedene mögliche Werte einer Variable zu speichern, so kann es nützlich für die Geschwindigkeit der Verifikation sein, wenn man effiziente Algorithmen finden kann, um Operationen gleichzeitig auf den einzelnen Werten der Menge auszuführen.
Für eine binäre Operation $\circ$ wird also ein effizienter Algorithmus auf Entscheidungsdiagrammen gesucht, der für zwei Mengen $A$ und $B$ die Menge
\[ \{ a\circ b\ |\ a\in A,b\in B\} \]
berechnet.

Um beispielsweise die elementweise Addition von zwei Mengen angeben zu können, definiert man zunächst eine Hilfsfunktion "`plus"', die zusätzlich noch einen Parameter enthält, die den Übertrag der letzten Bit-Addition speichert.
Addiert man nun zwei Diagramme, deren Hauptknoten die selbe Markierung $x$ haben, so ergibt sich ein neues Diagramm mit dem Hauptknoten der Markierung $x$, der wie folgt aufgebaut ist:
\[ \textrm{plus}\Bigg(\bdd{plus1_lhs},\bdd{plus1_rhs},0\Bigg) = \bdd{plus1_res} \]
Der linke Ast des Resultats muss hierbei den Fall behandeln, dass das Ergebnis der Addition im Bit $x$ den Wert 1 erhält.
Dies kann nur dann passieren, wenn genau eines der Argument-Bits 1 ist.
Hierfür gibt es trivialerweise zwei Möglichkeiten, bei der einen ist das erste Argument 1 und der linke Ast des ersten Argument wird ohne Überhang mit dem rechten Ast des zweiten Arguments addiert; bei der anderen Möglichkeit ist es genau andersrum.
Da beide Möglichkeiten auftreten können, müssen die Ergebnisse vereinigt werden, was auf Entscheidungsdiagramm-Ebene einer Disjunktion entspricht.
Für den rechten Ast des Resultats muss man nun die Möglichkeiten betrachten, bei denen das Bit $x$ den Wert 0 erhalten kann.
Dies kann geschehen, wenn beide Bits der Argumente 0 sind, oder wenn beide 1 sind.
Im letzteren Fall muss dann beim rekursiven Aufruf der Überhang hinzugefügt werden.

Wird bei der Addition bereits ein Überhang berücksichtigt, so ändert sich die Argumentation leicht, ist aber im wesentlichen symmetrisch:
\[ \textrm{plus}\Bigg(\bdd{plus1_lhs},\bdd{plus1_rhs},1\Bigg) = \bdd{plus2_res} \]

Im Fall, dass der Knoten eines der Argumente niedrigere Ordnung besitzt -- hier wird nur der Fall betrachtet, dass das zweite Argument niedrigere Ordnung hat, da die Addition symmetrisch ist -- kann man sich überlegen, dass das zweite Argument hierbei äquivalent zu einem Diagramm ist, bei dem ein mit $x$ markierter Knoten oben steht, dessen Äste beide zum ursprünglichen Diagramm führen.
Setzt man dies dann in die erste Gleichung ein, so erhält man (für $y<x$):
\[ \textrm{plus}\Bigg(\bdd{plus1_lhs},\underbrace{\bdd{plus3_rhs}}_R,0\Bigg) = \bdd{plus3_res} \]
Die Gleichung mit Übertrag ist genau äquivalent zu den vorherigen Fällen und daher weg gelassen.

\subsection{Tupel}
Tupel lassen sich einfach binär kodieren, indem man die Binärkodierung aller Elemente hintereinander schreibt.
Diese Art der Kodierung hat den Vorteil, dass es einfach möglich ist, Restriktionen auf einzelnen Elementen zu formulieren.
Möchte man beispielsweise alle Tupel kodieren, deren zweites Element eine bestimmte Bedingung erfüllt, so reicht es, die Bedingung nur auf dem Element selbst zu übersetzen und alle Kennzeichnungen der Knoten in dem resultierenden Diagramm um die Bitbreite des ersten Elements zu erhöhen.
\chapter{Lösungsansatz}
\label{sec:solution}
Mit Hilfe von Kontrakten lässt sich das Verhalten von jeder Komponente in einem GALS-System beschreiben.
Die Komponenten sind in der synchronen Modellierungssprache SCADE (siehe Abschnitt \ref{sec:scade}) modelliert.
Eine Verifikation eines solchen GALS-Systems muss nun zwei Aufgaben erfüllen:
\begin{enumerate}
\item Die Korrektheit der angegebenen Kontrakte muss überprüft werden.
  Die Kontrakte sind korrekt, wenn das unter liegende SCADE Modell die Kontrakt-Formel erfüllt.
  Um diese Eigenschaft nachzuweisen wird der SCADE Design Verifier verwendet.
\item Es muss nachgewiesen werden, dass das Zusammenspiel der Kontrakte das globale Verifikationsziel erfüllt.
  Diese Eigenschaft wird überprüft, indem das System aus Komponent-Kontrakten in den Promela Formalismus übersetzt wird und die Gültigkeit der Formel mit SPIN nachgewiesen wird.
\end{enumerate}
\begin{figure}[h]
  \centering
  \includegraphics[scale=.33]{nomenclature}
  \caption{GTL Nomenklatur}
  \label{fig:nomenclature}
\end{figure}

\chapter{GTL -- GALS Translation Language}
\label{sec:gtl}
Um ein zu verifizierendes GALS-Modell vollständig zu definieren müssen die folgenden Eigenschaften spezifiziert werden:
\begin{itemize}
\item Die synchronen Komponenten, aus denen das Gesamtsystem zusammen gesetzt wird.
  Jede synchrone Komponente ist eine Instanz eines in einer synchronen Sprache spezifizierten Modells.
\item Die Verbindungen zwischen den Komponenten.
  Eine Verbindung verknüpft eine Ausgabevariable einer Komponente mit einer Eingabevariable einer anderen.
\item Die zu verifizierende Eigenschaft in Form einer LTL Formel über die Variablen der einzelnen Komponenten.
\end{itemize}

Die GTL-Sprache verwendet externe Formalismen, um die synchronen Komponenten zu beschreiben.
Auf die Implementierungen der synchronen Komponenten wird in der Beschreibung nur verwiesen.
Hierfür wird das Schlüsselwort \emph{model} verwendet.
Das folgende Codefragment deklariert eine Komponente mit dem Namen \emph{EntrySensor}, die im \emph{SCADE}-Formalismus definiert wurde:
\begin{lstlisting}[language=gtl]
  model[scade] EntrySensor("LightBarrier");
\end{lstlisting}
Die Ausdrücke in Klammern ("`LightBarrier"') machen formalismus-spezifische Angaben, in diesem Fall geben sie beispielsweise die Klasse des gewünschten Modells an.
Zu beachten ist, dass in der Komponentendeklaration keine Ein- oder Ausgabekanäle definiert werden.
Die GTL-Sprache setzt voraus, dass diese implizit in dem Modell-Formalismus vorhanden sind.

Um nun Zusicherungen zu formulieren, die die Komponente einhalten wird, kann man die Deklaration um so genannte Kontrakte erweitern:
\begin{lstlisting}[language=gtl]
  model[scade] EntrySensor("LightBarrier") {
    obscured and (next obscured) and (not offline) => alert;
  }
\end{lstlisting}
Dieser Kontrakt besagt beispielsweise, dass wenn der Sensor der Lichtschranke zwei Zeitschritte verdeckt ist und der Sensor nicht ausgeschaltet ist, auf jeden Fall Alarm ausgelöst wird.
Kontrakte dürfen nur Aussagen über Variablen der lokalen Komponente machen.

Die Verbindungen zwischen Komponenten werden durch \emph{connect}-Deklarationen angegeben.
Eine Verbindung gibt dabei eine Variable einer Komponente an, von der sie ausgeht und eine Variable eines anderen Modells, zu der sie geht.
\begin{lstlisting}[language=gtl]
  connect EntrySensor.alert TrapDoor.open;
\end{lstlisting}
Diese Verbindung gibt beispielsweise an, dass das Ausgabesignal \emph{alert} der Komponente \emph{EntrySensor} mit dem Eingabesignal \emph{open} der Komponente \emph{TrapDoor} verbunden sein soll.
Es können nur Ausgabesignale mit Eingabesignalen verknüpft werden.

Um nun Aussagen über das Gesamtsystem treffen zu können, verwendet man das \emph{verify}-Schlüsselwort.
Mit diesem lassen sich LTL-Formeln angeben, die Gültigkeit im System besitzen sollen.
\begin{lstlisting}[language=gtl]
  verify {
    always (System.offline => not TrapDoor.open);
  }
\end{lstlisting}
Im Gegensatz zu Kontrakten können diese Formeln Variablen aus mehreren Modellen enthalten.
Die gesamte Grammatik der GTL-Sprache ist in Abbildung \ref{fig:grammar} angegeben.

\begin{figure}
  \centering
  \begin{grammar}
    <declaration> ::= `model' `[' <id> `]' <id> `(' (<argument> (`,' <argument>)*)? `)' <model\_contract>
    \alt `connect' <id> `.' <id> <id> `.' <id> `;'
    \alt `verify' `{' (<formula> `;')* `}'
    
    <model\_contract> ::= `{' (<formula> `;')* `}'
    \alt `;'
    
    <formula> ::= <var>
    \alt <expr> `<' <expr>
    \alt <expr> `>' <expr>
    \alt <expr> `<=' <expr>
    \alt <expr> `>=' <expr>
    \alt <expr> `=' <expr>
    \alt <id> `in' `{' (<lit> (`,' <lit>)*)? `}'
    \alt `not' <formula>
    \alt <formula> `and' <formula>
    \alt <formula> `or' <formula>
    \alt <formula> `follows' <formula>
    \alt `always' <formula>
    \alt `next' <formula>
    \alt `finally' <int> <formula>
    \alt `exists' <id> `=' <lit> `:' <formula>
    \alt `(' <formula> `)'
    
    <lit> ::= <int>
    \alt <var>
    
    <var> ::= <id>
    \alt <id> `.' <id>

    <id> ::= (`a'-`z' `A'-`Z' `0'-`9')+
    
    <int> ::= (`0'-`9')+

    <expr> ::= <lit>
    \alt <expr> `+' <expr>
    \alt <expr> `-' <expr>
    \alt <expr> `*' <expr>
    \alt <expr> `/' <expr>
    \alt `(' <expr> `)'
  \end{grammar}
  \caption{GTL Grammatik}
  \label{fig:grammar}
\end{figure}

\section{Formeln}
Die Ausdrücke, die zur Spezifikation von Kontrakten sowie zur Formulierung von einzuhaltenden Bedingungen verwendet werden, stellen eine Untermenge der so genannten LTL-Formeln (LTL steht für "`linear temporal logic"') dar.
Da der SCADE Design Verifier nicht in der Lage ist, Liveness-Eigenschaften zu verifizieren, können in den Formeln allerdings keine \emph{until}-Konstrukte verwendet werden.

\section{Operationelle Semantik}
\subsection{Variablen}
In den LTL-Formeln der Semantik können frühere Werte von Variablen referenziert werden.
Hierfür werden die Variablen mit einer natürlichen Zahl versehen, die die Anzahl von Schritten angibt, die in die Vergangenheit geschaut werden soll.

Variablen kommen in der operationellen Semantik in zwei Formen vor:
In der zu verifizierenden Formel kommen sie qualifiziert mit dem Modellnamen vor, die Variablen sind also Element der Menge $\mathit{Id}\times\mathit{Id}\times\mathbb{N}$.
In Modell-Kontrakten ist der Modellname klar, daher sind die Variablen hier unqualifiziert und damit Element der Menge $\mathit{Id}\times\mathbb{N}$.
\subsection{Aufbau}
\label{sec:sos_defs}
Die Semantik eines GTL-Modells wird durch ein Tripel angegeben, bestehend aus der Menge der Modelle, der Menge der Verbindungen zwischen den Variablen, sowie der LTL-Formel, deren Gültigkeit verifiziert werden soll.
\[ \mathit{GTL} = \mathcal{P}(\mathcal{M})\times\mathcal{P}(\mathcal{C})\times\mathit{LTL}(\mathit{Id}\times\mathit{Id}\times\mathbb{N}) \]
Die LTL-Formel ist über Paare von Namen definiert, wobei der erste den Namen des Modells angibt und der zweite den der Variable im Modell.
Eine Modellkonfiguration $\mathcal{M}$ ist gegeben durch einen Namen, den synchronen Automaten, den Kontrakt und Initialisierungswerten für die Variablen.
\[ \mathcal{M} = \mathit{Id}\times\mathcal{A}\times\mathit{LTL}(\mathit{Id})\times\mathcal{P}(\mathcal{D}) \]
Die Initialisierungswerte werden über eine Abbildung von Variablennamen auf Werte $\mathit{Val}$ dargestellt:
\[ \mathcal{D} = \mathit{Id}\rightarrow\mathit{Val} \]
Die Verbindungen zwischen Modellen werden als Viertupel dargestellt, die das Modell und die Variable angibt, von dem die Verbindung ausgeht und Modell und Variable, in dem die Verbindung endet.
\[ \mathcal{C} = \mathit{Id}\times\mathit{Id}\times\mathit{Id}\times\mathit{Id} \]
\subsection{Ableitungsregeln}
Um nun die Ableitungsregeln zu formulieren zu können, die angeben, wie ein gegebenes Textmodell in ein GTL-Modell übersetzt wird, müssen zuerst die Ableitungsarten eingeführt werden.
Zunächst gibt es die globale Ableitungsrelation $\vdash$.
Die Aussage $\gamma,T\vdash m$ sagt also aus:
Gegeben ein Textmodell $\gamma$ und eine Typenbindung $T : \mathit{Id}\times\mathit{Id}\rightarrow\mathit{Type}\times\{\mathit{Inp},\mathit{Outp}\}$ lässt sich das GTL-Modell $m : \mathit{GTL}$ ableiten.

Die Ableitung $\vdash_C$ leitet den Kontrakt von Modellen sowie die Initialisierungswerte ihrer Variablen her.
Die Aussage $\gamma,T\vdash_C (c,d)$ gibt also an: Gegeben eine textuelle Repräsentation $\gamma$ und eine Typbindung $T$ lässt sich der Kontrakt(LTL-Formel) $c$ und die Initialisierung $d : \mathit{Id}\rightarrow \mathit{Val}$ ableiten.

Formeln werden mithilfe von $\vdash_V$ abgeleitet.
Diese Relation gibt an, ob sich ein Ausdruck zu einer Formel eines bestimmten Typs ableiten lässt.
Die Aussage $e,T\vdash_V (f,t)$ sagt also aus, dass gegeben die Typisierung $T$ und den Ausdruck $e$ sich die LTL-Formel $f$ vom Typ $t$ herleiten lässt.

Um die korrekte Typisierung von Modellen sicher stellen zu können, benötigt man noch eine weitere Ableitungsart:
$\vdash_T$ gibt an, ob ein synchrones Modell, dass durch die übergebenen Parameter spezifiziert wird, eine gegebene Typisierung erfüllt und einem gegebenen Automaten entspricht.
Die Aussage
\[ \textrm{scade},(\textrm{"`model.scade"'})\vdash_T T,a \]
bedeutet also: Das Scade-Modell "`model.scade"' erfüllt die Typenbindung $T$ und entspricht dem synchronen Automaten $a$.
Da diese Ableitung abhängig vom synchronen Formalismus ist, wird sie hier nicht explizit angegeben.

Zunächst wird definiert, wie sich ein leeres Textmodell herleiten lässt; die Modelle und Verbindungen sind leer, die zu verifizierende Formel ist $\top$, also immer wahr.
\[
\inference[empty]{}{\epsilon,T\vdash (\emptyset,\emptyset,\top)}
\]
Damit eine Modelldeklaration gültig ist, muss sich der Inhalt des Kontraktes $c$ mit $\vdash_C$ ableiten lassen und das referenzierte synchrone Modell wohlgetypt im Verhältnis zum restlichen Modell sein:
\[
\inference[model]{\alpha,T \vdash(ms,cs,vs) & c,\{ (v,t,d)\ |\ (n,v,t,d)\in T \}\vdash_C f,d & \beta,\mathit{args}\vdash_T T,a}{\textbf{model[}\beta\textbf{]}\ n\textbf{(}\mathit{args}\textbf{)}\ \textbf{\{} c\textbf{\}}\ \alpha,T\vdash(ms\cup\{(n,a,f,d)\},cs,vs)}
\]
Eine Deklaration einer Verbindung ist gültig, wenn beide Variablen den gleichen Typ besitzen.
Außerdem muss die erste Variable eine Ausgabe-, die andere eine Eingabe-Variable sein.
\[
\inference[connect]{\alpha,T\vdash(ms,cs,vs) & T(cm_f,cv_f)=(t,\mathit{Outp}) & T(cm_t,cv_t) = (t,\mathit{Inp})}{\textbf{connect}\ cm_f\textbf{.}cv_f\ cm_t\textbf{.}cv_t\textbf{;}\ \alpha,T\vdash(ms,cs\cup\{(cm_f,cv_f,cm_t,cv_t)\},vs)}
\]
Eine Verifikationsblock muss sich mit $\vdash_C$ zu einer LTL-Formel ableiten lassen.
Gibt es mehr als einen Block, so werden die Formeln per Konjunktion zusammengefasst.
\[
\inference[verify]{\alpha,T\vdash(ms,cs,vs) & c,T\vdash_C (f,\emptyset)}{\textbf{verify}\ \textbf{\{}c\textbf{\}}\ \alpha,T\vdash (ms,cs,vs\land f))}
\]
Ein leerer Kontrakt erlaubt dem Prozess jedes Verhalten und wird daher durch die LTL-Formel $\top$ repräsentiert.
\[
\inference[$\epsilon$-contract]{}{\epsilon,T\vdash_C (\top,\emptyset)}
\]
Eine gültige Initialisierungsdeklaration liegt dann vor, wenn der Typ der initialisierten Variable dem des Wertes entspricht.
\[
\inference[init]{\alpha,T\vdash_C (c,i) & d\in T(c)}{\textbf{init}\ v\ d\textbf{;}\ \alpha,T\vdash_C (c,i\cup\{(v,d)\})}
\]
Eine Kontraktformel ist gültig, wenn sie sich erfolgreich zu einer LTL-Formel des Typs "`bool"' ableiten lässt.
Zu beachten ist, dass das Schlüsselwort \emph{contract} auch weggelassen werden kann.
\[
\inference[formula]{\alpha,T\vdash_C (c,i) & e,T,\emptyset\vdash_V (f,\textrm{bool})}{\textbf{contract}\ e\textbf{;}\ \alpha,T\vdash_C (c\land f,i)}
\]
Eine Variable kann zwei verschiedene Bedeutungen haben:
Entweder ist sie eine durch einen $\exists$-Quantor gebundene Variable (1) oder eine normale qualifizierte oder unqualifizierte Variable.
In beiden Fällen ist der Typ des Ausdrucks der Typ der Variable.
\[
\begin{array}{ll}
  \inference[var(1)]{(v,q,n)\in\gamma & T(q)=t}{v,T,\gamma\vdash_V (q^n,t)} &
  \inference[var(2)]{\lnot\exists q,n: (v,q,n)\in\gamma & T(v)=t}{v,T,\gamma\vdash_V (v,t)}
\end{array}
\]
Die Regeln für die logischen Konnektoren \emph{not}, \emph{true}, \emph{false}, \emph{and}, \emph{or} und \emph{implies} sind genau wie in der normalen Logik definiert:
\[
\begin{array}{ll}
  \inference[not]{e,T,\gamma\vdash_V (e',\textrm{bool})}{\textbf{not}\ e,T,\gamma\vdash_V (\lnot e',\textrm{bool})} &
  \inference[and]{e_1,T,\gamma\vdash_V (e_1',\textrm{bool}) & e_2,T,\gamma\vdash_V (e_2',\textrm{bool})}{e_1\ \textbf{and}\ e_2,T,\gamma\vdash_V (e_1'\land e_2',\textrm{bool})}\\[20pt]
  \inference[true]{}{\textbf{true},T,\gamma\vdash_V(\top,\textrm{bool})} &
  \inference[or]{e_1,T,\gamma\vdash_V (e_1',\textrm{bool}) & e_2,T,\gamma\vdash_V (e_2',\textrm{bool})}{e_1\ \textbf{or}\ e_2,T,\gamma\vdash_V (e_1'\lor e_2',\textrm{bool})}\\[20pt]
  \inference[false]{}{\textbf{false},T,\gamma\vdash_V(\top,\textrm{bool})} &
  \inference[implies]{e_1,T,\gamma\vdash_V (e_1',\textrm{bool}) & e_2,T,\gamma\vdash_V (e_2',\textrm{bool})}{e_1\ \textbf{implies}\ e_2,T,\gamma\vdash_V (\lnot e_1'\lor e_2',\textrm{bool})}
\end{array}
\]
Der $\exists$-Quantor bindet eine neue Variable an die aktuelle Version einer existierenden.
Das bedeutet, dass auf frühere Werte einer Variable zurück gegriffen werden kann, wenn die Variable in einem \emph{next}-Kontext verwendet wird.
\[
\inference[exists]{f,T,\gamma\cup\{(u,e,0)\}\vdash_V f'}{\textbf{exists}\ u\textbf{=}e\textbf{:}\ f,T,\gamma\vdash_V f'}
\]
Der \emph{next}-Operator wird in sein LTL-Äquivalent übersetzt.
Allerdings müssen die gebundenen Variablen auf den neuen Kontext angepasst werden, indem sie auf einen Wert früher in der Geschichte referenziert werden.
\[
\inference[next]{f,T,\{ (u,e,n+1)\ |\ (u,e,n)\in\gamma\}\vdash_V (f',\textrm{bool})}{\textbf{next}\ f,T,\gamma\vdash_V (\bigcirc f',\textrm{bool})}
\]
Tritt ein \emph{always}-Operator auf, so werden für die Ableitung der Unterformel alle Bindungen entfernt, da gebundene Variablen nicht innerhalb eines \emph{always}-Operators auftauchen dürfen.
\[
\inference[always]{f,T,\emptyset\vdash_V (f',\textrm{bool})}{\textbf{always}\ f,T,\gamma\vdash_V (\lnot(\top U (\lnot f')),\textrm{bool})}
\]
Der \emph{finally}-Operator ist nur eingeführt, um die Aussage "`Innerhalb der nächsten $i$ Schritte passiert $f$"' einfacher zu formulieren.
Er kann rekursiv mithilfe des \emph{or}- und \emph{next}-Operators definiert werden.
\[
\inference[finally-0]{f,T,\gamma\vdash_V f'}{\textbf{finally}0\ f,T,\gamma\vdash_V f'}
\]
\[
\inference[finally-i]{f\ \textbf{or}\ (\textbf{next}\ \textbf{finally}(i-1)\ f),T,\gamma\vdash_V f'}{\textbf{finally}i\ f,T,\gamma\vdash_V f'}
\]
Gleichheitstests können auf allen Variablen gleichen Typs durchgeführt werden.
\[
\inference[equal]{f,T,\gamma\vdash_V (f',t) & g,T,\gamma\vdash_V (g',t)}{f=g,T,\gamma\vdash_V(f'=g',\textrm{bool})}
\]
\[
\inference[nequal]{f,T,\gamma\vdash_V (f',t) & g,T,\gamma\vdash_V (g',t)}{f\textbf{!=}g,T,\gamma\vdash_V(\lnot(f'=g'),\textrm{bool})}
\]
Während Datentyp-spezifische Relationen wie $<$ oder $>$ nur auf speziellen Typen ausgeführt werden können (in diesem Fall "`int"').
\[
\inference[lesser]{f,T,\gamma\vdash_V (f',\textrm{int}) & g,T,\gamma\vdash_V (g',\textrm{int})}{f<g,T,\gamma\vdash_V(f'<g',\textrm{bool})}
\]
\[
\inference[greater]{f,T,\gamma\vdash_V (f',\textrm{int}) & g,T,\gamma\vdash_V (g',\textrm{int})}{f>g,T,\gamma\vdash_V(f'>g',\textrm{bool})}
\]
Statt Variablen können für Integer auch Konstanten verwendet werden. 
\[
\inference[const]{n\in\mathbb{N}}{n,T,\gamma\vdash_V (n,\textrm{int})}
\]
Einfache arithmetische Operationen werden ebenfalls unterstützt.
\[
\inference[plus]{f,T,\gamma\vdash_V (f',\textrm{int}) & g,T,\gamma\vdash_V (g',\textrm{int})}{f\textbf{+}g,T,\gamma\vdash_V (f'+g',\textrm{int})}
\]
\[
\inference[minus]{f,T,\gamma\vdash_V (f',\textrm{int}) & g,T,\gamma\vdash_V (g',\textrm{int})}{f\textbf{-}g,T,\gamma\vdash_V (f'-g',\textrm{int})}
\]
\[
\inference[times]{f,T,\gamma\vdash_V (f',\textrm{int}) & g,T,\gamma\vdash_V (g',\textrm{int})}{f\textbf{*}g,T,\gamma\vdash_V (f'\cdot g',\textrm{int})}
\]
\[
\inference[div]{f,T,\gamma\vdash_V (f',\textrm{int}) & g,T,\gamma\vdash_V (g',\textrm{int})}{f\textbf{/}g,T,\gamma\vdash_V (\frac{f'}{g'},\textrm{int})}
\]
Um die Aussage "`Variable $v$ hat den Wert $s_1$ oder $s_2$ oder\dots"' abzukürzen, kann man eine Menge von Werten angeben, die die Variable annehmen darf.
\[
\inference[elem]{\forall i\in\{0,\dots,n\}: s_i,T,\gamma\vdash_V (s_i',t) & T(v)=t}{v\ \textbf{in}\ \textbf{\{}s_0,\dots,s_n\textbf{\}},T,\gamma\vdash_V (\bigvee_{i\in\{0,\dots,n\}} v=s_i',\textrm{bool}) }
\]
\subsection{Interpretation als GALS-System}
%Um das in diesem Abschnitt definierte GTL-System als ein GALS System wie in Abschnitt \ref{sec:gals_formal_definition} zu betrachten, müssen die im System enthaltenen Formeln mithilfe des in Abschnitt \ref{sec:ltl-translation} angegebenen Algorithmus in Automaten übersetzt werden.
Gegeben ein GTL-Modell $(m,c,v)$ kann man ein GALS-System $(A,p,C)$ konstruieren, das die synchronen Komponenten enthält.
Die Automatenmenge enthält die Namen aller Komponenten des GTL-Modells.
\[ A = \{ \mathit{name}\ |\ (\mathit{name},\_,\_,\_)\in m \} \]
Die Abbildung der Automatennamen auf Automaten ordnet den synchronen Automaten zu.
\[ p(n) = \left\{\begin{array}{lr}
    a & (n,a,\_,\_)\in m\\
    \bot & \textrm{sonst}
  \end{array}\right. \]
Für die Verbindungen benötigt man eine Hilfsfunktion $r : \mathit{Id}\times\mathit{Id}\rightarrow \mathbb{N}$, die einer Komponentenvariable die entsprechende Position im Eingabe-/Ausgabe-Tupel zuordnet.
\[ C = \{(a_f,r(a_f,v_f),a_t,r(a_t,v_t)\ |\ (a_f,v_f,a_t,v_t)\in c\} \]
Um das GALS-System zu erhalten, das das Zusammenspiel der Kontrakte repräsentiert muss man statt der synchronen Automaten die mithilfe des in Abschnitt \ref{sec:ltl-translation} angebenen Algorithmus übersetzten LTL-Kontraktformeln verwenden.

\begin{tikzpicture}
  \node[draw] (promela) at (1.5,1) {Promela};
  \node[draw] (gtl) at (0,2) {GTL};
  \node[draw] (scade) at (3,2) {Scade};
  \node[draw] (verifier) at (1.5,0) {Verifier};
  \draw[->,blue] (scade) |- (promela);
  \draw[->,blue] (gtl) |- (promela);
  \draw[->,blue] (promela) -- (verifier);
\end{tikzpicture}

\begin{tikzpicture}
  \node[draw] (gtl) at (0,2) {GTL};
  \node[draw] (scade) at (3,2) {Scade};
  \node[draw] (promela) at (0,1) {Promela};
  \node[draw] (testnode) at (3,1) {Scade Testnode};
  \node[draw] (verifier) at (0,0) {Verifier};
  \draw[->,blue] (gtl) -- (promela);
  \draw[->,blue] (scade) -- (testnode);
  \draw[->,blue] (gtl) -- (testnode);
  \draw[->,blue] (promela) -- (verifier);
\end{tikzpicture}


\chapter{Übersetzung}
\label{sec:translation}
Dieser Abschnitt beschäftigt sich mit der Übersetzung der GTL in verschiedene Zielformalismen.
Im Rahmen dieser Arbeit wurden drei verschiedene Übersetzungsmethoden entwickelt und zwei Optimierungsstrategien implementiert:
\begin{itemize}
\item Die erste Übersetzungsmethode verwendet die von SCADE generierten C-Modelle um ein Gesamtsystem in Promela zusammen zu setzen.
\item Die Kontrakte der Komponenten können mithilfe des SCADE-Design-Verifiers überprüft werden.
\item Das Kontraktsystem kann nach Promela übersetzt werden.
  \begin{itemize}
  \item Statische BDD können verwendet werden, um den Zustandsraum bei der Verifikation zu verkleinern.
  \item Dynamische BDD dienen dem selben Zweck, haben aber weniger Einschränkungen.
  \end{itemize}
\end{itemize}
Zunächst wird eine allgemeine Konstruktion angegeben, mit der ein GALS-System nach Promela übersetzt werden kann.
Diese wird dann verwendet, um die anderen Übersetzungsmethoden zu erklären.
Dazu verwendet die allgemeine Konstruktion drei Übersetzungsfunktionen $\llbracket\rrbracket_C$, $\llbracket\rrbracket_A$ und $\llbracket\rrbracket_D$, die von den konkreten Übersetzungen bereit gestellt werden müssen.
Da die SCADE-Übersetzung keinen Promela-Code generiert, verwendet sie als einzige Übersetzung auch nicht die allgemeine Übersetzungsmethode.
\section{Promela-C-Integration}
\begin{figure}
  \centering
  \begin{tikzpicture}
    \node[draw] (gtl) at (0,2) {GTL};
    \node[draw] (scade) at (3,2) {Scade};
    \node[draw] (promela) at (0,1) {Promela};
    \node[draw] (c) at (3,1) {C};
    \node[draw] (verifier) at (1.5,0) {Verifier};
    \draw[->,blue] (gtl) -- (promela);
    \draw[->,blue] (scade) -- (c);
    \draw[->,blue] (promela) |- (verifier);
    \draw[->,blue] (c) |- (verifier);
  \end{tikzpicture}
  \caption{Simulation durch C-Integration von Promela}
\end{figure}

Diese Übersetzungsmethode führt keine Optimierungen oder Abstraktionen durch, sondern simuliert das Modell exakt so, wie durch die Spezifikation angegeben.
Die Kontrakte für die synchronen Komponenten werden also ignoriert.

Zunächst wird jede synchrone Komponente mit Hilfe des SCADE-Compilers nach C übersetzt.
Da der Übersetzungsprozess für jede Komponente einzeln durchgeführt werden muss, muss darauf geachtet werden, dass die Modelle keine Namenskonflikte aufweisen oder die gleichen Sub-Komponenten enthalten\cite{scade_c_integration}.

Dabei generiert der Compiler für jede Komponente zwei Datenstrukturen, die eine enthält die Eingabevariablen, die andere die Ausgabevariablen sowie den internen Zustand der Komponente.
Außerdem werden zwei Funktionen erstellt:
Die erste initialisiert den internen Zustand der Komponente, die zweite führt einen einzelnen Berechnungsschritt der Komponente durch.

Jede synchrone Komponente wird nun durch einen Promela-Prozess repräsentiert, der zunächst die Datenstrukturen initialisiert und dann in jedem Schritt die Eingabevariablen in die C-Datenstruktur kopiert, die Schrittfunktion aufruft und die Ergebnisse in die Ausgabevariablen schreibt.

\section{SCADE Übersetzung}
Diese Übersetzungsmethode wird benutzt, um die Korrektheit der Kontrakte für synchrone Komponenten zu verifizieren.
Dazu wird für jede Komponente ein SCADE-Knoten erstellt, der dann im Design-Verifier der SCADE-Suite auf Korrektheit getestet werden kann.
Der generierte Testknoten überwacht alle Ein- und Ausgänge der Komponente und gibt eine boolesche Variable zurück, die wahr ist, wenn die geforderte Eigenschaft erfüllt ist und falsch ist, wenn sie verletzt ist.
Der Design-Verifier kann nun benutzt werden, um sicher zu stellen, dass die ausgegebene Variable immer wahr ist.

Der Büchi-Automat, der die zu verifizierende Eigenschaft der Komponente darstellt, wird hierfür in eine SCADE-Zustandsmaschine übersetzt.
Jeder Zustand der Maschine definiert einen Wert für die Resultat-Variable.
Die Transitionen von Zustand $A$ in Zustand $B$ erhalten genau die Bedingungen, die in Zustand $B$ gelten müssen.
Zusätzlich zu den vom Büchi-Automaten vorgesehenen Transitionen erhält jeder Zustand noch eine Transition in den Fehlerzustand:
In diesem hat die Resultat-Variable den Wert falsch.
Die Transition ist von niedrigster Priorität, wird also nur dann geschaltet, wenn keine andere Transition schalten kann.
\section{Native Promela Übersetzung}
Diese Übersetzung verwendet die Kontrakte, um die Komponenten zu repräsentieren, aber führt keinerlei Optimierungen durch.

Jedes Atom in einem Zustand, dass eine Ausgabevariable $o$ determiniert, wird in ein \emph{Einschränkungs-Tupel} $(l_u,l_l,v_a,v_f,e_{eq},e_{neq})$ übersetzt, wobei die Variablen folgende Bedeutung haben:
\begin{itemize}
\item $l_u$ enthält alle Ausdrücke, die eine obere Schranke für die Variable angeben.
  Es gilt:
  \[ \forall l\in l_u: o < l \]
\item Analog enthält $l_l$ alle Ausdrücke, die eine untere Schranke darstellen.
\item $v_a$ enthält eine Menge von Werten, die von der Variable angenommen werden dürfen.
  Es gilt:
  \[ o\in v_a \]
\item Werte, die nicht angenommen werden dürfen, werden in $v_f$ gesammelt.
  \[ o\not\in v_f \]
\item $e_{eq}$ gibt eine Menge von Ausdrücken an, die gleich der Variable seien müssen:
  \[ \forall e\in e_{eq}: o=e \]
\item Genauso gibt $e_{neq}$ eine Menge von Ausdrücken an, die ungleich zu der Variable sein müssen:
  \[ \forall e\in e_{neq}: o\neq e \]
\end{itemize}
Wird eine Ausgabevariable von mehreren Atomen determiniert, so müssen die resultierenden Tupel zusammen geführt werden.
Gegeben zwei Tupel
\begin{align*}
  T_1 &= (l_u,l_l,v_a,v_f,e_{eq},e_{neq})\\
  T_2 &= (l_u',l_l',v_a',v_f',e_{eq}',e_{neq}')
\end{align*}
ergibt sich das Tupel, dass beide Einschränkungen erfasst als
\[ T_1\oplus T_2 = (l_u\cup l_u',l_l\cup l_l',v_a\cap v_a',v_f\cup v_f',e_{eq}\cup e_{eq}',e_{neq}\cup e_{neq}') \]
Für eine Ausgabevariable $o$ ergibt sich die Funktion $t$, die das entsprechende Einschränkungstupel berechnet durch
\begin{align*}
  t(o < e) &= (\{e\},\emptyset,\emptyset,\emptyset,\emptyset,\emptyset)\\
  t(o\leq e) &= (\{e+1\},\emptyset,\emptyset,\emptyset,\emptyset,\emptyset)\\
  t(o > e) &= (\emptyset,\{e\},\emptyset,\emptyset,\emptyset,\emptyset)\\
  t(o \geq e) &= (\emptyset,\{e-1\},\emptyset,\emptyset,\emptyset,\emptyset)\\
  t(o = e) &= (\emptyset,\emptyset,\emptyset,\emptyset,\{e\},\emptyset)\\
  t(o\neq e) &= (\emptyset,\emptyset,\emptyset,\emptyset,\emptyset,\{e\})\\
  t(o\in i) &= (\emptyset,\emptyset,i,\emptyset,\emptyset,\emptyset)\\
  t(o\not\in i) &= (\emptyset,\emptyset,\emptyset,i,\emptyset,\emptyset)\\
\end{align*}
Enthält ein Atom keine Ausgabevariable, so wird es in einen äquivalenten Promela-Ausdruck übersetzt.
Enthalten mehrere Atome keine Ausgabevariable, so werden die resultierenden Ausdrücke per Konjunktion verknüpft.

Eine Menge von Atomen $a$ wird also in ein Tupel $(p,e)$ aus einer Abbildung von Ausgabe-Variablen auf Einschränkungstupel $p$ und einem Promela-Ausdruck $e$ übersetzt.
Die Übersetzung $\llbracket \rrbracket_C$ liefert also nur den Promela-Ausdruck $e$, der die Bedingungen auf den Eingabe-Variablen repräsentiert:
\[ \llbracket a\rrbracket_C = e \]
Beispielsweise wird die Bedingung $x\leq 10\land x\neq y$ in den folgenden Promela-Code übersetzt:
\begin{lstlisting}[language=promela]
  x <= 10 && x!=y
\end{lstlisting}

Die Übersetzung der Einschränkungstupel gemäß der Semantik $\llbracket\rrbracket_A$ ist etwas komplizierter, da nicht-deterministisch alle möglichen Belegungen für die Ausgabevariable generiert werden müssen.
Dazu wird zunächst die Ausgabevariable $o$ auf die größte untere Schranke gesetzt, indem die Schranken miteinander verglichen werden.
Existiert keine untere Schranke, so wird der kleinste Wert des Wertebereichs der Variable gewählt.
Ansonsten wird mit Promela \emph{If}-Anweisungen ein Entscheidungsbaum aufgebaut.
Existieren also beispielsweise die drei unteren Schranken $l_l = \{ e1,e2,e3\}$, so sieht die Übersetzung wie folgt aus:
\begin{lstlisting}[language=promela,numbers=left,caption={Berechnung der unteren Schranke}]
if :: e1 < e2;
      if :: e1 < e3;
            o = e1
         :: else;
            o = e3
      fi
   :: else;
      if :: e2 < e3;
            o = e2
         :: else;
            o = e3
      fi
fi
\end{lstlisting}
Danach wird in einer \emph{Do}-Schleife die Ausgabevariable so lange hoch gezählt, bis eine der oberen Schranken erreicht wird.
Für jeden Wert muss nun noch geprüft werden, ob er die restlichen Bedingungen des Einschränkungstupels erfüllt.
Gilt also beispielsweise $l_u = \{ u1,u2 \}$ sowie $v_f=\{ f1, f2 \}$, so ergibt sich der folgende Code:
\begin{lstlisting}[language=promela,numbers=left,firstnumber=last,caption={Generierung von möglichen Werten}]
do :: o<u1 && o<u2;
      o = o+1
   :: o==f1 || o==f2;
      skip
   :: else;
      break
od
\end{lstlisting}
Da sich der gesamte Code-Block zur Ausgabe-Erzeugung innerhalb eines \emph{atomic}-Blocks befindet und der Ablauf an keiner Stelle blockieren kann, läuft der gesamte Code in einem Schritt ab.

Zusätzlich zu den Semantiken $\llbracket\rrbracket_C$ und $\llbracket\rrbracket_A$ muss noch die Bijektion $i$ und $\llbracket\rrbracket_D$ definiert werden (vgl. Seite \pageref{sec:translation-correctness}).
In dieser Übersetzung ist diese sehr einfach, da die Werte der Verbindungen direkt in Variablen gespeichert werden.
Die Bijektion konstruiert also nur eine Zuordnung $\sigma_e$, die die Werte für jede Verbindung aus dem Zustands- und Eingabevektor extrahiert.
Es gilt also:
\[ i((v_0,\dots,v_n),(v_{n+1},\dots,v_m)) = \sigma_e \]
wobei
\[ \sigma_e(j) \overset{def}{=} v_j \]
Da nur genau so viele Variablen deklariert werden, wie auch Verbindungen existieren ist es leicht einzusehen, dass $i$ in der Tat bijektiv ist.

Nun ist noch zu zeigen, dass $i$ die verlangten Eigenschaften 2.-4. von Seite \pageref{sec:bijection_conditions} aufweist.
Für ein $i(s,\beta)=\sigma_e$ muss nachgewiesen werden, dass die Äquivalenz aus (\ref{eq:assert1})
\[ \mathit{exec}(\llbracket \mu(q')\rrbracket_C,\sigma_e)\Leftrightarrow (\forall q\in Q^a: \delta^a(q,(s|^a,\beta|^a)) = (q',o)) \]
für alle Zustände $q'\in Q^a$ erfüllt ist.
Da $\llbracket\mu(q')\rrbracket_C$ eine Promela-Anweisung generiert, die nur ausgeführt werden kann, wenn die Belegungen in $\sigma_e$ den Bedingungen aus $\mu(q')$ entsprechen, existiert auch nur in diesem Fall ein Übergang zu dem Zustand $q'$.
Genauso kann die Anweisung $\llbracket\mu(q')\rrbracket_C$ nur dann ausführbar sein, wenn es einen entsprechenden Übergang gibt, da die Bedingungen aus dem Zustand $q'$ hierfür alle erfüllt sein müssen.

Desweiteren muss noch die Kompatibilität der Bijektion $i$ mit der Semantik $\llbracket\rrbracket_A$ gezeigt werden.
Hierzu muss die Korrektheit der Äquivalenz aus Gleichung (\ref{eq:assert2}) nachgewiesen werden.
Seien hierfür wieder $s$, $\beta$ und $\sigma_e$ mit 
\[ i(s,\beta) = \sigma_e \]
gegeben.
Existiert nun ein Übergang im Promela-Modell mit
\[ \xymatrix{ \left<\sigma_e,L\right> \ar[rr]^-{\llbracket \mu(q')\rrbracket_A} & & \left<\sigma_e',L'\right> } \]
so ergibt sich, dass hierbei genau die Variablen verändert werden, die in $\mu(q')$ vorkommen und Ausgabe-Variablen sind.
Nach der Konstruktion durch die Einschränkungstupel (Definition von $\llbracket\rrbracket_A$) ergibt sich außerdem, dass die nicht-deterministischen Zuweisungen alle innerhalb der durch das für die Übersetzung gegebene GTL-Modell definierten Wertebereiche liegen.
Die Existenz des entsprechenden Übergangs in dem Promela-Modell ist durch die Korrektheit der Übersetzung der Einschränkungstupel gesichert.

Die Default-Werte für die Variablen werden mit der Semantik $\llbracket\rrbracket_D$ definiert.
Diese weist den angegebenen Verbindungen einfach den angegebenen Wert zu.
Für die initiale Umgebung gilt dann:
\[ \llbracket \alpha\rrbracket_D \overset{def}{=} \sigma_e \]
wobei
\[ \sigma_e(j) = \alpha(j) \]
es ist leicht einzusehen, dass diese Definition die Forderung aus Gleichung \ref{eq:assert0} erfüllt.
\section{Übersetzungskonstruktion}
Gegeben ein wie in Abschnitt \ref{sec:sos_defs} spezifiziertes System $s\in \mathcal{S}$ mit $s=(ms,cs,vs)$ muss nun eine Übersetzung in ein äquivalentes Promela-Modell gefunden werden, die die Semantik des Systems erhält.
Für jede Komponente $(m,(\mathit{contr},\mathit{init}))\in ms$ mit dem Namen $m$, dem Kontrakt $\mathit{contr}$ und der Initialisierung $\mathit{init}$ wird nun der Kontrakt in einen äquivalenten Büchi-Automaten $(Q,\Sigma,\delta,\mu,q_0,\emptyset)$ übersetzt (Siehe Abschnitt \ref{sec:ltl-translation}).
Hierbei ist zu beachten, dass die generierten Automaten Bedingungen auf den Variablen als Ein- und Ausgabesymbole verwenden, da Relationen wie $x\leq y$ von dem LTL-Übersetzungsalgorithmus als atomare Aussagen betrachtet werden.

Für die Übersetzung werden die Funktionen $\llbracket\rrbracket_C$ und $\llbracket\rrbracket_A$ benötigt.
Die Funktion $\llbracket\rrbracket_C$ generiert aus den übergebenen Atomen einen Promela-Ausdruck, der abhängig vom globalen Zustand testet, ob alle Bedingungen, die durch die übergebenen Atome an die Eingabe-Variablen gestellt sind, erfüllt sind.
Diese Anweisung muss blockieren, bis die Bedingungen erfüllt sind und muss seiteneffektfrei sein, den globalen Zustand also nicht verändern.
Ähnlich dazu generiert die Funktion $\llbracket\rrbracket_A$ eine Anweisung, die die Ausgabevariablen entsprechend den übergebenen Atomen anpasst und damit den globalen Zustand verändert.
Die generierte Anweisung darf niemals blockieren.

Für jede Komponente wird nun ein äquivalenter Prozess wie folgt definiert:
\begin{lstlisting}[language=Promela,mathescape=true,numbers=left,numberstyle=\small,caption={Komponenten-Übersetzung als Promela-Prozess}]
proctype $m$() {
  if $[ \forall i\in q_0:$
  :: atomic {
       $\llbracket\mu(i)\rrbracket_C$;
       $\llbracket\mu(i)\rrbracket_A$;
       goto st_$i$
     }
  $]$ fi;
  $[ \forall q\in Q:$
  st_$q$: if $[\forall q'\in Q,q\delta q':$
  :: atomic {
       $\llbracket\mu(i)\rrbracket_C$;
       $\llbracket\mu(i)\rrbracket_A$;
       goto st_$q'$
     }
  $]$ fi;
  $]$
}
\end{lstlisting}
%Hierbei gibt $\llbracket\rrbracket_C$ einen Promela-Ausdruck an, der abhängig vom globalen Zustand testet, ob alle Bedingungen, die durch die übergebenen Atome spezifiziert sind, erfüllt sind.
%Die Anweisung muss blockieren, bis die Bedingungen erfüllt sind und darf den globalen Zustand nicht verändern.
%Ähnlich dazu generiert die Funktion $\llbracket\rrbracket_A$ eine Anweisung, die den globalen Zustand anhand der übergebenen Atome transformiert.
%Die generierte Anweisung darf nicht blockieren.

Die zu verifizierende Formel $v$ wird negiert ebenfalls in einen Büchi-Automaten $(Q,\Sigma,\delta,\mu,q_0,F)$ übersetzt und in eine äquivalente Promela \emph{never}-Deklaration übersetzt:
\begin{lstlisting}[language=Promela,mathescape=true,numbers=left,numberstyle=\small,caption={Verifikationsziel-Übersetzung als \emph{never}-Prozess}]
never {
  if $[ \forall i\in q_0:$
  :: atomic {
    $\llbracket \mu(i) \rrbracket_C$;
    goto st_$i$
  }
  $]$
  fi;
  $[ \forall q\in Q:$
  $[ q\in F:$ accept_$q$: $]$
  st_$q$:
    if $[ \forall q'\in Q,q\delta q':$
    :: atomic {
      $\llbracket \mu(q') \rrbracket_C$;
      goto st_$q'$
    }
    $]$
  $]$
}
\end{lstlisting}

Um den Initialzustand zu erreichen, wird vor dem Starten aller Prozesse noch eine Initialisierung durchgeführt.
Hierfür muss die konkrete Übersetzung die Funktion $\llbracket\rrbracket_D$ bereit stellen.
Diese generiert, gegeben eine Komponente, eine Variable und einen Initialisierungswert für die Variable, eine Anweisung, die den globalen Zustand so verändert, dass die Variable nun den entsprechenden Wert besitzt.
\begin{lstlisting}[language=Promela,mathescape=true,numbers=left,numberstyle=\small,caption={Initialisierungsprozess}]
init {
  $[ \forall (m,a,f,d)\in ms:$
    $[ \forall (v,val)\in d:$
    $\llbracket m,v,val\rrbracket_D$
    $]$
  $]$
  atomic {
  $[ \forall (m,a,f,d)\in ms:$
    run $m$();
  $]$
  }
}
\end{lstlisting}

\subsection{Korrektheit der Übersetzung}
Um zu beweisen, dass die angegebene Promela-Übersetzung korrekt ist, muss gezeigt werden, dass eine Semantik des Systems, repräsentiert durch eine Untermenge des vollständigen Transitionssystems $T'\subseteq T$ (Siehe Abschnitt \ref{sec:semantic}), mit der Semantik des übersetzten Promela-Modells übereinstimmt.
Um dies zu zeigen, wird die Promela-Semantik verwendet, wie sie in \cite{Gallardo04formalaspects} beschrieben ist.
Da in dieser Semantik gefordert ist, dass jede Anweisung ein implizites Label erhält, werden folgende Labels für die Anweisungen vergeben:
\begin{itemize}
\item Die \emph{If}-Anweisung in Zeile 2 erhält das Label \emph{Start}
\item In diesem Zweig wird der Ausdruck in Zeile 4 mit dem Label \emph{CI\_$i$} versehen
\item Die nachfolgende Zuweisung in Zeile 5 wird mit dem Label \emph{AI\_$i$} gekennzeichnet
\item Der If-Zweig in Zeile 12 kann über das Label \emph{C\_$q$\_$q'$} angesprungen werden
\item Die darauf folgende Anweisung in Zeile 13 bekommt das Label \emph{A\_$q$\_$q'$} zugewiesen
\end{itemize}

Mit diesen Kennzeichnungen ergibt sich nun die \emph{next}-Funktion der Semantik wie folgt:

\begin{tabular}{|c|c|}
  \hline
  $L$ & $\textrm{next}(L)$\\
  \hline
  CI\_$q$ & A\_$q$\\
  A\_$q$ & st\_$q$\\
  C\_$q$\_$q'$ & A\_$q$\_$q'$\\
  A\_$q$\_$q'$ & st\_$q'$\\
  \hline
\end{tabular}

Auch die erforderliche $g$-Funktion, die die Zweige einer \emph{If}-Anweisung angibt, kann somit hergeleitet werden als:

\begin{tabular}{|c|c|}
  \hline
  $L$ & $g(L)$\\
  \hline
  Start & $\{ \textrm{CI\_}i\ |\ i\in q_0 \}$\\
  st\_$q$ & $\{ \textrm{C\_}q\textrm{\_}q'\ |\ q'\in Q, q\delta q' \}$\\
  \hline
\end{tabular}

Für den Ausführungsmodus(\emph{mode}) ergibt sich:

\begin{tabular}{|c|c|}
  \hline
  $L$ & $\textrm{mode}(L)$\\
  \hline
  Start & ilv\\
  CI\_$i$ & atm\\
  AI\_$i$ & atm\\
  st\_$q$ & ilv\\
  C\_$q$\_$q'$ & atm\\
  A\_$q$\_$q'$ & atm\\
  \hline
\end{tabular}

Zunächst ist es nützlich ein paar allgemeine Aussagen aufzustellen, die die Verifikation der Richtigkeit der Übersetzung erleichtern.
\begin{enumerate}
\item Es ist leicht einzusehen, dass für alle Prozesse die Umgebung $\phi_e$ gleich ist, da die Prozesse keine lokalen Variablen deklarieren.
\end{enumerate}

Um nun zeigen zu können, dass das definierte GTL-System $(ms,cs,vs)$ bisimular zum übersetzten Promela Modell $tr(ms,cs,vs)$ ist, müssen folgende Anforderungen an die Semantik gestellt werden:
\begin{enumerate}
\item Es muss eine Bijektion $i$ zwischen Zuständen der Verbindungen sowie Eingaben $s\in S_C(\mathcal{G})\times I(\mathcal{G})$ und der Promela-Umgebung $\sigma_e$ existieren.
\item Die Definitionen $\llbracket \alpha \rrbracket_D$ müssen eine initiale Umgebung $\sigma_e^0$ definieren, die isomorph zum Initialzustand $\alpha$ ist: 
  \[ i(\alpha) = \sigma_e^0 \]
\item Befinden sich beide Systeme in isomorphen Zuständen, so wird der von $\llbracket \rrbracket_C$ erzeugte Ausdruck genau dann wahr, wenn es im abstrakten Modell einen entsprechenden Übergang zwischen den Zuständen gibt:
  \begin{equation}
    \begin{array}{rc}
      i(s,\beta) = \sigma_e \Rightarrow \forall q'\in Q^a: &
      \mathit{exec}(\llbracket \mu(q')\rrbracket_C,\sigma_e)) \\
      & \Leftrightarrow\\
      & (\forall q\in Q^a: \delta^a(q,(s|^a,\beta|^a))=(q',o))
    \end{array}
    \label{eq:assert1}
  \end{equation}
\item Die Anweisung, die von $\llbracket \rrbracket_A$ generiert wird, darf nie blockieren und muss isomorphe Zustände beibehalten:
  \begin{equation}
    \begin{array}{rc}
      i(s,\beta) = \sigma_e \Rightarrow \forall q'\in Q^a: &
      \xymatrix { \left<\sigma_e,L\right> \ar[rr]^-{\llbracket \mu(q')\rrbracket_A} & & \left<\sigma_e',L'\right> } \\
      & \Leftrightarrow\\
      & (\forall q\in Q^a: \delta^a(q,(s|^a,\beta|^a)) = (q',o)\\
      & \land\\
      & i(s[a\mapsto o],\beta[a\mapsto o]) = \sigma_e')
    \end{array}
    \label{eq:assert2}
  \end{equation}
\end{enumerate}
Nun kann man die Relation $\cong$ angeben, die Zustände des abstrakten Modells mit Zuständen des übersetzten Promela-Modells in Relation setzt.
Diese wird wie folgt definiert:
Zwei Zustände stehen genau dann in Relation, wenn ihre globalen Zustände isomorph sind und sich jeder Prozess des Promela-Modells am Label befindet, das mit dem Zustand im abstrakten Modell korrespondiert, oder sich am Label \emph{Start} befindet und der abstrakte Prozess im Zustand $\const{init}$ ist.
\[
\begin{array}{c}
  (q_0,\dots,q_N,c_0,\dots)\cong \gamma\\
  \Leftrightarrow\\
  i((c_0,\dots))=\gamma(0).\sigma_e\\
  \land\\
  \forall j\in\{1\dots N\}: (\gamma(j).\sigma_l = \textrm{st\_}q_j \lor (\gamma(j).\sigma_l = \textrm{Start}\land q_j=\const{init}))
\end{array}
\]
Nun muss gezeigt werden, dass es sich bei der eben definierten Relation tatsächlich um eine Bisimulationsrelation handelt.
Hierfür muss nachgewiesen werden, dass es für jede Transition, die ein bisimularer Zustand durchführen kann, eine Transition des anderen Zustands gibt und die Zielzustände der beiden Transitionen auch wieder bisimular sind.

Betrachten wir also einen Zustand des abstrakten Modells $s=(q_0,\dots,q_N,c_0,\dots)$ und einen Zustand des Promela-Modells $\gamma$.
Sind diese Zustände bisimular, so gilt nach Konstruktion
\[ i((c_0,\dots)) = \gamma(0).\sigma_e \]
und für jeden Prozesszustand $q_j$ entweder
\[ \gamma(j).\sigma_l = \textrm{st\_}q_j \]
oder
\[ \gamma(j).\sigma_l = \textrm{Start} \land q_j = \const{init} \]
Gilt der erste Fall, so lässt sich herleiten
\[ \inference[IfDo-proc]{
  \inference[Basic-proc]{\mathit{exec}(\textrm{C\_}q_j\textrm{\_}q_j',\gamma(j).\sigma_e) & \mathit{next}(\textrm{C\_}q_j\textrm{\_}q_j') = \textrm{A\_}q_j\textrm{\_}q_j'}{
  \xymatrix{ \left<\gamma(j).\sigma_e,\textrm{C\_}q_j\textrm{\_}q_j'\right>\ar@{|->}[r]^-{\llbracket \mu(q_j')\rrbracket_C} & _{proc}
    \left<\gamma(j).\sigma_e,\textrm{A\_}q_j\textrm{\_}q_j'\right>}
  }
  }
  { \xymatrix{ \left<\gamma(j).\sigma_e,\textrm{st\_}q_j\right> \ar@{|->}[r]^-{\llbracket \mu(q_j')\rrbracket_C} & _{proc}
      \left<\gamma(j).\sigma_e,\textrm{A\_}q_j\textrm{\_}q_j'\right>}
  }
\]
Weiterhin ist nach Voraussetzung bekannt, dass die Anweisung $\llbracket \mu(q_j')\rrbracket_A$ nie blockieren darf, also lässt sich herleiten
\[
  \xymatrix{ \left<\gamma(j).\sigma_e,\textrm{A\_}q_j\textrm{\_}q_j'\right> \ar@{|->}[r]^-{\llbracket \mu(q_j')\rrbracket_A} & _{proc}
    \left<\sigma_e',\textrm{st\_}q_j'\right> }
\]
Aus der Promela-Semantik lässt sich nun für ein $\gamma$ mit $\gamma(j).\sigma_l = \textrm{st\_}q_j$ herleiten:
\[
\inference[Atm-mod]{
  \inference[Single-int]{
    \xymatrix{ \left<\gamma(j).\sigma_e,\textrm{st\_}q_j\right> \ar@{|->}[r]^-{\llbracket \mu(q_j')\rrbracket_C} & _{proc}
      \left<\gamma(j).\sigma_e,\textrm{A\_}q_j\textrm{\_}q_j'\right> }
  }{
    \xymatrix{ \gamma \ar@{|->}[r]^-{\llbracket \mu(q_j')\rrbracket_C} & _{int}
      \gamma'
    }
  } & mode(\textrm{C\_}q_j\textrm{\_}q_j') = \mathit{atm}
}{
  \xymatrix{ \gamma \ar@{|->}[r]^-{\mathit{atm}_j} & _{mod}
    \gamma'
  }
}
\]
wobei $\gamma' = \gamma[\left<\gamma(j).\sigma_e,\textrm{A\_}q_j\textrm{\_}q_j'\right>/j]$.
Nach der gleichen Regel lässt sich auch herleiten
\[
\inference[Atm-mod]{
  \inference[Single-int]{
    \xymatrix{ \left<\gamma(j).\sigma_e,\textrm{A\_}q_j\textrm{\_}q_j'\right> \ar@{|->}[r]^-{\llbracket \mu(q_j')\rrbracket_A} & _{proc}
      \left<\sigma_e',\textrm{st\_}q_j'\right> }
  }{
    \xymatrix{ \gamma'\ar@{|->}[r]^-{\llbracket \mu(q_j')\rrbracket_A} & _{int}
      \gamma''
    }
  } & mode(\textrm{A\_}q_j\textrm{\_}q_j') = \mathit{atm}
}{
  \xymatrix{ \gamma'\ar@{|->}[r]^-{\mathit{atm}_j} & _{mod}
    \gamma''
  }
}
\]
wobei $\gamma'' = \gamma[\left<\sigma_e',\textrm{st\_}q_j'\right>/j]$.
Nun lassen sich die beiden atomaren Transitionen zusammenfassen:
\[
\inference[Atm-sim]{
  \xymatrix{ \gamma \ar@{|->}[r]^-{\mathit{atm}_j} & _{mod}
    \gamma'
  } &
  \xymatrix{ \gamma'\ar@{|->}[r]^-{\mathit{atm}_j} & _{mod}
    \gamma''
  }
}{
  \xymatrix{ \gamma \ar@{|->}[r]^-{\mathit{atm}_j} & _{sim}
    \gamma''
  }
}
\]
Fasst man nun alle hier angegebenen Ableitungsschritte zusammen, so ergibt sich
\[
\inference{
  exec(\textrm{C\_}q_j\textrm{\_}q_j',\gamma(j).\sigma_e) &
  \xymatrix{ \left<\gamma(j).\sigma_e,\textrm{A\_}q_j\textrm{\_}q_j'\right> \ar@{|->}[r]^-{\llbracket \mu(q_j')\rrbracket_A} & _{proc}
    \left<\sigma_e',\textrm{st\_}q_j'\right> }
}{
  \xymatrix{ \gamma \ar@{|->}[r]^-{\mathit{atm}_j} & _{sim}
    \gamma[\left<\sigma_e',\textrm{st\_}q_j'\right>/j]
  }
}
\]
Diese zwei Vorbedingungen sind nach Gleichung \ref{eq:assert1} und \ref{eq:assert2} genau dann erfüllt, wenn
\[ \delta^a(q_j,((c_0,\dots)|^a,\beta|^a)) = (q_j',o) \]
gilt.
Dies ist äquivalent dazu dass
\[ \lambda(((q_0,\dots,q_j,\dots,q_N),c),\beta,j) = (((q_0,\dots,q_j',\dots,q_N),c[a\mapsto o]),(\bot,\dots,\bot)[a\mapsto o]) \]
gilt.
Nach Gleichung \ref{eq:assert2} gilt außerdem
\[ i(c[a\mapsto o],\beta) = \sigma_e' \]
wobit die Bisimularität für diesen Fall gezeigt ist, denn
\[
\begin{array}{c}
  i(c[a\mapsto o],\beta) = \sigma_e' \land ((q_0,\dots,q_j,\dots,q_N),c)\cong\gamma\\
  \Rightarrow\\
  ((q_0,\dots,q_j',\dots,q_N),c[a\mapsto o])\cong \gamma[\left<\sigma_e',\textrm{st\_}q_j'\right>/j]
\end{array}
\]

Der Beweis für den Fall, dass sich der Prozess am Label \emph{Start} befindet ist, bis auf Änderung der Label-Namen äquivalent und daher ausgelassen.

\section{Optimierungen}
\subsection{Abstraktion durch statische BDD}
Um die in der GTL-Spezifikation gegebenen Kontrakte für die Komponenten des GALS-Systems für die Verifikation benutzen zu können, müssen diese in den Promela-Formalismus übertragen werden.
Durch die LTL$\rightarrow$Büchi-Übersetzung (Siehe Abschnitt \ref{sec:ltl-translation}) liegen die abstrahierten Komponenten bereits als Büchi-Automaten vor.
Die Zustände der Automaten enthalten aber nicht nur konkrete Wertzuweisungen für die Variablen der Komponenten, sondern auch Einschränkungen der Wertebereiche, die sogar von anderen Variablen abhängen können (Zum Beispiel Relationen wie $x < y + 3$).
Um zumindest Einschränkungen der Wertebereiche, die nicht von anderen Variablen abhängen (Zum Beispiel $x < 5$), übersetzen zu können, kann man BDDs benutzen.

Dazu wird jede Relation der Form $x R c$, wobei $x$ eine Variable und $c$ eine Konstante ist in ein BDD übersetzt.
Diese Übersetzungsmethode ist somit eingeschränkt auf diese Art von Relationen.
Relationen zwischen mehreren Ein- oder Ausgabe-Variablen sind daher nicht möglich.
Das BDD kann nun statt der Werte, die es repräsentiert zur Verifikation verwendet werden.
Es wird allerdings nicht das BDD selbst verwendet, sondern ein Identifier, der es repräsentiert.

Da die Verifikation so allerdings keinen direkten Zugriff mehr auf die Information hat, welches BDD von einem Identifier repräsentiert wird, muss vor der Verifikation eine Tabelle erstellt werden, in der gespeichert wird, welche BDD kompatibel miteinander sind, das heißt, welche von den BDD repräsentierte Mengen Untermengen von welchen anderen BDD-Mengen sind, damit die Bedingungen von Zustandsübergängen geprüft werden können.
\subsubsection{Korrektheit}
Um die formale Korrektheit der Übersetzung durch statische BDDs zu beweisen, müssen zunächst die Symbole $\Sigma$, $\llbracket\rrbracket_C$, $\llbracket\rrbracket_A$ sowie $\llbracket\rrbracket_D$ definiert werden.
Die Semantiksymbole $\Sigma$ sind hierbei eine Menge von Relationen, die höchstens eine Variable enthalten können, also z.B. "`$\{x<5,y=4\}$"'.

Durch einfache Umformungen lässt sich also jedes Semantiksymbol $\eta\in\Sigma$ als eine Abbildung von Variablen $V$ auf ein Entscheidungsdiagramm, dass die Relationen auf der entsprechenden Variable repräsentiert darstellen:
\[ \eta : V\rightarrow \mathbb{BDD} \]
Ist keine Relation für eine Variable angegeben, so ist die Variable unbeschränkt und durch das BDD repräsentiert, dass alle möglichen Werte enthält.

Die Funktion $\llbracket\rrbracket_C$ betrachtet die Eingabevariablen des übergebenen Semantiksymbols $\eta$, also die Menge
\[ \{ (v,\eta(v))\ |\ v\in \mathit{Inp}_i \} \]
Für jedes dieser Tupel wird nun ein Promela Ausdruck generiert, der überprüft, ob die Variable $v$ mit einem zu $\eta(v)$ kompatiblen BDD belegt ist.
Kompatibel bedeutet hierbei, dass die Gleichung $v\cap\eta(v)\neq \emptyset$ erfüllt ist.

Die Anweisungen, die von der $\llbracket\rrbracket_A$-Funktion generiert werden, weisen den Ausgabevariablen der Komponente neue BDDs zu.
Das bedeutet, dass für jedes Tupel der Menge
\[ \{ (v,\eta(v))\ |\ v\in \mathit{Out}_i \} \]
die Anweisung
\begin{lstlisting}[language=promela,mathescape=true]
  v = $\eta(v)$;
\end{lstlisting}
generiert wird.

Für jede Variable der Komponente wird durch $\llbracket\rrbracket_D$ eine globale Integer Variable generiert, die die Repräsentation des entsprechenden BDDs enthält.
Sind die Variablen von zwei Komponenten durch eine \emph{connect}-Deklaration verbunden, so wird nur eine Variable für die Eingangsvariable generiert, damit das Modell nicht unnötig vergrößert wird.
Schreibt der Ausgabeprozess auf seine Variable, so wird das Ergebnis stattdessen direkt in die gemeinsam verwendete Variable geschrieben.
\section{Abstraktion durch dynamische BDD}
Um die Probleme der statischen BDD-Übersetzung zu lösen, bietet es sich an, die Berechnung der BDDs erst während der Verifikation durchzuführen.
Der Vorteil ist, dass man nun auch mehrere Variablen miteinander in Beziehung setzten kann und auch Modelle verifizieren kann, die Zyklen enthalten.
Es kann allerdings passieren, dass bei der Verifikation mehrfach die gleiche BDD Operation ausgeführt wird; dies lässt sich aber durch den Einsatz eines Operationscaches vermeiden.

Für die Implementierung der dynamischen BDDs wurde die C-Bibliothek \emph{CUDD}
\footnote{Die Abkürzung steht für {\bf C}olorado {\bf U}niversity {\bf D}ecision {\bf D}iagram Package.
  Ein Benutzerhandbuch, sowie die Quellen sind online erhältlich\cite{cudd}.
  Die Bibliothek ist unter einer BSD-artigen Lizenz veröffentlicht.
}
gewählt, da diese im Gegensatz zu anderen Bibliotheken Zugriff auf ihre internen Schnittstellen bietet und somit sehr gut zu erweitern ist.
\section{Fehlereingrenzung}
\label{sec:error-refinement}
Das Ergebnis einer Verifikation, die die Kontrakt-Spezifikationen zur Optimierung verwendet sind eine oder mehrere Fehlerspuren.
Diese geben eine zeitlich geordnete Kette von Bedingungen über die Variablen der Komponenten des Systems.
Ein Beispiel für eine solche Spur ist die Kette
\[ \left[ (a<3,b\in \{3,5,6\}), (a > 4), (b\neq 5) \right] \]
Diese Kette von Bedingungen spezifiziert aber nicht ein Verhalten, sondern mehrere.
Die folgenden Verhalten des Systems sind beispielsweise spezifiziert:
\[ \left[ (a=2,b=3), (a=6), (b = 1) \right] \]
\[ \left[ (a=1,b=3), (a=5), (b = 1) \right] \]
Aus dieser Menge von spezifizierten Verhaltensweisen müssen nun nicht alle einen echten Fehler des Systems darstellen.
Tatsächlich reicht es, wenn ein Verhalten einen Fehler hervorruft.
Möglich ist aber auch, dass die Kontrakte dem System ein Verhalten erlauben, was das echte System niemals erzeugt.
In diesem Fall ist die Spezifikation des Systems zu grob und die Fehlerspuren nicht immer echte Fehler.

Um nun herauszufinden, welche konkreten Fehlerspuren das spezifizierte System nun tatsächlich hat, wird eine erneute Verifikation durchgeführt.
Diesmal wird das echte Systemverhalten als Grundlage herangezogen, wobei aber das Verhalten auf die Spuren begrenzt wird, die durch die Fehlerspur angegeben werden.
Meldet die Verifikation des so eingegrenzten Systems ebenfalls einen Fehler, so erhält man nicht nur die Bestätigung, dass die vorher erzeugte Fehlerspur echt ist, sondern auch ein konkretes Systemverhalten, dass zu einem Fehler führt.
Zeigt sich kein Fehler, so kann dies ein Hinweis sein, dass nicht genügend scharfe Kontrakte formuliert wurden.

\begin{figure}[h]
  \centering
  \begin{tikzpicture}
    \node[draw] (gtl) at (0,10) {GTL};
    \node[draw] (scade) at (4.5,10) {Scade};
    \node[draw] (pr1) at (0,9) {\begin{tabular}{c}Promela\\(Abstraktion)\end{tabular}};
    \node[draw] (ver1) at (0,8) {Verifier};
    \node[draw] (trace1) at (0,7) {\begin{tabular}{c}Fehlerspur\\(abstrakt)\end{tabular}};
    \node[draw] (pr2) at (3.5,7) {\begin{tabular}{c}Promela\\(C-Integration)\end{tabular}};
    \node[draw] (pr3) at (1.5,5.5) {\begin{tabular}{c}Promela\\(eingeschränkt)\end{tabular}};
    \node[draw] (ver2) at (1.5,4) {Verifier};
    \node[draw] (trace2) at (1.5,3) {Fehlerspur};
    \draw[->,blue] (gtl) -- (pr1);
    \draw[->,blue] (pr1) -- (ver1);
    \draw[->,blue] (ver1) -- (trace1);
    \draw[->,blue] (trace1) -| (pr3);
    \draw[->,blue] (pr2) -| (pr3);
    \draw[->,blue] (gtl) -| (pr2);
    \draw[->,blue] (scade) -| (pr2);
    \draw[->,blue] (pr3) -- (ver2);
    \draw[->,blue] (ver2) -- (trace2);
  \end{tikzpicture}
  \caption{Fehlereingrenzung}
  \label{fig:error-refinement}
\end{figure}

Um diese Technik konkret umsetzen zu können, müssen an vielen Stellen der Implementierung Erweiterungen vorgenommen werden.
Hierzu betrachten wir Abbildung \ref{fig:error-refinement}:
\begin{itemize}
\item Zunächst muss der aus der GTL-Spezifikation generierte Promela-Code in der Lage sein, Fehlerspuren zu erzeugen, die Aufschluss darüber geben, welche Transitionen des generierten Büchi-Automaten zu dem Fehler führten.
  Zwar erzeugt auch SPIN schon Fehlerspuren, jedoch geben diese Auskunft über die Ausführungsposition im Quelltext und sind daher extrem schwer zurück auf die Zustände des Büchi-Automaten zu rechnen.
  Die Lösung besteht darin, im generierten Quelltext vor jedem Betreten eines Zustands mit der \emph{printf}-Anweisung eine Ausgabe zu erzeugen, die den Namen des betretenden Zustands enthält.
  Spielt man eine generierte Fehlerspur nun in SPIN ab, so erhält man genau die Ausgaben für die betretenden Zustände und kann eine genaue Fehlerspur für den Büchi-Automaten berechnen.
\item Die Fehlerspur wird kodiert als eine Liste, in der jedes Element einem Zeitschritt entspricht.
  Ein Element enthält eine Abbildung von den Variablen des Systems auf BDDs, die den erlaubten Wertebereich dieser Variablen zum Zeitpunkt des Elements festlegen.
\item Die so generierte Fehlerspur muss jetzt noch benutzt werden, um in der Verifikation per C-Integration die Pfade zu beschränken.
  Dies geschieht, indem die Schnittmenge des generierte Büchi-Automat für die zu verifizierende Eigenschaft und des Automaten aus der Fehlerspur gebildet wird.
  Damit werden im Modell nur Pfade verifiziert, die der Fehlerspur entsprechen.
\end{itemize}

\chapter{Implementierung}
\label{sec:implementation}


Die Implementierung besteht zum einen aus der eigentlichen Anwengung -- \emph{gtl} -- und zum anderen aus verschiedenen Bibliotheken, die zusätzlich entwickelt werden mussten.
Diese sind:
\begin{itemize}
\item \emph{language-promela} -- Stellt Datenstrukturen für den Promela-Syntax bereit, formatiert Promela-Quelltext für die Ausgabe und parst Promela-Code.
\item \emph{language-scade} -- Ein Parser und Code-Generator für den SCADE-Syntax.
\item \emph{bdd} -- Eine Bibliothek, die binäre Entscheidungsdiagramme ("`binary decision diagrams"' -- BDD) implementiert.
\end{itemize}

\begin{figure}[h]
  \centering
  \begin{tikzpicture}
  \node[tape,draw,tape bend top=none] (gtl) at (0,0) {GTL};
  \node[rectangle,draw] (parser) at (0,-1) {Parser};
  \node[rounded rectangle,draw] (gtl ast) at (0,-2) {GTL AST};
  \node[rectangle,draw] (type checker) at (0,-3) {Type checker};
  \node[tape,draw,tape bend top=none] (scade) at (4,-3) {SCADE};
  \node[rounded rectangle,draw] (gtl spec) at (0,-4) {GTL Spec};
  \node[tape,draw,tape bend top=none] (promela1) at (-3,-5) {Promela};
  \node[cylinder,draw,shape border rotate=90,aspect=0.25] (cudd) at (-5,-6) {CUDD};
  \node[cloud,draw,cloud ignores aspect,inner sep=0em] (dyn bdd) at (-3,-8) {\begin{tabular}{c}Dynamic\\BDD\\Verifier\end{tabular}};
  \node[tape,draw,tape bend top=none] (promela2) at (0,-5) {Promela};
  \node[rectangle,draw] (kcg) at (4,-4) {KCG};
  \node[tape,draw,tape bend top=none] (c) at (4,-5) {C};
  \node[cloud,draw,cloud ignores aspect,inner sep=0em] (native) at (0,-8) {\begin{tabular}{c}Native\\Verifier\end{tabular}};
  \node[tape,draw,tape bend top=none] (scade2) at (2,-5) {\begin{tabular}{c}SCADE\\Testnode\end{tabular}};
  \node[cloud,draw,cloud ignores aspect,inner sep=0em] (scade verifier) at (6,-7) {\begin{tabular}{c}SCADE\\Verifier\end{tabular}};
  \draw[->] (gtl) -- (parser);
  \draw[->] (parser) -- (gtl ast);
  \draw[->] (gtl ast) -- (type checker);
  \draw[->] (scade) -- (type checker);
  \draw[->] (type checker) -- (gtl spec);
  \draw[->] (gtl spec) -| (promela1);
  \draw[->] (gtl spec) -- (promela2);
  \draw[->] (gtl spec) -| (scade2);
  \draw[->] (cudd) -| (dyn bdd);
  \draw[->] (promela1) -- (dyn bdd);
  \draw[->] (scade) -- (kcg);
  \draw[->] (kcg) -- (c);
  \draw[->] (promela2) -- (native);
  \draw[->] (c) |- (native);
  \draw[->] (scade) -| (scade verifier);
  \draw[->] (scade2) |- (scade verifier);
\end{tikzpicture}

  \caption{GTL Implementierung}
  \label{fig:gtl_implementation}
\end{figure}

Abbildung \ref{fig:gtl_implementation} zeigt den Datenfluss der \emph{gtl}-Anwendung.
Zunächst wird mithilfe des Parsers eine textuelle GTL-Repräsentation in einen abstrakten Syntax-Baum\footnote{englisch: abstract syntax tree, AST} transformiert.
Der Parser wird im Abschnitt \ref{module:Language.GTL.Parser} beschrieben, der Syntax-Baum in \ref{module:Language.GTL.Parser.Syntax}.
Daraufhin wird der Syntax-Baum an die Typüberprüfung weiter gereicht.
Diese extrahiert die Typinformationen aus den verwendeten synchronen Komponenten und überprüft, ob alle Kontrakte und Verifikationsformeln wohl-getypt sind (Siehe Abschnitt \ref{sec:sos}).
Für das SCADE-Backend müssen also die definierenden Dateien geparst werden, die Modelle in dem entstandenen Syntax-Baum gefunden werden und die SCADE-Typen in GTL-Typen umgewandelt werden.
Daraufhin wird der Syntax-Baum in eine Instanz des \emph{GTLSpec}-Datentyps umgewandelt (Beschrieben in Abschnitt \ref{module:Language.GTL.Model}).

Ab hier entscheidet sich nun, welche Transformation vom Benutzer gewählt wurde.
Für die SCADE-Verifikation der Komponenten wird für jede Komponente ein SCADE-Testknoten erzeugt, der dann zusammen mit dem Quelltext der Modelle mit dem SCADE Design-Verifier geprüft wird (Beschrieben in Abschnitt \ref{module:Language.GTL.Backend.Scade}).

Für die C-Übersetzung wird der SCADE Code-Generator KCG aufgerufen, der wie in Abschnitt \ref{sec:c_integration} beschrieben C-Code für alle Modelle liefert.
Es wird dann Promela-Code generiert, der die einzelnen C-Code-Modelle vereint.
Die Implementierung dieses Verfahrens wird in Abschnitt \ref{module:Language.GTL.PromelaCIntegration} genauer beschrieben.

Für die Übersetzung der Kontrakte mithilfe von binären Entscheidungsdiagrammen, wie in Abschnitt \ref{sec:bdd} beschrieben, wird Promela-Code generiert und dann gegen die CUDD-Bibliothek gelinkt.
Genauere Details des Verfahrens sind in Abschnitt \ref{module:Language.GTL.PromelaDynamicBDD} angegeben.
\chapter{Fallbeispiele}
\label{sec:case_studies}
\section{Quelle-Senke Beispiel}
Dieses minimalistische Beispiel soll die grundsätzliche Funktionsweise des Verifikationsalgorithmus und der BDD-Optimierung erläutern und demonstrieren, wie durch die Optimierung der Zustandsraum von Modellen reduziert werden kann.

Das System besteht aus zwei Komponenten:
Die Quelle hat einen Ausgang und produziert gibt die Sequenz der natürlichen Zahlen modulo 10 zurück, also
\[ 0,1,2,3,4,5,6,7,8,9,0,1,2,\dots \]
Der Ausgang der Quelle ist mit der Eingabe der Senke verbunden.
Diese prüft, ob der Wert an ihrem Eingang kleiner als 12 ist und setzt dann den Ausgang auf den Wert 0, andernfalls auf 1.
Zu verifizieren ist in diesem Beispiel, dass der Ausgabewert der Senke stehts 0 ist.

Für den Kontrakt bietet sich also an zu spezifizieren, dass die Quelle stets Werte produziert, die kleiner als 10 sind.
Die Senke produziert für Werte kleiner 10 stets die Ausgabe 0, was sich ebenfalls als Kontrakt formulieren lässt.
Die resultierende GTL-Beschreibung sieht wie folgt aus:
\begin{lstlisting}[language=gtl]
model[scade] source("source_sink.scade","Source") {
  contract always outp < 10;
  init outp 9;
}

model[scade] sink("source_sink.scade","Sink") {
  init outp 0;
  contract always (inp < 10 => outp=0);
}

connect source.outp sink.inp;

verify {
  always sink.outp=0;
}
\end{lstlisting}
Das Kontrakt-Automaten System, dass sich aus dieser Beschreibung ergibt ist in Abbildung \ref{fig:source_sink_automata} gezeigt.

\begin{figure}[h]
  \centering
  \begin{tikzpicture}
    \draw[color=red!50,fill=red!20,thick] (178.82bp,1.5bp) -- (216.82bp,1.5bp) -- (216.82bp,63.5bp) -- (178.82bp,63.5bp) -- (178.82bp,1.5bp);
\draw[color=red!50,fill=red!20,thick] (216.82bp,1.5bp) -- (356.82bp,1.5bp) -- (356.82bp,63.5bp) -- (216.82bp,63.5bp) -- (216.82bp,1.5bp);
\draw[color=red!50,fill=red!20,thick] (356.82bp,1.5bp) -- (404.82bp,1.5bp) -- (404.82bp,63.5bp) -- (356.82bp,63.5bp) -- (356.82bp,1.5bp);
\draw (197.82bp,32.5bp) node {inp};
\draw (380.82bp,32.5bp) node {outp};
\begin{scope}[shift={(220.375bp,6.2044999999999995bp)}]
\draw [color=blue!50,very thick,fill=blue!20](88.888bp,39.091bp) ellipse (25.99992bp and 12.49992bp);
\draw (88.888bp,39.091bp) node {$\begin{array}{c}inp\geq 10\end{array}$};
\draw [color=blue!50,very thick,fill=blue!20](27.0bp,32.748bp) ellipse (25.99992bp and 12.49992bp);
\draw (27.0bp,32.748bp) node {$\begin{array}{c}outp=1\end{array}$};
\draw [fill](63.529bp,3.0bp) ellipse (2.000016bp and 2.000016bp);
\draw [-,thick] (111.53bp,45.642bp) .. controls (122.82bp,46.548bp) and (132.89bp,44.365bp) .. (132.89bp,39.091bp);
\draw [-,thick] (132.89bp,39.091bp) .. controls (132.89bp,35.548bp) and (128.34bp,33.4bp) .. (121.93bp,32.646bp);
\draw [-,thick] (63.359bp,36.475bp) .. controls (63.134bp,36.452bp) and (62.909bp,36.429bp) .. (62.683bp,36.406bp);
\draw [-,thick] (52.53bp,35.365bp) .. controls (52.754bp,35.388bp) and (52.979bp,35.411bp) .. (53.205bp,35.434bp);
\draw [-,thick] (49.639bp,39.299bp) .. controls (60.929bp,40.205bp) and (71.0bp,38.022bp) .. (71.0bp,32.748bp);
\draw [-,thick] (71.0bp,32.748bp) .. controls (71.0bp,29.205bp) and (66.454bp,27.057bp) .. (60.046bp,26.304bp);
\draw [-,thick] (64.618bp,4.5508bp) .. controls (66.427bp,7.1253bp) and (70.259bp,12.578bp) .. (74.354bp,18.407bp);
\draw [-,thick] (61.959bp,4.2783bp) .. controls (59.405bp,6.3579bp) and (54.054bp,10.716bp) .. (48.288bp,15.411bp);
\draw [-latex,thick] (121.93bp,32.646bp) -- (111.53bp,32.541bp);
\draw [-latex,thick] (62.683bp,36.406bp) -- (52.471bp,35.359bp);
\draw [-latex,thick] (53.205bp,35.434bp) -- (63.417bp,36.481bp);
\draw [-latex,thick] (60.046bp,26.304bp) -- (49.639bp,26.198bp);
\draw [-latex,thick] (74.354bp,18.407bp) -- (80.338bp,26.923bp);
\draw [-latex,thick] (48.288bp,15.411bp) -- (40.394bp,21.84bp);
\end{scope}

\draw[color=red!50,fill=red!20,thick] (1.0bp,14.5bp) -- (117.0bp,14.5bp) -- (117.0bp,50.5bp) -- (1.0bp,50.5bp) -- (1.0bp,14.5bp);
\draw[color=red!50,fill=red!20,thick] (117.0bp,14.5bp) -- (165.0bp,14.5bp) -- (165.0bp,50.5bp) -- (117.0bp,50.5bp) -- (117.0bp,14.5bp);
\draw (141.0bp,32.5bp) node {outp};
\begin{scope}[shift={(5.840000000000003bp,19.0bp)}]
\draw [color=blue!50,very thick,fill=blue!20](59.321bp,13.5bp) ellipse (29.00016bp and 12.49992bp);
\draw (59.321bp,13.5bp) node {$\begin{array}{c}outp<10\end{array}$};
\draw [fill](3.0bp,13.5bp) ellipse (2.000016bp and 2.000016bp);
\draw [-,thick] (83.985bp,20.085bp) .. controls (95.848bp,20.898bp) and (106.32bp,18.703bp) .. (106.32bp,13.5bp);
\draw [-,thick] (106.32bp,13.5bp) .. controls (106.32bp,9.9229bp) and (101.37bp,7.7675bp) .. (94.422bp,7.0339bp);
\draw [-,thick] (4.8739bp,13.5bp) .. controls (7.6696bp,13.5bp) and (13.321bp,13.5bp) .. (19.967bp,13.5bp);
\draw [-latex,thick] (94.422bp,7.0339bp) -- (83.985bp,6.9148bp);
\draw [-latex,thick] (19.967bp,13.5bp) -- (30.196bp,13.5bp);
\end{scope}

\draw [-,thick] (165.0bp,32.5bp) .. controls (165.0bp,32.5bp) and (166.63bp,32.5bp) .. (168.78bp,32.5bp);
\draw [-latex,thick] (168.78bp,32.5bp) -- (178.82bp,32.5bp);


  \end{tikzpicture}
  \caption{Quelle-Senke Kontrakt-Automaten}
  \label{fig:source_sink_automata}
\end{figure}

Wenig erstaunlich ist nun das Ergebnis der Verifikation:
Da die BDD-Abstraktion die Bedingung $\mathit{outp}<10$ als eine Transition betrachtet, ist das resultierende Transitionssystem in der BDD-Verifikation bedeutend kleiner als in der C-Integration (Siehe Tabelle \ref{tab:source_sink_verifikation}).

\begin{table}
  \begin{tabular}{|l|r|r|r|}
    \hline
    \textbf{Modus} & \textbf{Zustände} & \textbf{Transitionen} & \textbf{Speicherverbrauch}\\
    \hline
    Native & 25 & 49 & 4,653 MB\\
    BDD & 6 & 11 & 4,653 MB\\
    \hline
  \end{tabular}
  \caption{Quelle-Senke Verifikationsergebnisse}
  \label{tab:source_sink_verifikation}
\end{table}

\chapter{Ausblick}
\label{sec:conclusion}
Die vorliegende Arbeit zeigt den aktuellen Entwicklungsstand der GTL-Spezifikationssprache (April 2011).
Im Rahmen des \emph{VerSyKo}-Projekts wird jedoch weiter an Verifikationsalgorithmen für verteilte GALS-Architekturen geforscht.
Die Ergebnisse dieser Forschung werden sehr wahrscheinlich großen Einfluss auf die weitere Entwicklung der GTL-Sprache haben.
Unter anderem werden folgende Gebiete untersucht:
\begin{itemize}
\item Inkorporation von zeit-basierter Modelierung und Verifikation.
  Hierbei soll die Verifikation von Realzeit-Eigenschaften ermöglicht werden.
\item Vervollständigung des Datenmodells.
  Momentan unterstützt die GTL Sprache nur einen Bruchteil der Datentypen der zugrundeliegenden Formalismen.
  Eine Erweiterung der Datentypen ermöglicht die Verifikation von realistischeren Modellen.
\item Verwendung von anderen synchronen Formalismen.
  Der Verifikationsalgorithmus des GTL-Programms erlaubt die gleichzeitige Verwendung von verschiedenen synchronen Formalismen zur Spezifikation von Komponenten.
  Die Erweiterung um Formalismen außer SCADE ist ein weiteres Ziel der Entwicklung.
\end{itemize}

\bibliography{lit}
\begin{appendix}
  \chapter{Installation}
Hier wird beschrieben, wie das GTL-Tool installiert werden kann.
\section{Voraussetzungen}
Die folgenden Software-Pakete werden für die Kompilierung der Software benötigt:
\begin{itemize}
\item Der Haskell-Compiler \emph{GHC}\footnote{Glasgow Haskell Compiler}.
  Es bietet sich die Installation der "`Haskell Platform"' an, die neben dem Compiler noch benötigte Distributionstools und Bibliotheken mitbringt.
  Die Linux-Distributionen "`Ubuntu"', "`Debian"', "`Fedora"', "`Arch Linux"', "`Gentoo"' und "`NixOS"' bieten diese bereits über ihre interne Paketverwaltung an.
  Für andere Systeme ist die Plattform unter \url{http://hackage.haskell.org/platform/} erhältlich.
\item Die Haskell-Bibliotheken "`binary"' und "`bzlib"', welche beide über die Haskell-Pa\-ket\-ver\-wal\-tung "`hackage"' verfügbar sind und damit automatisch bei der Kompilierung heruntergeladen und installiert werden.
\end{itemize}
Für die Nutzung aller Features des GTL-Tools sind außerdem die folgenden Softwarekomponenten erforderlich:
\begin{itemize}
\item Der Model-Checker \emph{SPIN}, erhältlich unter \url{http://spinroot.com}.
  Die Software ist in C geschrieben und hat keine externen Abhängigkeiten und sollte daher auf fast jeder Plattform verfügbar sein.
\item Die BDD-Bibliothek \emph{CUDD}, die man unter \url{http://vlsi.colorado.edu/~fabio/CUDD/} in Quelltextform findet.
\item Für die Überprüfung der synchronen Komponenten ist das Entwicklungswerkzeug \emph{SCADE} erforderlich.
  Als einzige Komponente ist diese nicht frei verfügbar, sondern muss von der Firma "`Esterel Technologies"' (\url{http://esterel-technologies.com} lizensiert werden.
\end{itemize}
Soll die aktuellste Version der Quelltexte bezogen werden, so benötigt man außerdem die Versionsverwaltung \emph{git}, erhältlich unter \url{http://git-scm.com/}.
\section{Quelltext beziehen}
Für den Quelltext zu der Anwendung entpackt man entweder die mitgelieferten Archive, oder lädt den aktuellsten Entwicklungsstand der Pakete per \emph{git} herunter.
Der entsprechende Befehl lautet
\begin{lstlisting}[language=bash,mathescape=true]
git checkout https://github.com/hguenther/$name$.git
\end{lstlisting}
Wobei $name$ den Namen des Paketes bezeichnet.
Die zu herunterladenen Pakete sind:
\begin{itemize}
\item language-scade
\item language-promela
\item bdd
\item gtl
\end{itemize}
\section{Kompilieren}
In jedem Quelltext-Verzeichnis muss nun der Befehl
\begin{lstlisting}[language=bash]
cabal install
\end{lstlisting}
ausgeführt werden.
Hierbei ist zu beachten, dass das Paket "`gtl"' als letztes installiert werden muss, da es von allen anderen abhängt.
\section{Verwendung}
Das Resultat der oben angegebenen Installationsschritte ist die Anwendung "`gtl"', die im Plattform-abhängigen Anwendungsverzeichnis zu finden ist.
Um die Optionen und Parameter der Anwendung zu begutachten lässt sich die Anwendung mit dem Parameter "`\verb|--help|"' starten:
\begin{verbatim}
gtl --help
\end{verbatim}
Liegt eine GTL-Spezifikation im in Abschnitt \ref{sec:grammar} beschriebenen Format vor, so lassen sich die verschiedenen Verifikationsmodi wie folgt ausführen ("`\verb|spec.gtl| ist hier immer die GTL-Spezifikationsdatei):
\begin{itemize}
\item Die SCADE-Verifikation der einzelnen Komponenten wird mit
\begin{verbatim}
gtl -m local spec.gtl
\end{verbatim}
ausgeführt.
Die SCADE-Modelle werden dabei aus der Datei extrahiert, die im ersten Parameter des Modells übergeben wird.
Der Name in der SCADE-Datei wird über den zweiten Parameter festgelegt.
Der folgende Code gibt also an, dass sich das gesuchte Beispiel in der Datei "`\verb|ExampleCar.scade|"' befindet und dort den Namen "`\verb|Car|"' trägt.
\begin{lstlisting}[language=gtl]
model[scade] car("ExampleCar.scade","Car");
\end{lstlisting}
\item Um ein Promela-Modell zu erhalten, dass die Kontrakte nativ übersetzt zu erhalten, wird
\begin{verbatim}
gtl -m native spec.gtl
\end{verbatim}
ausgeführt.
Das erzeugte Promela-Modell kann nun mit SPIN verifiziert werden.
\item Die dynamische BDD-Optimierung lässt sich wie folgt aufrufen:
\begin{verbatim}
gtl -m promela-buddy spec.gtl
\end{verbatim}
\item Die C-Integration der Komponenten wird durchgeführt mit
\begin{verbatim}
gtl -m native-c spec.gtl
\end{verbatim}
Man erhält hierbei ein Promela-Modell, dass den C-Code der Komponenten einbindet.
Die Kontrakte der Komponenten werden nicht verwendet.
Es müssen bei der Kompilierung des Verifikations-Binaries die Ordner, die den generierten C-Quelltext enthalten mit angegeben werden.


Hat man durch die Verifikation mit SPIN eines im zweiten oder dritten Punkt erzeugten GALS Modell in Promela eine Fehlerspur "`\verb|error.gtltrace|"' erhalten, so kann man diese in der C-Integration verwenden, um zu überprüfen, ob es sich um einen echten Fehler oder eine Unterspezifikation der Kontrakte handelt, indem man
\begin{verbatim}
gtl -m native-c spec.gtl -t error.gtltrace
\end{verbatim}
verwendet.
\end{itemize}

  \chapter{Quelltextdokumentation}
Diese Dokumentation wurde automatisch mit einer modifizierten Version des \emph{haddock}-Dokumentations-Tools\footnote{Erhältlich unter \url{http://haskell.org/haddock}.} aus den Quelltexten der für die Arbeit erstellten Anwendung generiert.
Die Dokumentation enthält Informationen über alle Datentypen, Klassen und Funktionen der Anwendung und gibt Aufschlüsse über die interne Funktionsweise der Anwendung.
\haddockmoduleheading{Language.GTL.Backend}
\label{module:Language.GTL.Backend}
\haddockbeginheader
{\haddockverb\begin{verbatim}
module Language.GTL.Backend (
    ModelInterface, 
    GTLBackend(GTLBackendModel,
               backendName,
               initBackend,
               typeCheckInterface,
               cInterface,
               backendVerify), 
    CInterface(CInterface,
               cIFaceIncludes,
               cIFaceStateType,
               cIFaceInputType,
               cIFaceStateInit,
               cIFaceIterate,
               cIFaceGetOutputVar,
               cIFaceGetInputVar,
               cIFaceTranslateType), 
    mergeTypes
  ) where\end{verbatim}}
\haddockendheader

Provides an abstraction over many different synchronous formalisms.
\par

\begin{haddockdesc}
\item[\begin{tabular}{@{}l}
type\ ModelInterface\ =\ (Map\ String\ TypeRep,\ Map\ String\ TypeRep)
\end{tabular}]
\end{haddockdesc}
\begin{haddockdesc}
\item[\begin{tabular}{@{}l}
class\ GTLBackend\ b\ where
\end{tabular}]\haddockbegindoc
A GTLBackend is a synchronous formalism that can be used to specify models and perform verification.
\par

\haddockpremethods{}\textbf{Methods}
\begin{haddockdesc}
\item[\begin{tabular}{@{}l}
backendName\ ::\ b\ ->\ String
\end{tabular}]\haddockbegindoc
The name of the backend. Used to determine which backend to load.
\par

\end{haddockdesc}
\begin{haddockdesc}
\item[\begin{tabular}{@{}l}
initBackend\ ::\ b\ ->\ {\char 91}String{\char 93}\ ->\ IO\ (GTLBackendModel\ b)
\end{tabular}]\haddockbegindoc
Initialize a backend with a list of parameters
\par

\end{haddockdesc}
\begin{haddockdesc}
\item[\begin{tabular}{@{}l}
typeCheckInterface
\end{tabular}]\haddockbegindoc
\haddockbeginargs
\haddockdecltt{::} & \haddockdecltt{b} & The backend
 \\
                                         \haddockdecltt{->} & \haddockdecltt{GTLBackendModel b} & The backend data
 \\
                                                                                                  \haddockdecltt{->} & \haddockdecltt{ModelInterface} & A type mapping for the in- and outputs
 \\
                                                                                                                                                        \haddockdecltt{->} & \haddockdecltt{Either String ModelInterface} & \\
\end{tabulary}\par
Perform type checking on the synchronized model
\par

\end{haddockdesc}
\begin{haddockdesc}
\item[\begin{tabular}{@{}l}
cInterface
\end{tabular}]\haddockbegindoc
\haddockbeginargs
\haddockdecltt{::} & \haddockdecltt{b} & The backend
 \\
                                         \haddockdecltt{->} & \haddockdecltt{GTLBackendModel b} & The backend data
 \\
                                                                                                  \haddockdecltt{->} & \haddockdecltt{CInterface} & \\
\end{tabulary}\par
Get the C-interface of a GTL model
\par

\end{haddockdesc}
\begin{haddockdesc}
\item[\begin{tabular}{@{}l}
backendVerify\ ::\ b\\\ \ \ \ \ \ \ \ \ \ \ \ \ \ \ \ \ ->\ GTLBackendModel\ b\ ->\ Expr\ String\ Bool\ ->\ IO\ (Maybe\ Bool)
\end{tabular}]\haddockbegindoc
Perform a backend-specific model checking algorithm.
   Returns \haddockid{Nothing} if the result is undecidable and \haddockid{Just} \haddockid{True}, if the verification goal holds.
\par

\end{haddockdesc}
\end{haddockdesc}
\begin{haddockdesc}
\item[\begin{tabular}{@{}l}
instance\ GTLBackend\ Scade
\end{tabular}]
\end{haddockdesc}
\begin{haddockdesc}
\item[\begin{tabular}{@{}l}
data\ CInterface
\end{tabular}]\haddockbegindoc
\haddockbeginconstrs
\haddockdecltt{=} & \haddockdecltt{CInterface} & \\
                    \{ & \haddockdecltt{cIFaceIncludes :: [String]} & A list of C-headers to be included
 \\
                    , & \haddockdecltt{cIFaceStateType :: [String]} & A list of C-types that together form the signature of the state of the state machine
 \\
                    , & \haddockdecltt{cIFaceInputType :: [String]} & The type signature of the input variables. Input variables aren't considered state.
 \\
                    , & \haddockdecltt{cIFaceStateInit :: [String]
                                                          -> String} & Generate a call to initialize the state machine
 \\
                    , & \haddockdecltt{cIFaceIterate :: [String]
                                                        -> [String]
                                                           -> String} & Perform one iteration of the state machine
 \\
                    , & \haddockdecltt{cIFaceGetOutputVar :: [String]
                                                             -> String
                                                                -> String} & Extract an output variable from the machine state
 \\
                    , & \haddockdecltt{cIFaceGetInputVar :: [String]
                                                            -> String
                                                               -> String} & Extract an input variable from the state machine
 \\
                    , & \haddockdecltt{cIFaceTranslateType :: TypeRep
                                                              -> String} & Translate a haskell type to C
 \\
                    \} &
\end{tabulary}\par
A C-interface is information that is needed to integrate a C-state machine.
\par

\end{haddockdesc}
\begin{haddockdesc}
\item[\begin{tabular}{@{}l}
mergeTypes\ ::\ Map\ String\ TypeRep\\\ \ \ \ \ \ \ \ \ \ \ \ \ \ ->\ Map\ String\ TypeRep\ ->\ Either\ String\ (Map\ String\ TypeRep)
\end{tabular}]\haddockbegindoc
Merge two type-mappings into one, report conflicting types
\par

\end{haddockdesc}
\haddockmoduleheading{Language.GTL.Backend.All}
\label{module:Language.GTL.Backend.All}
\haddockbeginheader
{\haddockverb\begin{verbatim}
module Language.GTL.Backend.All (
    AllBackend(AllBackend, allTypecheck, allCInterface, allVerifyLocal), 
    initAllBackend
  ) where\end{verbatim}}
\haddockendheader

\begin{haddockdesc}
\item[\begin{tabular}{@{}l}
data\ AllBackend
\end{tabular}]\haddockbegindoc
\haddockbeginconstrs
\haddockdecltt{=} & \haddockdecltt{AllBackend} & \\
                    \{ & \haddockdecltt{allTypecheck :: ModelInterface
                                                        -> Either String ModelInterface} & \\
                    , & \haddockdecltt{allCInterface :: CInterface} & \\
                    , & \haddockdecltt{allVerifyLocal :: Expr String Bool
                                                         -> IO (Maybe Bool)} & \\
                    \} &
\end{tabulary}\par
\end{haddockdesc}
\begin{haddockdesc}
\item[
initAllBackend\ ::\ String\ ->\ {\char 91}String{\char 93}\ ->\ IO\ (Maybe\ AllBackend)
]
\end{haddockdesc}
\haddockmoduleheading{Language.GTL.Backend.Scade}
\label{module:Language.GTL.Backend.Scade}
\haddockbeginheader
{\haddockverb\begin{verbatim}
module Language.GTL.Backend.Scade (
    Scade(Scade),  scadeTranslateTypeC,  scadeTypeToGTL,  scadeTypeMap, 
    scadeInterface,  buildTest,  buchiToScade,  startState,  failTransition, 
    failState,  buchiToStates,  stateToTransition,  litToExpr,  relToExpr, 
    relsToExpr
  ) where\end{verbatim}}
\haddockendheader

\begin{haddockdesc}
\item[\begin{tabular}{@{}l}
data\ Scade
\end{tabular}]\haddockbegindoc
\haddockbeginconstrs
\haddockdecltt{=} & \haddockdecltt{Scade} & \\
\end{tabulary}\par
\end{haddockdesc}
\begin{haddockdesc}
\item[\begin{tabular}{@{}l}
instance\ Show\ Scade\\instance\ GTLBackend\ Scade
\end{tabular}]
\end{haddockdesc}
\begin{haddockdesc}
\item[
scadeTranslateTypeC\ ::\ TypeRep\ ->\ String
]
\item[
scadeTypeToGTL\ ::\ TypeExpr\ ->\ Maybe\ TypeRep
]
\item[
scadeTypeMap\ ::\ {\char 91}(String,\ TypeExpr){\char 93}\\\ \ \ \ \ \ \ \ \ \ \ \ \ \ \ \ ->\ Either\ String\ (Map\ String\ TypeRep)
]
\end{haddockdesc}
\begin{haddockdesc}
\item[\begin{tabular}{@{}l}
scadeInterface
\end{tabular}]\haddockbegindoc
\haddockbeginargs
\haddockdecltt{::} & \haddockdecltt{String} & The name of the Scade model to analyze
 \\
                                              \haddockdecltt{->} & \haddockdecltt{[Declaration]} & The parsed source code
 \\
                                                                                                   \haddockdecltt{->} & \haddockdecltt{([(String, TypeExpr)], [(String, TypeExpr)])} & \\
\end{tabulary}\par
Extract type information from a SCADE model.
   Returns two list of variable-type pairs, one for the input variables, one for the outputs.
\par

\end{haddockdesc}
\begin{haddockdesc}
\item[\begin{tabular}{@{}l}
buildTest
\end{tabular}]\haddockbegindoc
\haddockbeginargs
\haddockdecltt{::} & \haddockdecltt{String} & Name of the SCADE node
 \\
                                              \haddockdecltt{->} & \haddockdecltt{[VarDecl]} & Input variables of the node
 \\
                                                                                               \haddockdecltt{->} & \haddockdecltt{[VarDecl]} & Output variables of the node
 \\
                                                                                                                                                \haddockdecltt{->} & \haddockdecltt{Declaration} & \\
\end{tabulary}\par
Constructs a SCADE node that connects the testnode with the actual implementation SCADE node.
\par

\end{haddockdesc}
\begin{haddockdesc}
\item[\begin{tabular}{@{}l}
buchiToScade
\end{tabular}]\haddockbegindoc
\haddockbeginargs
\haddockdecltt{::} & \haddockdecltt{String} & Name of the resulting SCADE node
 \\
                                              \haddockdecltt{->} & \haddockdecltt{Map String TypeExpr} & Input variables
 \\
                                                                                                         \haddockdecltt{->} & \haddockdecltt{Map String TypeExpr} & Output variables
 \\
                                                                                                                                                                    \haddockdecltt{->} & \haddockdecltt{Buchi (Set (GTLAtom String))} & The buchi automaton
 \\
                                                                                                                                                                                                                                        \haddockdecltt{->} & \haddockdecltt{Declaration} & \\
\end{tabulary}\par
Convert a buchi automaton to SCADE.
\par

\end{haddockdesc}
\begin{haddockdesc}
\item[\begin{tabular}{@{}l}
startState\ ::\ Buchi\ (Set\ (GTLAtom\ String))\ ->\ State
\end{tabular}]\haddockbegindoc
The starting state for a contract automaton.
\par

\end{haddockdesc}
\begin{haddockdesc}
\item[\begin{tabular}{@{}l}
failTransition\ ::\ Transition
\end{tabular}]\haddockbegindoc
Constructs a transition into the \haddockid{failState}.
\par

\end{haddockdesc}
\begin{haddockdesc}
\item[\begin{tabular}{@{}l}
failState\ ::\ State
\end{tabular}]\haddockbegindoc
The state which is entered when a contract is violated.
   There is no transition out of this state.
\par

\end{haddockdesc}
\begin{haddockdesc}
\item[\begin{tabular}{@{}l}
buchiToStates\ ::\ Buchi\ (Set\ (GTLAtom\ String))\ ->\ {\char 91}State{\char 93}
\end{tabular}]\haddockbegindoc
Translates a buchi automaton into a list of SCADE automaton states.
\par

\end{haddockdesc}
\begin{haddockdesc}
\item[\begin{tabular}{@{}l}
stateToTransition\ ::\ Integer\\\ \ \ \ \ \ \ \ \ \ \ \ \ \ \ \ \ \ \ \ \ ->\ BuchiState\ st\ (Set\ (GTLAtom\ String))\ f\ ->\ Transition
\end{tabular}]\haddockbegindoc
Given a state this function creates a transition into the state.
\par

\end{haddockdesc}
\begin{haddockdesc}
\item[
litToExpr\ ::\ Integral\ a\ =>\ Expr\ String\ a\ ->\ Expr
]
\item[
relToExpr\ ::\ GTLAtom\ String\ ->\ Expr
]
\item[
relsToExpr\ ::\ {\char 91}GTLAtom\ String{\char 93}\ ->\ Expr
]
\end{haddockdesc}
\haddockmoduleheading{Language.GTL.ErrorRefiner}
\label{module:Language.GTL.ErrorRefiner}
\haddockbeginheader
{\haddockverb\begin{verbatim}
module Language.GTL.ErrorRefiner (
    Trace,  CNameGen,  parseTrace,  filterTraces,  writeTraces,  readBDDTraces, 
    atomToC,  relToC,  exprToC,  intOpToC,  traceToPromela,  traceElemToC, 
    traceToBuchi
  ) where\end{verbatim}}
\haddockendheader

This module helps with the generation, storing, analyzing and processing of
    trace files.
\par

\begin{haddockdesc}
\item[\begin{tabular}{@{}l}
type\ Trace\ =\ {\char 91}{\char 91}GTLAtom\ (String,\ String){\char 93}{\char 93}
\end{tabular}]\haddockbegindoc
A trace is a list of requirements.
   Each requirement corresponds to a step in the model.
   Each requirement is a list of atoms that have to be true in the corresponding step.
\par

\end{haddockdesc}
\begin{haddockdesc}
\item[\begin{tabular}{@{}l}
type\ CNameGen\ =\ String\ ->\ String\ ->\ Integer\ ->\ String
\end{tabular}]\haddockbegindoc
Converts GTL variables to C-names.
   Takes the component name, the variable name and the history level of the variable.
\par

\end{haddockdesc}
\begin{haddockdesc}
\item[\begin{tabular}{@{}l}
parseTrace
\end{tabular}]\haddockbegindoc
\haddockbeginargs
\haddockdecltt{::} & \haddockdecltt{FilePath} & The promela file of the model
 \\
                                                \haddockdecltt{->} & \haddockdecltt{FilePath} & The path to the promela trail file
 \\
                                                                                                \haddockdecltt{->} & \haddockdecltt{IO [(String, Integer)]} & \\
\end{tabulary}\par
Parse a SPIN trace file by calling it with the spin interpreter and parsing the output.
   Produces a list of tuples where the first component is the name of the component that
   just performed a step and the second one is the state number that it transitioned into.
\par

\end{haddockdesc}
\begin{haddockdesc}
\item[\begin{tabular}{@{}l}
filterTraces\ ::\ String\ ->\ {\char 91}String{\char 93}
\end{tabular}]\haddockbegindoc
Given the output of a spin verifier, extract the filenames of traces.
\par

\end{haddockdesc}
\begin{haddockdesc}
\item[\begin{tabular}{@{}l}
writeTraces\ ::\ FilePath\ ->\ {\char 91}Trace{\char 93}\ ->\ IO\ ()
\end{tabular}]\haddockbegindoc
Write a list of traces into a file.
\par

\end{haddockdesc}
\begin{haddockdesc}
\item[\begin{tabular}{@{}l}
readBDDTraces\ ::\ FilePath\ ->\ IO\ {\char 91}Trace{\char 93}
\end{tabular}]\haddockbegindoc
Read a list of traces from a file.
\par

\end{haddockdesc}
\begin{haddockdesc}
\item[\begin{tabular}{@{}l}
atomToC
\end{tabular}]\haddockbegindoc
\haddockbeginargs
\haddockdecltt{::} & \haddockdecltt{CNameGen} & Function to generate C-names
 \\
                                                \haddockdecltt{->} & \haddockdecltt{GTLAtom (String, String)} & GTL atom to convert
 \\
                                                                                                                \haddockdecltt{->} & \haddockdecltt{String} & \\
\end{tabulary}\par
Given a function to generate names, this function converts a GTL atom into a C-expression.
\par

\end{haddockdesc}
\begin{haddockdesc}
\item[\begin{tabular}{@{}l}
relToC\ ::\ Relation\ ->\ String
\end{tabular}]\haddockbegindoc
Convert a GTL relation to a C operator
\par

\end{haddockdesc}
\begin{haddockdesc}
\item[\begin{tabular}{@{}l}
exprToC\ ::\ CNameGen\ ->\ Expr\ (String,\ String)\ Int\ ->\ String
\end{tabular}]\haddockbegindoc
Convert a GTL expression to a C-expression
\par

\end{haddockdesc}
\begin{haddockdesc}
\item[\begin{tabular}{@{}l}
intOpToC\ ::\ IntOp\ ->\ String
\end{tabular}]\haddockbegindoc
Convert a GTL integer operator to a C-operator
\par

\end{haddockdesc}
\begin{haddockdesc}
\item[\begin{tabular}{@{}l}
traceToPromela\ ::\ CNameGen\ ->\ Trace\ ->\ {\char 91}Step{\char 93}
\end{tabular}]\haddockbegindoc
Convert a trace into a promela module that checks if everything conforms to the trace.
\par

\end{haddockdesc}
\begin{haddockdesc}
\item[\begin{tabular}{@{}l}
traceElemToC\ ::\ CNameGen\ ->\ {\char 91}GTLAtom\ (String,\ String){\char 93}\ ->\ String
\end{tabular}]\haddockbegindoc
Convert a element from a trace into a C-expression.
\par

\end{haddockdesc}
\begin{haddockdesc}
\item[\begin{tabular}{@{}l}
traceToBuchi\ ::\ CNameGen\ ->\ Trace\ ->\ Buchi\ (Maybe\ String)
\end{tabular}]\haddockbegindoc
Convert a trace into a buchi automaton that checks for conformance to that trace.
\par

\end{haddockdesc}
\haddockmoduleheading{Language.GTL.LTL}
\label{module:Language.GTL.LTL}
\haddockbeginheader
{\haddockverb\begin{verbatim}
module Language.GTL.LTL (
    LTL(Atom, Bin, Un, Ground),  BinOp(And, Or, Until, UntilOp), 
    UnOp(Not, Next),  GBuchi,  Buchi, 
    BuchiState(BuchiState, isStart, vars, finalSets, successors),  ltlToBuchi, 
    ltlToBuchiM,  translateGBA,  buchiProduct
  ) where\end{verbatim}}
\haddockendheader

Implements Linear Time Logic and its translation into Buchi-Automaton.
\par

\haddocksection{Formulas
}
\begin{haddockdesc}
\item[\begin{tabular}{@{}l}
data\ LTL\ a
\end{tabular}]\haddockbegindoc
\haddockbeginconstrs
\haddockdecltt{=} & \haddockdecltt{Atom a} & \\
\haddockdecltt{|} & \haddockdecltt{Bin BinOp (LTL a) (LTL a)} & \\
\haddockdecltt{|} & \haddockdecltt{Un UnOp (LTL a)} & \\
\haddockdecltt{|} & \haddockdecltt{Ground Bool} & \\
\end{tabulary}\par
A LTL formula with atoms of type \emph{a}.
\par

\end{haddockdesc}
\begin{haddockdesc}
\item[\begin{tabular}{@{}l}
instance\ Eq\ a\ =>\ Eq\ (LTL\ a)\\instance\ Ord\ a\ =>\ Ord\ (LTL\ a)\\instance\ Show\ a\ =>\ Show\ (LTL\ a)
\end{tabular}]
\end{haddockdesc}
\begin{haddockdesc}
\item[\begin{tabular}{@{}l}
data\ BinOp
\end{tabular}]\haddockbegindoc
\haddockbeginconstrs
\haddockdecltt{=} & \haddockdecltt{And} & \\
\haddockdecltt{|} & \haddockdecltt{Or} & \\
\haddockdecltt{|} & \haddockdecltt{Until} & \\
\haddockdecltt{|} & \haddockdecltt{UntilOp} & \\
\end{tabulary}\par
Minimal set of binary operators for LTL.
\par

\end{haddockdesc}
\begin{haddockdesc}
\item[\begin{tabular}{@{}l}
instance\ Eq\ BinOp\\instance\ Ord\ BinOp\\instance\ Show\ BinOp
\end{tabular}]
\end{haddockdesc}
\begin{haddockdesc}
\item[\begin{tabular}{@{}l}
data\ UnOp
\end{tabular}]\haddockbegindoc
\haddockbeginconstrs
\haddockdecltt{=} & \haddockdecltt{Not} & \\
\haddockdecltt{|} & \haddockdecltt{Next} & \\
\end{tabulary}\par
Unary operators for LTL.
\par

\end{haddockdesc}
\begin{haddockdesc}
\item[\begin{tabular}{@{}l}
instance\ Eq\ UnOp\\instance\ Ord\ UnOp\\instance\ Show\ UnOp
\end{tabular}]
\end{haddockdesc}
\haddocksection{Buchi translation
}
\begin{haddockdesc}
\item[\begin{tabular}{@{}l}
type\ GBuchi\ st\ a\ f\ =\ Map\ st\ (BuchiState\ st\ a\ f)
\end{tabular}]\haddockbegindoc
A buchi automaton parametrized over the state identifier \emph{st}, the variable type \emph{a} and the final set type \emph{f}
\par

\end{haddockdesc}
\begin{haddockdesc}
\item[\begin{tabular}{@{}l}
type\ Buchi\ a\ =\ GBuchi\ Integer\ a\ (Set\ Integer)
\end{tabular}]\haddockbegindoc
A simple generalized buchi automaton.
\par

\end{haddockdesc}
\begin{haddockdesc}
\item[\begin{tabular}{@{}l}
data\ BuchiState\ st\ a\ f
\end{tabular}]\haddockbegindoc
\haddockbeginconstrs
\haddockdecltt{=} & \haddockdecltt{BuchiState} & \\
                    \{ & \haddockdecltt{isStart :: Bool} & Is the state an initial state?
 \\
                    , & \haddockdecltt{vars :: a} & The variables that must be true in this state.
 \\
                    , & \haddockdecltt{finalSets :: f} & In which final sets is this state a member?
 \\
                    , & \haddockdecltt{successors :: Set st} & All following states
 \\
                    \} &
\end{tabulary}\par
A state representation of a buchi automaton.
\par

\end{haddockdesc}
\begin{haddockdesc}
\item[\begin{tabular}{@{}l}
instance\ (Show\ st,\ Show\ a,\ Show\ f)\ =>\ Show\ (BuchiState\ st\ a\ f)
\end{tabular}]
\end{haddockdesc}
\begin{haddockdesc}
\item[\begin{tabular}{@{}l}
ltlToBuchi\ ::\ (Ord\ a,\ Show\ a)\ =>\ LTL\ a\ ->\ Buchi\ (Map\ a\ Bool)
\end{tabular}]\haddockbegindoc
Converts a LTL formula to a generalized buchi automaton.
\par

\end{haddockdesc}
\begin{haddockdesc}
\item[\begin{tabular}{@{}l}
ltlToBuchiM\ ::\ (Ord\ a,\ Monad\ m,\ Show\ a)\ =>\ ({\char 91}(a,\ Bool){\char 93}\ ->\ m\ b)\\\ \ \ \ \ \ \ \ \ \ \ \ \ \ \ \ \ \ \ \ \ \ \ \ \ \ \ \ \ \ \ \ \ \ \ \ \ \ \ \ \ \ \ ->\ LTL\ a\ ->\ m\ (Buchi\ b)
\end{tabular}]\haddockbegindoc
Same as \haddockid{ltlToBuchi} but also allows the user to construct the variable type and runs in a monad.
\par

\end{haddockdesc}
\begin{haddockdesc}
\item[\begin{tabular}{@{}l}
translateGBA\ ::\ (Ord\ st,\ Ord\ f)\ =>\ GBuchi\ st\ a\ (Set\ f)\\\ \ \ \ \ \ \ \ \ \ \ \ \ \ \ \ \ \ \ \ \ \ \ \ \ \ \ \ \ \ \ \ \ \ \ ->\ GBuchi\ (st,\ Int)\ a\ Bool
\end{tabular}]\haddockbegindoc
Transforms a generalized buchi automaton into a regular one.
\par

\end{haddockdesc}
\begin{haddockdesc}
\item[\begin{tabular}{@{}l}
buchiProduct
\end{tabular}]\haddockbegindoc
\haddockbeginargs
\haddockdecltt{::} & \haddockdecltt{(Ord st1, Ord f1, Ord st2, Ord f2)} \\
                     \haddockdecltt{=>} & \haddockdecltt{GBuchi st1 a (Set f1)} & First buchi automaton
 \\
                                                                                  \haddockdecltt{->} & \haddockdecltt{GBuchi st2 b (Set f2)} & Second buchi automaton
 \\
                                                                                                                                               \haddockdecltt{->} & \haddockdecltt{GBuchi (st1, st2) (a, b) (Set (Either f1 f2))} & \\
\end{tabulary}\par
Calculates the product automaton of two given buchi automatons.
\par

\end{haddockdesc}
\haddockmoduleheading{Language.GTL.Model}
\label{module:Language.GTL.Model}
\haddockbeginheader
{\haddockverb\begin{verbatim}
module Language.GTL.Model (
    GTLModel(GTLModel,
             gtlModelContract,
             gtlModelBackend,
             gtlModelInput,
             gtlModelOutput,
             gtlModelDefaults), 
    GTLSpec(GTLSpec, gtlSpecModels, gtlSpecVerify, gtlSpecConnections), 
    gtlParseModel,  gtlParseSpec
  ) where\end{verbatim}}
\haddockendheader

\begin{haddockdesc}
\item[\begin{tabular}{@{}l}
data\ GTLModel
\end{tabular}]\haddockbegindoc
\haddockbeginconstrs
\haddockdecltt{=} & \haddockdecltt{GTLModel} & \\
                    \{ & \haddockdecltt{gtlModelContract :: Expr String Bool} & \\
                    , & \haddockdecltt{gtlModelBackend :: AllBackend} & \\
                    , & \haddockdecltt{gtlModelInput :: Map String TypeRep} & \\
                    , & \haddockdecltt{gtlModelOutput :: Map String TypeRep} & \\
                    , & \haddockdecltt{gtlModelDefaults :: Map String (Maybe Dynamic)} & \\
                    \} &
\end{tabulary}\par
\end{haddockdesc}
\begin{haddockdesc}
\item[\begin{tabular}{@{}l}
data\ GTLSpec
\end{tabular}]\haddockbegindoc
\haddockbeginconstrs
\haddockdecltt{=} & \haddockdecltt{GTLSpec} & \\
                    \{ & \haddockdecltt{gtlSpecModels :: Map String GTLModel} & \\
                    , & \haddockdecltt{gtlSpecVerify :: Expr (String, String) Bool} & \\
                    , & \haddockdecltt{gtlSpecConnections :: [(String, String, String, String)]} & \\
                    \} &
\end{tabulary}\par
\end{haddockdesc}
\begin{haddockdesc}
\item[
gtlParseModel\ ::\ ModelDecl\ ->\ IO\ (Either\ String\ (String,\ GTLModel))
]
\item[
gtlParseSpec\ ::\ {\char 91}Declaration{\char 93}\ ->\ IO\ (Either\ String\ GTLSpec)
]
\end{haddockdesc}
\haddockmoduleheading{Language.GTL.Parser}
\label{module:Language.GTL.Parser}
\haddockbeginheader
{\haddockverb\begin{verbatim}
module Language.GTL.Parser (
    gtl
  ) where\end{verbatim}}
\haddockendheader

Implements a parser for the GTL specification language.
\par

\begin{haddockdesc}
\item[
gtl\ ::\ {\char 91}Token{\char 93}\ ->\ {\char 91}Declaration{\char 93}
]
\end{haddockdesc}
\haddockmoduleheading{Language.GTL.Parser.Lexer}
\label{module:Language.GTL.Parser.Lexer}
\haddockbeginheader
{\haddockverb\begin{verbatim}
module Language.GTL.Parser.Lexer (
    lexGTL
  ) where\end{verbatim}}
\haddockendheader

The GTL Lexer  
\par

\begin{haddockdesc}
\item[\begin{tabular}{@{}l}
lexGTL\ ::\ String\ ->\ {\char 91}Token{\char 93}
\end{tabular}]\haddockbegindoc
Convert GTL code lazily into a list of tokens.
\par

\end{haddockdesc}
\haddockmoduleheading{Language.GTL.Parser.Syntax}
\label{module:Language.GTL.Parser.Syntax}
\haddockbeginheader
{\haddockverb\begin{verbatim}
module Language.GTL.Parser.Syntax (
    Declaration(Model, Connect, Verify), 
    ModelDecl(ModelDecl,
              modelName,
              modelType,
              modelArgs,
              modelContract,
              modelInits,
              modelInputs,
              modelOutputs), 
    ConnectDecl(ConnectDecl,
                connectFromModel,
                connectFromVariable,
                connectToModel,
                connectToVariable), 
    VerifyDecl(VerifyDecl, verifyFormulas), 
    GExpr(GBin, GUn, GConst, GConstBool, GVar, GSet, GExists), 
    InitExpr(InitAll, InitOne)
  ) where\end{verbatim}}
\haddockendheader

Data types representing a parsed GTL file.
\par

\begin{haddockdesc}
\item[\begin{tabular}{@{}l}
data\ Declaration
\end{tabular}]\haddockbegindoc
\haddockbeginconstrs
\haddockdecltt{=} & \haddockdecltt{Model ModelDecl} & Declares a model.
 \\
\haddockdecltt{|} & \haddockdecltt{Connect ConnectDecl} & Declares a connection between two models.
 \\
\haddockdecltt{|} & \haddockdecltt{Verify VerifyDecl} & Declares a property that needs to be verified.
 \\
\end{tabulary}\par
A GTL file is a list of declarations.
\par

\end{haddockdesc}
\begin{haddockdesc}
\item[\begin{tabular}{@{}l}
instance\ Show\ Declaration
\end{tabular}]
\end{haddockdesc}
\begin{haddockdesc}
\item[\begin{tabular}{@{}l}
data\ ModelDecl
\end{tabular}]\haddockbegindoc
\haddockbeginconstrs
\haddockdecltt{=} & \haddockdecltt{ModelDecl} & \\
                    \{ & \haddockdecltt{modelName :: String} & The name of the model in the GTL formalism.
 \\
                    , & \haddockdecltt{modelType :: String} & The synchronous formalism the model is written in (for example \emph{scade})
 \\
                    , & \haddockdecltt{modelArgs :: [String]} & Arguments specific to the synchronous formalism, for example in which file the model is specified etc.
 \\
                    , & \haddockdecltt{modelContract :: [GExpr]} & A list of contracts that this model fulfills.
 \\
                    , & \haddockdecltt{modelInits :: [(String, InitExpr)]} & A list of initializations for the variables of the model.
 \\
                    , & \haddockdecltt{modelInputs :: Map String String} & Declared inputs of the model with their corresponding type
 \\
                    , & \haddockdecltt{modelOutputs :: Map String String} & Declared outputs of a model
 \\
                    \} &
\end{tabulary}\par
Declares a synchronous model.
\par

\end{haddockdesc}
\begin{haddockdesc}
\item[\begin{tabular}{@{}l}
instance\ Show\ ModelDecl
\end{tabular}]
\end{haddockdesc}
\begin{haddockdesc}
\item[\begin{tabular}{@{}l}
data\ ConnectDecl
\end{tabular}]\haddockbegindoc
\haddockbeginconstrs
\haddockdecltt{=} & \haddockdecltt{ConnectDecl} & \\
                    \{ & \haddockdecltt{connectFromModel :: String} & Model of the source variable
 \\
                    , & \haddockdecltt{connectFromVariable :: String} & Name of the source variable
 \\
                    , & \haddockdecltt{connectToModel :: String} & Model of the target variable
 \\
                    , & \haddockdecltt{connectToVariable :: String} & Name of the target variable
 \\
                    \} &
\end{tabulary}\par
Declares a connection between two variables
\par

\end{haddockdesc}
\begin{haddockdesc}
\item[\begin{tabular}{@{}l}
instance\ Show\ ConnectDecl
\end{tabular}]
\end{haddockdesc}
\begin{haddockdesc}
\item[\begin{tabular}{@{}l}
data\ VerifyDecl
\end{tabular}]\haddockbegindoc
\haddockbeginconstrs
\haddockdecltt{=} & \haddockdecltt{VerifyDecl} & \\
                    \{ & \haddockdecltt{verifyFormulas :: [GExpr]} & The formulas to be verified.
 \\
                    \} &
\end{tabulary}\par
A list of formulas to verify.
\par

\end{haddockdesc}
\begin{haddockdesc}
\item[\begin{tabular}{@{}l}
instance\ Show\ VerifyDecl
\end{tabular}]
\end{haddockdesc}
\begin{haddockdesc}
\item[\begin{tabular}{@{}l}
data\ GExpr
\end{tabular}]\haddockbegindoc
\haddockbeginconstrs
\haddockdecltt{=} & \haddockdecltt{GBin BinOp GExpr GExpr} & \\
\haddockdecltt{|} & \haddockdecltt{GUn UnOp GExpr} & \\
\haddockdecltt{|} & \haddockdecltt{GConst Int} & \\
\haddockdecltt{|} & \haddockdecltt{GConstBool Bool} & \\
\haddockdecltt{|} & \haddockdecltt{GVar (Maybe String) String} & \\
\haddockdecltt{|} & \haddockdecltt{GSet [Integer]} & \\
\haddockdecltt{|} & \haddockdecltt{GExists String (Maybe String) String GExpr} & \\
\end{tabulary}\par
An untyped expression type.
   Used internally in the parser.
\par

\end{haddockdesc}
\begin{haddockdesc}
\item[\begin{tabular}{@{}l}
instance\ Eq\ GExpr\\instance\ Ord\ GExpr\\instance\ Show\ GExpr
\end{tabular}]
\end{haddockdesc}
\begin{haddockdesc}
\item[\begin{tabular}{@{}l}
data\ InitExpr
\end{tabular}]\haddockbegindoc
\haddockbeginconstrs
\haddockdecltt{=} & \haddockdecltt{InitAll} & The variable is initialized with all possible values.
 \\
\haddockdecltt{|} & \haddockdecltt{InitOne Integer} & The variable is initialized with a specific value.
 \\
\end{tabulary}\par
Information about the initialization of a variable.
\par

\end{haddockdesc}
\begin{haddockdesc}
\item[\begin{tabular}{@{}l}
instance\ Eq\ InitExpr\\instance\ Show\ InitExpr
\end{tabular}]
\end{haddockdesc}
\haddockmoduleheading{Language.GTL.Parser.Token}
\label{module:Language.GTL.Parser.Token}
\haddockbeginheader
{\haddockverb\begin{verbatim}
module Language.GTL.Parser.Token (
    Token(Identifier,
          Key,
          Bracket,
          Dot,
          Semicolon,
          Colon,
          Comma,
          ConstString,
          ConstInt,
          Unary,
          Binary), 
    KeyWord(KeyAll,
            KeyConnect,
            KeyContract,
            KeyModel,
            KeyOutput,
            KeyInit,
            KeyInput,
            KeyVerify,
            KeyExists), 
    BracketType(Parentheses, Square, Curly), 
    UnOp(GOpAlways, GOpNext, GOpNot, GOpFinally), 
    BinOp(GOpAnd,
          GOpOr,
          GOpImplies,
          GOpIn,
          GOpNotIn,
          GOpLessThan,
          GOpLessThanEqual,
          GOpGreaterThan,
          GOpGreaterThanEqual,
          GOpEqual,
          GOpNEqual,
          GOpPlus,
          GOpMinus,
          GOpMult,
          GOpDiv)
  ) where\end{verbatim}}
\haddockendheader

\begin{haddockdesc}
\item[\begin{tabular}{@{}l}
data\ Token
\end{tabular}]\haddockbegindoc
\haddockbeginconstrs
\haddockdecltt{=} & \haddockdecltt{Identifier String} & \\
\haddockdecltt{|} & \haddockdecltt{Key KeyWord} & \\
\haddockdecltt{|} & \haddockdecltt{Bracket BracketType Bool} & \\
\haddockdecltt{|} & \haddockdecltt{Dot} & \\
\haddockdecltt{|} & \haddockdecltt{Semicolon} & \\
\haddockdecltt{|} & \haddockdecltt{Colon} & \\
\haddockdecltt{|} & \haddockdecltt{Comma} & \\
\haddockdecltt{|} & \haddockdecltt{ConstString String} & \\
\haddockdecltt{|} & \haddockdecltt{ConstInt Integer} & \\
\haddockdecltt{|} & \haddockdecltt{Unary UnOp} & \\
\haddockdecltt{|} & \haddockdecltt{Binary BinOp} & \\
\end{tabulary}\par
\end{haddockdesc}
\begin{haddockdesc}
\item[\begin{tabular}{@{}l}
instance\ Show\ Token
\end{tabular}]
\end{haddockdesc}
\begin{haddockdesc}
\item[\begin{tabular}{@{}l}
data\ KeyWord
\end{tabular}]\haddockbegindoc
\haddockbeginconstrs
\haddockdecltt{=} & \haddockdecltt{KeyAll} & \\
\haddockdecltt{|} & \haddockdecltt{KeyConnect} & \\
\haddockdecltt{|} & \haddockdecltt{KeyContract} & \\
\haddockdecltt{|} & \haddockdecltt{KeyModel} & \\
\haddockdecltt{|} & \haddockdecltt{KeyOutput} & \\
\haddockdecltt{|} & \haddockdecltt{KeyInit} & \\
\haddockdecltt{|} & \haddockdecltt{KeyInput} & \\
\haddockdecltt{|} & \haddockdecltt{KeyVerify} & \\
\haddockdecltt{|} & \haddockdecltt{KeyExists} & \\
\end{tabulary}\par
\end{haddockdesc}
\begin{haddockdesc}
\item[\begin{tabular}{@{}l}
instance\ Show\ KeyWord
\end{tabular}]
\end{haddockdesc}
\begin{haddockdesc}
\item[\begin{tabular}{@{}l}
data\ BracketType
\end{tabular}]\haddockbegindoc
\haddockbeginconstrs
\haddockdecltt{=} & \haddockdecltt{Parentheses} & \\
\haddockdecltt{|} & \haddockdecltt{Square} & \\
\haddockdecltt{|} & \haddockdecltt{Curly} & \\
\end{tabulary}\par
\end{haddockdesc}
\begin{haddockdesc}
\item[\begin{tabular}{@{}l}
instance\ Show\ BracketType
\end{tabular}]
\end{haddockdesc}
\begin{haddockdesc}
\item[\begin{tabular}{@{}l}
data\ UnOp
\end{tabular}]\haddockbegindoc
\haddockbeginconstrs
\haddockdecltt{=} & \haddockdecltt{GOpAlways} & \\
\haddockdecltt{|} & \haddockdecltt{GOpNext} & \\
\haddockdecltt{|} & \haddockdecltt{GOpNot} & \\
\haddockdecltt{|} & \haddockdecltt{GOpFinally (Maybe Integer)} & \\
\end{tabulary}\par
\end{haddockdesc}
\begin{haddockdesc}
\item[\begin{tabular}{@{}l}
instance\ Eq\ UnOp\\instance\ Ord\ UnOp\\instance\ Show\ UnOp
\end{tabular}]
\end{haddockdesc}
\begin{haddockdesc}
\item[\begin{tabular}{@{}l}
data\ BinOp
\end{tabular}]\haddockbegindoc
\haddockbeginconstrs
\haddockdecltt{=} & \haddockdecltt{GOpAnd} & \\
\haddockdecltt{|} & \haddockdecltt{GOpOr} & \\
\haddockdecltt{|} & \haddockdecltt{GOpImplies} & \\
\haddockdecltt{|} & \haddockdecltt{GOpIn} & \\
\haddockdecltt{|} & \haddockdecltt{GOpNotIn} & \\
\haddockdecltt{|} & \haddockdecltt{GOpLessThan} & \\
\haddockdecltt{|} & \haddockdecltt{GOpLessThanEqual} & \\
\haddockdecltt{|} & \haddockdecltt{GOpGreaterThan} & \\
\haddockdecltt{|} & \haddockdecltt{GOpGreaterThanEqual} & \\
\haddockdecltt{|} & \haddockdecltt{GOpEqual} & \\
\haddockdecltt{|} & \haddockdecltt{GOpNEqual} & \\
\haddockdecltt{|} & \haddockdecltt{GOpPlus} & \\
\haddockdecltt{|} & \haddockdecltt{GOpMinus} & \\
\haddockdecltt{|} & \haddockdecltt{GOpMult} & \\
\haddockdecltt{|} & \haddockdecltt{GOpDiv} & \\
\end{tabulary}\par
\end{haddockdesc}
\begin{haddockdesc}
\item[\begin{tabular}{@{}l}
instance\ Eq\ BinOp\\instance\ Ord\ BinOp\\instance\ Show\ BinOp
\end{tabular}]
\end{haddockdesc}
\haddockmoduleheading{Language.GTL.PrettyPrinter}
\label{module:Language.GTL.PrettyPrinter}
\haddockbeginheader
{\haddockverb\begin{verbatim}
module Language.GTL.PrettyPrinter (
    getDotBoundingBox,  buchiToDot,  gtlToTikz,  modelToTikz,  pointToTikz, 
    dotToTikz,  splinePoints,  atomToLatex,  estimateWidth
  ) where\end{verbatim}}
\haddockendheader

This module provides functions to render GTL specifications to Tikz Latex drawing commands.
    It can thus be used to get a pretty image for a GTL file.
\par

\begin{haddockdesc}
\item[\begin{tabular}{@{}l}
getDotBoundingBox\ ::\ DotGraph\ a\ ->\ Rect
\end{tabular}]\haddockbegindoc
Get the bounding box of a preprocessed graph.
\par

\end{haddockdesc}
\begin{haddockdesc}
\item[\begin{tabular}{@{}l}
buchiToDot\ ::\ GBuchi\ Integer\ (Map\ (GTLAtom\ String)\ Bool)\ f\\\ \ \ \ \ \ \ \ \ \ \ \ \ \ ->\ DotGraph\ String
\end{tabular}]\haddockbegindoc
Convert a Buchi automaton into a Dot graph.
\par

\end{haddockdesc}
\begin{haddockdesc}
\item[\begin{tabular}{@{}l}
gtlToTikz\ ::\ GTLSpec\ ->\ IO\ String
\end{tabular}]\haddockbegindoc
Convert a GTL specification to Tikz drawing commands.
   This needs to be IO because it calls graphviz programs to preprocess the picture.
\par

\end{haddockdesc}
\begin{haddockdesc}
\item[\begin{tabular}{@{}l}
modelToTikz\ ::\ GTLModel\ ->\ IO\ (String,\ Double,\ Double)
\end{tabular}]\haddockbegindoc
Convert a single model into Tikz drawing commands.
   Also returns the width and height of the bounding box for the rendered picture.
\par

\end{haddockdesc}
\begin{haddockdesc}
\item[\begin{tabular}{@{}l}
pointToTikz\ ::\ Point\ ->\ String
\end{tabular}]\haddockbegindoc
Helper function to render a graphviz point in Tikz.
\par

\end{haddockdesc}
\begin{haddockdesc}
\item[\begin{tabular}{@{}l}
dotToTikz
\end{tabular}]\haddockbegindoc
\haddockbeginargs
\haddockdecltt{::} & \haddockdecltt{(Show a, Ord a)} \\
                     \haddockdecltt{=>} & \haddockdecltt{Maybe (Map a (Map String TypeRep, Map String TypeRep, String, Double, Double))} & Can provide interfaces for the contained models if needed.
 \\
                                                                                                                                           \haddockdecltt{->} & \haddockdecltt{DotGraph a} & \\
                                                                                                                                                                                             \haddockdecltt{->} & \haddockdecltt{String} & \\
\end{tabulary}\par
Convert a graphviz graph to Tikz drawing commands.
\par

\end{haddockdesc}
\begin{haddockdesc}
\item[\begin{tabular}{@{}l}
splinePoints\ ::\ {\char 91}a{\char 93}\ ->\ {\char 91}(a,\ a,\ a,\ a){\char 93}
\end{tabular}]\haddockbegindoc
Convert a list of points into a spline by grouping them.
\par

\end{haddockdesc}
\begin{haddockdesc}
\item[\begin{tabular}{@{}l}
atomToLatex\ ::\ GTLAtom\ String\ ->\ String
\end{tabular}]\haddockbegindoc
Render a GTL atom to LaTeX.
\par

\end{haddockdesc}
\begin{haddockdesc}
\item[\begin{tabular}{@{}l}
estimateWidth\ ::\ GTLAtom\ String\ ->\ Int
\end{tabular}]\haddockbegindoc
Estimate the visible width of a LaTeX rendering of a GTL atom in characters.
\par

\end{haddockdesc}
\haddockmoduleheading{Language.GTL.PromelaCIntegration}
\label{module:Language.GTL.PromelaCIntegration}
\haddockbeginheader
{\haddockverb\begin{verbatim}
module Language.GTL.PromelaCIntegration (
    translateGTL,  varName,  neverClaim,  generatePromelaCode,  connectionMap
  ) where\end{verbatim}}
\haddockendheader

Verifies a GTL specification by converting the components to C-code and
    simulating all possible runs.
\par

\begin{haddockdesc}
\item[\begin{tabular}{@{}l}
translateGTL
\end{tabular}]\haddockbegindoc
\haddockbeginargs
\haddockdecltt{::} & \haddockdecltt{Maybe FilePath} & An optional path to a trace file
 \\
                                                      \haddockdecltt{->} & \haddockdecltt{[Declaration]} & The GTL code
 \\
                                                                                                           \haddockdecltt{->} & \haddockdecltt{[Declaration]} & The SCADE code for the components
 \\
                                                                                                                                                                \haddockdecltt{->} & \haddockdecltt{IO String} & \\
\end{tabulary}\par
Compile a GTL declaration into a promela module simulating the specified model.
   Optionally takes a trace that is used to restrict the execution.
   Outputs promela code.
\par

\end{haddockdesc}
\begin{haddockdesc}
\item[\begin{tabular}{@{}l}
varName
\end{tabular}]\haddockbegindoc
\haddockbeginargs
\haddockdecltt{::} & \haddockdecltt{String} & The component name
 \\
                                              \haddockdecltt{->} & \haddockdecltt{String} & The variable name
 \\
                                                                                            \haddockdecltt{->} & \haddockdecltt{Integer} & The history level of the variable
 \\
                                                                                                                                           \haddockdecltt{->} & \haddockdecltt{String} & \\
\end{tabulary}\par
Convert a GTL name into a C-name.
\par

\end{haddockdesc}
\begin{haddockdesc}
\item[\begin{tabular}{@{}l}
neverClaim
\end{tabular}]\haddockbegindoc
\haddockbeginargs
\haddockdecltt{::} & \haddockdecltt{Trace} & The trace
 \\
                                             \haddockdecltt{->} & \haddockdecltt{Expr (String, String) Bool} & The verify expression
 \\
                                                                                                               \haddockdecltt{->} & \haddockdecltt{Module} & \\
\end{tabulary}\par
Convert a trace and a verify expression into a promela never claim.
   If you don't want to include a trace, give an empty one `{\char 91}{\char 93}'.
\par

\end{haddockdesc}
\begin{haddockdesc}
\item[\begin{tabular}{@{}l}
generatePromelaCode
\end{tabular}]\haddockbegindoc
\haddockbeginargs
\haddockdecltt{::} & \haddockdecltt{TypeMap} & Contains type informations extracted from the SCADE specification
 \\
                                               \haddockdecltt{->} & \haddockdecltt{[((String, String), (String, String))]} & A list of connections between variables
 \\
                                                                                                                             \haddockdecltt{->} & \haddockdecltt{Map (String, String) Integer} & A mapping that gives the maximum history level for each variable involved
 \\
                                                                                                                                                                                                 \haddockdecltt{->} & \haddockdecltt{[Module]} & \\
\end{tabulary}\par
Create promela processes for each component in a GTL specification.
\par

\end{haddockdesc}
\begin{haddockdesc}
\item[\begin{tabular}{@{}l}
connectionMap
\end{tabular}]\haddockbegindoc
\haddockbeginargs
\haddockdecltt{::} & \haddockdecltt{[Declaration]} & GTL specification
 \\
                                                     \haddockdecltt{->} & \haddockdecltt{TypeMap} & Type informations
 \\
                                                                                                    \haddockdecltt{->} & \haddockdecltt{[((String, String), (String, String))]} & \\
\end{tabulary}\par
Get a list of connections between variables for a given GTL specification.
   Also type-checks the connections.
\par

\end{haddockdesc}
\haddockmoduleheading{Language.GTL.PromelaDynamicBDD}
\label{module:Language.GTL.PromelaDynamicBDD}
\haddockbeginheader
{\haddockverb\begin{verbatim}
module Language.GTL.PromelaDynamicBDD (
    TransModel(TransModel,
               varsInit,
               varsIn,
               varsOut,
               stateMachine,
               checkFunctions), 
    TransProgram(TransProgram, transModels, transClaims, claimChecks), 
    deleteTmp,  verifyModel,  traceToAtoms,  varName,  translateContracts, 
    translateModel,  translateNever,  AtomCache,  OutputMapping,  parseGTLAtom, 
    parseGTLRelation,  createBDDAssign,  createBDDCompare,  BDDConst(bddConst), 
    createBDDExpr,  buildTransProgram
  ) where\end{verbatim}}
\haddockendheader

Implements a verification mechanism that abstracts components by using their
    contract to build a state machine that acts on BDD.
\par

\begin{haddockdesc}
\item[\begin{tabular}{@{}l}
data\ TransModel
\end{tabular}]\haddockbegindoc
\haddockbeginconstrs
\haddockdecltt{=} & \haddockdecltt{TransModel} & \\
                    \{ & \haddockdecltt{varsInit :: Map String String} & \\
                    , & \haddockdecltt{varsIn :: Map String Integer} & \\
                    , & \haddockdecltt{varsOut :: Map String (Map (Maybe (String, String)) (Set Integer))} & \\
                    , & \haddockdecltt{stateMachine :: Buchi ([Integer], [Integer], [GTLAtom String])} & \\
                    , & \haddockdecltt{checkFunctions :: [String]} & \\
                    \} &
\end{tabulary}\par
An internal representation of a translated GTL model.
\par

\end{haddockdesc}
\begin{haddockdesc}
\item[\begin{tabular}{@{}l}
instance\ Show\ TransModel
\end{tabular}]
\end{haddockdesc}
\begin{haddockdesc}
\item[\begin{tabular}{@{}l}
data\ TransProgram
\end{tabular}]\haddockbegindoc
\haddockbeginconstrs
\haddockdecltt{=} & \haddockdecltt{TransProgram} & \\
                    \{ & \haddockdecltt{transModels :: Map String TransModel} & \\
                    , & \haddockdecltt{transClaims :: [Buchi [Integer]]} & \\
                    , & \haddockdecltt{claimChecks :: [String]} & \\
                    \} &
\end{tabulary}\par
An internal representation of a translated GTL program.
\par

\end{haddockdesc}
\begin{haddockdesc}
\item[\begin{tabular}{@{}l}
instance\ Show\ TransProgram
\end{tabular}]
\end{haddockdesc}
\begin{haddockdesc}
\item[\begin{tabular}{@{}l}
deleteTmp\ ::\ FilePath\ ->\ IO\ ()
\end{tabular}]\haddockbegindoc
Helper function to securely delete temporary files.
   Deletes a file if it exists, if not, ignore it.
\par

\end{haddockdesc}
\begin{haddockdesc}
\item[\begin{tabular}{@{}l}
verifyModel
\end{tabular}]\haddockbegindoc
\haddockbeginargs
\haddockdecltt{::} & \haddockdecltt{Bool} & Keep temporary files generated by spin, gcc, etc.
 \\
                                            \haddockdecltt{->} & \haddockdecltt{String} & Name of the GTL file without extension
 \\
                                                                                          \haddockdecltt{->} & \haddockdecltt{[Declaration]} & The SCADE source code
 \\
                                                                                                                                               \haddockdecltt{->} & \haddockdecltt{[Declaration]} & The GTL file contents
 \\
                                                                                                                                                                                                    \haddockdecltt{->} & \haddockdecltt{IO ()} & \\
\end{tabulary}\par
Do a complete verification of a given GTL file
\par

\end{haddockdesc}
\begin{haddockdesc}
\item[\begin{tabular}{@{}l}
traceToAtoms
\end{tabular}]\haddockbegindoc
\haddockbeginargs
\haddockdecltt{::} & \haddockdecltt{TransProgram} & The program to work on
 \\
                                                    \haddockdecltt{->} & \haddockdecltt{[(String, Integer)]} & The transitions, given in the form (model,transition-number)
 \\
                                                                                                               \haddockdecltt{->} & \haddockdecltt{Trace} & \\
\end{tabulary}\par
Given a list of transitions, give a list of atoms that have to hold for each transition.
\par

\end{haddockdesc}
\begin{haddockdesc}
\item[\begin{tabular}{@{}l}
varName\ ::\ Bool\ ->\ String\ ->\ String\ ->\ Integer\ ->\ String
\end{tabular}]\haddockbegindoc
Helper function to convert the name of a GTL variable into the
   translated C-representation.
\par

\end{haddockdesc}
\begin{haddockdesc}
\item[\begin{tabular}{@{}l}
translateContracts\ ::\ TransProgram\ ->\ {\char 91}Module{\char 93}
\end{tabular}]\haddockbegindoc
Convert a translated GTL program into a PROMELA module.
\par

\end{haddockdesc}
\begin{haddockdesc}
\item[\begin{tabular}{@{}l}
translateModel
\end{tabular}]\haddockbegindoc
\haddockbeginargs
\haddockdecltt{::} & \haddockdecltt{String} & The name of the model
 \\
                                              \haddockdecltt{->} & \haddockdecltt{TransModel} & The actual model
 \\
                                                                                                \haddockdecltt{->} & \haddockdecltt{[Step]} & \\
\end{tabulary}\par
Convert a translated GTL model into a PROMELA process body.
\par

\end{haddockdesc}
\begin{haddockdesc}
\item[\begin{tabular}{@{}l}
translateNever\ ::\ Buchi\ {\char 91}Integer{\char 93}\ ->\ {\char 91}Step{\char 93}
\end{tabular}]\haddockbegindoc
Translate a buchi automaton representing a verify expression into a never claim.
\par

\end{haddockdesc}
\begin{haddockdesc}
\item[\begin{tabular}{@{}l}
type\ AtomCache\ =\ Map\ (GTLAtom\ (Maybe\ String,\ String))\ (Integer,\ Bool,\ String)
\end{tabular}]\haddockbegindoc
A cache that maps atoms to C-functions that represent them.
   The C-functions are encoded by a unique number, whether they are a test- or
   assignment-function and their source code representation.
\par

\end{haddockdesc}
\begin{haddockdesc}
\item[\begin{tabular}{@{}l}
type\ OutputMapping\ =\ Map\ String\ (Map\ (Maybe\ (String,\ String))\ (Set\ Integer))
\end{tabular}]\haddockbegindoc
A map from component names to output variable informations.
\par

\end{haddockdesc}
\begin{haddockdesc}
\item[\begin{tabular}{@{}l}
parseGTLAtom
\end{tabular}]\haddockbegindoc
\haddockbeginargs
\haddockdecltt{::} & \haddockdecltt{AtomCache} & A cache of already parsed atoms
 \\
                                                 \haddockdecltt{->} & \haddockdecltt{Maybe (String, OutputMapping)} & Informations about the containing component
 \\
                                                                                                                      \haddockdecltt{->} & \haddockdecltt{GTLAtom (Maybe String, String)} & The atom to parse
 \\
                                                                                                                                                                                            \haddockdecltt{->} & \haddockdecltt{((Integer, Bool), AtomCache)} & \\
\end{tabulary}\par
Parse a GTL atom to a C-function.
   Returns the unique number of the function and whether its a test- or assignment-function.
\par

\end{haddockdesc}
\begin{haddockdesc}
\item[\begin{tabular}{@{}l}
parseGTLRelation
\end{tabular}]\haddockbegindoc
\haddockbeginargs
\haddockdecltt{::} & \haddockdecltt{BDDConst a} \\
                     \haddockdecltt{=>} & \haddockdecltt{AtomCache} & A cache of parsed atoms
 \\
                                                                      \haddockdecltt{->} & \haddockdecltt{Maybe (String, OutputMapping)} & Informations about the containing component
 \\
                                                                                                                                           \haddockdecltt{->} & \haddockdecltt{Relation} & The relation type to parse
 \\
                                                                                                                                                                                           \haddockdecltt{->} & \haddockdecltt{Expr (Maybe String, String) a} & Left hand side of the relation
 \\
                                                                                                                                                                                                                                                                \haddockdecltt{->} & \haddockdecltt{Expr (Maybe String, String) a} & Right hand side of the relation
 \\
                                                                                                                                                                                                                                                                                                                                     \haddockdecltt{->} & \haddockdecltt{(Integer, Bool, String)} & \\
\end{tabulary}\par
Parse a GTL relation into a C-Function.
   Returns a unique number for the resulting function, whether its a test- or assignment function and
   its source-code representation.
\par

\end{haddockdesc}
\begin{haddockdesc}
\item[\begin{tabular}{@{}l}
createBDDAssign
\end{tabular}]\haddockbegindoc
\haddockbeginargs
\haddockdecltt{::} & \haddockdecltt{BDDConst a} \\
                     \haddockdecltt{=>} & \haddockdecltt{Integer} & How many temporary variables have been used so far?
 \\
                                                                    \haddockdecltt{->} & \haddockdecltt{String} & The current component name
 \\
                                                                                                                  \haddockdecltt{->} & \haddockdecltt{String} & The name of the target variable
 \\
                                                                                                                                                                \haddockdecltt{->} & \haddockdecltt{Map (Maybe (String, String)) (Set Integer)} & A mapping of output variables
 \\
                                                                                                                                                                                                                                                  \haddockdecltt{->} & \haddockdecltt{Relation} & The relation used to assign the BDD
 \\
                                                                                                                                                                                                                                                                                                  \haddockdecltt{->} & \haddockdecltt{Expr (Maybe String, String) a} & The expression to assign the BDD with
 \\
                                                                                                                                                                                                                                                                                                                                                                       \haddockdecltt{->} & \haddockdecltt{String} & \\
\end{tabulary}\par
Create a BDD assignment
\par

\end{haddockdesc}
\begin{haddockdesc}
\item[\begin{tabular}{@{}l}
createBDDCompare
\end{tabular}]\haddockbegindoc
\haddockbeginargs
\haddockdecltt{::} & \haddockdecltt{BDDConst a} \\
                     \haddockdecltt{=>} & \haddockdecltt{Integer} & How many temporary variables have been used?
 \\
                                                                    \haddockdecltt{->} & \haddockdecltt{Maybe String} & If the comparision is part of a contract, give the name of the component, otherwise \haddockid{Nothing} \\
                                                                                                                        \haddockdecltt{->} & \haddockdecltt{Relation} & The relation used to compare the BDDs
 \\
                                                                                                                                                                        \haddockdecltt{->} & \haddockdecltt{Expr (Maybe String, String) a} & Expression representing BDD 1
 \\
                                                                                                                                                                                                                                             \haddockdecltt{->} & \haddockdecltt{Expr (Maybe String, String) a} & Expression representing BDD 2
 \\
                                                                                                                                                                                                                                                                                                                  \haddockdecltt{->} & \haddockdecltt{String} & \\
\end{tabulary}\par
Create a comparison operation between two BDD.
\par

\end{haddockdesc}
\begin{haddockdesc}
\item[\begin{tabular}{@{}l}
class\ BDDConst\ t\ where
\end{tabular}]\haddockbegindoc
A class of types that have a representation as a BDD.
\par

\haddockpremethods{}\textbf{Methods}
\begin{haddockdesc}
\item[\begin{tabular}{@{}l}
bddConst\ ::\ t\ ->\ String
\end{tabular}]\haddockbegindoc
Convert a value to the BDD C-representation.
\par

\end{haddockdesc}
\end{haddockdesc}
\begin{haddockdesc}
\item[\begin{tabular}{@{}l}
instance\ BDDConst\ Bool\\instance\ BDDConst\ Int
\end{tabular}]
\end{haddockdesc}
\begin{haddockdesc}
\item[\begin{tabular}{@{}l}
createBDDExpr
\end{tabular}]\haddockbegindoc
\haddockbeginargs
\haddockdecltt{::} & \haddockdecltt{BDDConst a} \\
                     \haddockdecltt{=>} & \haddockdecltt{Integer} & The current number of temporary variables
 \\
                                                                    \haddockdecltt{->} & \haddockdecltt{Maybe String} & The current component
 \\
                                                                                                                        \haddockdecltt{->} & \haddockdecltt{Expr (Maybe String, String) a} & The GTL expression
 \\
                                                                                                                                                                                             \haddockdecltt{->} & \haddockdecltt{([String], [String], Integer, String)} & \\
\end{tabulary}\par
Convert a GTL expression into a C-expression.
   Returns a list of statements that have to be executed before the expression,
   one that has to be executed afterwards, a number of temporary variables used
   and the resulting C-expression.
\par

\end{haddockdesc}
\begin{haddockdesc}
\item[\begin{tabular}{@{}l}
buildTransProgram
\end{tabular}]\haddockbegindoc
\haddockbeginargs
\haddockdecltt{::} & \haddockdecltt{[Declaration]} & The SCADE file
 \\
                                                     \haddockdecltt{->} & \haddockdecltt{[Declaration]} & The GTL file
 \\
                                                                                                          \haddockdecltt{->} & \haddockdecltt{TransProgram} & \\
\end{tabulary}\par
Generate a program from a given SCADE- and GTL-file.
\par

\end{haddockdesc}
\haddockmoduleheading{Language.GTL.Translation}
\label{module:Language.GTL.Translation}
\haddockbeginheader
{\haddockverb\begin{verbatim}
module Language.GTL.Translation (
    GTLAtom(GTLRel, GTLElem, GTLVar),  mapGTLVars,  gtlAtomNot,  gtlToBuchi, 
    gtlsToBuchi,  getAtomVars,  gtlToLTL
  ) where\end{verbatim}}
\haddockendheader

Translates GTL expressions into LTL formula.
\par

\begin{haddockdesc}
\item[\begin{tabular}{@{}l}
data\ GTLAtom\ v
\end{tabular}]\haddockbegindoc
\haddockbeginconstrs
\haddockdecltt{=} & \haddockdecltt{GTLRel Relation (Expr v Int) (Expr v Int)} & \\
\haddockdecltt{|} & \haddockdecltt{GTLElem v [Integer] Bool} & \\
\haddockdecltt{|} & \haddockdecltt{GTLVar v Integer Bool} & \\
\end{tabulary}\par
A representation of GTL expressions that can't be further translated into LTL
   and thus have to be used as atoms.
\par

\end{haddockdesc}
\begin{haddockdesc}
\item[\begin{tabular}{@{}l}
instance\ Eq\ v\ =>\ Eq\ (GTLAtom\ v)\\instance\ Ord\ v\ =>\ Ord\ (GTLAtom\ v)\\instance\ Show\ v\ =>\ Show\ (GTLAtom\ v)\\instance\ Binary\ v\ =>\ Binary\ (GTLAtom\ v)
\end{tabular}]
\end{haddockdesc}
\begin{haddockdesc}
\item[\begin{tabular}{@{}l}
mapGTLVars\ ::\ (v\ ->\ w)\ ->\ GTLAtom\ v\ ->\ GTLAtom\ w
\end{tabular}]\haddockbegindoc
Applies a function to every variable in the atom.
\par

\end{haddockdesc}
\begin{haddockdesc}
\item[\begin{tabular}{@{}l}
gtlAtomNot\ ::\ GTLAtom\ v\ ->\ GTLAtom\ v
\end{tabular}]\haddockbegindoc
Negate a GTL atom.
\par

\end{haddockdesc}
\begin{haddockdesc}
\item[\begin{tabular}{@{}l}
gtlToBuchi\ ::\ (Monad\ m,\ Ord\ v,\ Show\ v)\ =>\ ({\char 91}GTLAtom\ v{\char 93}\ ->\ m\ a)\\\ \ \ \ \ \ \ \ \ \ \ \ \ \ \ \ \ \ \ \ \ \ \ \ \ \ \ \ \ \ \ \ \ \ \ \ \ \ \ \ \ \ ->\ Expr\ v\ Bool\ ->\ m\ (Buchi\ a)
\end{tabular}]\haddockbegindoc
Translates a GTL expression into a buchi automaton.
   Needs a user supplied function that converts a list of atoms that have to be
   true into the variable type of the buchi automaton.
\par

\end{haddockdesc}
\begin{haddockdesc}
\item[\begin{tabular}{@{}l}
gtlsToBuchi\ ::\ (Monad\ m,\ Ord\ v,\ Show\ v)\ =>\ ({\char 91}GTLAtom\ v{\char 93}\ ->\ m\ a)\\\ \ \ \ \ \ \ \ \ \ \ \ \ \ \ \ \ \ \ \ \ \ \ \ \ \ \ \ \ \ \ \ \ \ \ \ \ \ \ \ \ \ \ ->\ {\char 91}Expr\ v\ Bool{\char 93}\ ->\ m\ (Buchi\ a)
\end{tabular}]\haddockbegindoc
Like \haddockid{gtlToBuchi} but takes more than one formula.
\par

\end{haddockdesc}
\begin{haddockdesc}
\item[\begin{tabular}{@{}l}
getAtomVars\ ::\ GTLAtom\ v\ ->\ {\char 91}(v,\ Integer){\char 93}
\end{tabular}]\haddockbegindoc
Extract all variables with their history level from an atom.
\par

\end{haddockdesc}
\begin{haddockdesc}
\item[\begin{tabular}{@{}l}
gtlToLTL\ ::\ Expr\ v\ Bool\ ->\ LTL\ (GTLAtom\ v)
\end{tabular}]\haddockbegindoc
Translate a GTL expression into a LTL formula.
\par

\end{haddockdesc}

\end{appendix}
\end{document}
