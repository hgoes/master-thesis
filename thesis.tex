\documentclass[10pt]{scrbook}

\usepackage[bookmarks=true,colorlinks=false]{hyperref} % PDF-Inhaltsverzeichnis und Links

\usepackage[utf8]{inputenc} % UTF-8 Kodierung
\usepackage[ngerman]{babel} % Deutsche Silbentrennung

\usepackage[T1]{fontenc}
\usepackage[lf]{venturis} % Eine schöne Schriftart
\usepackage{graphicx} % For includegraphics
\usepackage[nounderscore]{syntax} % Syntax im BNF-Stil (Niemals das "nounderscore entfernen, das macht alles kaputt!)
\usepackage{amssymb} % Für \mathbb
\usepackage[numbers]{natbib} % Literaturverzeichnis
\usepackage{tikz} % Für Grafiken und Diagramme
\usepackage{listings} % Für Quellcode-Listings (Promela,GTL etc.)
\usepackage{stmaryrd} % Für \llbracket,\rrbracket
\usepackage{semantic} % Strukturelle Operationelle Semantik
\usepackage[all]{xy} % XY-Matrix Umgebung für Diagramme
\usepackage{clrscode3e} % Für Pseudo-Code im Cormen-Stil
\usepackage{tocbibind} % Bindet das Literaturverzeichnis ins Inhaltsverzeichnis ein

\definecolorset{RGB}{}{}{d1,11,64,12;d2,39,140,41;d3,77,57,153;d4,89,49,16;d5,217,124,49}

\bibliographystyle{alphadin}

\title{Verifikation von GALS Systemen}
\author{Henning Günther}

\lstdefinelanguage{gtl}{
  morekeywords={model,and,or,next,not,connect,verify,always},
  sensitive=true,
  morestring=[b]",
  otherkeywords={.,=>}
}

\newcommand{\bdd}[1]{\vcenter{\hbox{\includegraphics[scale=.4]{#1}}}}

\begin{document}

\maketitle

\tableofcontents

\chapter{Einleitung}
Hier ein bisschen Motivation zum Thema
\chapter{GALS-Architekturen}
\section{GALS-Architekturen}
Ein GALS System -- GALS steht für "`Globally asynchronous, locally synchronous"' -- besteht aus mehreren synchronen Komponenten, die asynchron miteinander kommunizieren.
Formal kann ein synchrones System $\alpha$ vollständig als ein 4-Tupel
\[ \alpha = (I,O,S,t,f) \]
mit den folgenden Komponenten beschrieben werden ($V$ sei hier die Menge aller möglichen Variablen):
\begin{itemize}
\item Eine Menge von Eingabevariablen $I\subseteq V$, also beispielsweise $I=\{x,y\}$, wenn ein System mit zwei Eingabevariablen spezifiziert werden soll.
\item Eine Menge von Ausgabevariablen $O\subseteq V$.
\item Die Zustandsmenge $S$, die alle möglichen internen Zustände des Systems darstellt.
\item Einer Zuordnungsfunktion $t : I\cup O\rightarrow {Set}$ die jeder Variable einen Typen zuordnet.
\item Der eigentlichen Auswertungsfunktion $f : S\times \left(\prod_{v\in I} t(v)\right)\rightarrow S\times\left(\prod_{v\in O} t(v)\right)$, die also jeder Kombination aus Eingaben und internen Zuständen eine Ausgabe und einen neuen Zustand zuordnet.
\end{itemize}
Aus dieser Definition ergibt sich also, dass synchrone Komponenten deterministisch sind, also bei gleichem internen Zustand und gleicher Eingabe immer die gleiche Ausgabe produzieren.

Ein GALS-System $\gamma$ lässt sich mit dieser Definition darstellen als ein Tupel der Form
\[ \gamma = (\mathcal{A},\mathcal{C}) \]
wobei die Komponenten folgende Bedeutung haben:
\begin{itemize}
\item $\mathcal{A}$ ist eine Menge von synchronen Komponenten.
\item Die Relation $\mathcal{C}\subseteq V\times V$ gibt an, welche Ausgangsvariablen mit welchen Eingangsvariablen verknüpft sind.
\end{itemize}

\subsection{Semantik}
Diese Defintionen geben natürlich noch keinen Aufschluss über die Interpretationsweise der so spezifizierten GALS-Systeme.
Sie geben lediglich Aufschluss über die Verknüpfung der einzelnen synchronen Komponenten, nicht aber über deren Ausführungsweise.
Tatsächlich bieten sich eine Vielzahl von Möglichkeiten an, ein gegebenes GALS-System auszuführen.
Einige davon sollen hier vorgestellt werden.

\subsubsection{Synchrone Ausführung}
Es ist möglich, ein GALS-System als ein synchrones System zu interpretieren, in dem alle Komponenten gleichzeitig ihre Berechnungsschritte ausführen.
%BLA BLA BLA
\chapter{Model-Checking}
\section{SPIN}
\label{sec:spin}
SPIN ist ein Model-Checker für asynchrone Software-Modelle, die in der Sprache Promela ({\bf Pro}cess {\bf Me}ta {\bf La}nguage) definiert sind~\cite{spinbook}.
Das Werkzeug verwendet "`Explicit-State Model Checking"'-Techniken, berechnet also jeden möglichen Zustand des Systems und überprüft, ob die zu verifizierenden Eigenschaften erfüllt sind.
SPIN unterstützt eine Reihe von Optimierungstechniken, darunter
\begin{itemize}
\item "`Partial order reduction"' um den Zustandsraum von Modellen zu verkleinern~\cite{partial_order_reduction}.
\item Zustandskompressionstechniken, die den Speicherbedarf von Verifikationen senken können~\cite{spin_state_compression}.
\item Nutzung von Multi-Prozessor-Systemen zur Geschwindigkeitsverbesserung~\cite{spin_multi_core}.
\end{itemize}
Es werden sowohl die Verifikation von \emph{Safety}-Eigenschaften ("`Es passiert nie etwas schlimmes"') als auch von \emph{Liveness}-Eigenschaften ("`Es passiert immer mal wieder etwas gutes"') unterstützt.
Die Verifikation von Modellen erfolgt nicht direkt durch SPIN selbst, sondern es wird Code für einen domänenspezifischen Modell-Checker generiert.
\section{SCADE}
\label{sec:scade}
SCADE ist ein Formalismus und Werkzeug zur Erstellung sicherheitskritischer Softwaresysteme\footnote{SCADE wird von der Firma \emph{Esterel Technologies} vertrieben, zu finden unter \url{http://esterel-technologies.com}}.
Zentraler Bestandteil ist die auf der synchronen Sprache Lustre~\cite{lustre} aufbauende Beschreibungssprache.
Die Sprache ist Datenfluss-orientiert, es werden also Gleichungen für die Ein- und Ausgabevariablen formuliert.
Es stehen aber auch andere Mechanismen, wie zum Beispiel State-Machines zur Verfügung.

Außerdem beinhaltet die SCADE Suite noch einen Model-Checker, den \emph{Design Verifier}, mit dessen Hilfe sich Eigenschaften von Modellen nachprüfen lassen\footnote{Entwickelt von der Firma \emph{Prover}, zu finden unter \url{http://prover.com}}.
\chapter{GTL -- GALS Translation Language}
Um ein zu verifizierendes GALS-Modell vollständig zu definieren müssen die folgenden Eigenschaften spezifiziert werden:
\begin{itemize}
\item Die synchronen Komponenten, aus denen das Gesamtsystem zusammen gesetzt wird.
  Jede synchrone Komponente ist eine Instanz eines in einer synchronen Sprache spezifizierten Modells.
\item Die Verbindungen zwischen den Komponenten.
  Eine Verbindung verknüpft eine Ausgabevariable einer Komponente mit einer Eingabevariable einer anderen.
\item Die zu verifizierende Eigenschaft in Form einer LTL Formel über die Variablen der einzelnen Komponenten.
\end{itemize}

Die GTL-Sprache verwendet externe Formalismen, um die synchronen Komponenten zu beschreiben.
Auf die Implementierungen der synchronen Komponenten wird in der Beschreibung nur verwiesen.
Hierfür wird das Schlüsselwort \emph{model} verwendet.
Das folgende Codefragment deklariert eine Komponente mit dem Namen \emph{EntrySensor}, die im \emph{SCADE}-Formalismus definiert wurde:
\begin{lstlisting}[language=gtl]
  model[scade] EntrySensor("LightBarrier");
\end{lstlisting}
Die Ausdrücke in Klammern ("`LightBarrier"') machen formalismus-spezifische Angaben, in diesem Fall geben sie beispielsweise die Klasse des gewünschten Modells an.
Zu beachten ist, dass in der Komponentendeklaration keine Ein- oder Ausgabekanäle definiert werden.
Die GTL-Sprache setzt voraus, dass diese implizit in dem Modell-Formalismus vorhanden sind.

Um nun Zusicherungen zu formulieren, die die Komponente einhalten wird, kann man die Deklaration um so genannte Kontrakte erweitern:
\begin{lstlisting}[language=gtl]
  model[scade] EntrySensor("LightBarrier") {
    obscured and (next obscured) and (not offline) => alert;
  }
\end{lstlisting}
Dieser Kontrakt besagt beispielsweise, dass wenn der Sensor der Lichtschranke zwei Zeitschritte verdeckt ist und der Sensor nicht ausgeschaltet ist, auf jeden Fall Alarm ausgelöst wird.
Kontrakte dürfen nur Aussagen über Variablen der lokalen Komponente machen.

Die Verbindungen zwischen Komponenten werden durch \emph{connect}-Deklarationen angegeben.
Eine Verbindung gibt dabei eine Variable einer Komponente an, von der sie ausgeht und eine Variable eines anderen Modells, zu der sie geht.
\begin{lstlisting}[language=gtl]
  connect EntrySensor.alert TrapDoor.open;
\end{lstlisting}
Diese Verbindung gibt beispielsweise an, dass das Ausgabesignal \emph{alert} der Komponente \emph{EntrySensor} mit dem Eingabesignal \emph{open} der Komponente \emph{TrapDoor} verbunden sein soll.
Es können nur Ausgabesignale mit Eingabesignalen verknüpft werden.

Um nun Aussagen über das Gesamtsystem treffen zu können, verwendet man das \emph{verify}-Schlüsselwort.
Mit diesem lassen sich LTL-Formeln angeben, die Gültigkeit im System besitzen sollen.
\begin{lstlisting}[language=gtl]
  verify {
    always (System.offline => not TrapDoor.open);
  }
\end{lstlisting}
Im Gegensatz zu Kontrakten können diese Formeln Variablen aus mehreren Modellen enthalten.
Die gesamte Grammatik der GTL-Sprache ist im nächsten Abschnitt \ref{sec:grammar} angegeben.

\section{Grammatik}
\label{sec:grammar}
Eine globale Deklaration ist entweder eine Modell-Deklaration, eine Verbindungsdeklaration oder eine Formel, die das Gesamtsystem erfüllen muss.
\begin{grammar}
  <declaration> ::= <model\_decl>
  \alt <connect\_decl>
  \alt <verify\_decl>
\end{grammar}
Eine Modell Deklaration besteht aus dem Schlüsselwort \emph{model}, gefolgt von dem Namen des synchronen Formalismus, in dem das Modell definiert ist (beispielsweise \emph{scade}).
Danach folgt der Name der Komponente und eine Liste von Argumenten, die spezifisch für den synchronen Formalismus angeben, wie das Modell zu laden ist.
Als letztes kommt der Rumpf der Modelldeklaration, der die Kontrakte spezifiziert, die das Modell erfüllt.
\begin{grammar}
  <model\_decl> ::= `model' `[' <id> `]' <id> `(' (<argument> (`,' <argument>)*)? `)' <model\_contract>
\end{grammar}
Ein Kontrakt ist eine Liste von LTL-Formeln, die die Komponente erfüllt oder Initialisierungswerten für die Variablen der Komponente.
Erfüllt das Modell keine Formel, so kann die Modell-Deklaration auch mit einem Semikolon beendet werden.
\begin{grammar}
  <model\_contract> ::= `{' (<contract\_body> `;')* `}'
  \alt `;'

  <contract\_body> ::= <formula>
  \alt `init' <id> <int>
\end{grammar}
Eine Verbindungsdeklaration besteht aus dem Schlüsselwort \emph{connect} und der Angabe einer Variable der Quellkomponente und der Zielkomponente.
Die Variable wird dabei durch den Namen der Komponente, einen Punkt und den Namen der Variable spezifiziert.
\begin{grammar}
  <connect\_decl> ::= `connect' <id> `.' <id> <id> `.' <id> `;'
\end{grammar}
Formeln, die über das Gesamtsystem verifiziert werden sollen, lassen sich in \emph{verify}-Blöcken spezifizieren.
Diese enthalten eine Liste von Formeln.
\begin{grammar}
  <verify\_decl> ::= `verify' `{' (<formula> `;')* `}'
\end{grammar}
Eine Formel kann verschiedene Formen annehmen:
\begin{itemize}
\item Als atomare Aussage besteht sie aus einer Variable.
\item Als Relation setzt sie zwei Ausdrücke in eine Beziehung ("`kleiner"', "`größer"', "`gleich"' etc.).
\item Um anzugeben, dass eine Variable einen Wert aus einer Menge annehmen kann, lässt sich das \emph{in}-Schlüsselwort verwenden.
\item Formeln können mit den logischen Operatoren \emph{and}, \emph{or}, \emph{implies} sowie \emph{not} wieder zu neuen Formeln zusammen gesetzt werden.
\item Mithilfe der temporallogischen Operatoren \emph{always}, \emph{next} und \emph{finally} können neue Formeln aus anderen gebildet werden.
\item Das Konstrukt \emph{exists} erlaubt die Bindung einer Variable an den aktuellen Wert einer anderen und somit die Referenzierung von früheren Werten.
\end{itemize}
\begin{grammar}
  <formula> ::= <var>
  \alt <expr> `<' <expr>
  \alt <expr> `>' <expr>
  \alt <expr> `<=' <expr>
  \alt <expr> `>=' <expr>
  \alt <expr> `=' <expr>
  \alt <var> `in' `{' (<lit> (`,' <lit>)*)? `}'
  \alt `not' <formula>
  \alt <formula> `and' <formula>
  \alt <formula> `or' <formula>
  \alt <formula> `implies' <formula>
  \alt `always' <formula>
  \alt `next' <formula>
  \alt `finally' <int> <formula>
  \alt `exists' <id> `=' <lit> `:' <formula>
  \alt `(' <formula> `)'
  
  <lit> ::= <int>
  \alt <var>
\end{grammar}
Ein Ausdruck ist entweder eine numerische Konstante, eine Variable oder eine arithmetische Operation (Addition, Subtraktion, Multiplikation oder Division werden unterstützt) von zwei Ausdrücken.
\begin{grammar}
  <expr> ::= <lit>
  \alt <expr> `+' <expr>
  \alt <expr> `-' <expr>
  \alt <expr> `*' <expr>
  \alt <expr> `/' <expr>
  \alt `(' <expr> `)'
\end{grammar}
Eine Variable kann unqualifiziert sein und nur ihren Namen angeben, oder zusätzlich noch den Namen der Komponente aus der sie stammt angeben und damit als qualifiziert gelten.
\begin{grammar}  
  <var> ::= <id>
  \alt <id> `.' <id>
  
  <id> ::= (`a'-`z' `A'-`Z' `0'-`9')+
  
  <int> ::= (`0'-`9')+
\end{grammar}


\section{Formeln}
\label{sec:formula}
Die Ausdrücke, die zur Spezifikation von Kontrakten sowie zur Formulierung von einzuhaltenden Bedingungen verwendet werden, stellen eine Untermenge der so genannten LTL-Formeln (LTL steht für "`linear temporal logic"') dar.
Allerdings sind die Atome der LTL-Formel nicht nur boolesche Variablen und Konstanten, sondern auch Relationen zwischen numerischen Variablen, Ausdrücken und/oder Konstanten.

Da der SCADE Design Verifier nicht in der Lage ist, Liveness-Eigenschaften zu verifizieren, können in den Formeln allerdings keine \emph{until}-Konstrukte verwendet werden.

\section{Operationelle Semantik}
\subsection{Variablen}
In den LTL-Formeln der Semantik können frühere Werte von Variablen referenziert werden.
Hierfür werden die Variablen mit einer natürlichen Zahl versehen, die die Anzahl von Schritten angibt, die in die Vergangenheit geschaut werden soll.

Variablen kommen in der operationellen Semantik in zwei Formen vor:
In der zu verifizierenden Formel kommen sie qualifiziert mit dem Modellnamen vor, die Variablen sind also Element der Menge $\mathit{Id}\times\mathit{Id}\times\mathbb{N}$.
In Modell-Kontrakten ist der Modellname klar, daher sind die Variablen hier unqualifiziert und damit Element der Menge $\mathit{Id}\times\mathbb{N}$.
\subsection{Aufbau}
\label{sec:sos_defs}
Die Semantik eines GTL-Modells wird durch ein Tripel angegeben, bestehend aus der Menge der Modelle, der Menge der Verbindungen zwischen den Variablen, sowie der LTL-Formel, deren Gültigkeit verifiziert werden soll.
\[ \mathit{GTL} = \mathcal{P}(\mathcal{M})\times\mathcal{P}(\mathcal{C})\times\mathit{LTL}(\mathit{Id}\times\mathit{Id}\times\mathbb{N}) \]
Die LTL-Formel ist über Paare von Namen definiert, wobei der erste den Namen des Modells angibt und der zweite den der Variable im Modell.
Eine Modellkonfiguration $\mathcal{M}$ ist gegeben durch einen Namen, den synchronen Automaten, den Kontrakt und Initialisierungswerten für die Variablen.
\[ \mathcal{M} = \mathit{Id}\times\mathcal{A}\times\mathit{LTL}(\mathit{Id})\times\mathcal{P}(\mathcal{D}) \]
Die Initialisierungswerte werden über eine Abbildung von Variablennamen auf Werte $\mathit{Val}$ dargestellt:
\[ \mathcal{D} = \mathit{Id}\rightarrow\mathit{Val} \]
Die Verbindungen zwischen Modellen werden als Viertupel dargestellt, die das Modell und die Variable angibt, von dem die Verbindung ausgeht und Modell und Variable, in dem die Verbindung endet.
\[ \mathcal{C} = \mathit{Id}\times\mathit{Id}\times\mathit{Id}\times\mathit{Id} \]
\subsection{Ableitungsregeln}
Um nun die Ableitungsregeln zu formulieren zu können, die angeben, wie ein gegebenes Textmodell in ein GTL-Modell übersetzt wird, müssen zuerst die Ableitungsarten eingeführt werden.
Zunächst gibt es die globale Ableitungsrelation $\vdash$.
Die Aussage $\gamma,T\vdash m$ sagt also aus:
Gegeben ein Textmodell $\gamma$ und eine Typenbindung $T : \mathit{Id}\times\mathit{Id}\rightarrow\mathit{Type}\times\{\mathit{Inp},\mathit{Outp}\}$ lässt sich das GTL-Modell $m : \mathit{GTL}$ ableiten.

Die Ableitung $\vdash_C$ leitet den Kontrakt von Modellen sowie die Initialisierungswerte ihrer Variablen her.
Die Aussage $\gamma,T\vdash_C (c,d)$ gibt also an: Gegeben eine textuelle Repräsentation $\gamma$ und eine Typbindung $T$ lässt sich der Kontrakt(LTL-Formel) $c$ und die Initialisierung $d : \mathit{Id}\rightarrow \mathit{Val}$ ableiten.

Formeln werden mithilfe von $\vdash_V$ abgeleitet.
Diese Relation gibt an, ob sich ein Ausdruck zu einer Formel eines bestimmten Typs ableiten lässt.
Die Aussage $e,T\vdash_V (f,t)$ sagt also aus, dass gegeben die Typisierung $T$ und den Ausdruck $e$ sich die LTL-Formel $f$ vom Typ $t$ herleiten lässt.

Um die korrekte Typisierung von Modellen sicher stellen zu können, benötigt man noch eine weitere Ableitungsart:
$\vdash_T$ gibt an, ob ein synchrones Modell, dass durch die übergebenen Parameter spezifiziert wird, eine gegebene Typisierung erfüllt und einem gegebenen Automaten entspricht.
Die Aussage
\[ \textrm{scade},(\textrm{"`model.scade"'})\vdash_T T,a \]
bedeutet also: Das Scade-Modell "`model.scade"' erfüllt die Typenbindung $T$ und entspricht dem synchronen Automaten $a$.
Da diese Ableitung abhängig vom synchronen Formalismus ist, wird sie hier nicht explizit angegeben.

Zunächst wird definiert, wie sich ein leeres Textmodell herleiten lässt; die Modelle und Verbindungen sind leer, die zu verifizierende Formel ist $\top$, also immer wahr.
\[
\inference[empty]{}{\epsilon,T\vdash (\emptyset,\emptyset,\top)}
\]
Damit eine Modelldeklaration gültig ist, muss sich der Inhalt des Kontraktes $c$ mit $\vdash_C$ ableiten lassen und das referenzierte synchrone Modell wohlgetypt im Verhältnis zum restlichen Modell sein:
\[
\inference[model]{\alpha,T \vdash(ms,cs,vs) & c,\{ (v,t,d)\ |\ (n,v,t,d)\in T \}\vdash_C f,d & \beta,\mathit{args}\vdash_T T,a}{\textbf{model[}\beta\textbf{]}\ n\textbf{(}\mathit{args}\textbf{)}\ \textbf{\{} c\textbf{\}}\ \alpha,T\vdash(ms\cup\{(n,a,f,d)\},cs,vs)}
\]
Eine Deklaration einer Verbindung ist gültig, wenn beide Variablen den gleichen Typ besitzen.
Außerdem muss die erste Variable eine Ausgabe-, die andere eine Eingabe-Variable sein.
\[
\inference[connect]{\alpha,T\vdash(ms,cs,vs) & T(cm_f,cv_f)=(t,\mathit{Outp}) & T(cm_t,cv_t) = (t,\mathit{Inp})}{\textbf{connect}\ cm_f\textbf{.}cv_f\ cm_t\textbf{.}cv_t\textbf{;}\ \alpha,T\vdash(ms,cs\cup\{(cm_f,cv_f,cm_t,cv_t)\},vs)}
\]
Eine Verifikationsblock muss sich mit $\vdash_C$ zu einer LTL-Formel ableiten lassen.
Gibt es mehr als einen Block, so werden die Formeln per Konjunktion zusammengefasst.
\[
\inference[verify]{\alpha,T\vdash(ms,cs,vs) & c,T\vdash_C (f,\emptyset)}{\textbf{verify}\ \textbf{\{}c\textbf{\}}\ \alpha,T\vdash (ms,cs,vs\land f))}
\]
Ein leerer Kontrakt erlaubt dem Prozess jedes Verhalten und wird daher durch die LTL-Formel $\top$ repräsentiert.
\[
\inference[$\epsilon$-contract]{}{\epsilon,T\vdash_C (\top,\emptyset)}
\]
Eine gültige Initialisierungsdeklaration liegt dann vor, wenn der Typ der initialisierten Variable dem des Wertes entspricht.
\[
\inference[init]{\alpha,T\vdash_C (c,i) & d\in T(c)}{\textbf{init}\ v\ d\textbf{;}\ \alpha,T\vdash_C (c,i\cup\{(v,d)\})}
\]
Eine Kontraktformel ist gültig, wenn sie sich erfolgreich zu einer LTL-Formel des Typs "`bool"' ableiten lässt.
Zu beachten ist, dass das Schlüsselwort \emph{contract} auch weggelassen werden kann.
\[
\inference[formula]{\alpha,T\vdash_C (c,i) & e,T,\emptyset\vdash_V (f,\textrm{bool})}{\textbf{contract}\ e\textbf{;}\ \alpha,T\vdash_C (c\land f,i)}
\]
Eine Variable kann zwei verschiedene Bedeutungen haben:
Entweder ist sie eine durch einen $\exists$-Quantor gebundene Variable (1) oder eine normale qualifizierte oder unqualifizierte Variable.
In beiden Fällen ist der Typ des Ausdrucks der Typ der Variable.
\[
\begin{array}{ll}
  \inference[var(1)]{(v,q,n)\in\gamma & T(q)=t}{v,T,\gamma\vdash_V (q^n,t)} &
  \inference[var(2)]{\lnot\exists q,n: (v,q,n)\in\gamma & T(v)=t}{v,T,\gamma\vdash_V (v,t)}
\end{array}
\]
Die Regeln für die logischen Konnektoren \emph{not}, \emph{true}, \emph{false}, \emph{and}, \emph{or} und \emph{implies} sind genau wie in der normalen Logik definiert:
\[
\begin{array}{ll}
  \inference[not]{e,T,\gamma\vdash_V (e',\textrm{bool})}{\textbf{not}\ e,T,\gamma\vdash_V (\lnot e',\textrm{bool})} &
  \inference[and]{e_1,T,\gamma\vdash_V (e_1',\textrm{bool}) & e_2,T,\gamma\vdash_V (e_2',\textrm{bool})}{e_1\ \textbf{and}\ e_2,T,\gamma\vdash_V (e_1'\land e_2',\textrm{bool})}\\[20pt]
  \inference[true]{}{\textbf{true},T,\gamma\vdash_V(\top,\textrm{bool})} &
  \inference[or]{e_1,T,\gamma\vdash_V (e_1',\textrm{bool}) & e_2,T,\gamma\vdash_V (e_2',\textrm{bool})}{e_1\ \textbf{or}\ e_2,T,\gamma\vdash_V (e_1'\lor e_2',\textrm{bool})}\\[20pt]
  \inference[false]{}{\textbf{false},T,\gamma\vdash_V(\bot,\textrm{bool})} &
  \inference[impl]{e_1,T,\gamma\vdash_V (e_1',\textrm{bool}) & e_2,T,\gamma\vdash_V (e_2',\textrm{bool})}{e_1\ \textbf{implies}\ e_2,T,\gamma\vdash_V (\lnot e_1'\lor e_2',\textrm{bool})}
\end{array}
\]
Der $\exists$-Quantor bindet eine neue Variable an die aktuelle Version einer existierenden.
Das bedeutet, dass auf frühere Werte einer Variable zurück gegriffen werden kann, wenn die Variable in einem \emph{next}-Kontext verwendet wird.
\[
\inference[exists]{f,T,\gamma\cup\{(u,e,0)\}\vdash_V f'}{\textbf{exists}\ u\textbf{=}e\textbf{:}\ f,T,\gamma\vdash_V f'}
\]
Der \emph{next}-Operator wird in sein LTL-Äquivalent übersetzt.
Allerdings müssen die gebundenen Variablen auf den neuen Kontext angepasst werden, indem sie auf einen Wert früher in der Geschichte referenziert werden.
\[
\inference[next]{f,T,\{ (u,e,n+1)\ |\ (u,e,n)\in\gamma\}\vdash_V (f',\textrm{bool})}{\textbf{next}\ f,T,\gamma\vdash_V (\bigcirc f',\textrm{bool})}
\]
Tritt ein \emph{always}-Operator auf, so werden für die Ableitung der Unterformel alle Bindungen entfernt, da gebundene Variablen nicht innerhalb eines \emph{always}-Operators auftauchen dürfen.
\[
\inference[always]{f,T,\emptyset\vdash_V (f',\textrm{bool})}{\textbf{always}\ f,T,\gamma\vdash_V (\lnot(\top U (\lnot f')),\textrm{bool})}
\]
Der \emph{finally}-Operator ist nur eingeführt, um die Aussage "`Innerhalb der nächsten $i$ Schritte passiert $f$"' einfacher zu formulieren.
Er kann rekursiv mithilfe des \emph{or}- und \emph{next}-Operators definiert werden.
\[
\inference[finally-0]{f,T,\gamma\vdash_V f'}{\textbf{finally}0\ f,T,\gamma\vdash_V f'}
\]
\[
\inference[finally-i]{f\ \textbf{or}\ (\textbf{next}\ \textbf{finally}(i-1)\ f),T,\gamma\vdash_V f'}{\textbf{finally}i\ f,T,\gamma\vdash_V f'}
\]
Gleichheitstests können auf allen Variablen gleichen Typs durchgeführt werden.
\[
\inference[equal]{f,T,\gamma\vdash_V (f',t) & g,T,\gamma\vdash_V (g',t)}{f=g,T,\gamma\vdash_V(f'=g',\textrm{bool})}
\]
\[
\inference[nequal]{f,T,\gamma\vdash_V (f',t) & g,T,\gamma\vdash_V (g',t)}{f\textbf{!=}g,T,\gamma\vdash_V(\lnot(f'=g'),\textrm{bool})}
\]
Während Datentyp-spezifische Relationen wie $<$ oder $>$ nur auf speziellen Typen ausgeführt werden können (in diesem Fall "`int"').
\[
\inference[lesser]{f,T,\gamma\vdash_V (f',\textrm{int}) & g,T,\gamma\vdash_V (g',\textrm{int})}{f<g,T,\gamma\vdash_V(f'<g',\textrm{bool})}
\]
\[
\inference[greater]{f,T,\gamma\vdash_V (f',\textrm{int}) & g,T,\gamma\vdash_V (g',\textrm{int})}{f>g,T,\gamma\vdash_V(f'>g',\textrm{bool})}
\]
Statt Variablen können für Integer auch Konstanten verwendet werden. 
\[
\inference[const]{n\in\mathbb{N}}{n,T,\gamma\vdash_V (n,\textrm{int})}
\]
Einfache arithmetische Operationen werden ebenfalls unterstützt.
\[
\inference[plus]{f,T,\gamma\vdash_V (f',\textrm{int}) & g,T,\gamma\vdash_V (g',\textrm{int})}{f\textbf{+}g,T,\gamma\vdash_V (f'+g',\textrm{int})}
\]
\[
\inference[minus]{f,T,\gamma\vdash_V (f',\textrm{int}) & g,T,\gamma\vdash_V (g',\textrm{int})}{f\textbf{-}g,T,\gamma\vdash_V (f'-g',\textrm{int})}
\]
\[
\inference[times]{f,T,\gamma\vdash_V (f',\textrm{int}) & g,T,\gamma\vdash_V (g',\textrm{int})}{f\textbf{*}g,T,\gamma\vdash_V (f'\cdot g',\textrm{int})}
\]
\[
\inference[div]{f,T,\gamma\vdash_V (f',\textrm{int}) & g,T,\gamma\vdash_V (g',\textrm{int})}{f\textbf{/}g,T,\gamma\vdash_V (\frac{f'}{g'},\textrm{int})}
\]
Um die Aussage "`Variable $v$ hat den Wert $s_1$ oder $s_2$ oder\dots"' abzukürzen, kann man eine Menge von Werten angeben, die die Variable annehmen darf.
\[
\inference[elem]{\forall i\in\{0,\dots,n\}: s_i,T,\gamma\vdash_V (s_i',t) & T(v)=t}{v\ \textbf{in}\ \textbf{\{}s_0,\dots,s_n\textbf{\}},T,\gamma\vdash_V (\bigvee_{i\in\{0,\dots,n\}} v=s_i',\textrm{bool}) }
\]
\subsection{Interpretation als GALS-System}
%Um das in diesem Abschnitt definierte GTL-System als ein GALS System wie in Abschnitt \ref{sec:gals_formal_definition} zu betrachten, müssen die im System enthaltenen Formeln mithilfe des in Abschnitt \ref{sec:ltl-translation} angegebenen Algorithmus in Automaten übersetzt werden.
Gegeben ein GTL-Modell $(m,c,v)$ kann man ein GALS-System $(A,p,C)$ konstruieren, das die synchronen Komponenten enthält.
Die Automatenmenge enthält die Namen aller Komponenten des GTL-Modells.
\[ A = \{ \mathit{name}\ |\ (\mathit{name},\_,\_,\_)\in m \} \]
Die Abbildung der Automatennamen auf Automaten ordnet den synchronen Automaten zu.
\[ p(n) = \left\{\begin{array}{lr}
    a & (n,a,\_,\_)\in m\\
    \bot & \textrm{sonst}
  \end{array}\right. \]
Für die Verbindungen benötigt man eine Hilfsfunktion $r : \mathit{Id}\times\mathit{Id}\rightarrow \mathbb{N}$, die einer Komponentenvariable die entsprechende Position im Eingabe-/Ausgabe-Tupel zuordnet.
\[ C = \{(a_f,r(a_f,v_f),a_t,r(a_t,v_t)\ |\ (a_f,v_f,a_t,v_t)\in c\} \]
Um das GALS-System zu erhalten, das das Zusammenspiel der Kontrakte repräsentiert muss man statt der synchronen Automaten die mithilfe des in Abschnitt \ref{sec:ltl-translation} angebenen Algorithmus übersetzten LTL-Kontraktformeln verwenden.
\subsection{Interpretation der Kontrakte als GALS-System}
Anstatt die synchronen Automaten der Komponenten für die Komposition des GALS-Systems zu verwenden, ist es auch möglich, die Kontrakte wie in Abschnitt \ref{sec:contracts} erläutert als obere Schranke für das System-Verhalten zu verwenden.
Da jeder Kontrakt durch eine LTL-Formel beschrieben wird, lässt sich mit dem Algorithmus aus Abschnitt \ref{sec:ltl-translation} ein äquivalenter Büchi-Automat konstruieren.
Dieser enthält in jedem Zustand eine Menge von atomaren Aussagen, die gelten müssen, damit der Zustand betreten werden darf.
Für eine Menge von Variablen $V$ und die Typzuordnung $T : V\rightarrow \mathit{Type}$ erzeugt jede atomare Aussage $f$ eine Menge von Zuordnungen $l(f)$, die die entsprechende Aussage erfüllen.
Für die atomare Aussage $x\leq y$ ergibt sich mit der Variablenmenge $\{x,y\}$ und der Typzuordnung $T = \{ (x,\mathbb{N}),(y,\mathbb{N}) \}$ die Zuordnungen
\[ \{ \{ (x,0), (y,0) \}, \{ (x,0), (y,1) \}, \{ (x,0), (y,2) \},\dots,\{ (x,1), (y,1) \}, \{ (x,1), (y,2) \},\dots \} \]

Gegeben ein Büchi-Automat $(Q,\Sigma,\delta,\mu,q_0,\emptyset)$ lässt sich ein äquivalenter nicht-de\-ter\-mi\-nis\-tischer Mealy Automat $(Q\cup\{\const{init}\},\Sigma',\Omega,\delta',\{\const{init}\})$ erzeugen mit:
\[
\begin{array}{c}
  \forall q,q'\in Q: q\neq \const{init}\Rightarrow\\
  (q\delta q'\land \mu(q') = p) \Leftrightarrow (\forall a\in l(p): (q,\mathit{inp}(a))\delta'(q',\mathit{outp}(a)))
\end{array}
\]
\[
\forall q\in q_0: \mu(q) = p \Leftrightarrow (\forall a\in l(p): (\const{init},\mathit{inp}(a))\delta'(q,\mathit{outp}(a)))
\]
wobei $\mathit{inp}$ den Vektor von Eingaben aus einer Belegung extrahiert und $\mathit{outp}$ den Vektor von Ausgaben.

%\begin{tikzpicture}
%  \node[draw] (promela) at (1.5,1) {Promela};
%  \node[draw] (gtl) at (0,2) {GTL};
%  \node[draw] (scade) at (3,2) {Scade};
%  \node[draw] (verifier) at (1.5,0) {Verifier};
%  \draw[->,blue] (scade) |- (promela);
%  \draw[->,blue] (gtl) |- (promela);
%  \draw[->,blue] (promela) -- (verifier);
%\end{tikzpicture}

%\begin{tikzpicture}
%  \node[draw] (gtl) at (0,2) {GTL};
%  \node[draw] (scade) at (3,2) {Scade};
%  \node[draw] (promela) at (0,1) {Promela};
%  \node[draw] (testnode) at (3,1) {Scade Testnode};
%  \node[draw] (verifier) at (0,0) {Verifier};
%  \draw[->,blue] (gtl) -- (promela);
%  \draw[->,blue] (scade) -- (testnode);
%  \draw[->,blue] (gtl) -- (testnode);
%  \draw[->,blue] (promela) -- (verifier);
%\end{tikzpicture}


\chapter{Modelltransformationen}
\section{LTL -- Linear temporal logic}
LTL-Formeln können benutzt werden, um das zeitabhängige Verhalten von Systemen zu beschreiben\cite{ltlbasics}.
Die LTL-Logik stellt eine Erweiterung der Aussagenlogik dar, so dass nicht nur Aussagen über den jetzigen Zustand getroffen werden können, sondern auch über noch folgende Zustände.
Die Aussagenlogik wird um folgende Operatoren erweitert:
\begin{itemize}
\item Der \emph{next}-Operator (Auch mit $\bigcirc$ bezeichnet) sagt aus, dass eine Formel im nächstens Zustand gilt.
  Die Formel $\bigcirc\varphi$ spezifiziert also Pfade der Form
  \[ \xymatrix @R=0em {
      \bullet \ar[r] & \bullet \ar[r] & \bullet \ar@{-->}[r] &\\
      & \varphi & &
  }
    \]
\item Mit dem \emph{always}-Operator ($\square$) versehene Formeln gelten sowohl im aktuellen wie auch in allen folgenden Zuständen.
  Die Formel $\square\varphi$ erlaubt also alle Pfade der Form
  \[ \xymatrix @R=0em {
      \bullet \ar[r] & \bullet \ar[r] & \bullet \ar@{-->}[r] &\\
      \varphi & \varphi & \varphi &
  }
    \]
\item Der \emph{finally}-Operator ($\diamond$) gibt an, dass eine Formel irgendwann in der Zukunft einmal gelten wird.
  Die Formel $\diamond\varphi$ spezifiziert zum Beispiel Pfade der Form
  \[ \xymatrix @R=0em {
    \bullet \ar[r] & \bullet \ar@{-->}[r] & \bullet \ar[r] & \bullet \ar@{-->}[r] & \\
    & & \varphi & &
  } \]
\item Formeln die gelten sollen, bis eine bestimmte Bedingung erfüllt ist, lassen sich mit dem \emph{until}-Operator ($U$) angeben.
  Wird beispielsweise gefordert, dass die Formel $\varphi$ gilt, bis $\psi$ gilt, so lässt sich dies schreiben als $\varphi U\psi$.
  Ein Beispielpfad für diese Formel ist
  \[ \xymatrix @R=0em {
    \bullet \ar[r] & \bullet \ar@{-->}[r] & \bullet \ar[r] & \bullet \ar@{-->}[r] & \\
    \varphi & \varphi & \varphi & \psi &
  } \]
\end{itemize}
Um die vollständige Mächtigkeit von LTL zu erreichen reicht es allerdings auch schon, nur die Operatoren $\bigcirc$ und $U$ zu haben, denn der \emph{finally}-Operator lässt sich ausdrücken als
\[ \diamond\varphi = \top U \varphi \]
und der \emph{always}-Operator als
\[ \square\varphi = \lnot (\top U \lnot\varphi) \]
Alle anderen Operatoren sind also zwar nützlich, aber nicht benötigt.
\section{Büchi-Automaten}
Büchi-Automaten stellen eine Erweiterung von endlichen Automaten auf unendliche Eingaben dar\cite{buchibasics}.
Anders als endliche Automaten, die eine Eingabe akzeptieren, wenn die Ausführung in einem finalen Zustand endet, akzeptiert ein Büchi-Automat eine Eingabe genau dann, wenn die Ausführung unendlich oft einen finalen Zustand erreicht.

Formal ist ein Büchi-Automat ein Tupel
\[ A = (Q,\Sigma,\delta,\mu,q_0,F) \]
wobei die Symbole folgende Bedeutung haben:
\begin{itemize}
  \item $Q$ ist die Menge der Zustände des Automaten.
  \item Die Menge von atomaren Aussagen $\Sigma$, die gültig oder ungültig sein können.
  \item $\delta\subseteq Q\times Q$ ist die Übergangsrelation des Automaten.
  \item $\mu : Q\rightarrow\Sigma$ gibt an, welche Aussagen in einem Zustand gelten müssen.
  \item $q_0\subseteq Q$ ist die Startzustandsmenge.
  \item $F\subseteq Q$ ist die Finalzustandsmenge.
\end{itemize}
Eine Folge von Aussagen $a_0a_1a_2\dots$ wird nun also genau dann akzeptiert, wenn es eine Folge von Zuständen $q_0q_1q_2\dots$ gibt, wobei stets gilt $q_n\delta q_{n+1}$ und in der mindestens ein Zustand aus $F$ unendlich oft vorkommt.
Außerdem muss jede Aussage $a_n$ mit der Menge von Aussagen $\mu(q_n)$ kompatibel sein.

\subsection{Verallgemeinerter Büchi-Automat}
Ein Verallgemeinerter Büchi-Automat\footnote{Im englischen als "`generalized buchi automaton"' bezeichnet und als \emph{GBA} abgekürzt.} unterscheidet sich von einem normalen Büchi-Automaten durch die Definition des Akzeptanzverhaltens.
Während ein Büchi-Automat akzeptiert, wenn unendlich oft ein Finalzustand betreten wird, ist $F$ hier eine Menge von Finalzustandsmengen ($F\subseteq\mathcal{P}(Q)$).
Der Automat akzeptiert nun, wenn aus jeder Finalzustandsmenge mindestens ein Zustand unendlich oft betreten wird.

Ein verallgemeinerter Büchi-Automat lässt sich in einen normalen Büchi-Automaten übersetzen, indem man für jede Finalzustandsmenge eine "`Ebene"' einführt.
Jede Ebene enthält die gleichen Zustände und Transitionen wie der ursprüngliche Automat, nur dass beim Verlassen von den Finalzuständen der aktuellen Finalzustandsmenge die nächste Ebene betreten wird.
Dadurch wird erreicht, dass ein Zyklus nur dann zustande kommt, wenn ein Zustand aus allen Finalzustandsmengen betreten wird.

Für einen verallgemeinerten Büchi-Automaten $(Q,\Sigma,\delta,\mu,q_0,F)$ konstruiert man also einen Büchi-Automaten $(Q',\Sigma,\delta',\mu',q_0',F')$.
Dazu benötigt man zunächst eine Hilfsfunktion $i : F\rightarrow F$, die zyklisch durch alle Finalzustandsmengen geht.
\begin{align*}
  Q' &= \{ (q,f)\ |\ q\in Q, f\in F \}\\
  \delta' &= \{ ((q_1,f_1),(q_2,f_2))\ |\ (q_1,q_2)\in\delta, (q_1\in f_1\land i(f_1)=f_2)\lor f_1=f_2\}\\
  \mu'((q,f)) &= \mu(q)\\
  F' &= \{ (q,f)\ |\ f\in F, q\in f \}
\end{align*}

\section{LTL Übersetzung}
Um LTL Formeln einfacher übersetzen zu können und zu kanonisieren, werden diese in Büchi-Automaten übersetzt.
Diese Übersetzung benutzt den in \cite{Gerth95simpleon-the-fly} beschriebenen Algorithmus.
\section{Übersetzungskonstruktion}
Gegeben ein wie in Abschnitt \ref{sec:sos_defs} spezifiziertes System $s\in \mathcal{S}$ mit $s=(ms,cs,vs)$ muss nun eine Übersetzung in ein äquivalentes Promela-Modell gefunden werden, die die Semantik des Systems erhält.
Für jede Komponente $(m,(\mathit{contr},\mathit{init}))\in ms$ mit dem Namen $m$, dem Kontrakt $\mathit{contr}$ und der Initialisierung $\mathit{init}$ wird nun der Kontrakt in einen äquivalenten Büchi-Automaten $(Q,\Sigma,\delta,\mu,q_0,\emptyset)$ übersetzt (Siehe Abschnitt \ref{sec:ltl-translation}).
Hierbei ist zu beachten, dass die generierten Automaten Bedingungen auf den Variablen als Ein- und Ausgabesymbole verwenden, da Relationen wie $x\leq y$ von dem LTL-Übersetzungsalgorithmus als atomare Aussagen betrachtet werden.

Für die Übersetzung werden die Funktionen $\llbracket\rrbracket_C$ und $\llbracket\rrbracket_A$ benötigt.
Die Funktion $\llbracket\rrbracket_C$ generiert aus den übergebenen Atomen einen Promela-Ausdruck, der abhängig vom globalen Zustand testet, ob alle Bedingungen, die durch die übergebenen Atome an die Eingabe-Variablen gestellt sind, erfüllt sind.
Diese Anweisung muss blockieren, bis die Bedingungen erfüllt sind und muss seiteneffektfrei sein, den globalen Zustand also nicht verändern.
Ähnlich dazu generiert die Funktion $\llbracket\rrbracket_A$ eine Anweisung, die die Ausgabevariablen entsprechend den übergebenen Atomen anpasst und damit den globalen Zustand verändert.
Die generierte Anweisung darf niemals blockieren.

Für jede Komponente wird nun ein äquivalenter Prozess wie folgt definiert:
\begin{lstlisting}[language=Promela,mathescape=true,numbers=left,numberstyle=\small,caption={Komponenten-Übersetzung als Promela-Prozess},label=lst:component]
proctype $m$() {
  if $[ \forall i\in q_0:$
  :: atomic {
       $\llbracket\mu(i)\rrbracket_C$;
       $\llbracket\mu(i)\rrbracket_A$;
       goto st_$i$
     }
  $]$ fi;
  $[ \forall q\in Q:$
  st_$q$: if $[\forall q'\in Q,q\delta q':$
  :: atomic {
       $\llbracket\mu(i)\rrbracket_C$;
       $\llbracket\mu(i)\rrbracket_A$;
       goto st_$q'$
     }
  $]$ fi;
  $]$
}
\end{lstlisting}
%Hierbei gibt $\llbracket\rrbracket_C$ einen Promela-Ausdruck an, der abhängig vom globalen Zustand testet, ob alle Bedingungen, die durch die übergebenen Atome spezifiziert sind, erfüllt sind.
%Die Anweisung muss blockieren, bis die Bedingungen erfüllt sind und darf den globalen Zustand nicht verändern.
%Ähnlich dazu generiert die Funktion $\llbracket\rrbracket_A$ eine Anweisung, die den globalen Zustand anhand der übergebenen Atome transformiert.
%Die generierte Anweisung darf nicht blockieren.

Die zu verifizierende Formel $v$ wird negiert ebenfalls in einen Büchi-Automaten $(Q,\Sigma,\delta,\mu,q_0,F)$ übersetzt und in eine äquivalente Promela \emph{never}-Deklaration übersetzt:
\begin{lstlisting}[language=Promela,mathescape=true,numbers=left,numberstyle=\small,caption={Verifikationsziel-Übersetzung als \emph{never}-Prozess}]
never {
  if $[ \forall i\in q_0:$
  :: atomic {
    $\llbracket \mu(i) \rrbracket_C$;
    goto st_$i$
  }
  $]$
  fi;
  $[ \forall q\in Q:$
  $[ q\in F:$ accept_$q$: $]$
  st_$q$:
    if $[ \forall q'\in Q,q\delta q':$
    :: atomic {
      $\llbracket \mu(q') \rrbracket_C$;
      goto st_$q'$
    }
    $]$
  $]$
}
\end{lstlisting}

Um den Initialzustand zu erreichen, wird vor dem Starten aller Prozesse noch eine Initialisierung durchgeführt.
Hierfür muss die konkrete Übersetzung die Funktion $\llbracket\rrbracket_D$ bereit stellen.
Diese generiert, gegeben eine Komponente, eine Variable und einen Initialisierungswert für die Variable, eine Anweisung, die den globalen Zustand so verändert, dass die Variable nun den entsprechenden Wert besitzt.
\begin{lstlisting}[language=Promela,mathescape=true,numbers=left,numberstyle=\small,caption={Initialisierungsprozess}]
init {
  $[ \forall (m,a,f,d)\in ms:$
    $[ \forall (v,val)\in d:$
    $\llbracket m,v,val\rrbracket_D$
    $]$
  $]$
  atomic {
  $[ \forall (m,a,f,d)\in ms:$
    run $m$();
  $]$
  }
}
\end{lstlisting}

\subsection{Korrektheit der Übersetzung}
Um zu beweisen, dass die angegebene Promela-Übersetzung korrekt ist, muss gezeigt werden, dass eine Semantik des Systems, repräsentiert durch eine Untermenge des vollständigen Transitionssystems $T'\subseteq T$ (Siehe Abschnitt \ref{sec:semantic}), mit der Semantik des übersetzten Promela-Modells bisimular ist.
Um dies zu zeigen, wird die Promela-Semantik verwendet, wie sie in \cite{Gallardo04formalaspects} beschrieben ist.
Da in dieser Semantik gefordert ist, dass jede Anweisung ein implizites Label erhält, werden folgende Labels für die Anweisungen in \lstlistingname~\ref{lst:component} (Seite \pageref{lst:component}) vergeben:
\begin{itemize}
\item Die \emph{If}-Anweisung in Zeile 2 erhält das Label \emph{Start}.
\item In diesem Zweig wird der Ausdruck in Zeile 4 mit dem Label \emph{CI\_$i$} versehen.
\item Die nachfolgende Zuweisung in Zeile 5 wird mit dem Label \emph{AI\_$i$} gekennzeichnet.
\item Der If-Zweig in Zeile 12 kann über das Label \emph{C\_$q$\_$q'$} angesprungen werden.
\item Die darauf folgende Anweisung in Zeile 13 bekommt das Label \emph{A\_$q$\_$q'$} zugewiesen.
\end{itemize}

Mit diesen Kennzeichnungen ergibt sich nun die \emph{next}-Funktion der Semantik wie folgt:

\begin{tabular}{|c|c|}
  \hline
  $L$ & $\textrm{next}(L)$\\
  \hline
  CI\_$q$ & A\_$q$\\
  A\_$q$ & st\_$q$\\
  C\_$q$\_$q'$ & A\_$q$\_$q'$\\
  A\_$q$\_$q'$ & st\_$q'$\\
  \hline
\end{tabular}

Auch die erforderliche $g$-Funktion, die die Zweige einer \emph{If}-Anweisung angibt, kann somit hergeleitet werden als:

\begin{tabular}{|c|c|}
  \hline
  $L$ & $g(L)$\\
  \hline
  Start & $\{ \textrm{CI\_}i\ |\ i\in q_0 \}$\\
  st\_$q$ & $\{ \textrm{C\_}q\textrm{\_}q'\ |\ q'\in Q, q\delta q' \}$\\
  \hline
\end{tabular}

Für den Ausführungsmodus(\emph{mode}) ergibt sich:

\begin{tabular}{|c|c|}
  \hline
  $L$ & $\textrm{mode}(L)$\\
  \hline
  Start & ilv\\
  CI\_$i$ & atm\\
  AI\_$i$ & atm\\
  st\_$q$ & ilv\\
  C\_$q$\_$q'$ & atm\\
  A\_$q$\_$q'$ & atm\\
  \hline
\end{tabular}

Zunächst ist es nützlich ein paar allgemeine Aussagen aufzustellen, die die Verifikation der Korrektheit der Übersetzung erleichtern.
Es ist leicht einzusehen, dass für alle Prozesse die Umgebung $\phi_e$ gleich ist, da die Prozesse keine lokalen Variablen deklarieren.
Daher kann für die Betrachtung der Gesamtumgebung des Promela-Systems die Umgebung eines beliebigen Prozesses herangezogen werden.

Um nun zu zeigen, dass das definierte GTL-System $(ms,cs,vs)$ bisimilar zum übersetzten Promela Modell ist, werden folgende Anforderungen an die Semantik gestellt:
\begin{enumerate}
\item Es muss eine Bijektion $i$ zwischen Zuständen der Verbindungen sowie Eingaben $s\in S_C(\mathcal{G})\times I(\mathcal{G})$ und der Promela-Umgebung $\sigma_e$ existieren.
\item Die Definitionen $\llbracket \alpha \rrbracket_D$ müssen eine initiale Umgebung $\sigma_e^0$ definieren, die isomorph zum Initialzustand $\alpha$ ist: 
  \[ i(\alpha) = \sigma_e^0 \]
\item Befinden sich beide Systeme in isomorphen Zuständen, so wird der von $\llbracket \rrbracket_C$ erzeugte Ausdruck genau dann wahr, wenn es im abstrakten Modell einen entsprechenden Übergang zwischen den Zuständen gibt:
  \begin{equation}
    i(s,\beta) = \sigma_e \Rightarrow \forall q'\in Q^a: \left(
      \begin{array}{c}
        \mathit{exec}(\llbracket \mu(q')\rrbracket_C,\sigma_e)) \\
        \Leftrightarrow\\
        (\forall q\in Q^a: \delta^a(q,(s|^a,\beta|^a))=(q',o))
    \end{array}\right)
    \label{eq:assert1}
  \end{equation}
\item Die Anweisung, die von $\llbracket \rrbracket_A$ generiert wird, darf nie blockieren und muss isomorphe Zustände beibehalten:
  \begin{equation}
    i(s,\beta) = \sigma_e \Rightarrow \forall q'\in Q^a: \left(
      \begin{array}{c}
        \xymatrix{ \left<\sigma_e,L\right> \ar[rr]^-{\llbracket \mu(q')\rrbracket_A} & & \left<\sigma_e',L'\right> } \\
        \Leftrightarrow\\
        (\forall q\in Q^a: \delta^a(q,(s|^a,\beta|^a)) = (q',o)\\
        \land\\
        i(s[a\mapsto o],\beta[a\mapsto o]) = \sigma_e')
      \end{array}\right)
     \label{eq:assert2}
  \end{equation}
\end{enumerate}
Nun kann man die Relation $\cong$ angeben, die Zustände des abstrakten Modells mit Zuständen des übersetzten Promela-Modells in Relation setzt.
Diese wird wie folgt definiert:
Zwei Zustände stehen genau dann in Relation, wenn ihre globalen Zustände isomorph sind und sich jeder Prozess des Promela-Modells am Label befindet, das mit dem Zustand im abstrakten Modell korrespondiert, oder sich am Label \emph{Start} befindet und der abstrakte Prozess im Zustand $\const{init}$ ist.
\[
\begin{array}{c}
  (q_0,\dots,q_N,c_0,\dots)\cong \gamma\\
  \Leftrightarrow\\
  i((c_0,\dots))=\gamma(0).\sigma_e\\
  \land\\
  \forall j\in\{1\dots N\}: (\gamma(j).\sigma_l = \textrm{st\_}q_j \lor (\gamma(j).\sigma_l = \textrm{Start}\land q_j=\const{init}))
\end{array}
\]
Nun muss gezeigt werden, dass es sich bei der eben definierten Relation tatsächlich um eine Bisimulationsrelation handelt.
Hierfür muss nachgewiesen werden, dass es für jede Transition, die ein bisimilarer Zustand durchführen kann, eine Transition des anderen Zustands gibt und die Zielzustände der beiden Transitionen auch wieder bisimilar sind.

Betrachten wir also einen Zustand des abstrakten Modells $s=(q_0,\dots,q_N,c_0,\dots)$ und einen Zustand des Promela-Modells $\gamma$.
Sind diese Zustände bisimular, so gilt nach Konstruktion
\[ i((c_0,\dots)) = \gamma(0).\sigma_e \]
und für jeden Prozesszustand $q_j$ entweder
\[ \gamma(j).\sigma_l = \textrm{st\_}q_j \]
oder
\[ \gamma(j).\sigma_l = \textrm{Start} \land q_j = \const{init} \]
Nun wird gezeigt, dass falls einer dieser Zustände eine Transition in einen neuen Zustand zulässt, es einen Übergang im abstrakten Modell gibt, der in einen Zustand führt, der mit dem neuen Zustand des Promela-Modells bisimilar ist.
Gilt der erste Fall, so lässt sich herleiten
\[ \inference[IfDo-proc]{
  \inference[Basic-proc]{\mathit{exec}(\textrm{C\_}q_j\textrm{\_}q_j',\gamma(j).\sigma_e) & \mathit{next}(\textrm{C\_}q_j\textrm{\_}q_j') = \textrm{A\_}q_j\textrm{\_}q_j'}{
  \xymatrix{ \left<\gamma(j).\sigma_e,\textrm{C\_}q_j\textrm{\_}q_j'\right>\ar@{|->}[r]^-{\llbracket \mu(q_j')\rrbracket_C} & _{proc}
    \left<\gamma(j).\sigma_e,\textrm{A\_}q_j\textrm{\_}q_j'\right>}
  }
  }
  { \xymatrix{ \left<\gamma(j).\sigma_e,\textrm{st\_}q_j\right> \ar@{|->}[r]^-{\llbracket \mu(q_j')\rrbracket_C} & _{proc}
      \left<\gamma(j).\sigma_e,\textrm{A\_}q_j\textrm{\_}q_j'\right>}
  }
\]
Weiterhin ist nach Voraussetzung bekannt, dass die Anweisung $\llbracket \mu(q_j')\rrbracket_A$ nie blockieren darf, also lässt sich herleiten
\[
  \xymatrix{ \left<\gamma(j).\sigma_e,\textrm{A\_}q_j\textrm{\_}q_j'\right> \ar@{|->}[r]^-{\llbracket \mu(q_j')\rrbracket_A} & _{proc}
    \left<\sigma_e',\textrm{st\_}q_j'\right> }
\]
Aus der Promela-Semantik lässt sich nun für ein $\gamma$ mit $\gamma(j).\sigma_l = \textrm{st\_}q_j$ herleiten:
\[
\inference[Atm-mod]{
  \inference[Single-int]{
    \xymatrix{ \left<\gamma(j).\sigma_e,\textrm{st\_}q_j\right> \ar@{|->}[r]^-{\llbracket \mu(q_j')\rrbracket_C} & _{proc}
      \left<\gamma(j).\sigma_e,\textrm{A\_}q_j\textrm{\_}q_j'\right> }
  }{
    \xymatrix{ \gamma \ar@{|->}[r]^-{\llbracket \mu(q_j')\rrbracket_C} & _{int}
      \gamma'
    }
  } & mode(\textrm{C\_}q_j\textrm{\_}q_j') = \mathit{atm}
}{
  \xymatrix{ \gamma \ar@{|->}[r]^-{\mathit{atm}_j} & _{mod}
    \gamma'
  }
}
\]
wobei $\gamma' = \gamma[\left<\gamma(j).\sigma_e,\textrm{A\_}q_j\textrm{\_}q_j'\right>/j]$.
Nach der gleichen Regel lässt sich auch herleiten
\[
\inference[Atm-mod]{
  \inference[Single-int]{
    \xymatrix{ \left<\gamma(j).\sigma_e,\textrm{A\_}q_j\textrm{\_}q_j'\right> \ar@{|->}[r]^-{\llbracket \mu(q_j')\rrbracket_A} & _{proc}
      \left<\sigma_e',\textrm{st\_}q_j'\right> }
  }{
    \xymatrix{ \gamma'\ar@{|->}[r]^-{\llbracket \mu(q_j')\rrbracket_A} & _{int}
      \gamma''
    }
  } & mode(\textrm{A\_}q_j\textrm{\_}q_j') = \mathit{atm}
}{
  \xymatrix{ \gamma'\ar@{|->}[r]^-{\mathit{atm}_j} & _{mod}
    \gamma''
  }
}
\]
wobei $\gamma'' = \gamma[\left<\sigma_e',\textrm{st\_}q_j'\right>/j]$.
Nun lassen sich die beiden atomaren Transitionen zusammenfassen:
\[
\inference[Atm-sim]{
  \xymatrix{ \gamma \ar@{|->}[r]^-{\mathit{atm}_j} & _{mod}
    \gamma'
  } &
  \xymatrix{ \gamma'\ar@{|->}[r]^-{\mathit{atm}_j} & _{mod}
    \gamma''
  }
}{
  \xymatrix{ \gamma \ar@{|->}[r]^-{\mathit{atm}_j} & _{sim}
    \gamma''
  }
}
\]
Fasst man nun alle hier angegebenen Ableitungsschritte zusammen, so ergibt sich
\[
\inference{
  exec(\textrm{C\_}q_j\textrm{\_}q_j',\gamma(j).\sigma_e) &
  \xymatrix{ \left<\gamma(j).\sigma_e,\textrm{A\_}q_j\textrm{\_}q_j'\right> \ar@{|->}[r]^-{\llbracket \mu(q_j')\rrbracket_A} & _{proc}
    \left<\sigma_e',\textrm{st\_}q_j'\right> }
}{
  \xymatrix{ \gamma \ar@{|->}[r]^-{\mathit{atm}_j} & _{sim}
    \gamma[\left<\sigma_e',\textrm{st\_}q_j'\right>/j]
  }
}
\]
Diese zwei Vorbedingungen sind nach Gleichung \ref{eq:assert1} und \ref{eq:assert2} genau dann erfüllt, wenn
\[ \delta^a(q_j,((c_0,\dots)|^a,\beta|^a)) = (q_j',o) \]
gilt.
Dies ist äquivalent dazu dass
\[ \lambda(((q_0,\dots,q_j,\dots,q_N),c),j,\beta) = (((q_0,\dots,q_j',\dots,q_N),c[a\mapsto o]),(\bot,\dots,\bot)[a\mapsto o]) \]
gilt.
Nach Gleichung \ref{eq:assert2} gilt außerdem, dass der veränderte globale Zustand $\sigma_e'$ isomorph zum abstrakten Zustand $c[a\mapsto o]$ ist:
\[ i(c[a\mapsto o],\beta) = \sigma_e' \]
womit die Bisimularität für diesen Fall gezeigt ist, denn
\[
\begin{array}{c}
  i(c[a\mapsto o],\beta) = \sigma_e' \land ((q_0,\dots,q_j,\dots,q_N),c)\cong\gamma\\
  \Rightarrow\\
  ((q_0,\dots,q_j',\dots,q_N),c[a\mapsto o])\cong \gamma[\left<\sigma_e',\textrm{st\_}q_j'\right>/j]
\end{array}
\]

Der Beweis für den Fall, dass sich der Prozess am Label \emph{Start} befindet ist, bis auf Änderung der Label-Namen analog und daher ausgelassen.

\chapter{Implementierung}


Die Implementierung besteht zum einen aus der eigentlichen Anwengung -- \emph{gtl} -- und zum anderen aus verschiedenen Bibliotheken, die zusätzlich entwickelt werden mussten.
Diese sind:
\begin{itemize}
\item \emph{language-promela} -- Stellt Datenstrukturen für den Promela-Syntax bereit, formatiert Promela-Quelltext für die Ausgabe und parst Promela-Code.
\item \emph{language-scade} -- Ein Parser und Code-Generator für den SCADE-Syntax.
\item \emph{bdd} -- Eine Bibliothek, die binäre Entscheidungsdiagramme ("`binary decision diagrams"' -- BDD) implementiert.
\end{itemize}

\begin{figure}[h]
  \centering
  \begin{tikzpicture}
  \node[tape,draw,fill=d1!40,tape bend top=none] (gtl) at (0,0) {GTL};
  \node[rectangle,draw,fill=d2!40] (parser) at (0,-1) {Parser};
  \node[rounded rectangle,draw,fill=d3!40] (gtl ast) at (0,-2) {GTL AST};
  \node[rectangle,draw,fill=d2!40] (type checker) at (0,-3) {Type checker};
  \node[tape,draw,fill=d1!40,tape bend top=none] (scade) at (4,-3) {SCADE};
  \node[rounded rectangle,draw,fill=d3!40] (gtl spec) at (0,-4) {GTL Spec};
  \node[tape,draw,fill=d1!40,tape bend top=none] (promela1) at (-3,-5) {Promela};
  \node[cylinder,draw,fill=d5!40,shape border rotate=90,aspect=0.25] (cudd) at (-5,-6) {CUDD};
  \node[cloud,draw,fill=d4!40,cloud ignores aspect,inner sep=0em] (dyn bdd) at (-3,-8) {\begin{tabular}{c}Dynamic\\BDD\\Verifier\end{tabular}};
  \node[tape,draw,fill=d1!40,tape bend top=none] (promela2) at (0,-5) {Promela};
  \node[rectangle,draw,fill=d2!40] (kcg) at (4,-4) {KCG};
  \node[tape,draw,fill=d1!40,tape bend top=none] (c) at (4,-5) {C};
  \node[cloud,draw,fill=d4!40,cloud ignores aspect,inner sep=0em] (native) at (0,-8) {\begin{tabular}{c}Native\\Verifier\end{tabular}};
  \node[tape,draw,fill=d1!40,tape bend top=none] (scade2) at (2,-5) {\begin{tabular}{c}SCADE\\Testnode\end{tabular}};
  \node[cloud,draw,fill=d4!40,cloud ignores aspect,inner sep=0em] (scade verifier) at (6,-7) {\begin{tabular}{c}SCADE\\Design\\Verifier\end{tabular}};
  \draw[->] (gtl) -- (parser);
  \draw[->] (parser) -- (gtl ast);
  \draw[->] (gtl ast) -- (type checker);
  \draw[->] (scade) -- (type checker);
  \draw[->] (type checker) -- (gtl spec);
  \draw[->] (gtl spec) -| (promela1);
  \draw[->] (gtl spec) -- (promela2);
  \draw[->] (gtl spec) -| (scade2);
  \draw[->] (cudd) -| (dyn bdd);
  \draw[->] (promela1) -- (dyn bdd);
  \draw[->] (scade) -- (kcg);
  \draw[->] (kcg) -- (c);
  \draw[->] (promela2) -- (native);
  \draw[->] (c) |- (native);
  \draw[->] (scade) -| (scade verifier);
  \draw[->] (scade2) |- (scade verifier);
\end{tikzpicture}

  \caption{GTL Implementierung}
  \label{fig:gtl_implementation}
\end{figure}

Abbildung \ref{fig:gtl_implementation} zeigt den Datenfluss der \emph{gtl}-Anwendung.
Zunächst wird mithilfe des Parsers eine textuelle GTL-Repräsentation in einen abstrakten Syntax-Baum\footnote{englisch: abstract syntax tree, AST} transformiert.
Der Parser wird im Abschnitt \ref{module:Language.GTL.Parser} beschrieben, der Syntax-Baum in \ref{module:Language.GTL.Parser.Syntax}.
Daraufhin wird der Syntax-Baum an die Typüberprüfung weiter gereicht.
Diese extrahiert die Typinformationen aus den verwendeten synchronen Komponenten und überprüft, ob alle Kontrakte und Verifikationsformeln wohl-getypt sind (Siehe Abschnitt \ref{sec:sos}).
Für das SCADE-Backend müssen also die definierenden Dateien geparst werden, die Modelle in dem entstandenen Syntax-Baum gefunden werden und die SCADE-Typen in GTL-Typen umgewandelt werden.
Daraufhin wird der Syntax-Baum in eine Instanz des \emph{GTLSpec}-Datentyps umgewandelt (Beschrieben in Abschnitt \ref{module:Language.GTL.Model}).

Ab hier entscheidet sich nun, welche Transformation vom Benutzer gewählt wurde.
Für die SCADE-Verifikation der Komponenten wird für jede Komponente ein SCADE-Testknoten erzeugt, der dann zusammen mit dem Quelltext der Modelle mit dem SCADE Design-Verifier geprüft wird (Beschrieben in Abschnitt \ref{module:Language.GTL.Backend.Scade}).

Für die C-Übersetzung wird der SCADE Code-Generator KCG aufgerufen, der wie in Abschnitt \ref{sec:c_integration} beschrieben C-Code für alle Modelle liefert.
Es wird dann Promela-Code generiert, der die einzelnen C-Code-Modelle vereint.
Die Implementierung dieses Verfahrens wird in Abschnitt \ref{module:Language.GTL.PromelaCIntegration} genauer beschrieben.

Für die Übersetzung der Kontrakte mithilfe von binären Entscheidungsdiagrammen, wie in Abschnitt \ref{sec:bdd} beschrieben, wird Promela-Code generiert und dann gegen die CUDD-Bibliothek gelinkt.
Genauere Details des Verfahrens sind in Abschnitt \ref{module:Language.GTL.PromelaDynamicBDD} angegeben.
\section{Binäre Entscheidungsdiagramme}
Versucht man, in einem Programm Funktionen genau wie Daten zu behandeln, so stößt man auf verschiedene Probleme:
\begin{itemize}
\item Wird die Funktion durch ihren Code repräsentiert, so ist die Darstellung nicht nur abhängig von der Wahl der Programmiersprache, sondern auch uneindeutig, da es in einer Turing-vollständigen Sprache unendlich viele Quelltexte gibt, die eine gegebene Funktion kodieren.
\item Verwendet man zur Repräsentation die Wertetabelle der Funktion, so ist zwar eine eindeutige Kodierung sichergestellt, allerdings kann schon die Kodierung einer Funktion mit sehr kleinem Wertebereich enorm viel Speicher veranschlagen (Die Kodierung einer Funktion von 32-bit Integer nach Bool würde zum Beispiel schon 0.5 GB Daten benötigen).
\end{itemize}
Um diese Probleme zu lösen, kann man binäre Entscheidungsdiagramme (BDD)\footnote{In der englisch-sprachigen Literatur als "`binary decision diagrams"' bezeichnet und mit BDD abgekürzt} verwenden.
Diese lassen sich verwenden, um Funktionen der Form
\[ f : \mathbb{B}^n\rightarrow\mathbb{B} \]
eindeutig zu kodieren (Wobei $n$ beliebig, aber endlich ist).

Ein binäres Entscheidungsdiagramm ist ein gerichteter, azyklischer Graph des folgenden Aufbaus:
\begin{itemize}
\item Den einfachsten Fall stellen die Diagramme dar, die nur aus den Symbolen $\top$ oder $\bot$ bestehen (Abbildung \ref{fig:easy_bdd}).
  \begin{figure}[!h]
    \centering
    \begin{tabular}{cc}
      \includegraphics[scale=.5]{top} & \includegraphics[scale=.5]{bot}
    \end{tabular}
    \caption{Die einfachsten zwei Entscheidungsdiagramme}
    \label{fig:easy_bdd}
  \end{figure}
  Diese repräsentieren die Funktion, die für alle Eingaben wahr ist ($\top$) und die Funktion, die für alle Eingaben unwahr ist ($\bot$).
\item Möchte man die Funktion, die sich, falls die Variable $\alpha$ wahr ist, wie die Funktion $f_1$ und ansonsten wie $f_2$ verhält, kodieren, so erstellt man einen neuen Knoten mit der Bezeichnung $\alpha$ und verbindet ihn mit einer durchzogenen Linie mit dem Diagramm für $f_1$ und mit einer gestrichelten Linie mit dem Diagramm von $f_2$ (Abbildung \ref{fig:con_bdd}).
  \begin{figure}[!h]
    \centering
    \includegraphics[scale=.5]{con}
    \caption{Zusammengesetztes Entscheidungsdiagramm}
    \label{fig:con_bdd}
  \end{figure}
\end{itemize}
Mit diesen einfachen Konstruktionsregeln lassen sich nun beliebige Funktionen konstruieren.
Beispielsweise kann die Funktion $(\alpha\land\beta)\lor \gamma$ wie in Abbildung \ref{fig:example1_bdd} kodiert werden.
\begin{figure}[h]
  \centering
  \includegraphics[scale=.5]{example1}
  \caption{Entscheidungsdiagramm der Funktion $(\alpha\land\beta)\lor \gamma$}
  \label{fig:example1_bdd}
\end{figure}
Diese Art der Kodierung hat nun aber das folgende Problem: Sie ist nicht eindeutig.
Beispielsweise stellt das Diagramm in Abbildung \ref{fig:example2_bdd} die selbe Funktion dar.
\begin{figure}[h]
  \centering
  \includegraphics[scale=.5]{example2}
  \caption{Äquivalentes Entscheidungsdiagramm der Funktion $(\alpha\land\beta)\lor \gamma$}
  \label{fig:example2_bdd}
\end{figure}
Das liegt daran, dass das Diagramm in Abbildung \ref{fig:example1_bdd} viele redundante Knoten enthält:
Beide Äste des linken $\gamma$-Knoten führen zum gleichen Knoten, dass heißt an dieser Stelle spielt die Belegung der Variablen keine Rolle.
Die anderen beiden $\gamma$-Knoten sind äquivalent, da ihre Kinderknoten gleich sind.
Entfernt man alle diese Redundanzen, so erhält man das \emph{reduzierte} Entscheidungsdiagramm in Abbildung \ref{fig:example3_bdd}.
\begin{figure}[h]
  \centering
  \includegraphics[scale=.5]{example3}
  \caption{Reduziertes Entscheidungsdiagramm der Funktion $(\alpha\land\beta)\lor \gamma$}
  \label{fig:example3_bdd}
\end{figure}

Die Bedingung, dass das Diagramm keine redundanten Knoten aufweisen darf, reicht allerdings noch nicht aus, um eine Eindeutigkeit zu erzwingen, wie das Diagramm in Abbildung \ref{fig:example4_bdd} zeigt.
\begin{figure}[h]
  \centering
  \includegraphics[scale=.5]{example4}
  \caption{Äquivalentes reduziertes Entscheidungsdiagramm der Funktion $(\alpha\land\beta)\lor \gamma$}
  \label{fig:example4_bdd}
\end{figure}

Um dieses Problem zu lösen, kann man fordern, dass die Diagramme zusätzlich auch \emph{geordnet} sind, dass heißt es gibt eine totale Ordnung auf den Variablen und Variablen höherer Ordnung haben Verbindungen zu Variablen niedriger Ordnung, aber nicht umgekehrt.
Das Diagramm in Abbildung \ref{fig:example3_bdd} hat also die Ordnung $\alpha > \beta > \gamma$, während in Abbildung \ref{fig:example4_bdd} die Ordnung $\gamma > \alpha > \beta$ eingehalten wird.

Die so eingeführten \emph{geordneten}, \emph{reduzierten} binären Entscheidungsdiagramme haben damit eine Reihe von Vorteilen gegenüber anderen Funktionskodierungen:
\begin{itemize}
\item Sie sind eindeutig, jede Funktion hat genau ein Entscheidungsdiagramm.
  Außerdem sind sie schon eindeutig über ihren Anfangsknoten definiert, so dass ein Test auf Gleichheit in konstanter Zeit möglich ist (Zum Vergleich: Bei der Kodierung als Wertetabelle benötigt man $2^n$ Vergleiche wobei $n$ die Anzahl der Variablen ist und bei der Kodierung als Quelltext ist ein Test auf Äquivalenz in vielen Fällen prinzipiell unmöglich\footnote{Gezeigt durch die Unentscheidbarkeit des Halteproblems\cite{halteproblem}}).
\item Eine Auswertung der Funktion ist effizient möglich:
  An jedem Knoten wird entschieden, ob die entsprechende Variable wahr oder falsch ist und der entsprechende Ast verfolgt.
  Endet man bei dem Knoten $\top$, so ist das Ergebnis der Funktion wahr, endet man bei $\bot$, so ist es falsch.
\item Für viele Funktionen ist das entsprechende Entscheidungsdiagramm sehr klein.
  Allerdings lässt sich zeigen, dass es Funktionen gibt, für die keine effiziente Repräsentation als Entscheidungsdiagramm existiert, wie zum Beispiel die Multiplikationsfunktion\cite{Bryant98onthe}.
\item Speichert man mehrere Entscheidungsdiagramme, so kann man den Speicherbedarf enorm verringern, indem man gleiche Knoten zwischen Diagrammen teilt.
\end{itemize}

In dem Rest dieser Arbeit sind mit dem Begriff "`Entscheidungsdiagramm"' immer geordnete, reduzierte und geteilte Entscheidungsdiagramme gemeint.
\section{Promela-C-Integration}
\begin{figure}
  \centering
  \begin{tikzpicture}
    \node[draw] (gtl) at (0,2) {GTL};
    \node[draw] (scade) at (3,2) {Scade};
    \node[draw] (promela) at (0,1) {Promela};
    \node[draw] (c) at (3,1) {C};
    \node[draw] (verifier) at (1.5,0) {Verifier};
    \draw[->,blue] (gtl) -- (promela);
    \draw[->,blue] (scade) -- (c);
    \draw[->,blue] (promela) |- (verifier);
    \draw[->,blue] (c) |- (verifier);
  \end{tikzpicture}
  \caption{Simulation durch C-Integration von Promela}
\end{figure}

Diese Übersetzungsmethode führt keine Optimierungen oder Abstraktionen durch, sondern simuliert das Modell exakt so, wie durch die Spezifikation angegeben.
Die Kontrakte für die synchronen Komponenten werden also ignoriert.
Um die Komponenten zu simulieren reichen die Informationen in der GTL-Spezifikation nicht aus, denn die Modell sind hier nur als Referenzen vorhanden.
Die eigentliche Implementierung findet in den synchronen Formalismen statt, also in dieser Arbeit in SCADE.
Die SCADE-Komponenten müssen also nach Promela übersetzt werden, damit das Gesamtmodell simuliert und verifiziert werden kann.
Eine direkte Übersetzung ist zwar prinzipiell möglich, aber aufgrund der Mächtigkeit der SCADE Sprache mit sehr viel Aufwand verbunden.
Einfacher ist es, die von SCADE angebotene C-Übersetzung zu verwenden und den generierten C-Code in Promela einzubinden, was SPIN seit Version 4 erlaubt.

Zunächst wird jede synchrone Komponente mit Hilfe des SCADE-Compilers nach C übersetzt.
Da der Übersetzungsprozess für jede Komponente einzeln durchgeführt werden muss, muss darauf geachtet werden, dass die Modelle keine Namenskonflikte aufweisen oder die gleichen Sub-Komponenten enthalten\cite{scade_c_integration}.

Der Compiler generiert nun für jede Komponente zwei Datenstrukturen, die erste enthält die Eingabevariablen und erhält das Namensprefix "`inC\_"', die zweite enthält die Ausgabevariablen sowie den internen Zustand der Komponente und ist mit dem Prefix "`outC\_"' versehen.
Außerdem werden zwei Funktionen erstellt:
Die erste initialisiert den internen Zustand der Komponente und hat das Suffix "`\_reset"', die zweite führt einen einzelnen Berechnungsschritt der Komponente durch und ist genau wie die Komponente benannt.

%Jede synchrone Komponente wird nun durch einen Promela-Prozess repräsentiert, der zunächst die Datenstrukturen initialisiert und dann in jedem Schritt die Eingabevariablen in die C-Datenstruktur kopiert, die Schrittfunktion aufruft und die Ergebnisse in die Ausgabevariablen schreibt.
Die Übersetzung geht nun wie folgt vor:
Zunächst wird für jeden Prozess eine Instanz der Zustands-Datenstruktur zum globalen Zustandsvektor hinzugefügt.
Dies geschieht über die Verwendung des "`c\_state"'-Konstruktes in Promela.
Eine Komponente "`Engine"' würde beispielsweise den folgenden Code generieren:
\begin{lstlisting}[language=promela]
c_state "outC_Engine Engine_state" "Global"
\end{lstlisting}
Dies fügt dem Zustandsvektor die Variable "`Engine\_state"' hinzu.
Das Schlüsselwort "`Global"' führt dazu, dass auf die Variable von jedem Prozess aus zugegriffen werden kann.

Die Eingabe-Datenstruktur der Komponenten ist für den Zustand des Gesamtsystems nicht entscheidend, sondern wird nur benötigt, um die Eingabesignale der Komponente vor dem Berechnungsschritt zu sammeln.
Deswegen ist es ausreichend, die Struktur mit dem "`c\_decl"'-Konstrukt zu deklarieren:
\begin{lstlisting}[language=promela]
c_decl {
  inC_Engine Engine_input;
}
\end{lstlisting}

Zur korrekten Simulation der synchronen Komponente müssen nun folgende Dinge geschehen:
\begin{enumerate}
\item Am Anfang der Simulation muss die Zustands-Datenstruktur der Komponente initialisiert werden.
  Hierfür generiert der Code-Generator eine "`reset"' Funktion:
  \begin{lstlisting}[language=promela]
c_code {
  Engine_reset(&now.Engine_state);
}
  \end{lstlisting}
\item In jedem Berechnungsschritt müssen die Eingaben für die Komponente aus den Aus\-ga\-be-Da\-ten\-struk\-tu\-ren der verbundenen Komponenten kopiert werden.
  \begin{lstlisting}[language=promela]
c_code {
  Engine_input.power = now.Power_state.on;
  Engine_input.mode = now.SpeedRegulator_state.output;
}
  \end{lstlisting}
\item Die generierte Schrittfunktion der Komponente muss aufgerufen werden.
  \begin{lstlisting}[language=promela]
c_code {
  Engine(&Engine_input,&now.Engine_state);
}
  \end{lstlisting}
\end{enumerate}
Die gesamte Übersetzung des vorgestellten Modells sieht also so aus:
\begin{lstlisting}[language=promela]
c_code {
  \#include "Engine.h"
}

c_state "outC_Engine Engine_state" "Global"

c_decl {
  inC_Engine Engine_input;
}

proctype Power { ... }
proctype SpeedRegulator { ... }

proctype Engine {
  c_code {
    Engine_reset(&now.Engine_state);
  };
  do
  :: c_code {
       Engine(&Engine_input,&now.Engine_state);
     }
  od
}
\end{lstlisting}
Problematisch bei der C-Übersetzung ist allerdings zu sehen, dass der übersetzte Code alle Ausgabevariablen der Komponente in den Zustandsraum der Komponente übernimmt, obwohl sie eventuell gar nicht für den Folgezustand der Komponente entscheidend sind.
Das kann dazu führen, dass die Verifikation mit der C-Übersetzung mehr Zustände generiert als es eine äquivalente, auf direkter Übersetzung basierende Übersetzung tun würde.
Das Problem kann für die C-Übersetzung nicht umgangen werden, da die Information, ob eine Ausgabevariable für den nächsten Zustand verantwortlich ist, tiefergehende Analysen des SCADE Modells erfordern würden, die auf einer quasi-Übersetzung des SCADE-Codes nach Promela gleich käme.
\section{SCADE Übersetzung}
Diese Übersetzungsmethode wird benutzt, um die Korrektheit der Kontrakte für synchrone Komponenten zu verifizieren.
Dazu wird für jede Komponente ein SCADE-Knoten erstellt, der dann im Design-Verifier der SCADE-Suite auf Korrektheit getestet werden kann.

\subsection{Abstraktion durch statische BDD}
Um die in der GTL-Spezifikation gegebenen Kontrakte für die Komponenten des GALS-Systems für die Verifikation benutzen zu können, müssen diese in den Promela-Formalismus übertragen werden.
Durch die LTL$\rightarrow$Büchi-Übersetzung (Siehe Abschnitt \ref{sec:ltl-translation}) liegen die abstrahierten Komponenten bereits als Büchi-Automaten vor.
Die Zustände der Automaten enthalten aber nicht nur konkrete Wertzuweisungen für die Variablen der Komponenten, sondern auch Einschränkungen der Wertebereiche, die sogar von anderen Variablen abhängen können (Zum Beispiel Relationen wie $x < y + 3$).
Um zumindest Einschränkungen der Wertebereiche, die nicht von anderen Variablen abhängen (Zum Beispiel $x < 5$), übersetzen zu können, kann man BDDs benutzen.

Dazu wird jede Relation der Form $x R c$, wobei $x$ eine Variable und $c$ eine Konstante ist in ein BDD übersetzt.
Das BDD kann nun statt der Werte, die es repräsentiert zur Verifikation verwendet werden.
Da die Verifikation allerdings keinen direkten Zugriff mehr auf die Information hat, welches BDD von einem Wert repräsentiert wird, muss vor der Verifikation eine Tabelle erstellt werden, in der gespeichert wird, welche BDD miteinander kompatibel sind, also zwischen welchen die Implikation "`$\Rightarrow$"' gilt.
\subsubsection{Korrektheit}
Um die formale Korrektheit der Übersetzung durch statische BDDs zu beweisen, müssen zunächst die Symbole $\Sigma$, $\llbracket\rrbracket_C$, $\llbracket\rrbracket_A$ sowie $\llbracket\rrbracket_D$ definiert werden.
Die Semantiksymbole $\Sigma$ sind hierbei eine Menge von Relationen, die höchstens eine Variable enthalten können, also z.B. "`$\{x<5,y=4\}$"'.

Durch einfache Umformungen lässt sich also jedes Semantiksymbol $\eta\in\Sigma$ als eine Abbildung von Variablen $V$ auf ein Entscheidungsdiagramm, dass die Relationen auf der entsprechenden Variable repräsentiert darstellen:
\[ \eta : V\rightarrow \mathbb{BDD} \]
Ist keine Relation für eine Variable angegeben, so ist die Variable unbeschränkt und durch das BDD repräsentiert, dass alle möglichen Werte enthält.

Die Funktion $\llbracket\rrbracket_C$ betrachtet die Eingabevariablen des übergebenen Semantiksymbols $\eta$, also die Menge
\[ \{ (v,\eta(v))\ |\ v\in \mathit{Inp}_i \} \]
Für jedes dieser Tupel wird nun ein Promela Ausdruck generiert, der überprüft, ob die Variable $v$ mit einem zu $\eta(v)$ kompatiblen BDD belegt ist.
Kompatibel bedeutet hierbei, dass die Gleichung $v\cap\eta(v)\neq \emptyset$ erfüllt ist.

Die Anweisungen, die von der $\llbracket\rrbracket_A$-Funktion generiert werden, weisen den Ausgabevariablen der Komponente neue BDDs zu.
Das bedeutet, dass für jedes Tupel der Menge
\[ \{ (v,\eta(v))\ |\ v\in \mathit{Out}_i \} \]
die Anweisung
\begin{lstlisting}[language=promela,mathescape=true]
  v = $\eta(v)$;
\end{lstlisting}
generiert wird.

Für jede Variable der Komponente wird durch $\llbracket\rrbracket_D$ eine globale Integer Variable generiert, die die Repräsentation des entsprechenden BDDs enthält.
Sind die Variablen von zwei Komponenten durch eine \emph{connect}-Deklaration verbunden, so wird nur eine Variable für die Eingangsvariable generiert, damit das Modell nicht unnötig vergrößert wird.
Schreibt der Ausgabeprozess auf seine Variable, so wird das Ergebnis stattdessen direkt in die gemeinsam verwendete Variable geschrieben.
\section{Abstraktion durch dynamische BDD}
\section{Fehlereingrenzung}
Das Ergebnis einer Verifikation, die die Kontrakt-Spezifikationen zur Optimierung verwendet sind eine oder mehrere Fehlerspuren.
Diese geben eine zeitlich geordnete Kette von Bedingungen über die Variablen der Komponenten des Systems.
Ein Beispiel für eine solche Spur ist die Kette
\[ \left[ (a<3,b\in \{3,5,6\}), (a > 4), (b\neq 5) \right] \]
Diese Kette von Bedingungen spezifiziert aber nicht ein Verhalten, sondern mehrere.
Die folgenden Verhalten des Systems sind beispielsweise spezifiziert:
\[ \left[ (a=2,b=3), (a=6), (b = 1) \right] \]
\[ \left[ (a=1,b=3), (a=5), (b = 1) \right] \]
Aus dieser Menge von spezifizierten Verhaltensweisen müssen nun nicht alle einen echten Fehler des Systems darstellen.
Tatsächlich reicht es, wenn ein Verhalten einen Fehler hervorruft.
Möglich ist aber auch, dass die Kontrakte dem System ein Verhalten erlauben, was das echte System niemals erzeugt.
In diesem Fall ist die Spezifikation des Systems zu grob und die Fehlerspuren nicht immer echte Fehler.

Um nun herauszufinden, welche konkreten Fehlerspuren das spezifizierte System nun tatsächlich hat, wird eine erneute Verifikation durchgeführt.
Diesmal wird das echte Systemverhalten als Grundlage herangezogen, wobei aber das Verhalten auf die Spuren begrenzt wird, die durch die Fehlerspur angegeben werden.
Meldet die Verifikation des so eingegrenzten Systems ebenfalls einen Fehler, so erhält man nicht nur die Bestätigung, dass die vorher erzeugte Fehlerspur echt ist, sondern auch ein konkretes Systemverhalten, dass zu einem Fehler führt.
Zeigt sich kein Fehler, so kann dies ein Hinweis sein, dass nicht genügend scharfe Kontrakte formuliert wurden.

\begin{figure}
  \centering
  \begin{tikzpicture}
    \node[draw] (gtl) at (0,10) {GTL};
    \node[draw] (scade) at (4.5,10) {Scade};
    \node[draw] (pr1) at (0,9) {\begin{tabular}{c}Promela\\(Abstraktion)\end{tabular}};
    \node[draw] (ver1) at (0,8) {Verifier};
    \node[draw] (trace1) at (0,7) {\begin{tabular}{c}Fehlerspur\\(abstrakt)\end{tabular}};
    \node[draw] (pr2) at (3.5,7) {\begin{tabular}{c}Promela\\(C-Integration)\end{tabular}};
    \node[draw] (pr3) at (1.5,5.5) {\begin{tabular}{c}Promela\\(eingeschränkt)\end{tabular}};
    \node[draw] (ver2) at (1.5,4) {Verifier};
    \node[draw] (trace2) at (1.5,3) {Fehlerspur};
    \draw[->,blue] (gtl) -- (pr1);
    \draw[->,blue] (pr1) -- (ver1);
    \draw[->,blue] (ver1) -- (trace1);
    \draw[->,blue] (trace1) -| (pr3);
    \draw[->,blue] (pr2) -| (pr3);
    \draw[->,blue] (gtl) -| (pr2);
    \draw[->,blue] (scade) -| (pr2);
    \draw[->,blue] (pr3) -- (ver2);
    \draw[->,blue] (ver2) -- (trace2);
  \end{tikzpicture}
  \caption{Fehlereingrenzung}
\end{figure}

\chapter{Fallbeispiele}
\chapter{Ausblick}
\bibliography{lit}

\end{document}
