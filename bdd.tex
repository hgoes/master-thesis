\section{Binäre Entscheidungsdiagramme}
Versucht man, in einem Programm Funktionen genau wie Daten zu behandeln, so stößt man auf verschiedene Probleme:
\begin{itemize}
\item Wird die Funktion durch ihren Code repräsentiert, so ist die Darstellung nicht nur abhängig von der Wahl der Programmiersprache, sondern auch uneindeutig, da es in einer Turing-vollständigen Sprache unendlich viele Quelltexte gibt, die eine gegebene Funktion kodieren.
\item Verwendet man zur Repräsentation die Wertetabelle der Funktion, so ist zwar eine eindeutige Kodierung sichergestellt, allerdings kann schon die Kodierung einer Funktion mit sehr kleinem Wertebereich enorm viel Speicher veranschlagen (Die Kodierung einer Funktion von 32-bit Integer nach Bool würde zum Beispiel schon 0.5 GB Daten benötigen).
\end{itemize}
Um diese Probleme zu lösen, kann man binäre Entscheidungsdiagramme (BDD)\footnote{In der englisch-sprachigen Literatur als "`binary decision diagrams"' bezeichnet und mit BDD abgekürzt} verwenden.
Diese lassen sich verwenden, um Funktionen der Form
\[ f : \mathbb{B}^n\rightarrow\mathbb{B} \]
eindeutig zu kodieren (Wobei $n$ beliebig, aber endlich ist).

Ein binäres Entscheidungsdiagramm ist ein gerichteter, azyklischer Graph des folgenden Aufbaus:
\begin{itemize}
\item Den einfachsten Fall stellen die Diagramme dar, die nur aus den Symbolen $\top$ oder $\bot$ bestehen (Abbildung \ref{fig:easy_bdd}).
  \begin{figure}[!h]
    \centering
    \begin{tabular}{cc}
      \includegraphics[scale=.5]{top} & \includegraphics[scale=.5]{bot}
    \end{tabular}
    \caption{Die einfachsten zwei Entscheidungsdiagramme}
    \label{fig:easy_bdd}
  \end{figure}
  Diese repräsentieren die Funktion, die für alle Eingaben wahr ist ($\top$) und die Funktion, die für alle Eingaben unwahr ist ($\bot$).
\item Möchte man die Funktion, die sich, falls die Variable $\alpha$ wahr ist, wie die Funktion $f_1$ und ansonsten wie $f_2$ verhält, kodieren, so erstellt man einen neuen Knoten mit der Bezeichnung $\alpha$ und verbindet ihn mit einer durchzogenen Linie mit dem Diagramm für $f_1$ und mit einer gestrichelten Linie mit dem Diagramm von $f_2$ (Abbildung \ref{fig:con_bdd}).
  \begin{figure}[!h]
    \centering
    \includegraphics[scale=.5]{con}
    \caption{Zusammengesetztes Entscheidungsdiagramm}
    \label{fig:con_bdd}
  \end{figure}
\end{itemize}
Mit diesen einfachen Konstruktionsregeln lassen sich nun beliebige Funktionen konstruieren.
Beispielsweise kann die Funktion $(\alpha\land\beta)\lor \gamma$ wie in Abbildung \ref{fig:example1_bdd} kodiert werden.
\begin{figure}[h]
  \centering
  \includegraphics[scale=.5]{example1}
  \caption{Entscheidungsdiagramm der Funktion $(\alpha\land\beta)\lor \gamma$}
  \label{fig:example1_bdd}
\end{figure}
Diese Art der Kodierung hat nun aber das folgende Problem: Sie ist nicht eindeutig.
Beispielsweise stellt das Diagramm in Abbildung \ref{fig:example2_bdd} die selbe Funktion dar.
\begin{figure}[h]
  \centering
  \includegraphics[scale=.5]{example2}
  \caption{Äquivalentes Entscheidungsdiagramm der Funktion $(\alpha\land\beta)\lor \gamma$}
  \label{fig:example2_bdd}
\end{figure}
Das liegt daran, dass das Diagramm in Abbildung \ref{fig:example1_bdd} viele redundante Knoten enthält:
Beide Äste des linken $\gamma$-Knoten führen zum gleichen Knoten, dass heißt an dieser Stelle spielt die Belegung der Variablen keine Rolle.
Die anderen beiden $\gamma$-Knoten sind äquivalent, da ihre Kinderknoten gleich sind.
Entfernt man alle diese Redundanzen, so erhält man das \emph{reduzierte} Entscheidungsdiagramm in Abbildung \ref{fig:example3_bdd}.
\begin{figure}[h]
  \centering
  \includegraphics[scale=.5]{example3}
  \caption{Reduziertes Entscheidungsdiagramm der Funktion $(\alpha\land\beta)\lor \gamma$}
  \label{fig:example3_bdd}
\end{figure}

Die Bedingung, dass das Diagramm keine redundanten Knoten aufweisen darf, reicht allerdings noch nicht aus, um eine Eindeutigkeit zu erzwingen, wie das Diagramm in Abbildung \ref{fig:example4_bdd} zeigt.
\begin{figure}[h]
  \centering
  \includegraphics[scale=.5]{example4}
  \caption{Äquivalentes reduziertes Entscheidungsdiagramm der Funktion $(\alpha\land\beta)\lor \gamma$}
  \label{fig:example4_bdd}
\end{figure}

Um dieses Problem zu lösen, kann man fordern, dass die Diagramme zusätzlich auch \emph{geordnet} sind, dass heißt es gibt eine totale Ordnung auf den Variablen und Variablen höherer Ordnung haben Verbindungen zu Variablen niedriger Ordnung, aber nicht umgekehrt.
Das Diagramm in Abbildung \ref{fig:example3_bdd} hat also die Ordnung $\alpha > \beta > \gamma$, während in Abbildung \ref{fig:example4_bdd} die Ordnung $\gamma > \alpha > \beta$ eingehalten wird.

Die so eingeführten \emph{geordneten}, \emph{reduzierten} binären Entscheidungsdiagramme haben damit eine Reihe von Vorteilen gegenüber anderen Funktionskodierungen:
\begin{itemize}
\item Sie sind eindeutig, jede Funktion hat genau ein Entscheidungsdiagramm.
  Außerdem sind sie schon eindeutig über ihren Anfangsknoten definiert, so dass ein Test auf Gleichheit in konstanter Zeit möglich ist (Zum Vergleich: Bei der Kodierung als Wertetabelle benötigt man $2^n$ Vergleiche wobei $n$ die Anzahl der Variablen ist und bei der Kodierung als Quelltext ist ein Test auf Äquivalenz in vielen Fällen prinzipiell unmöglich\footnote{Gezeigt durch die Unentscheidbarkeit des Halteproblems\cite{halteproblem}}).
\item Eine Auswertung der Funktion ist effizient möglich:
  An jedem Knoten wird entschieden, ob die entsprechende Variable wahr oder falsch ist und der entsprechende Ast verfolgt.
  Endet man bei dem Knoten $\top$, so ist das Ergebnis der Funktion wahr, endet man bei $\bot$, so ist es falsch.
\item Für viele Funktionen ist das entsprechende Entscheidungsdiagramm sehr klein.
  Allerdings lässt sich zeigen, dass es Funktionen gibt, für die keine effiziente Repräsentation als Entscheidungsdiagramm existiert, wie zum Beispiel die Multiplikationsfunktion\cite{Bryant98onthe}.
\item Speichert man mehrere Entscheidungsdiagramme, so kann man den Speicherbedarf enorm verringern, indem man gleiche Knoten zwischen Diagrammen teilt.
\end{itemize}

In dem Rest dieser Arbeit sind mit dem Begriff "`Entscheidungsdiagramm"' immer geordnete, reduzierte und geteilte Entscheidungsdiagramme gemeint.
Mehr zu Entscheidungsdiagrammen im allgemeinen lässt sich in \cite{knuth2011computer} finden.
\section{Datentypen als Entscheidungsdiagramme}
Da Enscheidungsdiagramme binäre Funktionen auf boolesche Werte kodieren, lassen sie sich auch verwenden, um Mengen von endlichen Datentypen zu kodieren.
Das ist möglich, weil sich jeder endliche Datentyp binär kodieren lässt (Notfalls durch Durchnummerierung aller möglichen Werte).
Die entsprechende Funktion gibt also \emph{wahr} zurück, wenn sich das Element, dass der binären Kodierung entspricht sich in der Menge befindet.

Ist also ein endlicher Datentyp $T$ zusammen mit einer Kodierungsfunktion $c : T\rightarrow \{0,1\}^n$ gegeben, so lässt sich ein BDD für eine beliebige Menge $Q\subseteq T$ konstruieren, indem man für jedes Element $q\in Q$ der Menge das BDD konstruiert, dass die Funktion repräsentiert, die nur bei dem Wert $c(q)$ wahr ist.
Die so generierten BDD werden dann per Disjunktion zu einem BDD zusammengefügt.
Diese Konstruktion ist aber extrem ineffizient, da sehr viele BDDs erstellt und sofort wieder verworfen werden.
Effizienter ist es, das BDD mit einem \emph{divide-and-conquer}-Algorithmus zu erstellen:
Dieser erhält eine Menge der noch zu kodierenden Werte sowie die Bitposition, an der gerade kodiert wird (beginnend bei Bit null).
Der Algorithmus teilt nun die Menge in zwei Mengen:
Die eine enthält die Werte deren Kodierung an der aktuellen Bitposition eine null aufweisen, die andere Menge enthält die restlichen Werte.
Auf diesen Mengen wird der Algorithmus nun rekursiv aufgerufen.
Die vollständige Implementierung ist in Abbildung \ref{alg:create_bdd} zu sehen.

\begin{figure}[h]
\centering
\begin{codebox}
\Procname{$\proc{CreateBDD}(Q,\id{Pos})$}
\li \If $Q\isequal\emptyset$\Then
\li \Return $\proc{Leaf}(0)$
\End
\li \If $\id{Pos}\isequal n$\Then
\li \Return $\proc{Leaf}(1)$
\End
\li $Q_l\gets \{ v\ |\ v\in Q, c(v)[\id{Pos}]\isequal 0 \}$
\li $Q_r\gets \{ v\ |\ v\in Q, c(v)[\id{Pos}]\isequal 1 \}$
\li \Return $\proc{Node}(\id{Pos},\proc{CreateBDD}(Q_l,\id{Pos}+1),\proc{CreateBDD}(Q_r,\id{Pos}+1))$
\end{codebox}
\caption{Eine Funktion, um endliche Mengen in BDDs umzuwandeln}
\label{alg:create_bdd}
\end{figure}

Um ein BDD nun wieder in eine Menge zu verwandeln, kann man das Diagramm rekursiv durchlaufen.
Die entsprechende Funktion erhält das aktuell betrachtete Teildiagramm sowie die Bitposition und den binären Wert der bis zu dieser Position kodiert wurde.
Enspricht das Diagramm dem Nullblatt, so bedeutet dies, dass vom Teildiagramm keine Werte kodiert werden, es wird also die leere Menge zurück gegeben.
Ist die Bitposition am Ende des kodierbaren Bereichs angelangt, so wird der aktuelle Wert von dem Teildiagramm kodiert und damit als einzelnes Element dekodiert und zurück gegeben.
Entspricht der aktuelle Hauptknoten des Diagramms nicht der aktuellen Bitposition, so ist der Wert an der aktuellen Bitposition egal, das Ergebnis ist also die Vereinigung der Aufrufe mit Wert null und eins an der aktuellen Bitposition.
Ansonsten wird der linke Ast des Knotens mit dem Wert eins belegt, der rechte mit null und die Ergebnisse der rekursiven Aufrufe vereinigt.
Der vollständige Algorithmus ist in Abbildung \ref{alg:decode_bdd} angegeben.

\begin{figure}[h]
\begin{codebox}
\Procname{$\proc{DecodeBDD}(\id{Tree},\id{Pos},\id{Value})$}
\li \If $\id{Tree}\isequal\proc{Leaf}(0)$\Then
\li \Return $\emptyset$
\li \ElseIf $\id{Pos}\isequal n$\Then
\li \Return $\{c^{-1}(\id{Value})\}$
\li \ElseIf $\id{Tree}\isequal\id{Leaf}(1)\mbox{ or }\attrib{Tree}{id}<\id{Pos}$\Then
\li \Return $\proc{DecodeBDD}(\id{Tree},\id{Pos}+1,\id{Value}\|(1<<\id{Pos}))\ \cup$
\Startalign{\Return}
\> $\proc{DecodeBDD}(\id{Tree},\id{Pos}+1,\id{Value})$
\Stopalign
\li \Else
\li \Return $\proc{DecodeBDD}(\attrib{Tree}{left},\id{Pos}+1,\id{Value}\|(1<<\id{Pos}))\ \cup$
\Startalign{\Return}
\> $\proc{DecodeBDD}(\attrib{Tree}{right},\id{Pos}+1,\id{Value})$
\Stopalign
\End
\end{codebox}
\caption{Algorithmus um BDDs in Mengen umzuwandeln}
\label{alg:decode_bdd}
\end{figure}

Im folgenden sollen sowohl Datentypen wie auch Algorithmen auf ihnen in BDD-Form realisiert werden.
\subsection{Integer}
Ganze Zahlen mit oder ohne Vorzeichen lassen sich sehr leicht binär kodieren.
Kodiert man natürliche Zahlen beispielsweise mit den drei Bits $b_0$, $b_1$ und $b_2$ (Wobei $b_0$ das niederwertigste Bit kodiert), so lässt sich die Menge $\{2,3,5\}$ auffassen als das Diagramm in Abbildung \ref{fig:example_set1_bdd}.
\begin{figure}[h]
  \centering
  \includegraphics[scale=.5]{example_set1}
  \caption{Entscheidungsdiagramm der Menge $\{2,3,5\}$}
  \label{fig:example_set1_bdd}
\end{figure}

Verwendet man Mengen, um verschiedene mögliche Werte einer Variable zu speichern, so kann es nützlich für die Geschwindigkeit der Verifikation sein, wenn man effiziente Algorithmen finden kann, um Operationen gleichzeitig auf den einzelnen Werten der Menge auszuführen.
Für eine binäre Operation $\circ$ wird also ein effizienter Algorithmus auf Entscheidungsdiagrammen gesucht, der für zwei Mengen $A$ und $B$ die Menge
\[ \{ a\circ b\ |\ a\in A,b\in B\} \]
berechnet.
Im folgenden wird die Implementierung einiger einfacher Operationen erläutert.

\subsubsection{Addition}
Um beispielsweise die elementweise Addition von zwei Mengen angeben zu können, definiert man zunächst eine Hilfsfunktion "`plus"', die zusätzlich noch einen Parameter enthält, die den Übertrag der letzten Bit-Addition speichert.
Addiert man nun zwei Diagramme, deren Hauptknoten die selbe Markierung $x$ haben, so ergibt sich ein neues Diagramm mit dem Hauptknoten der Markierung $x$, der wie folgt aufgebaut ist:
\[ \textrm{plus}\Bigg(\bdd{plus1_lhs},\bdd{plus1_rhs},0\Bigg) = \bdd{plus1_res} \]
Der linke Ast des Resultats muss hierbei den Fall behandeln, dass das Ergebnis der Addition im Bit $x$ den Wert 1 erhält.
Dies kann nur dann passieren, wenn genau eines der Argument-Bits 1 ist.
Hierfür gibt es trivialerweise zwei Möglichkeiten, bei der einen ist das erste Argument 1 und der linke Ast des ersten Argument wird ohne Überhang mit dem rechten Ast des zweiten Arguments addiert; bei der anderen Möglichkeit ist es genau andersrum.
Da beide Möglichkeiten auftreten können, müssen die Ergebnisse vereinigt werden, was auf Entscheidungsdiagramm-Ebene einer Disjunktion entspricht.
Für den rechten Ast des Resultats muss man nun die Möglichkeiten betrachten, bei denen das Bit $x$ den Wert 0 erhalten kann.
Dies kann geschehen, wenn beide Bits der Argumente 0 sind, oder wenn beide 1 sind.
Im letzteren Fall muss dann beim rekursiven Aufruf der Überhang hinzugefügt werden.

Wird bei der Addition bereits ein Überhang berücksichtigt, so ändert sich die Argumentation leicht, ist aber im wesentlichen symmetrisch:
\[ \textrm{plus}\Bigg(\bdd{plus1_lhs},\bdd{plus1_rhs},1\Bigg) = \bdd{plus2_res} \]

Im Fall, dass der Knoten eines der Argumente niedrigere Ordnung besitzt -- hier wird nur der Fall betrachtet, dass das zweite Argument niedrigere Ordnung hat, da die Addition symmetrisch ist -- kann man sich überlegen, dass das zweite Argument hierbei äquivalent zu einem Diagramm ist, bei dem ein mit $x$ markierter Knoten oben steht, dessen Äste beide zum ursprünglichen Diagramm führen.
Setzt man dies dann in die erste Gleichung ein, so erhält man (für $y<x$):
\[ \textrm{plus}\Bigg(\bdd{plus1_lhs},\underbrace{\bdd{plus3_rhs}}_R,0\Bigg) = \bdd{plus3_res} \]
Die Gleichung mit Übertrag ist genau äquivalent zu den vorherigen Fällen und daher weg gelassen.
Um das Ergebnis der Addition zweier BDD auszurechnen ruft man nun die definierte Hilfsfunktion mit dem Parameter 0 für den Übertrag auf.
\subsubsection{Subtraktion}
Für die Subtraktion muss der Code für die Addition nur leicht verändert werden:
\[ \textrm{minus}\Bigg(\bdd{plus1_lhs},\bdd{plus1_rhs},0\Bigg) = \bdd{minus1_res} \]
\[ \textrm{minus}\Bigg(\bdd{plus1_lhs},\bdd{plus1_rhs},1\Bigg) = \bdd{minus2_res} \]
\[ \textrm{minus}\Bigg(\bdd{plus1_lhs},\underbrace{\bdd{plus3_rhs}}_R,0\Bigg) = \bdd{plus3_res} \]
\subsection{Tupel}
Tupel lassen sich einfach binär kodieren, indem man die Binärkodierung aller Elemente hintereinander schreibt.
Diese Art der Kodierung hat den Vorteil, dass es einfach möglich ist, Restriktionen auf einzelnen Elementen zu formulieren.
Möchte man beispielsweise alle Tupel kodieren, deren zweites Element eine bestimmte Bedingung erfüllt, so reicht es, die Bedingung nur auf dem Element selbst zu übersetzen und alle Kennzeichnungen der Knoten in dem resultierenden Diagramm um die Bitbreite des ersten Elements zu erhöhen.