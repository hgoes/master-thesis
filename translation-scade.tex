\section{SCADE Übersetzung}
Diese Übersetzungsmethode wird benutzt, um die Korrektheit der Kontrakte für synchrone Komponenten zu verifizieren.
Dazu wird für jede Komponente ein SCADE-Observer erstellt, der dann im Design-Verifier der SCADE-Suite auf Korrektheit getestet werden kann.
Der generierte Observer überwacht alle Ein- und Ausgänge der Komponente und gibt einen booleschen Ausgabefluss zurück, der wahr ist, wenn die geforderte Eigenschaft erfüllt ist und falsch ist, wenn sie verletzt ist.
Der Design-Verifier kann nun benutzt werden, um sicher zu stellen, dass der ausgegebene Datenfluss immer wahr ist.

Der Büchi-Automat, der die zu verifizierende Eigenschaft der Komponente darstellt, wird hierfür in eine SCADE-Zustandsmaschine übersetzt.
Jeder Zustand der Maschine definiert einen Wert für die Resultat-Variable.
Die Transitionen von Zustand $A$ in Zustand $B$ erhalten genau die Bedingungen, die in Zustand $B$ gelten müssen.
Zusätzlich zu den vom Büchi-Automaten vorgesehenen Transitionen erhält jeder Zustand noch eine Transition in den Fehlerzustand:
In diesem hat die Resultat-Variable den Wert falsch.
Die Transition ist von niedrigster Priorität, wird also nur dann geschaltet, wenn keine andere Transition schalten kann.
\subsection{Korrektheit}
Um die Korrektheit der Übersetzung nach SCADE nachzuweisen kann die strukturelle operationelle Semantik von Lustre aus \cite{functional_lustre} verwendet werden.
Da die momentane Übersetzungsmethode aber auf die Generierung von Automaten angewiesen ist, welche von der operationellen Semantik nicht behandelt werden, ist der Korrektheitsbeweis der Übersetzung für spätere Arbeiten weg gelassen.