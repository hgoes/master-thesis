\section{Native Promela Übersetzung}
Diese Übersetzung verwendet die Kontrakte, um die Komponenten zu repräsentieren, aber führt keinerlei Optimierungen durch.

Jedes Atom in einem Zustand, dass eine Ausgabevariable $o$ determiniert, wird in ein \emph{Einschränkungs-Tupel} $(l_u,l_l,v_a,v_f,e_{eq},e_{neq})$ übersetzt, wobei die Variablen folgende Bedeutung haben:
\begin{itemize}
\item $l_u$ enthält alle Ausdrücke, die eine obere Schranke für die Variable angeben.
  Es gilt:
  \[ \forall l\in l_u: o < l \]
\item Analog enthält $l_l$ alle Ausdrücke, die eine untere Schranke darstellen.
\item $v_a$ enthält eine Menge von Werten, die von der Variable angenommen werden dürfen.
  Es gilt:
  \[ o\in v_a \]
\item Werte, die nicht angenommen werden dürfen, werden in $v_f$ gesammelt.
  \[ o\not\in v_f \]
\item $e_{eq}$ gibt eine Menge von Ausdrücken an, die gleich der Variable seien müssen:
  \[ \forall e\in e_{eq}: o=e \]
\item Genauso gibt $e_{neq}$ eine Menge von Ausdrücken an, die ungleich zu der Variable sein müssen:
  \[ \forall e\in e_{neq}: o\neq e \]
\end{itemize}
Wird eine Ausgabevariable von mehreren Atomen determiniert, so müssen die resultierenden Tupel zusammen geführt werden.
Gegeben zwei Tupel
\begin{align*}
  T_1 &= (l_u,l_l,v_a,v_f,e_{eq},e_{neq})\\
  T_2 &= (l_u',l_l',v_a',v_f',e_{eq}',e_{neq}')
\end{align*}
ergibt sich das Tupel, dass beide Einschränkungen erfasst als
\[ T_1\oplus T_2 = (l_u\cup l_u',l_l\cup l_l',v_a\cap v_a',v_f\cup v_f',e_{eq}\cup e_{eq}',e_{neq}\cup e_{neq}') \]
Für eine Ausgabevariable $o$ ergibt sich die Funktion $t$, die das entsprechende Einschränkungstupel berechnet durch
\begin{align*}
  t(o < e) &= (\{e\},\emptyset,\emptyset,\emptyset,\emptyset,\emptyset)\\
  t(o\leq e) &= (\{e+1\},\emptyset,\emptyset,\emptyset,\emptyset,\emptyset)\\
  t(o > e) &= (\emptyset,\{e\},\emptyset,\emptyset,\emptyset,\emptyset)\\
  t(o \geq e) &= (\emptyset,\{e-1\},\emptyset,\emptyset,\emptyset,\emptyset)\\
  t(o = e) &= (\emptyset,\emptyset,\emptyset,\emptyset,\{e\},\emptyset)\\
  t(o\neq e) &= (\emptyset,\emptyset,\emptyset,\emptyset,\emptyset,\{e\})\\
  t(o\in i) &= (\emptyset,\emptyset,i,\emptyset,\emptyset,\emptyset)\\
  t(o\not\in i) &= (\emptyset,\emptyset,\emptyset,i,\emptyset,\emptyset)\\
\end{align*}
Enthält ein Atom keine Ausgabevariable, so wird es in einen äquivalenten Promela-Ausdruck übersetzt.
Enthalten mehrere Atome keine Ausgabevariable, so werden die resultierenden Ausdrücke per Konjunktion verknüpft.

Eine Menge von Atomen $a$ wird also in ein Tupel $(p,e)$ aus einer Abbildung von Ausgabe-Variablen auf Einschränkungstupel $p$ und einem Promela-Ausdruck $e$ übersetzt.
Die Übersetzung $\llbracket \rrbracket_C$ liefert also nur den Promela-Ausdruck $e$, der die Bedingungen auf den Eingabe-Variablen repräsentiert:
\[ \llbracket a\rrbracket_C = e \]
Beispielsweise wird die Bedingung $x\leq 10\land x\neq y$ in den folgenden Promela-Code übersetzt:
\begin{lstlisting}[language=promela]
  x <= 10 && x!=y
\end{lstlisting}

Die Übersetzung der Einschränkungstupel gemäß der Semantik $\llbracket\rrbracket_A$ ist etwas komplizierter, da nicht-deterministisch alle möglichen Belegungen für die Ausgabevariable generiert werden müssen.
Dazu wird zunächst die Ausgabevariable $o$ auf die größte untere Schranke gesetzt, indem die Schranken miteinander verglichen werden.
Existiert keine untere Schranke, so wird der kleinste Wert des Wertebereichs der Variable gewählt.
Ansonsten wird mit Promela \emph{If}-Anweisungen ein Entscheidungsbaum aufgebaut.
Existieren also beispielsweise die drei unteren Schranken $l_l = \{ e1,e2,e3\}$, so sieht die Übersetzung wie folgt aus:
\begin{lstlisting}[language=promela,numbers=left,caption={Berechnung der unteren Schranke}]
if :: e1 < e2;
      if :: e1 < e3;
            o = e1
         :: else;
            o = e3
      fi
   :: else;
      if :: e2 < e3;
            o = e2
         :: else;
            o = e3
      fi
fi
\end{lstlisting}
Danach wird in einer \emph{Do}-Schleife die Ausgabevariable so lange hoch gezählt, bis eine der oberen Schranken erreicht wird.
Für jeden Wert muss nun noch geprüft werden, ob er die restlichen Bedingungen des Einschränkungstupels erfüllt.
Gilt also beispielsweise $l_u = \{ u1,u2 \}$ sowie $v_f=\{ f1, f2 \}$, so ergibt sich der folgende Code:
\begin{lstlisting}[language=promela,numbers=left,firstnumber=last,caption={Generierung von möglichen Werten}]
do :: o<u1 && o<u2;
      o = o+1
   :: o==f1 || o==f2;
      skip
   :: else;
      break
od
\end{lstlisting}
Da sich der gesamte Code-Block zur Ausgabe-Erzeugung innerhalb eines \emph{atomic}-Blocks befindet und der Ablauf an keiner Stelle blockieren kann, läuft der gesamte Code in einem Schritt ab.

Zusätzlich zu den Semantiken $\llbracket\rrbracket_C$ und $\llbracket\rrbracket_A$ muss noch die Bijektion $i$ und $\llbracket\rrbracket_D$ definiert werden (vgl. Seite \pageref{sec:translation-correctness}).
In dieser Übersetzung ist diese sehr einfach, da die Werte der Verbindungen direkt in Variablen gespeichert werden.
Die Bijektion konstruiert also nur eine Zuordnung $\sigma_e$, die die Werte für jede Verbindung aus dem Zustands- und Eingabevektor extrahiert.
Es gilt also:
\[ i((v_0,\dots,v_n),(v_{n+1},\dots,v_m)) = \sigma_e \]
wobei
\[ \sigma_e(j) \overset{def}{=} v_j \]
Da nur genau so viele Variablen deklariert werden, wie auch Verbindungen existieren ist es leicht einzusehen, dass $i$ in der Tat bijektiv ist.

Nun ist noch zu zeigen, dass $i$ die verlangten Eigenschaften 2.-4. von Seite \pageref{sec:bijection_conditions} aufweist.
Für ein $i(s,\beta)=\sigma_e$ muss nachgewiesen werden, dass die Äquivalenz aus (\ref{eq:assert1})
\[ \mathit{exec}(\llbracket \mu(q')\rrbracket_C,\sigma_e)\Leftrightarrow (\forall q\in Q^a: \delta^a(q,(s|^a,\beta|^a)) = (q',o)) \]
für alle Zustände $q'\in Q^a$ erfüllt ist.
Da $\llbracket\mu(q')\rrbracket_C$ eine Promela-Anweisung generiert, die nur ausgeführt werden kann, wenn die Belegungen in $\sigma_e$ den Bedingungen aus $\mu(q')$ entsprechen, existiert auch nur in diesem Fall ein Übergang zu dem Zustand $q'$.
Genauso kann die Anweisung $\llbracket\mu(q')\rrbracket_C$ nur dann ausführbar sein, wenn es einen entsprechenden Übergang gibt, da die Bedingungen aus dem Zustand $q'$ hierfür alle erfüllt sein müssen.

Desweiteren muss noch die Kompatibilität der Bijektion $i$ mit der Semantik $\llbracket\rrbracket_A$ gezeigt werden.
Hierzu muss die Korrektheit der Äquivalenz aus Gleichung (\ref{eq:assert2}) nachgewiesen werden.
Seien hierfür wieder $s$, $\beta$ und $\sigma_e$ mit 
\[ i(s,\beta) = \sigma_e \]
gegeben.
Existiert nun ein Übergang im Promela-Modell mit
\[ \xymatrix{ \left<\sigma_e,L\right> \ar[rr]^-{\llbracket \mu(q')\rrbracket_A} & & \left<\sigma_e',L'\right> } \]
so ergibt sich, dass hierbei genau die Variablen verändert werden, die in $\mu(q')$ vorkommen und Ausgabe-Variablen sind.
Nach der Konstruktion durch die Einschränkungstupel (Definition von $\llbracket\rrbracket_A$) ergibt sich außerdem, dass die nicht-deterministischen Zuweisungen alle innerhalb der durch das für die Übersetzung gegebene GTL-Modell definierten Wertebereiche liegen.

Umgekehrt sichert die Korrektheit der Übersetzung der Einschränkungstupel, dass für jeden Zustandsübergang des abstrakten Modells das Promela-Modell den gewünschten Übergang durchführen kann.

Die Default-Werte für die Variablen werden mit der Semantik $\llbracket\rrbracket_D$ definiert.
Diese weist den angegebenen Verbindungen einfach den angegebenen Wert zu.
Für die initiale Umgebung gilt dann:
\[ \llbracket \alpha\rrbracket_D \overset{def}{=} \sigma_e \]
wobei
\[ \sigma_e(j) = \alpha(j) \]
es ist leicht einzusehen, dass diese Definition die Forderung aus Gleichung \ref{eq:assert0} erfüllt.