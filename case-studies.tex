\section{Quelle-Senke Beispiel}
Dieses minimalistische Beispiel soll die grundsätzliche Funktionsweise des Verifikationsalgorithmus und der BDD-Optimierung erläutern und demonstrieren, wie durch die Optimierung der Zustandsraum von Modellen reduziert werden kann.

Das System besteht aus zwei Komponenten:
Die Quelle hat einen Ausgang und produziert gibt die Sequenz der natürlichen Zahlen modulo 10 zurück, also
\[ 0,1,2,3,4,5,6,7,8,9,0,1,2,\dots \]
Der Ausgang der Quelle ist mit der Eingabe der Senke verbunden.
Diese prüft, ob der Wert an ihrem Eingang kleiner als 12 ist und setzt dann den Ausgang auf den Wert 0, andernfalls auf 1.
Zu verifizieren ist in diesem Beispiel, dass der Ausgabewert der Senke stehts 0 ist.

Für den Kontrakt bietet sich also an zu spezifizieren, dass die Quelle stets Werte produziert, die kleiner als 10 sind.
Die Senke produziert für Werte kleiner 10 stets die Ausgabe 0, was sich ebenfalls als Kontrakt formulieren lässt.
Die resultierende GTL-Beschreibung sieht wie folgt aus:
\begin{lstlisting}[language=gtl]
model[scade] source("source_sink.scade","Source") {
  contract always outp < 10;
  init outp 9;
}

model[scade] sink("source_sink.scade","Sink") {
  init outp 0;
  contract always (inp < 10 => outp=0);
}

connect source.outp sink.inp;

verify {
  always sink.outp=0;
}
\end{lstlisting}
Das Kontrakt-Automaten System, dass sich aus dieser Beschreibung ergibt ist in Abbildung \ref{fig:source_sink_automata} gezeigt.

\begin{figure}[h]
  \centering
  \begin{tikzpicture}
    \draw[color=d3,fill=d3!40,thick] (178.82bp,1.5bp) -- (216.82bp,1.5bp) -- (216.82bp,63.5bp) -- (178.82bp,63.5bp) -- (178.82bp,1.5bp);
\draw[color=d3,fill=d3!40,thick] (216.82bp,1.5bp) -- (356.82bp,1.5bp) -- (356.82bp,63.5bp) -- (216.82bp,63.5bp) -- (216.82bp,1.5bp);
\draw[color=d3,fill=d3!40,thick] (356.82bp,1.5bp) -- (404.82bp,1.5bp) -- (404.82bp,63.5bp) -- (356.82bp,63.5bp) -- (356.82bp,1.5bp);
\draw (197.82bp,32.5bp) node {inp};
\draw (380.82bp,32.5bp) node {outp};
\begin{scope}[shift={(220.375bp,6.2044999999999995bp)}]
\draw [color=d1,very thick,fill=d1!40](88.888bp,39.091bp) ellipse (25.99992bp and 12.49992bp);
\draw (88.888bp,39.091bp) node {$\begin{array}{c}inp\geq 10\end{array}$};
\draw [color=d1,very thick,fill=d1!40](27.0bp,32.748bp) ellipse (25.99992bp and 12.49992bp);
\draw (27.0bp,32.748bp) node {$\begin{array}{c}outp=1\end{array}$};
\draw [fill](63.529bp,3.0bp) ellipse (2.000016bp and 2.000016bp);
\draw [-,thick] (111.53bp,45.642bp) .. controls (122.82bp,46.548bp) and (132.89bp,44.365bp) .. (132.89bp,39.091bp);
\draw [-,thick] (132.89bp,39.091bp) .. controls (132.89bp,35.548bp) and (128.34bp,33.4bp) .. (121.93bp,32.646bp);
\draw [-,thick] (63.359bp,36.475bp) .. controls (63.134bp,36.452bp) and (62.909bp,36.429bp) .. (62.683bp,36.406bp);
\draw [-,thick] (52.53bp,35.365bp) .. controls (52.754bp,35.388bp) and (52.979bp,35.411bp) .. (53.205bp,35.434bp);
\draw [-,thick] (49.639bp,39.299bp) .. controls (60.929bp,40.205bp) and (71.0bp,38.022bp) .. (71.0bp,32.748bp);
\draw [-,thick] (71.0bp,32.748bp) .. controls (71.0bp,29.205bp) and (66.454bp,27.057bp) .. (60.046bp,26.304bp);
\draw [-,thick] (64.618bp,4.5508bp) .. controls (66.427bp,7.1253bp) and (70.259bp,12.578bp) .. (74.354bp,18.407bp);
\draw [-,thick] (61.959bp,4.2783bp) .. controls (59.405bp,6.3579bp) and (54.054bp,10.716bp) .. (48.288bp,15.411bp);
\draw [-latex,thick] (121.93bp,32.646bp) -- (111.53bp,32.541bp);
\draw [-latex,thick] (62.683bp,36.406bp) -- (52.471bp,35.359bp);
\draw [-latex,thick] (53.205bp,35.434bp) -- (63.417bp,36.481bp);
\draw [-latex,thick] (60.046bp,26.304bp) -- (49.639bp,26.198bp);
\draw [-latex,thick] (74.354bp,18.407bp) -- (80.338bp,26.923bp);
\draw [-latex,thick] (48.288bp,15.411bp) -- (40.394bp,21.84bp);
\end{scope}

\draw[color=d3,fill=d3!40,thick] (1.0bp,14.5bp) -- (117.0bp,14.5bp) -- (117.0bp,50.5bp) -- (1.0bp,50.5bp) -- (1.0bp,14.5bp);
\draw[color=d3,fill=d3!40,thick] (117.0bp,14.5bp) -- (165.0bp,14.5bp) -- (165.0bp,50.5bp) -- (117.0bp,50.5bp) -- (117.0bp,14.5bp);
\draw (141.0bp,32.5bp) node {outp};
\begin{scope}[shift={(5.840000000000003bp,19.0bp)}]
\draw [color=d1,very thick,fill=d1!40](59.321bp,13.5bp) ellipse (29.00016bp and 12.49992bp);
\draw (59.321bp,13.5bp) node {$\begin{array}{c}outp<10\end{array}$};
\draw [fill](3.0bp,13.5bp) ellipse (2.000016bp and 2.000016bp);
\draw [-,thick] (83.985bp,20.085bp) .. controls (95.848bp,20.898bp) and (106.32bp,18.703bp) .. (106.32bp,13.5bp);
\draw [-,thick] (106.32bp,13.5bp) .. controls (106.32bp,9.9229bp) and (101.37bp,7.7675bp) .. (94.422bp,7.0339bp);
\draw [-,thick] (4.8739bp,13.5bp) .. controls (7.6696bp,13.5bp) and (13.321bp,13.5bp) .. (19.967bp,13.5bp);
\draw [-latex,thick] (94.422bp,7.0339bp) -- (83.985bp,6.9148bp);
\draw [-latex,thick] (19.967bp,13.5bp) -- (30.196bp,13.5bp);
\end{scope}

\draw [-,thick] (165.0bp,32.5bp) .. controls (165.0bp,32.5bp) and (166.63bp,32.5bp) .. (168.78bp,32.5bp);
\draw [-latex,thick] (168.78bp,32.5bp) -- (178.82bp,32.5bp);


  \end{tikzpicture}
  \caption{Quelle-Senke Kontrakt-Automaten}
  \label{fig:source_sink_automata}
\end{figure}

Wenig erstaunlich ist nun das Ergebnis der Verifikation:
Da die BDD-Abstraktion die Bedingung $\mathit{outp}<10$ als eine Transition betrachtet, ist das resultierende Transitionssystem in der BDD-Verifikation bedeutend kleiner als in der C-Integration (Siehe Tabelle \ref{tab:source_sink_verifikation}).

\begin{table}
  \begin{tabular}{|l|r|r|r|}
    \hline
    \textbf{Modus} & \textbf{Zustände} & \textbf{Transitionen} & \textbf{Speicherverbrauch}\\
    \hline
    Native & 25 & 49 & 4,653 MB\\
    BDD & 6 & 11 & 4,653 MB\\
    \hline
  \end{tabular}
  \caption{Quelle-Senke Verifikationsergebnisse}
  \label{tab:source_sink_verifikation}
\end{table}
