\section{LTL -- Linear temporal logic}
LTL-Formeln können benutzt werden, um das zeitabhängige Verhalten von Systemen zu beschreiben~\cite{ltlbasics}.
Die LTL-Logik stellt eine Erweiterung der Aussagenlogik dar, so dass nicht nur Aussagen über den jetzigen Zustand getroffen werden können, sondern auch über noch folgende Zustände.
Die Aussagenlogik wird um folgende Operatoren erweitert:
\begin{itemize}
\item Der \emph{next}-Operator (Auch mit $\bigcirc$ bezeichnet) sagt aus, dass eine Formel im nächstens Zustand gilt.
  Die Formel $\bigcirc\varphi$ spezifiziert also Pfade der Form
  \[ \xymatrix @R=0em {
      \bullet \ar[r] & \bullet \ar[r] & \bullet \ar@{-->}[r] &\\
      & \varphi & &
  }
    \]
\item Mit dem \emph{always}-Operator ($\square$) versehene Formeln gelten sowohl im aktuellen wie auch in allen folgenden Zuständen.
  Die Formel $\square\varphi$ erlaubt also alle Pfade der Form
  \[ \xymatrix @R=0em {
      \bullet \ar[r] & \bullet \ar[r] & \bullet \ar@{-->}[r] &\\
      \varphi & \varphi & \varphi &
  }
    \]
\item Der \emph{finally}-Operator ($\diamond$) gibt an, dass eine Formel irgendwann in der Zukunft einmal gelten wird.
  Die Formel $\diamond\varphi$ spezifiziert zum Beispiel Pfade der Form
  \[ \xymatrix @R=0em {
    \bullet \ar[r] & \bullet \ar@{-->}[r] & \bullet \ar[r] & \bullet \ar@{-->}[r] & \\
    & & \varphi & &
  } \]
\item Formeln die gelten sollen, bis eine bestimmte Bedingung erfüllt ist, lassen sich mit dem \emph{until}-Operator ($U$) angeben.
  Wird beispielsweise gefordert, dass die Formel $\varphi$ gilt, bis $\psi$ gilt, so lässt sich dies schreiben als $\varphi U\psi$.
  Ein Beispielpfad für diese Formel ist
  \[ \xymatrix @R=0em {
    \bullet \ar[r] & \bullet \ar@{-->}[r] & \bullet \ar[r] & \bullet \ar@{-->}[r] & \\
    \varphi & \varphi & \varphi & \psi &
  } \]
\end{itemize}
Um die vollständige Mächtigkeit von LTL zu erreichen reicht es allerdings auch schon, nur die Operatoren $\bigcirc$ und $U$ zu haben, denn der \emph{finally}-Operator lässt sich ausdrücken als
\[ \diamond\varphi = \top U \varphi \]
und der \emph{always}-Operator als
\[ \square\varphi = \lnot (\top U \lnot\varphi) \]
Alle anderen Operatoren sind also zwar nützlich, aber nicht benötigt.