Mit Hilfe von Kontrakten lässt sich das Verhalten von jeder Komponente in einem GALS-System beschreiben.
Die Komponenten sind in der synchronen Modellierungssprache SCADE (siehe Abschnitt \ref{sec:scade}) modelliert.
Eine Verifikation eines solchen GALS-Systems muss nun zwei Aufgaben erfüllen:
\begin{enumerate}
\item Die Korrektheit der angegebenen Kontrakte muss überprüft werden.
  Die Kontrakte sind korrekt, wenn das unter liegende SCADE Modell die Kontrakt-Formel erfüllt.
  Um diese Eigenschaft nachzuweisen wird der SCADE Design Verifier verwendet.
\item Es muss nachgewiesen werden, dass das Zusammenspiel der Kontrakte das globale Verifikationsziel erfüllt.
  Diese Eigenschaft wird überprüft, indem das System aus Komponent-Kontrakten in den Promela Formalismus übersetzt wird und die Gültigkeit der Formel mit SPIN nachgewiesen wird.
\end{enumerate}
\begin{figure}[h]
  \centering
  \begin{tikzpicture}
  \node[tape,draw,tape bend top=none] (gtl) at (0,0) {GTL};
  \node[rectangle,draw] (parser) at (0,-1) {Parser};
  \node[rounded rectangle,draw] (gtl ast) at (0,-2) {GTL AST};
  \node[rectangle,draw] (type checker) at (0,-3) {Type checker};
  \node[tape,draw,tape bend top=none] (scade) at (4,-3) {SCADE};
  \node[rounded rectangle,draw] (gtl spec) at (0,-4) {GTL Spec};
  \node[tape,draw,tape bend top=none] (promela1) at (-3,-5) {Promela};
  \node[cylinder,draw,shape border rotate=90,aspect=0.25] (cudd) at (-5,-6) {CUDD};
  \node[cloud,draw,cloud ignores aspect,inner sep=0em] (dyn bdd) at (-3,-8) {\begin{tabular}{c}Dynamic\\BDD\\Verifier\end{tabular}};
  \node[tape,draw,tape bend top=none] (promela2) at (0,-5) {Promela};
  \node[rectangle,draw] (kcg) at (4,-4) {KCG};
  \node[tape,draw,tape bend top=none] (c) at (4,-5) {C};
  \node[cloud,draw,cloud ignores aspect,inner sep=0em] (native) at (0,-8) {\begin{tabular}{c}Native\\Verifier\end{tabular}};
  \node[tape,draw,tape bend top=none] (scade2) at (2,-5) {\begin{tabular}{c}SCADE\\Testnode\end{tabular}};
  \node[cloud,draw,cloud ignores aspect,inner sep=0em] (scade verifier) at (6,-7) {\begin{tabular}{c}SCADE\\Verifier\end{tabular}};
  \draw[->] (gtl) -- (parser);
  \draw[->] (parser) -- (gtl ast);
  \draw[->] (gtl ast) -- (type checker);
  \draw[->] (scade) -- (type checker);
  \draw[->] (type checker) -- (gtl spec);
  \draw[->] (gtl spec) -| (promela1);
  \draw[->] (gtl spec) -- (promela2);
  \draw[->] (gtl spec) -| (scade2);
  \draw[->] (cudd) -| (dyn bdd);
  \draw[->] (promela1) -- (dyn bdd);
  \draw[->] (scade) -- (kcg);
  \draw[->] (kcg) -- (c);
  \draw[->] (promela2) -- (native);
  \draw[->] (c) |- (native);
  \draw[->] (scade) -| (scade verifier);
  \draw[->] (scade2) |- (scade verifier);
\end{tikzpicture}

  \caption{GTL Implementierung}
  \label{fig:gtl_implementation2}
\end{figure}
