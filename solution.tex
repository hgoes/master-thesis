In dieser Arbeit sollen Kontrakte verwendet werden, um das Zu\-stands\-ex\-plo\-sions-Pro\-blem~\cite{state_explosion} bei der Verifikation von großen GALS-Systemen zu verhindern.

Mit Hilfe von Kontrakten lässt sich das Verhalten von jeder Komponente in einem GALS-System beschreiben.
Die Komponenten sind in der synchronen Modellierungssprache SCADE (siehe Abschnitt \ref{sec:scade}) modelliert.
Die Kontrakte treffen Aussagen über die Ein- und Ausgangsvariablen der Komponenten.
Kommunikation zwischen den Komponenten wird ermöglicht, indem Ausgabevariablen mit Eingabevariablen verbunden werden.
Eine Verbindung gibt dabei immer die Ursprungskomponente und ihre Ausgabevariable sowie die Zielkomponente mit ihrer Eingabevariable an.
Diese Zusammenhänge sind in Abbildung \ref{fig:nomenclature} gezeigt.
\begin{figure}
  \centering
  \includegraphics[scale=.33]{nomenclature}
  \caption{GTL Nomenklatur}
  \label{fig:nomenclature}
\end{figure}

Grafisch lässt sich eine einzelne Komponente in Anlehnung an die Notationssprache UML~\cite{uml} wie in Abbildung \ref{fig:example_generator} darstellen.
\begin{figure}
  \centering
  \includegraphics[scale=.33]{generator}
  \caption{Beispielkomponente}
  \label{fig:example_generator}
\end{figure}
Die abgebildete Komponente stellt dabei einen Generator dar, der einen booleschen Eingang "`on"' und einen numerischen Ausgang "`output"' besitzt.
Der Kontrakt sagt aus, dass falls der Eingang auf "`wahr"' gesetzt wird, der Ausgang immer höchstens den Wert 10 ausgibt.

Eine Verifikation eines solchen GALS-Systems muss nun zwei Aufgaben erfüllen:
\begin{enumerate}
\item Die Korrektheit der angegebenen Kontrakte muss überprüft werden.
  Die Kontrakte sind korrekt, wenn das unter liegende SCADE Modell die Kontrakt-Formel erfüllt.
  Um diese Eigenschaft nachzuweisen wird der SCADE Design Verifier verwendet.
\item Es muss nachgewiesen werden, dass das Zusammenspiel der Kontrakte das globale Verifikationsziel erfüllt.
  Diese Eigenschaft wird überprüft, indem das System aus Komponent-Kontrakten in den Promela Formalismus übersetzt wird und die Gültigkeit der Formel mit SPIN nachgewiesen wird.
\end{enumerate}

Die erste Eigenschaft wird überprüft, indem in SCADE ein Wrapper um die Komponente konstruiert wird, der alle Eingaben der Komponente repliziert und auf den Ausgaben die Tests durchführt, die der Kontrakt spezifiziert.
Der entsprechende Wrapper für die Beispielkomponente aus Abbildung \ref{fig:example_generator} ist in Abbildung \ref{fig:example_generator_test} skizziert.
\begin{figure}
  \centering
  \includegraphics[scale=.33]{generator_test}
  \caption{SCADE-Wrapper für Beispielkomponente}
  \label{fig:example_generator_test}
\end{figure}

Für den Nachweis der zweiten Eigenschaft ist es erforderlich, die Kontrakte in Komponenten zu übersetzen, die jedes Verhalten zeigen können, dass der Kontrakt ihnen vorgibt.
Hierfür werden im Promela-Formalismus nicht-deterministische Automaten konstruiert, deren Zusammenspiel dann mit SPIN überprüft wird.

\section{Offene Fragen}
Mit dem in dieser Arbeit vorgestellten Ansatz lassen sich Sicherheitseigenschaften von GALS-Systemen gut nachweisen, da für Sicherheitseigenschaften wichtig ist, dass Komponenten ein oder mehrere bestimmte Verhalten \emph{nicht} zeigen.
Die Kontrakte stellen daher auch sicher, dass eine Komponente nicht mehr Verhalten zeigt, als es der Kontrakt zulässt.

Bei der Verifikation von Lebendigkeitseigenschaften stellt sich jedoch ein Problem:
Lebendigkeitseigenschaften fordern, dass eine Komponente ein bestimmtes Verhalten garantiert besitzt.
Da die Korrektheit eines Kontraktes bedeutet, dass der Kontrakt mindestens so viel Verhalten wie die Komponente selbst besitzt, kann der Kontrakt Verhalten erlauben, dass die Lebendigkeitseigenschaft erfüllt.
Das bedeutet aber nicht zwangsläufig, dass die Komponente dieses Verhalten auch besitzt.

Dieses Problem wird in dieser Arbeit aber nicht behandelt sondern es wird davon ausgegangen, dass alle zu verifizierenden Eigenschaften Sicherheitseigenschaften sind.