Mit Hilfe von Kontrakten lässt sich das Verhalten von jeder Komponente in einem GALS-System beschreiben.
Die Komponenten sind in der synchronen Modellierungssprache SCADE (siehe Abschnitt \ref{sec:scade}) modelliert.
Die Kontrakte treffen Aussagen über die Ein- und Ausgangsvariablen der Komponenten.
Kommunikation zwischen den Komponenten wird ermöglicht, indem Ausgabevariablen mit Eingabevariablen verbunden werden.
Eine Verbindung gibt dabei immer die Ursprungskomponente und ihre Ausgabevariable sowie die Zielkomponente mit ihrer Eingabevariable an.
Diese Zusammenhänge sind in Abbildung \ref{fig:nomenclature} gezeigt.
\begin{figure}
  \centering
  \includegraphics[scale=.33]{nomenclature}
  \caption{GTL Nomenklatur}
  \label{fig:nomenclature}
\end{figure}

Grafisch lässt sich eine einzelne Komponente in Anlehnung an die Notationssprache UML~\cite{uml} wie in Abbildung \ref{fig:example_generator} darstellen.
\begin{figure}
  \centering
  \includegraphics[scale=.33]{generator}
  \caption{Beispielkomponente}
  \label{fig:example_generator}
\end{figure}
Die abgebildete Komponente stellt dabei einen Generator dar, der einen booleschen Eingang "`on"' und einen numerischen Ausgang "`output"' besitzt.
Der Kontrakt sagt aus, dass falls der Eingang auf "`wahr"' gesetzt wird, der Ausgang immer höchstens den Wert 10 ausgibt.

Eine Verifikation eines solchen GALS-Systems muss nun zwei Aufgaben erfüllen:
\begin{enumerate}
\item Die Korrektheit der angegebenen Kontrakte muss überprüft werden.
  Die Kontrakte sind korrekt, wenn das unter liegende SCADE Modell die Kontrakt-Formel erfüllt.
  Um diese Eigenschaft nachzuweisen wird der SCADE Design Verifier verwendet.
\item Es muss nachgewiesen werden, dass das Zusammenspiel der Kontrakte das globale Verifikationsziel erfüllt.
  Diese Eigenschaft wird überprüft, indem das System aus Komponent-Kontrakten in den Promela Formalismus übersetzt wird und die Gültigkeit der Formel mit SPIN nachgewiesen wird.
\end{enumerate}
