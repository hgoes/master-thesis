Mit Hilfe von Kontrakten lässt sich das Verhalten von jeder Komponente in einem GALS-System beschreiben.
Die Komponenten sind in der synchronen Modellierungssprache SCADE (siehe Abschnitt \ref{sec:scade}) modelliert.
Eine Verifikation eines solchen GALS-Systems muss nun zwei Aufgaben erfüllen:
\begin{enumerate}
\item Die Korrektheit der angegebenen Kontrakte muss überprüft werden.
  Die Kontrakte sind korrekt, wenn das unter liegende SCADE Modell die Kontrakt-Formel erfüllt.
  Um diese Eigenschaft nachzuweisen wird der SCADE Design Verifier verwendet.
\item Es muss nachgewiesen werden, dass das Zusammenspiel der Kontrakte das globale Verifikationsziel erfüllt.
  Diese Eigenschaft wird überprüft, indem das System aus Komponent-Kontrakten in den Promela Formalismus übersetzt wird und die Gültigkeit der Formel mit SPIN nachgewiesen wird.
\end{enumerate}
\begin{figure}[h]
  \centering
  \includegraphics[scale=.33]{nomenclature}
  \caption{GTL Nomenklatur}
  \label{fig:nomenclature}
\end{figure}
