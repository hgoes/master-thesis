\section{Übersetzungskonstruktion}
Gegeben ein wie in Abschnitt \ref{sec:sos_defs} spezifiziertes System $s\in \mathcal{S}$ mit $s=(ms,c,vs)$ muss nun eine Übersetzung in ein äquivalentes Promela-Modell gefunden werden, die die Semantik des Systems erhält.
Für jede Komponente $(m,(\mathit{contr},\mathit{init}))\in ms$ mit dem Namen $m$, dem Kontrakt $\mathit{contr}$ und der Initialisierung $\mathit{init}$ wird nun der Kontrakt in einen äquivalenten Büchi-Automaten $(Q,\Sigma,\delta,\mu,q_0,\emptyset)$ übersetzt (Siehe Abschnitt \ref{sec:ltl-translation}).
Hierbei ist zu beachten, dass die generierten Automaten Bedingungen auf den Variablen als Ein- und Ausgabesymbole verwenden, da Relationen wie $x\leq y$ von dem LTL-Übersetzungsalgorithmus als atomare Aussagen betrachtet werden.


Anschließend wird ein neuer Prozess wie folgt definiert:
\begin{lstlisting}[language=Promela,mathescape=true,numbers=left,numberstyle=\small]
  active proctype $m$() {
    if $[ \forall i\in q_0:$
    :: atomic {
      $\llbracket \mu(i) \rrbracket_C$;
      $\llbracket \mu(i) \rrbracket_A$;
      goto st_$i$
    }
    $]$
    fi;
    $[ \forall q\in Q:$
    st_$q$:
      if $[ \forall q'\in Q, q\delta q':$
      :: atomic {
        $\llbracket \mu(q') \rrbracket_C$;
        $\llbracket \mu(q') \rrbracket_A$;
        goto st_$q'$
      }
      $]$
    $]$
  }
\end{lstlisting}
Hierbei gibt $\llbracket\rrbracket_C$ einen Promela-Ausdruck an, der abhängig vom globalen Zustand testet, ob alle Bedingungen, die durch die übergebenen Atome spezifiziert sind, erfüllt sind.
Die Anweisung muss blockieren, bis die Bedingungen erfüllt sind und darf den globalen Zustand nicht verändern.
Ähnlich dazu generiert die Funktion $\llbracket\rrbracket_A$ eine Anweisung, die den globalen Zustand anhand der übergebenen Atome transformiert.
Die generierte Anweisung darf nicht blockieren.

Die zu verifizierende Formel $v$ wird negiert ebenfalls in einen Büchi-Automaten $(Q,\Sigma,\delta,\mu,q_0,F)$ übersetzt und in eine äquivalente Promela \emph{never}-Deklaration übersetzt:
\begin{lstlisting}[language=Promela,mathescape=true,numbers=left,numberstyle=\small]
  never {
    if $[ \forall i\in q_0:$
    :: atomic {
      $\llbracket \mu(i) \rrbracket_C$;
      goto st_$i$
    }
    $]$
    fi;
    $[ \forall q\in Q:$
    $[ q\in F:$ accept_$q$: $]$
    st_$q$:
      if $[ \forall q'\in Q,q\delta q':$
      :: atomic {
        $\llbracket \mu(q') \rrbracket_C$;
        goto st_$q'$
      }
      $]$
    $]$
  }
\end{lstlisting}
\subsection{Richtigkeit der Übersetzung}
Um zu beweisen, dass die angegebene Promela-Übersetzung korrekt ist, muss gezeigt werden, dass eine Semantik des Systems, repräsentiert durch eine Untermenge des vollständigen Transitionssystems $T'\subseteq T$ (Siehe Abschnitt \ref{sec:semantic}), mit der Semantik des übersetzten Promela-Modells übereinstimmt.
Um dies zu zeigen, wird die Promela-Semantik verwendet, wie sie in \cite{Gallardo04formalaspects} beschrieben ist.
Da in dieser Semantik gefordert ist, dass jede Anweisung ein implizites Label erhält, werden folgende Labels für die Anweisungen vergeben:
\begin{itemize}
\item Die \emph{If}-Anweisung in Zeile 2 erhält das Label \emph{Start}
\item In diesem Zweig wird der Ausdruck in Zeile 4 mit dem Label \emph{CI\_$n$} versehen
\item Die folgende Anweisung in Zeile 5 wird mit \emph{AI\_$n$} kenntlich gemacht
\item Der Ausdruck in diesem Zweig auf Zeile 14 bekommt die Kennzeichnung \emph{C\_$q$\_$q'$}
\item Die folgende Anweisung in Zeile 15 erhält das Label \emph{A\_$q$\_$q'$}
\end{itemize}

Mit diesen Kennzeichnungen ergibt sich nun die \emph{next}-Funktion der Semantik wie folgt:

\begin{tabular}{|c|c|}
  \hline
  $L$ & $\textrm{next}(L)$\\
  \hline
  CI\_$n$ & AI\_$n$\\
  AI\_$n$ & st\_$i$\\
  C\_$q$\_$q'$ & A\_$q$\_$q'$\\
  A\_$q$\_$q'$ & st\_$q'$\\
  \hline
\end{tabular}

Auch die erforderliche $g$-Funktion, die die Zweige einer \emph{If}-Anweisung angibt, kann somit hergeleitet werden als:

\begin{tabular}{|c|c|}
  \hline
  $L$ & $g(L)$\\
  \hline
  Start & $\{ \textrm{CI\_}0,\textrm{CI\_}1,\dots \}$\\
  st\_$q$ & $\{ \textrm{C\_}q\textrm{\_}0,\textrm{C\_}q\textrm{\_}1,\dots \}$\\
  \hline
\end{tabular}

Für den Ausführungsmodus(\emph{mode}) ergibt sich:

\begin{tabular}{|c|c|}
  \hline
  $L$ & $\textrm{mode}(L)$\\
  \hline
  Start & ilv\\
  CI\_$n$ & atm\\
  AI\_$n$ & atm\\
  st\_$q$ & ilv\\
  C\_$q$\_$q'$ & atm\\
  A\_$q$\_$q'$ & atm\\
  \hline
\end{tabular}

Zunächst ist es nützlich ein paar allgemeine Aussagen aufzustellen, die die Verifikation der Richtigkeit der Übersetzung erleichtern.
\begin{enumerate}
\item Es ist leicht einzusehen, dass für alle Prozesse die Umgebung $\phi_e$ gleich ist, da die Prozesse keine lokalen Variablen deklarieren.
\end{enumerate}

Um nun zeigen zu können, dass das definierte System $(\Sigma,D,V)$ mit der Semantik $(S,\llbracket \rrbracket_S,\alpha)$ bisimular zum übersetzten Promela Modell $tr(\Sigma,D,V)$ ist, müssen folgende Anforderungen an die Semantik gestellt werden:
\begin{enumerate}
\item Es muss eine Bijektion $i$ zwischen Zuständen $S$ und der Promela-Umgebung $\sigma_e$ existieren.
\item Die Definitionen $\llbracket \alpha \rrbracket_D$ müssen eine initiale Umgebung $\sigma_e^0$ definieren, die isomorph zum Initialzustand $\alpha$ ist: 
  \[ i(\alpha) = \sigma_e^0 \]
\item Befinden sich beide Systeme in isomorphen Zuständen, so wird der von $\llbracket \rrbracket_C$ erzeugte Ausdruck genau dann wahr, wenn es im abstrakten Modell einen entsprechenden Übergang zwischen den Zuständen gibt:
  \[ \begin{array}{rc}
    i(s) = \sigma_e \Rightarrow &
    (s,(p_1,\dots,p_i,\dots,p_N))\rightarrow (s',(p_1,\dots,p_i',\dots,p_N)) \\
    & \Leftrightarrow\\
    & exec(\llbracket \delta(p_i')\rrbracket_C,\sigma_e)
  \end{array} \]
\item Die Anweisung, die von $\llbracket \rrbracket_A$ generiert wird, darf nie blockieren und muss isomorphe Zustände beibehalten:
  \[ \begin{array}{rc}
    i(s) = \sigma_e \Rightarrow &
    (s,(p_1,\dots,p_i,\dots,p_N))\rightarrow (s',(p_1,\dots,p_i',\dots,p_N))\\
    & \Leftrightarrow\\
    & \xymatrix { \left<\sigma_e,L\right> \ar[r]^-{\llbracket \delta(p_i')\rrbracket_A} & \left<\sigma_e',L'\right> } \land i(s') = \sigma_e'
  \end{array} \]
\end{enumerate}
Nun kann man die Relation $\cong$ angeben, die Zustände des abstrakten Modells mit Zuständen des übersetzten Promela-Modells in Relation setzt.
Diese wird wie folgt definiert:
Zwei Zustände stehen genau dann in Relation, wenn ihre globalen Zustände isomorph sind und sich jeder Prozess des Promela-Modells am Label befindet, das mit dem Zustand im abstrakten Modell korrespondiert, oder sich am Label \emph{Start} befindet und der abstrakte Prozess im Zustand $\bot$ ist.
\[
\begin{array}{c}
  (s,(p_1,\dots,p_N))\cong \gamma\\
  \Leftrightarrow\\
  i(s)=\gamma(0).\sigma_e \land \forall i\in\{1\dots N\}: (\gamma(i).\sigma_l = \textrm{st\_}p_i \lor (\gamma(i).\sigma_l = \textrm{Start}\land p_i=\bot))
\end{array}
\]
Nun muss gezeigt werden, dass es sich bei der eben definierten Relation tatsächlich um eine Bisimulationsrelation handelt.
Hierfür muss nachgewiesen werden, dass es für jede Transition, die ein bisimularer Zustand durchführen kann, eine Transition des anderen Zustands gibt und die Zielzustände der beiden Transitionen auch wieder bisimular sind.

Betrachten wir also einen Zustand des abstrakten Modells $(s,(p_1,\dots,p_N))$ und einen Zustand des Promela-Modells $\gamma$.
Sind diese Zustände bisimular, so gilt nach Konstruktion
\[ i(s) = \gamma(0).\sigma_e \]
und für jeden Prozess $p_i$ entweder
\[ \gamma(i).\sigma_l = \textrm{st\_}p_i \]
oder
\[ \gamma(i).\sigma_l = \textrm{Start} \land p_i = \bot \]
Gilt der erste Fall, so lässt sich herleiten
\[ \inference[IfDo-proc]{
  \inference[Basic-proc]{exec(\textrm{C\_}p_i\textrm{\_}p_i',\gamma(i).\sigma_e) & next(\textrm{C\_}p_i\textrm{\_}p_i') = \textrm{A\_}p_i\textrm{\_}p_i'}{
  \xymatrix{ \left<\gamma(i).\sigma_e,\textrm{C\_}p_i\textrm{\_}p_i'\right>\ar@{|->}[r]^-{\llbracket \delta(p_i')\rrbracket_C} & _{proc}
    \left<\gamma(i).\sigma_e,\textrm{A\_}p_i\textrm{\_}p_i'\right>}
  }
  }
  { \xymatrix{ \left<\gamma(i).\sigma_e,\textrm{st\_}p_i\right> \ar@{|->}[r]^-{\llbracket \delta(p_i')\rrbracket_C} & _{proc}
      \left<\gamma(i).\sigma_e,\textrm{A\_}p_i\textrm{\_}p_i'\right>}
  }
\]
Weiterhin ist nach Voraussetzung bekannt, dass die Anweisung $\llbracket \delta(p_i')\rrbracket_A$ nie blockieren darf, also lässt sich herleiten
\[
  \xymatrix{ \left<\gamma(i).\sigma_e,\textrm{A\_}p_i\textrm{\_}p_i'\right> \ar@{|->}[r]^-{\llbracket \delta(p_i')\rrbracket_A} & _{proc}
    \left<\sigma_e',\textrm{st\_}p_i'\right> }
\]
Aus der Promela-Semantik lässt sich nun für ein $\gamma$ mit $\gamma(i).\sigma_l = \textrm{st\_}p_i$
\[
\inference[Atm-mod]{
  \inference[Single-int]{
    \xymatrix{ \left<\gamma(i).\sigma_e,\textrm{st\_}p_i\right> \ar@{|->}[r]^-{\llbracket \delta(p_i')\rrbracket_C} & _{proc}
      \left<\gamma(i).\sigma_e,\textrm{A\_}p_i\textrm{\_}p_i'\right> }
  }{
    \xymatrix{ \gamma \ar@{|->}[r]^-{\llbracket \delta(p_i')\rrbracket_C} & _{int}
      \gamma'
    }
  } & mode(\textrm{C\_}p_i\textrm{\_}p_i') = atm
}{
  \xymatrix{ \gamma \ar@{|->}[r]^-{atm_i} & _{mod}
    \gamma'
  }
}
\]
wobei $\gamma' = \gamma[\left<\gamma(i).\sigma_e,\textrm{A\_}p_i\textrm{\_}p_i'\right>/i]$.
Nach der gleichen Regel lässt sich auch herleiten
\[
\inference[Atm-mod]{
  \inference[Single-int]{
    \xymatrix{ \left<\gamma(i).\sigma_e,\textrm{A\_}p_i\textrm{\_}p_i'\right> \ar@{|->}[r]^-{\llbracket \delta(p_i')\rrbracket_A} & _{proc}
      \left<\sigma_e',\textrm{st\_}p_i'\right> }
  }{
    \xymatrix{ \gamma'\ar@{|->}[r]^-{\llbracket \delta(p_i')\rrbracket_A} & _{int}
      \gamma''
    }
  } & mode(\textrm{A\_}p_i\textrm{\_}p_i') = atm
}{
  \xymatrix{ \gamma'\ar@{|->}[r]^-{atm_i} & _{mod}
    \gamma''
  }
}
\]
wobei $\gamma'' = \gamma[\left<\sigma_e',\textrm{st\_}p_i'\right>/i]$.
Nun lassen sich die beiden atomaren Transitionen zusammenfassen:
\[
\inference[Atm-sim]{
  \xymatrix{ \gamma \ar@{|->}[r]^-{atm_i} & _{mod}
    \gamma'
  } &
  \xymatrix{ \gamma'\ar@{|->}[r]^-{atm_i} & _{mod}
    \gamma''
  }
}{
  \xymatrix{ \gamma \ar@{|->}[r]^-{atm_i} & _{sim}
    \gamma''
  }
}
\]
Fasst man nun alle hier angegebenen Ableitungsschritte zusammen, so ergibt sich
\[
\inference{
  exec(\textrm{C\_}p_i\textrm{\_}p_i',\gamma(i).\sigma_e) &
  \xymatrix{ \left<\gamma(i).\sigma_e,\textrm{A\_}p_i\textrm{\_}p_i'\right> \ar@{|->}[r]^-{\llbracket \delta(p_i')\rrbracket_A} & _{proc}
    \left<\sigma_e',\textrm{st\_}p_i'\right> }
}{
  \xymatrix{ \gamma \ar@{|->}[r]^-{atm_i} & _{sim}
    \gamma[\left<\sigma_e',\textrm{st\_}p_i'\right>/i]
  }
}
\]
Diese zwei Vorbedingungen sind genau dann erfüllt, wenn
\[ (s,(p_1,\dots,p_i,\dots,p_N))\rightarrow (s',(p_1,\dots,p_i',\dots,p_N)) \]
gilt.
Außerdem gilt dann $i(s') = \sigma_e'$ und die Bisimularität ist für diesen Fall gezeigt, denn
\[
\begin{array}{c}
  i(s') = \sigma_e' \land (s,(p_1,\dots,p_i,\dots,p_N))\cong\gamma\\
  \Rightarrow\\
  (s',(p_1,\dots,p_i',\dots,p_N))\cong \gamma[\left<\sigma_e',\textrm{st\_}p_i'\right>/i]
\end{array}
\]
Der Beweis für den Fall, dass sich der Prozess am Label \emph{Start} befindet ist, bis auf Änderung der Label-Namen äquivalent und daher ausgelassen.
