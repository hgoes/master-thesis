\section{Übersetzungskonstruktion}
Ein spezifiziertes System mit $N$ Prozessen wird dargestellt als
\[ (\Sigma,D,V) \]
mit den Variablen $V$, der Semantiksymbole $\Sigma$ und den Prozessen
\[ D = (P_1,\dots,P_N) \]
Jeder Prozess ist spezifiziert durch
\[ P_i = (Q_i,\delta_i,\Delta_i,I_i,F_i,Inp_i,Outp_i) \]
wobei $Q_i$ die Zustandsmenge des Prozesses ist, $\delta_i: Q_i\rightarrow \Sigma$ die Semantik für jeden Zustand angibt, $\Delta_i\subseteq Q_i\times Q_i$ die Übergänge zwischen Zuständen darstellt, $I_i,F_i\subseteq Q_i$ die Initial- beziehungsweise Finalzustände kodiert und $Inp_i,Outp_i\subseteq V$ die Ein- und Ausgabevariablen des Prozesses angeben.

Die Semantik eines Systems wird durch ein Tupel
\[ (S,\llbracket \rrbracket_S,\alpha) \]
angegeben.
Hierbei stellt $S$ die globale Zustandsmenge dar, $\llbracket \rrbracket_S:\Sigma\rightarrow (S\rightarrow S\cup \{\bot\})$ gibt eine Übersetzung der Semantiksymbole in Zustandsübergänge an und $\alpha\in S$ kennzeichnet den initialen globalen Zustand.

Der aktuelle Zustand eines Systems lässt sich dann angeben als Kombination des globalen, semantikabhängigen Zustands $s\in S$ und den Prozesszuständen $p_i\in Q_i$:
\[ (s,(p_1,\dots,p_N)) \]
Die Zustandsübergänge dieses Systems folgen nun einer einfachen Regel
\[ \inference[step]{p_i\Delta p_i'&\llbracket \delta_i(p_i')\rrbracket_S(s)\neq\bot}{(s,(p_1,\dots,p_i,\dots,p_n))\rightarrow (\llbracket\delta_i(p_i')\rrbracket_S(s),(p_1,\dots,p_i',\dots,p_n))} \]
Der Initialzustand ist gegeben durch
\[ \inference[init]{}{(\alpha,(\bot,\dots,\bot))} \]
Ein Prozess im initialen Zustand kann in jeden Startzustand schalten:
\[ \inference[start]{p_i=\bot&p_i'\in I_i&\llbracket \delta_i(p_i')\rrbracket_S(s)\neq\bot}{(s,(p_1,\dots,p_i,\dots,p_N))\rightarrow (\llbracket \delta_i(p_i')\rrbracket_S(s),(p_1,\dots,p_i',\dots,p_N))} \]

Um ein solches Modell nun nach Promela zu übersetzen, benötigt man zusätzlich noch drei weitere semantikspezifische Konstrukte:
\begin{enumerate}
\item Eine Funktion $\llbracket\rrbracket_C : \Sigma\rightarrow \textrm{PrExpr}$, die Semantiksymbole in Promela-Ausdrücke übersetzt, die genau dann ausgeführt werden können, wenn nach der Semantik ein Zustand betreten werden darf.
\item Eine Funktion $\llbracket\rrbracket_A : \Sigma\rightarrow \textrm{PrStmt}^{*}$, die für jedes Semantiksymbol die Anweisungen generiert, die den globalen Zustand transformieren.
  Die Anweisungen dürfen unter keinen Umständen blockieren.
\item Außerdem eine Funktion $\llbracket\rrbracket_D : S\rightarrow \textrm{PrDecl}^{*}$, die des globalen Startzustand in eine Folge von Promela-Deklarationen übersetzt.
  Diese sind allerdings in so fern eingeschränkt, dass sie nur Variablen oder Kanäle definieren dürfen, aber keine zusätzlichen Prozesse.
\end{enumerate}

Mit diesen Hilfsmitteln lässt sich nun die Übersetzung des Modells nach Promela darstellen als eine Funktion $tr(\Sigma,D,V)$, mit der folgenden Definition:
\begin{lstlisting}[language=Promela,mathescape=true,numbers=left,numberstyle=\small]
$\llbracket \alpha \rrbracket_D$
$[ \forall P=(Q,\delta,\Delta,I,F,Inp,Outp)\in D:$
active proctype $P$() {
  if $[ \forall i\in I:$
  :: atomic {
       $\llbracket \delta(i) \rrbracket_C$;
       $\llbracket \delta(i) \rrbracket_A$;
       goto st_$i$
     }
     $]$
  fi;
  $[ \forall q\in Q:$
  st_$q$: if $[ \forall q'\in Q, q\Delta q':$
        :: atomic {
             $\llbracket \delta(q) \rrbracket_C$;
             $\llbracket \delta(q) \rrbracket_A$;
             goto st_$q'$
           }
           $]$
        fi
  $]$
}
$]$
\end{lstlisting}
\subsection{Richtigkeit der Übersetzung}
Um zu beweisen, dass die angegebene Promela-Übersetzung korrekt ist, muss gezeigt werden, dass die eben definierte Semantik des Modells mit der Semantik des übersetzten Promela-Modells übereinstimmt.
Um dies zu zeigen, wird die Promela-Semantik verwendet, wie sie in \cite{Gallardo04formalaspects} beschrieben ist.
Da in dieser Semantik gefordert ist, dass jede Anweisung ein implizites Label erhält, werden folgende Labels für die Anweisungen vergeben:
\begin{itemize}
\item Die \emph{If}-Anweisung in Zeile 4 erhält das Label \emph{Start}
\item In diesem Zweig wird der Ausdruck in Zeile 6 mit dem Label \emph{CI\_$n$} versehen
\item Die folgende Anweisung in Zeile 7 wird mit \emph{AI\_$n$} kenntlich gemacht
\item Der Ausdruck in diesem Zweig auf Zeile 15 bekommt die Kennzeichnung \emph{C\_$q$\_$q'$}
\item Die folgende Anweisung in Zeile 16 erhält das Label \emph{A\_$q$\_$q'$}
\end{itemize}

Mit diesen Kennzeichnungen ergibt sich nun die \emph{next}-Funktion der Semantik wie folgt:

\begin{tabular}{|c|c|}
  \hline
  $L$ & $\textrm{next}(L)$\\
  \hline
  CI\_$n$ & AI\_$n$\\
  AI\_$n$ & st\_$i$\\
  C\_$q$\_$q'$ & A\_$q$\_$q'$\\
  A\_$q$\_$q'$ & st\_$q'$\\
  \hline
\end{tabular}

Auch die erforderliche $g$-Funktion, die die Zweige einer \emph{If}-Anweisung angibt, kann somit hergeleitet werden als:

\begin{tabular}{|c|c|}
  \hline
  $L$ & $g(L)$\\
  \hline
  Start & $\{ \textrm{CI\_}0,\textrm{CI\_}1,\dots \}$\\
  st\_$q$ & $\{ \textrm{C\_}q\textrm{\_}0,\textrm{C\_}q\textrm{\_}1,\dots \}$\\
  \hline
\end{tabular}

Für den Ausführungsmodus(\emph{mode}) ergibt sich:

\begin{tabular}{|c|c|}
  \hline
  $L$ & $\textrm{mode}(L)$\\
  \hline
  Start & ilv\\
  CI\_$n$ & atm\\
  AI\_$n$ & atm\\
  st\_$q$ & ilv\\
  C\_$q$\_$q'$ & atm\\
  A\_$q$\_$q'$ & atm\\
  \hline
\end{tabular}

Zunächst ist es nützlich ein paar allgemeine Aussagen aufzustellen, die die Verifikation der Richtigkeit der Übersetzung erleichtern.
\begin{enumerate}
\item Es ist leicht einzusehen, dass für alle Prozesse die Umgebung $\phi_e$ gleich ist, da die Prozesse keine lokalen Variablen deklarieren.
\end{enumerate}

Um nun zeigen zu können, dass das definierte System $(\Sigma,D,V)$ mit der Semantik $(S,\llbracket \rrbracket_S,\alpha)$ bisimular zum übersetzten Promela Modell $tr(\Sigma,D,V)$ ist, müssen folgende Anforderungen an die Semantik gestellt werden:
\begin{enumerate}
\item Es muss eine Bijektion $i$ zwischen Zuständen $S$ und der Promela-Umgebung $\sigma_e$ existieren.
\item Die Definitionen $\llbracket \alpha \rrbracket_D$ müssen eine initiale Umgebung $\sigma_e^0$ definieren, die isomorph zum Initialzustand $\alpha$ ist: 
  \[ i(\alpha) = \sigma_e^0 \]
\item Befinden sich beide Systeme in isomorphen Zuständen, so wird der von $\llbracket \rrbracket_C$ erzeugte Ausdruck genau dann wahr, wenn es im abstrakten Modell einen entsprechenden Übergang zwischen den Zuständen gibt:
  \[ \begin{array}{rc}
    i(s) = \sigma_e \Rightarrow &
    (s,(p_1,\dots,p_i,\dots,p_N))\rightarrow (s',(p_1,\dots,p_i',\dots,p_N)) \\
    & \Leftrightarrow\\
    & exec(\llbracket \varphi(p_i')\rrbracket_C,\sigma_e)
  \end{array} \]
\item Die Anweisung, die von $\llbracket \rrbracket_A$ generiert wird, darf nie blockieren und muss isomorphe Zustände beibehalten:
  \[ \begin{array}{rc}
    i(s) = \sigma_e \Rightarrow &
    (s,(p_1,\dots,p_i,\dots,p_N))\rightarrow (s',(p_1,\dots,p_i',\dots,p_N))\\
    & \Leftrightarrow\\
    & \xymatrix { \left<\sigma_e,L\right> \ar[r]^-{\llbracket \varphi(p_i')\rrbracket_A} & \left<\sigma_e',L'\right> } \land i(\sigma_e) = s'
  \end{array} \]
\end{enumerate}
Nun kann man die Relation $\cong$ angeben, die Zustände des abstrakten Modells mit Zuständen des übersetzten Promela-Modells in Relation setzt.
Diese wird wie folgt definiert:
Zwei Zustände stehen genau dann in Relation, wenn ihre globalen Zustände isomorph sind und sich jeder Prozess des Promela-Modells am Label befindet, das mit dem Zustand im abstrakten Modell korrespondiert, oder sich am Label \emph{Start} befindet und der abstrakte Prozess im Zustand $\bot$ ist.
\[
\begin{array}{c}
  (s,(p_1,\dots,p_N))\cong \gamma\\
  \Leftrightarrow\\
  i(s)=\gamma(0).\sigma_e \land \forall i\in\{1\dots N\}: (\gamma(i).\sigma_l = \textrm{st\_}p_i \lor (\gamma(i).\sigma_l = \textrm{Start}\land p_i=\bot))
\end{array}
\]
Nun muss gezeigt werden, dass es sich bei der eben definierten Relation tatsächlich um eine Bisimulationsrelation handelt.
Hierfür muss nachgewiesen werden, dass es für jede Transition, die ein bisimularer Zustand durchführen kann, eine Transition des anderen Zustands gibt und die Zielzustände der beiden Transitionen auch wieder bisimular sind.

Betrachten wir also einen Zustand des abstrakten Modells $(s,(p_1,\dots,p_N))$ und einen Zustand des Promela-Modells $\gamma$.
Sind diese Zustände bisimular, so gilt nach Konstruktion
\[ i(s) = \gamma(0).\sigma_e \]
und für jeden Prozess $p_i$ entweder
\[ \gamma(i).\sigma_l = \textrm{st\_}p_i \]
oder
\[ \gamma(i).\sigma_l = \textrm{Start} \land p_i = \bot \]
Gilt der erste Fall, so
\[ \inference[IfDo-proc]{
  I(\textrm{st\_}p_i)\in If &
  \xymatrix{ \left<\gamma(i).\sigma_e,\textrm{C\_}p_i\textrm{\_}p_i'\right>\ar[r]^-{\llbracket \varphi(p_i')\rrbracket_C}_>{proc} & \left< x \right>}
  }
  {\left<\gamma(i).\sigma_e,\textrm{st\_}p_i\right>}
\]

