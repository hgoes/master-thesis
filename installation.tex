\chapter{Installation}
Hier wird beschrieben, wie das GTL-Tool installiert werden kann.
\section{Voraussetzungen}
Die folgenden Software-Pakete werden für die Kompilierung der Software benötigt:
\begin{itemize}
\item Der Haskell-Compiler \emph{GHC}\footnote{Glasgow Haskell Compiler}.
  Es bietet sich die Installation der "`Haskell Platform"' an, die neben dem Compiler noch benötigte Distributionstools und Bibliotheken mitbringt.
  Die Linux-Distributionen "`Ubuntu"', "`Debian"', "`Fedora"', "`Arch Linux"', "`Gentoo"' und "`NixOS"' bieten diese bereits über ihre interne Paketverwaltung an.
  Für andere Systeme ist die Plattform unter \url{http://hackage.haskell.org/platform/} erhältlich.
\item Die Haskell-Bibliotheken "`binary"' und "`bzlib"', welche beide über die Haskell-Pa\-ket\-ver\-wal\-tung "`hackage"' verfügbar sind und damit automatisch bei der Kompilierung heruntergeladen und installiert werden.
\end{itemize}
Für die Nutzung aller Features des GTL-Tools sind außerdem die folgenden Softwarekomponenten erforderlich:
\begin{itemize}
\item Der Model-Checker \emph{SPIN}, erhältlich unter \url{http://spinroot.com}.
  Die Software ist in C geschrieben und hat keine externen Abhängigkeiten und sollte daher auf fast jeder Plattform verfügbar sein.
\item Die BDD-Bibliothek \emph{CUDD}, die man unter \url{http://vlsi.colorado.edu/~fabio/CUDD/} in Quelltextform findet.
\item Für die Überprüfung der synchronen Komponenten ist das Entwicklungswerkzeug \emph{SCADE} erforderlich.
  Als einzige Komponente ist diese nicht frei verfügbar, sondern muss von der Firma "`Esterel Technologies"' (\url{http://esterel-technologies.com} lizensiert werden.
\end{itemize}
Soll die aktuellste Version der Quelltexte bezogen werden, so benötigt man außerdem die Versionsverwaltung \emph{git}, erhältlich unter \url{http://git-scm.com/}.
\section{Quelltext beziehen}
Für den Quelltext zu der Anwendung entpackt man entweder die mitgelieferten Archive, oder lädt den aktuellsten Entwicklungsstand der Pakete per \emph{git} herunter.
Der entsprechende Befehl lautet
\begin{lstlisting}[language=bash,mathescape=true]
git checkout https://github.com/hguenther/$name$.git
\end{lstlisting}
Wobei $name$ den Namen des Paketes bezeichnet.
Die zu herunterladenen Pakete sind:
\begin{itemize}
\item language-scade
\item language-promela
\item bdd
\item gtl
\end{itemize}
\section{Kompilieren}
In jedem Quelltext-Verzeichnis muss nun der Befehl
\begin{lstlisting}[language=bash]
cabal install
\end{lstlisting}
ausgeführt werden.
Hierbei ist zu beachten, dass das Paket "`gtl"' als letztes installiert werden muss, da es von allen anderen abhängt.
\section{Verwendung}
Das Resultat der oben angegebenen Installationsschritte ist die Anwendung "`gtl"', die im Plattform-abhängigen Anwendungsverzeichnis zu finden ist.
Um die Optionen und Parameter der Anwendung zu begutachten lässt sich die Anwendung mit dem Parameter "`\verb|--help|"' starten:
\begin{verbatim}
gtl --help
\end{verbatim}
Liegt eine GTL-Spezifikation im in Abschnitt \ref{sec:grammar} beschriebenen Format vor, so lassen sich die verschiedenen Verifikationsmodi wie folgt ausführen ("`\verb|spec.gtl| ist hier immer die GTL-Spezifikationsdatei):
\begin{itemize}
\item Die SCADE-Verifikation der einzelnen Komponenten wird mit
\begin{verbatim}
gtl -m local spec.gtl
\end{verbatim}
ausgeführt.
Die SCADE-Modelle werden dabei aus der Datei extrahiert, die im ersten Parameter des Modells übergeben wird.
Der Name in der SCADE-Datei wird über den zweiten Parameter festgelegt.
Der folgende Code gibt also an, dass sich der gesuchte SCADE-Knoten in der Datei "`\verb|ExampleCar.scade|"' befindet und dort den Namen "`\verb|Car|"' trägt.
\begin{lstlisting}[language=gtl]
model[scade] car("ExampleCar.scade","Car");
\end{lstlisting}
\item Um ein Promela-Modell zu erhalten, dass die Kontrakte nativ übersetzt, zu erhalten, wird
\begin{verbatim}
gtl -m native spec.gtl
\end{verbatim}
ausgeführt.
Das erzeugte Promela-Modell kann nun mit SPIN verifiziert werden.
\item Die dynamische BDD-Optimierung lässt sich wie folgt aufrufen:
\begin{verbatim}
gtl -m promela-buddy spec.gtl
\end{verbatim}
\item Die C-Integration der Komponenten wird durchgeführt mit
\begin{verbatim}
gtl -m native-c spec.gtl
\end{verbatim}
Man erhält hierbei ein Promela-Modell, dass den C-Code der Komponenten einbindet.
Die Kontrakte der Komponenten werden nicht verwendet.
Es müssen bei der Kompilierung des Verifikations-Binaries die Ordner, die den generierten C-Quelltext enthalten mit angegeben werden.


Hat man durch die Verifikation mit SPIN eines im zweiten oder dritten Punkt erzeugten GALS-Modells in Promela eine Fehlerspur "`\verb|error.gtltrace|"' erhalten, so kann man diese in der C-Integration verwenden, um zu überprüfen, ob es sich um einen echten Fehler oder eine Unterspezifikation der Kontrakte handelt, indem man
\begin{verbatim}
gtl -m native-c spec.gtl -t error.gtltrace
\end{verbatim}
verwendet.
\end{itemize}
