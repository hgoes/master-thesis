\chapter{Installation}
Hier wird beschrieben, wie die im Rahmen dieser Arbeit entwickelte Software installiert werden kann.
\section{Voraussetzungen}
Die folgenden Software-Pakete werden für die Kompilierung der Software benötigt:
\begin{itemize}
\item Der Haskell-Compiler \emph{GHC}\footnote{Glasgow Haskell Compiler}.
  Es bietet sich die Installation der "`Haskell Platform"' an, die neben dem Compiler noch benötigte Distributionstools und Bibliotheken mitbringt.
  Die Linux-Distributionen "`Ubuntu"', "`Debian"', "`Fedora"', "`Arch Linux"', "`Gentoo"' und "`NixOS"' bieten diese bereits über ihre interne Paketverwaltung an.
  Für andere Systeme ist die Plattform unter \url{http://hackage.haskell.org/platform/} erhältlich.
\item Die Haskell-Bibliotheken "`binary"' und "`bzlib"', welche beide über die Haskell-Pa\-ket\-ver\-wal\-tung "`hackage"' verfügbar sind und damit automatisch bei der Kompilierung heruntergeladen und installiert werden.
\end{itemize}
Für die Nutzung aller Features des GTL-Tools sind außerdem die folgenden Softwarekomponenten erforderlich:
\begin{itemize}
\item Der Model-Checker \emph{SPIN}, erhältlich unter \url{http://spinroot.com}.
  Die Software ist in C geschrieben und hat keine externen Abhängigkeiten und sollte daher auf fast jeder Plattform verfügbar sein.
\item Die BDD-Bibliothek \emph{CUDD}, die man unter \url{http://vlsi.colorado.edu/~fabio/CUDD/} in Quelltextform findet.
\item Für die Überprüfung der synchronen Komponenten ist das Entwicklungswerkzeug \emph{SCADE} erforderlich.
  Als einzige Komponente ist diese nicht frei verfügbar, sondern muss von der Firma "`Esterel Technologies"' (\url{http://esterel-technologies.com} lizensiert werden.
\end{itemize}
Soll die aktuellste Version der Quelltexte bezogen werden, so benötigt man außerdem die Versionsverwaltung \emph{git}, erhältlich unter \url{http://git-scm.com/}.
\section{Quelltext beziehen}
Für den Quelltext zu der Anwendung entpackt man entweder die mitgelieferten Archive, oder lädt den aktuellsten Entwicklungsstand der Pakete per \emph{git} herunter.
Der entsprechende Befehl lautet
\begin{lstlisting}[language=bash,mathescape=true]
git checkout https://github.com/hguenther/$name$
\end{lstlisting}
Wobei $name$ den Namen des Paketes bezeichnet.
Die zu herunterladenen Pakete sind:
\begin{itemize}
\item language-scade
\item language-promela
\item bdd
\item gtl
\end{itemize}
\section{Kompilieren}
In jedem Quelltext-Verzeichnis muss nun der Befehl
\begin{lstlisting}[language=bash]
cabal install
\end{lstlisting}
ausgeführt werden.
Hierbei ist zu beachten, dass das Paket "`gtl"' als letztes installiert werden muss, da es von allen anderen abhängt.
